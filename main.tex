\documentclass{article}

\usepackage[utf8]{inputenc}
\usepackage[T1]{fontenc}
\usepackage[french]{babel}
\usepackage[a4paper, margin=2cm]{geometry}
\usepackage{amsmath}
\usepackage{amsthm}
\usepackage{amssymb}
\usepackage{enumitem}
\usepackage{stmaryrd}
\usepackage{color}
\usepackage{parskip}
\usepackage{tikz,tkz-tab}
\usepackage{titlesec}
\usepackage{tocloft}
\usepackage{fancyhdr}
\usepackage[normalem]{ulem}
\usepackage[hidelinks]{hyperref}

\AtBeginDocument{\def\labelitemi{$\bullet$}}
\setlist[itemize]{leftmargin=.5cm}
\setlist[enumerate]{leftmargin=.7cm}

\theoremstyle{definition}
\newtheorem*{thm}{Théorème}
\newtheorem*{dfn}{Définition}
\newtheorem*{lmm}{Lemme}
\newtheorem*{rem}{Remarque}
\newtheorem*{res}{Résultat}
\newtheorem*{prop}{Proposition}
\newtheorem{csq}{Conséquence}[subsection]
\renewcommand{\thecsq}{\arabic{csq}}
\newtheorem*{cor}{Corollaire}
\newtheorem*{exo}{Exercice}
\newtheorem*{notation}{Notation}

\renewcommand{\thesection}{\arabic{section}}
\renewcommand{\thesubsection}{(\emph{\alph{subsection}})}

\titleformat*{\section}{\Large\bfseries}
\titleformat*{\subsection}{\large\bfseries}

\setlength{\cftsecnumwidth}{0.6cm}
\renewcommand{\cftsecpresnum}{}
\renewcommand{\cftsecaftersnum}{}

\addto\captionsfrench{
  \renewcommand{\contentsname}%
    {Plan du cours}%
}

\pagestyle{fancy}
\fancyhf{}
\renewcommand{\sectionmark}[1]{\markright{\thesection\ --\ \textsc{#1}}}
\renewcommand{\subsectionmark}[1]{}
\renewcommand{\headrulewidth}{1pt}
\rhead{Page \thepage}
\cfoot{}
\lhead{\rightmark}
\renewcommand{\headrulewidth}{1pt}




\DeclareMathOperator{\dom}{dom}
\DeclareMathOperator{\ord}{ord}
\DeclareMathOperator{\Vect}{Vect}
\DeclareMathOperator{\img}{Im}
\DeclareMathOperator{\Ker}{Ker}

\newcommand{\todo}[1]{{\color{red}À faire: #1}}

\begin{document}

~

\vspace{1cm}

\begin{center}
\thispagestyle{empty}
\textbf{\LARGE Compléments sur les polynômes} \\[1em]
\end{center}
\tableofcontents
\thispagestyle{empty}

\newpage
\setcounter{page}{1}
\section{Rappels}

$\mathbb K$ désigne un corps (donc commutatif). Un polynôme de $\mathbb K[X]$ s'écrit $P$ ou $P(X)$, l'évaluation en $a\in\mathbb K$ de $P$ se note $P(a)$. L'application \[
    P\in\mathbb K[X] \longmapsto (a\longmapsto P(a)\in\mathbb K)
\]
est un morphisme d'algèbres qui est injectif si $\mathbb K$ est injectif. Pour $P\in\mathbb K[X]\setminus \{0\}$, $a\in\mathbb K$ est une racine de $P$ ssi \[
    P(a)=0\iff (X-a)\;|\;P
\]
et $a$ est racine de multiplicité $\alpha \geq 1$ ssi \[
    \begin{cases}
        P(a)=P'(a)=\cdots=P^{(\alpha - 1)}(a)=0 \\
        P^{(\alpha)}(a)\neq 0
    \end{cases}
    \iff \begin{cases}
        (X-a)^\alpha \; |\; P\\
        (X-a)^{\alpha+1}\;\not|\;P
    \end{cases}
\]
Si $a$ n'est pas racine de $P$, on convient que $a$ est racine de multiplicité $0$.

\begin{notation} ~
    \begin{itemize}
        \item
            On note $Z_\mathbb K(P)$ l'ensemble des racines de $P$ dans $\mathbb K$
        \item On note $\mathrm{mult}(P, a)$ la multiplicité de la racine $a$ dans $P$
    \end{itemize}
\end{notation}

\begin{thm}
    Soit $P\in \mathbb K[X]\setminus \{0\}$. Le polynôme $P$ a au plus $\deg P$ racines dans $\mathbb K$.
    En particulier, le polynôme nul est le seul polynôme qui a une infinité de racines.
\end{thm}

\begin{thm}
    Soit $P\in\mathbb K[X]$.
    \begin{enumerate}
        \item \[
                P(X)=\sum_{k=0}^{+\infty} \frac{P^{(k)}(0)}{k!}X^k
            \]
        \item \[
                \forall a\in\mathbb K, \quad P(X)=\sum_{k=0}^{+\infty}\frac{P^{(k)}(a)}{k!}(X-a)^k
            \]
    \end{enumerate}
\end{thm}
\begin{proof} ~
    \begin{enumerate}
        \item L'application \[
                \psi: P\longmapsto P(X)-\sum_{k=0}^{+\infty}\frac{P^{(k)}(a)}{k!}(X-a)^k
            \]
            est linéaire nulle sur les vecteurs de la base canonique.
    \end{enumerate}
\end{proof}

\begin{rem}
    On montre de même que \[
        P(a)=\sum_{k=0}^{+\infty}\frac{P^{(k)}(X)}{k!}(a-X)^k
    \]
\end{rem}

\section{Factorisation des polynomes}

\subsection{Rappels}

\begin{thm}
    Soit $P\in\mathbb K[X]$, $a_1, \cdots, a_p$ deux à deux distincts et $n_1, \cdots, n_p\in\mathbb N^\star$. Si $a_1, \cdots, a_p$ sont des racines de multiplicités $n_1, \cdots, n_p$ de $P$, alors \[
        (X-a)^{n_1}\cdots (X-a_p)^{n_p}\;|\;P
    \]
    En particulier, si $\deg P=\sum_i n_i$ alors il existe $\lambda\in\mathbb K^\star$ tel que \[
        P(X)=\lambda(X-a_1)^{n_1}\cdots (X-a_p)^{n_p}
    \]
\end{thm}

\subsection{Théorème de d'Alembert-Gauss}

\begin{thm}
    Tout polynome non constant de $\mathbb C[X]$ admet au moins une racine
\end{thm}

\begin{proof}[Preuve de Gauss]
    \begin{itemize}
        \item $P\in \mathbb C[X]$ non constant a même racine que $\frac1{\deg P}P$ donc on suppose $P$ unitaire.
        \item Si $P(X)=a_0+\cdots, a_{n-1}X^{n-1}+X^n$, et $M=\max |a_i|$, alors pour $|z|\geq M+2$, \begin{align*}
                |P(z)|&\geq |z|^n-M\frac{|z|^n-2}{|z|-1}=\frac{|z|^{n+1}-|z|^n-M|z|^n+M}{|z|-1}\\
                      &\geq \frac{|z|^n}{|z|-1} \left( |z|-1-M \right)\geq \frac{|z|^n}{|z|-1}\left(1-\frac M{M+1}\right)\geq \frac{(M+2)^n}{M+1}>M
            \end{align*}
        \item \begin{align*}
                \inf_{\mathbb C}|P|&=\min\left(\underbrace{\inf_{|z|>M+2}|P(z)|}_{>M}, \underbrace{\inf_{|z|\leq M+2}|P(z)|}_{\leq |P(0)|\leq M}\right) \\&=\inf _{|z|\leq M+2}|P(z)|
            \end{align*}
            Par compacité il existe $z_0\in\mathcal B_f(0, M+2)$ tel que $\inf_{\mathbb C}|P|=|P(z_0)|$.
        \item Si $P(z_0)\neq 0$ alors il existe $k\geq 1$ tel que \begin{align*}
                P(z_0+h)&=P(z_0)+\frac{P^{(k)}(z_0)}{k!}h^k+o(h^k), \quad P(z_0)\neq 0 \\
                        &=P(z_0)\left(1+h^k\frac{P^{(k)}(z_0)}{P(z_0)k!}+o(h^k) \right)
            \end{align*}
        \item En écrivant \[
                \frac{P^{(k)}(z_0)}{P(z_0)k!}=\rho e^{i\theta}, \quad h=\epsilon e^{-i\frac{\theta+\pi}k}
            \]
            on obtient \[
                P(z_0+\epsilon e^{-i\frac\theta k})=\underbrace{P(z_0)\underbrace{\left(1 \epsilon^k \rho + o(\epsilon^k)\right)}_{|\;\;|<1}}_{|\;\;|<|P(z_0)|}
            \]
            absurde.
    \end{itemize}
\end{proof}

\section{Polynômes sous contraintes}

\subsection{Contraintes sur les zéros}

\begin{exo}
    Si $Z_{\mathbb C}(P)=Z_{\mathbb C}(Q)$ et $Z_{\mathbb C}(P+1)=Z_{\mathbb C}(Q+1)$ pour $P,Q$ non constants de $\mathbb C[X]$, alors $P=Q$
\end{exo}

On suppose par l'absurde que $P\neq Q$ et par exemple $n=\deg P\geq \deg Q=m$. On note $u_1, \cdots, u_r$ les racines de $P$ (et donc de $Q$), $v_1, \cdots, v_s$ les racines de $P+1$ (donc de $Q+1$). Les $u_i$ et les $v_j$ sont tous distincts, donc \[
    \deg P=n\geq \deg(P-Q)\geq r+s \quad\text{car }P-Q\text{ s'annule en $u_i$ et $v_j$ et }P-Q\neq 0
\]
On a $P'=(P+1)'$ done $P'$ a $n-r$ racines dans les $u_i$ avec multiplicité, et $n-s$ dans les $n_j$ donc \[
    n-r+n-s\leq \deg P'=n-1\iff n-1\leq r + s
\]
Or $r+s\leq n$ absurde et donc $P = Q$

\subsection{Coefficients unitaires}

\begin{exo}
    Déterminer les $P\in\mathbb R[X]$ non constants avec les coefficients $\pm1$ et dont toutes les racines sont réelles.
\end{exo}

\begin{proof}[Résolution]
    On note $P$ un polynôme de degré $n\geq 2$ qui convient et quitte à changer $P$ en $-P$, $P$ est unitaire.
    \begin{itemize}
        \item $P(X)=X^n+p_{n-1}X^{n-1}+\cdots +p_0$
        \item $r_1, \cdots, r_n$ les racines (non nulles car $p_0\neq 0$) donne \[
                r_1^2+\cdots +r_n^2=p_{n-1}^2-2p_{n-2}>0 \implies p_{n-2}=-1 \text{ et }r_1^2+\cdots+r_n^2=3
            \]
        \item \[
                p_0 X^n P \left( \frac1X \right)=X^n+p_0p_1X^{n-1}+p_0p_2X^{n-2}+\cdots
            \]
            vérifie les mêmes propriétés donc \[
                \frac 1{r_1^2} +\cdots +\frac1{r_n^2}=3
            \]

        \item \[
                \sum_{i=1}^n\underbrace{\left(r_i^2+\frac1{r_i^2}\right)}_{\geq 2}=6 \implies n\leq 3
            \]
    \end{itemize}
    On doit étudier $12$ polynômes
    \begin{center}
        \begin{tabular}{lll}
            \sout{$X^2+X+1$} & \sout{$X^3+X^2+X+1$} & \sout{$X^3-X^2+X+1$} \\
            $X^2+X-1$ & \sout{$X^3+X^2+X-1$} & \sout{$X^3-X^2+X-1$} \\
            \sout{$X^2-X+1$} & \sout{$X^3+X^2-X+1$} & $X^3-X^2-X+1$ \\
            $X^2-X-1$ & $X^3+X^2-X-1$ & \sout{$X^3-X^2-X-1$}
        \end{tabular}
    \end{center}
\end{proof}

\subsection{Équation fonctionnelle}

\begin{exo}
    Déterminer les polynômes $P\in\mathbb C[X]$ tels que $P(X)P(X+2)+P(X^2)=0$
\end{exo}

\begin{proof}[Résolution] $-1$ et $0$ sont les deux polynômes constants solutions.

    Soit $P$ non constant solution et $a$ une racine. $P(a)=0\implies P(a^2)=0$ donc $P$ admet une infinité de racines sauf si $|a|=1$ ou $a=0$.

    Si $a=0$ est racine alors $P(-2)P(0)+P(4)=0$ donc $4$ est racine ce qui est absurde. Donc $|a|=1$.

    Si $a$ est racine alors $(a-2)^2$ racine donc $|a-2|=1=|a|$ donc $a=1$ (intersection de deux cercles), donc $1$ seule racine de $P$ et donc pour un $k\geq 1$, \[
        P=\lambda(X-1)^k
    \]
    En calculant, on trouve $\lambda^2+\lambda=0$ et $\lambda \neq 0$ donc $\lambda = -1$
\end{proof}

\begin{exo}
    Trouver les polynômes $P\in\mathbb R[X]$ tels que \[
        (X+1)P(X-1)-(X-1)P(X)
    \]
    est un polynôme constant
\end{exo}

\begin{proof}[Résolution]
    On suppose que $P$ convient et $(X+1)P(X-1)-(X-1)P(X)=\lambda$. \begin{itemize}
        \item $2P(-1)=\lambda$ et $2P(0)=\lambda$ donc $P(0)=P(-1)$
        \item $Q(X)=P(X)-\frac\lambda 2$ s'annule en $-1$ et $0$ donc $Q(X)=X(X+1)A(X)$ ie $P(X)=X(X+1)A(X)+\frac\lambda 2$.
        \item \[
                (X+1)((X-1)X A(X-1)+\frac\lambda2)-(X-1)X(X+1)A(X)-(X-1)\frac\lambda2=\lambda
            \]
            donc \[
                X(X-1)(X+1)(A(X-1)-A(X))=0 \implies A(X)=A(X-1)
            \]
            et donc $A$ est un polynôme constant. Donc $P$ est de la forme $P(X)=aX(X+1)+\mu, \quad a, \mu\in\mathbb R$
    \end{itemize}
\end{proof}

\section{Liens coefficients et racines}

\subsection{Relations de Viète}

Pour $x_1, \cdots, x_n\in\mathbb C$ on note: \[
    \sigma_i(x_1, \cdots, x_n)=\sum_{\substack{I\subset \llbracket 1, n\rrbracket\\|I|=i}}\prod_{l\in I}x_l
\]
Soit $P(X)=a_nX^n+\cdots+a_0$ un polynôme de degré $n$ et $x_1, \cdots, x_n$ ses racines dans $\mathbb C$. \begin{itemize}
    \item \begin{align*}
            P(X)=a_n(X-x_1)\cdots (X-x_n)=a_n(X^n-\sigma_1 X^{n-1}+ \sigma_2 X^{n-2}+\cdots (-1)^n\sigma_nX^{n-n})
        \end{align*}
    \item Ainsi, \[
            \forall i\in\llbracket 1, n\rrbracket, \quad (-1)^i\sigma_i=\frac{a_{n-i}}{a_n}
        \]
\end{itemize}
En particulier, on en déduit \[
    x_1+\cdots+x_n = -\frac{a_{n-1}}{a_n}
\]
et \[
    x_1\times \cdots\times x_n=(-1)^n\frac{a_0}{a_n}
\]

\subsection{Somme de Newton}

On note $P(X)=a_nX^n+\cdots+a_0$ un polynôme de $\mathbb C[X]$, $x_1, \cdots, x_n$ ses racines. On note $S_k=x_1^k+\cdots + x_n^k$ appelée $k-$ième somme de Newton.

Pour $k\geq n$, \begin{align*}
    a_nS_k+a_{n-1}S_{k-1}+\cdots + a_0S_{k-n}&=a_nx_1^k+\cdots +a_0x_1^{k-n}+\cdots +a_nx_n^k+\cdots +a_0x_n^{k-n} \\ &=x_1^{k-n}P(x_1)+\cdots + x_n^{k-n}P(x_n)=0
\end{align*}

Avec les relations de Viète, pour $k\geq n$, \[
    \underbrace{\sigma_0}_{=\; 0 \text{ convention}} S_k-\sigma_1S_{k-1}+\cdots + (-1)^nS_{k-n}=0
\]

Il est possible de calculer $S_1, \cdots, S_{n-1}$ par des formules de récurrence. \begin{itemize}
    \item $S_0=n$ clair. On va montrer que \[
            k\sigma_k=\sum_{i=1}^k(-1)^{i-1}\sigma_{k-i}S_i
        \]
    \item On note $P$ le polynôme $P/\dom P$\[
            P(x)=\prod_{i=1}^n(x-x_i)=\frac1{a_n}\sum_{k=0}^na_kx^k=\sigma_{k=0}^n\sigma_{n-k}(-1)^{n-k}x^k
        \]
        d'où \[
            x^nP\left(\frac1x\right)=\underbrace{\prod_{k=1}^n(1-xx_k)}_{Q(x)}=\sum_{k=0}^n\sigma_{n-k}(-1)^{n-k}x^{n-k}=\sum_{k=0}^n\sigma_k(-1)^kx^k
        \]
    \item \begin{align*}
            \sum_{k=0}^n k\sigma_k(-1)^kx^k&= x\sum_{k=1}^n \frac{Q(x)}{1-xxk}(-x_k) \\
                                           &=-\sum_{k=1}^n\frac{xx_k}{1-xx_k}Q(x) \\
                                           &=-\left(\sum_{k=1}^n\sum_{i\geq 0}(xx_k)^{i+1}\right)Q(x) \\
                                           &=-\sum_{i\geq 0}x^{i+1}S_{i+1}\sum_{l=0}^n(-1)^k\sigma_kx^k
        \end{align*}

        En identifiant le coefficient de $x^k$, \begin{align*}
            (-1)^kk\sigma_k=-\sum_{l=0}^{k-1}(-1)^l\sigma_lS_{k-l} \iff k\sigma_k &=-\sum_{l=0}^{k-1}(-1)^{k-l}\sigma_lS_{k-l} \\
                                                                                  & =\sum_{l=1}^k(-1)^{l-1}\sigma_{k-l}S_l
        \end{align*}
\end{itemize}

\section{Polynômes scindés sur $\mathbb R$}

\subsection{Une CNS}

\begin{exo}
    Montrer que $P\in\mathbb R[X]$ de degré $n$ est scindé sur $\mathbb R$ ssi \[
        \forall z\in\mathbb C, \quad |P(z)|\geq |\dom(P)|\times |\Im(z)|^n
    \]
\end{exo}

\begin{proof}[Résolution]
    \emph{($\Longrightarrow$)} clair, les racines complexes de $P$ ont une partie imaginaire nulle.

     \emph{($\implies$)} On écrit $P(X)=\dom(P)(X-a_1)\cdots (X-a_n), \quad a_i\in\mathbb R$. En injectant $z=a+ib$, \[
        |P(z)|=|\dom P|\times |a-a_1+ib| \cdots |a-a_n+ib|\geq |\dom P| |b|^n
    \]
\end{proof}

\subsection{Opérations stabilisant $\mathcal S$}

On note $\mathcal S$ l'ensemble des polynômes scindés de $\mathbb R[X]$. On note \[
    P(X)=\lambda (X-a_i)^{\alpha_1}\cdots (X-a_p)^{\alpha_p}, \quad a_1<\cdots <a_p
\]
et on a les propriétés de stabilité suivantes.

\vspace{.2cm}

 Rolle donne que $P'$ s'annule au moins une fois sur $]a_i, a_{i+1}[$ pour $1\geq i < p$. Il y a alors au total $(\alpha_1-1)+\cdots+(\alpha_p-1)+p-1=n-1$ racines réelles (avec multiplicité) et $\deg P'=n-1$ donc $P'$ est scindé et $\mathcal S$ est stable par dérivation.

\vspace{.2cm}

 Pour $a\in\mathbb R$ fixé, on note $g(x)=e^{ax}P(x)$, $g'(x)=(P'(x)+aP(x))e^{ax}$. $P'+aP$ s'annule en $a_i$ avec multiplicité $\geq \alpha_i-1$. Rolle donne que $g'$ s'annule au moins une fois sur $]a_i, a_{i+1}[$ donc $P'+aP$ aussi. Ce polynôme a $n-1$ racines réelles et est de degré $n$, donc par division la dernière racine est réelle. Donc $P'+aP\in\mathcal S$

\vspace{.2cm}

 Plus généralement, si $D$ est l'opérateur de dérivation polynomiale, pour $Q\in\mathcal S$ alors 
\begin{itemize}
    \item 
\[
    Q(X)=\lambda (X-b_1)\cdots (X-b_p)
\]
\item \[
        Q(D)=\lambda(D-b_1\mathrm{id})\circ \cdots \circ (D-b_p \mathrm{id})
    \]
\end{itemize}
et pour $P\in\mathcal S$, $Q(D)(P)\in\mathcal S$.

\subsection{Faisceau de polynômes à racines entrelacées (ENS)}

\begin{exo}
    On note $P, Q\in\mathbb R[X]$ deux polynômes scindés à racines simples distinctes tels qu'entre deux racines de l'un, il y en aie au moins une de l'autre. Montrer que $\lambda P+\mu Q$ est scindé sur $\mathbb R$
\end{exo}

\begin{proof}[Résolution]
    On suppose que $P$ a la plus petite racine, on note $x_1 <\cdots<x_n$ les racines de $P$. Sur $]x_1, x_2[$, il y a au moins une racine de $Q$ et au plus une (sinon $x_1$ et $x_2$ ne sont pas consécutives) donc les $n-1$ premières racines de $Q$ satisfont \[
        x_1<y_1<\cdots <x_{n-1}<y_{n-1}<x_n
    \]
    Il y a lors deux cas: $Q$ possède une racine $y_n>x_n$, ou $Q$ a $n-1$ racines réelles. On se place dans le premier cas (l'autre se traite similairement).
    
    Comme $\lambda, \mu\in\mathbb R$ sont quelconques, on peut supposer $P$ et $Q$ unitaires donc \[
        P(X)=(X-x_1)\cdots (X-x_n) \qquad Q(X)=(X-y_1)\cdots (X-y_n)
    \]
    Pour $\lambda, \mu\neq 0$, les racines réelles de $\lambda P+\mu Q$ sont différentes des $x_i, y_i$, donc sont les mêmes que celles de $\frac{\lambda P+\mu Q}{PQ}$ sur $\mathbb R\setminus \{x_i, y_i\}$. \[
        \varphi(x)=\frac{\lambda P(x)+\mu Q(x)}{P(x)Q(x)}=\frac\lambda{Q(x)}+\frac\mu{P(x)}
    \]
    Si $\lambda, \mu>0$, alors $\varphi$ s'annule $n$ fois car il y a changement de signe entre $]x_i, y_i[$ (les polynômes sont unitaires donc on connaît exactement les signes de $P$ et $Q$). Si $\lambda>0>\mu$, on trouve $n-1$ racines réelles de la même manière, et par division la dernière est aussi réelle. Les autres cas se traitent de la même manière.
\end{proof}

\section{Valeurs des polynômes}

\subsection{Polynôme à valeurs entières}

On va caractériser les polynômes de $\mathbb C[X]$ tels que $P(\mathbb Z)\subset \mathbb Z$

On définit \[
    \binom X0=1\qquad \text{ et }\qquad \forall n\geq 1,\quad \binom Xn=\frac1{n!}X(X-1)\cdots(X-n+1)
\]

\begin{lmm}
    Pour tout $n\in\mathbb N$ et pour tout $x\in\mathbb Z$, \[
        \binom xn\in\mathbb Z
    \]
\end{lmm}

\begin{proof}
    Pour $0\leq x\leq n-1$, $\binom xn=0$. Pour $x\geq n$, $\binom xn$ est un coef du binôme. Pour $x<0$, $(-1)^n\binom xn$ est un coef du binôme.
\end{proof}

 On introduit \[
    \Delta : P\in\mathbb K[X]\longmapsto P(X+1)-P(X)
\]
Pour $n\geq 1$, \[
    \Delta\binom Xn=\binom X{n-1}\qquad\text{ et }\qquad \Delta\binom X0=0
\]
et $\displaystyle \left(\binom Xn\right)_{n\geq 0}$ est une base de $\mathbb C[X]$ (degré échelonné). On suppose que $P\in\mathbb C[X]$ vérifie $P(\mathbb Z)\subset \mathbb Z$. On note $a_0, \cdots, a_n$ les coefficients de la décomposition de $P$ dans cette base. Alors \[
    P(0)=a_0\in\mathbb Z, \quad a_1=\Delta P(0)\in\mathbb Z, \cdots, \quad a_n=\Delta^nP(0)\in\mathbb Z
\]

Ainsi, $a_0, \cdots, a_n\in\mathbb Z$ et la réciproque est évidente. On a ainsi établi \[
    P(\mathbb Z)\subset \mathbb Z\iff \exists n\in\mathbb N, \exists, a_0, \cdots, a_n\in\mathbb Z, P(X)=\sum_{k=0}^na_k\binom Xk
\]

\begin{rem}
    On a aussi établi la formule de Gregory \[
        P(X)=\sum_{k\geq 0}\Delta^kP(0)\binom Xk
    \]
\end{rem}

\begin{exo}[ENS]
    On note $P\in\mathbb R[X]$ un polynôme de degré $n$ tel que $P(0), P(1), P(4), \cdots, P(n^2)\in\mathbb Z$. Montrer que $\forall a\in\mathbb Z, P(a^2)\subset \mathbb Z$.
\end{exo}

\begin{proof}[Résolution]
    $Q(X)=P((X-n)^2)$ de degré $2n$ envoie $2n+1$ valeurs de $\mathbb Z$ dans $\mathbb Z$ donc a des coefficients entiers dans la base explicitée précédemment donc est à valeurs dans $\mathbb Z$ ce qui conclut.
\end{proof}

\subsection{Polynôme stabilisant le cercle unité}

\begin{exo}
    Déterminer les polynômes $P\in\mathbb C[X]$ tels que $\mathbb U$ est stable par $P$.
\end{exo}

\begin{proof}[Résolution]
    Le cas des polynômes constants est immédiat. On note $P(z)=a_0+\cdots +a_nz^n$. On a \[
        P(e^{i\theta})\overline{P(e^{i\theta})}=1\implies P(e^{i\theta})\overline P(e^{-i\theta})=1 \implies P(e^i\theta)\overline P(e^{-i\theta})e^{in\theta}=e^{in\theta}
    \]
    Si on note $Q(X)=\overline P\left(\frac1X\right)X^n=\bar a_0X^n+\cdots +\bar a_n$ alors $PQ-X^n$ s'annule sur $\mathbb U$ donc est nul donc $P(X)\;|\;X^n$, $\deg P=n$ et $P=\lambda X^n$, où $\lambda=P(1)\in\mathbb U$. La réciproque est claire.
\end{proof}

\begin{exo}
    Variantes \begin{itemize}
        \item $P(\mathbb R)\subset \mathbb R[X]$. Solution: Formule de Gregory, interpolation de Lagrange, polynôme conjugué, ...
        \item $P(\mathbb Q)\subset \mathbb Q$. Solution: Formule de Gregory
        \item $P(\mathbb Q)\subset \mathbb R\setminus \mathbb Q$ et $P(\mathbb R\setminus \mathbb Q)\subset \mathbb Q$
        \item $P(\mathbb Q)=\mathbb Q$
    \end{itemize}
\end{exo}

\begin{proof}[Résolution]
    Résolution pour $P(\mathbb Q)=\mathbb Q$. Déjà $P\in\mathbb Q[X]$ (par la $2$-ième variante), donc il existe $d$ entier positif minimal tel que $Q=dP\in\mathbb Z[X]$. On se donne $v$ premier avec les coefs $a_i$ de $Q$ et $d$ et $v$ nombre premier. Il existe $\frac ab$ tel que $P(\frac ab)=\frac1v$ donc \[
        v(a_na^n+\cdots +a_0b^n)=db^n \implies v\;|\; b^n\implies v\;|\; b
    \]
    et si $n\geq 1$ alors $v^2\;|\;db^n$ et donc $v|a_na^n+\cdots$ donc $v\;|\;a$ absurde. Par suite, $n=1$, et la réciproque est évidente.
\end{proof}

\subsection{Polynômes positifs}

On va caractériser les polynômes tq $P(\mathbb R_+)\subset \mathbb R_+$.
On pose \[
    \mathcal E=\{P\in\mathbb R[X] \;/\; \exists A, B\in\mathbb R[X], P=A^2+XB^2\}
\]
et on note $\mathcal S$ l'ensemble des polynômes qui conviennent. On a clairement $\mathcal E\subset \mathcal S$, puis $\mathcal E$ est stable par produit car \[
    (A^2+XB^2)(A_1^2+XB_1^2)=(AA_1+XBB_1)^2+X(BA_1+AB_1)^2
\]
Soit $P\in\mathbb S$. Dans la décomposition de $P$ en irréductibles de $\mathbb R[X]$, si on note $\alpha_1, \cdots, \alpha_p$ les puissances des facteurs de degré $1$, on a $(X-a_i)$ change de signe si $\alpha_i$ impair et dans ce cas $a_i<0$ et $X-a_i=\sqrt{-a_i}^2+X1^2\in\mathcal E$. Les autres termes sont dans $\mathcal E$ et donc $\mathcal S=\mathcal E$.

\section{Localisation des racines}

\subsection{Théorème de Gauss-Lucas}

On note $P\in\mathbb C[X]$ avec $n=\deg P\geq 2$. On va montrer que si $z_1, \cdots, z_n$ sont les racines de $P$ (répétées avec multiplicités) alors les racines de $P'$ sont dans l'enveloppe convexe de $z_1, \cdots, s_n$. On a:
\[
    \frac{P'}P(z)=\frac{\alpha_1}{z-z_1}+\cdots +\frac{\alpha_p}{z-z_p}
\]
donc si $z$ est une racine de $P'$ différente des $z_1, \cdots, z_n$, alors \[
    \alpha_1\frac{z-z_1}{|z-z_1|^2}+\cdots+\alpha_p\frac{z-z_p}{|z-z_p|^2}=0 \iff z \left( \frac{\alpha_1}{|z-z_1|^2}+\cdots+\frac{\alpha_p}{|z-z_p|^2} \right)=\frac{\alpha_1}{|z-z_1|^2}z_1+\cdots+\frac{\alpha_p}{|z-z_p|^2}z_p
\]
Donc $z$ est bien un barycentre de $z_1, \cdots, z_n$.

\subsection{Borne de Cauchy}

On note $P(z)=z^n+a_{n-1}z^{n-1}+\cdots+a_0\in\mathbb C[X]$ et on pose $M=\max_{i\leq n-1}\limits |a_i|$. On note $z\in\mathbb C$ tel que $|z|>M+1$. Par l'absurde si $P(z)=0$ alors \begin{align*}
    |z^n|=|a_{n-1}z^{n-1}+\cdots +a_0|\leq M\frac{|z|^n-1}{|z|-1} &\implies |z|^{n+1}-|z|^n\leq M|z|^n-M \\ &\implies |z| |z|^n\leq (M+1)|z|^n \\ &\implies |z|\leq M+1\quad\text{ absurde }
\end{align*}
Bilan les racines de $P$ sont dans le disque $D(0, M+1)$

\subsection{Enestrom-Kakeya}

\begin{res}[Cauchy]
    Si $P(x)=x^n-b_{n-1}x^{n-1}-\cdots -b_0$ est un polynôme tel que les $b_i$ sont positifs non tous nuls alors $P$ a une seule racine $a$ dans $\mathbb R_+^\star$, $a$ est racine simple et $\mathbb Z_{\mathbb C}(P)\subset \mathcal B_f(0, a)$.
\end{res}

\begin{proof}
    Pour $x>0$, $f(x)=-\frac{P(x)}{x^n}=\frac{b_0}{x^n}+\cdots + \frac{b_{n-1}}{x}-1$ est décroissante.
    
    \begin{center}
        \begin{tikzpicture}
            \tkzTabInit{$x$ / 1 , $f(x)$ / 1}{$0$, $a$, $+\infty$}
            \tkzTabVar{+/ $+\infty$, R /, -/ $-1$}
        \end{tikzpicture}
    \end{center}
    Donc $f$ a une unique racine $a>0$, puis \[
        f'(a)=-n\frac{b_0}{x^{n+1}}-\cdots-\frac{b_{n-1}}{x^2}<0 \implies P'(a)\neq 0
    \]
    donc $a$ est racine simple de $P$. Si $z_0$ est une racine de $P$ dans $\mathbb C$ alors $P(z_0)=0$ donc \[
        |z_0^n|=|b_{n-1}z_0^{n-1}\cdots b_0|\leq b_{n-1}|z_0^n|+b_0
    \]
    donc $P(|z_0|)\leq 0$ et $f(|z_0|)\geq 0$ donc $|z_0|\leq a$
\end{proof}

\begin{res}[Enestrom-Kakeya]
    Si $P(x)=a_{n-1}x^{n-1}+\cdots +a_0$ est un polynôme à coefficients strictement positifs, alors si $e$ est une racine complexe de $P$, \[
        \delta=\min_{i\leq n-1}\frac{a_{i-1}}{a_i} \leq |z|\leq \max_{i\leq n-1}\frac{a_{i-1}}{a_i}=\gamma
    \]
\end{res}

\begin{proof}
    On considère le polynôme \begin{align*}
        Q(x)=(x-\gamma)P(x)&=a_{n-1}x^{n}+(a_{n-2}-\gamma a_{n-1})x^{n-1}+\cdots +(a_0-\gamma a_1)x -\gamma a_0\\ &= a_{n-1}\underbrace{\left(x^n-\frac{\gamma a_{n-1}-a_{n-2}}{a_{n-1}}x^{n-1} -\cdots -\frac{\gamma a_1-a_0}{a_{n-1}}x-\frac{\gamma a_0}{a_{n-1}}\right)}_{\text{vérifie les hypothèses du résultat précédent donc a une unique racine réelle $>0$, $\gamma$}}
    \end{align*}
    d'où la seconde inégalité. La première s'obtient en remarquant que $z$ racine de $P$ entraîne $z$ non nul et $\frac 1z$ racine de $X^{n-1}P\left(\frac1X\right)$ auquel on applique le même raisonnement.
\end{proof}

\subsection{Disques de Gershgorin}

On se donne $A\in\mathcal M_n(\mathbb C)$ et $\lambda\neq a_{1, 1}, \cdots, a_{n, n}$. On note $D=\mathrm{diag} (a_{1, 1}, \cdots, a_{n, n})$ et $E=A-D$. Si $\lambda$ est valeur propre, alors \[
    A_\lambda=\underbrace{A-\lambda I_n}_{\text{non inversible}}=\underbrace{D-\lambda I_n}_{\text{inversible}}+E
\]
Puis $\exists X\in\mathcal M_{n, 1}(\mathbb C), X\neq 0 / A_\lambda X=0$ donc \begin{align*}
    (D-\lambda I_n)X+EX=0&\iff X=-(D-\lambda I_n)^{-1}EX \\&\iff \forall i\in\llbracket 1, n\rrbracket, \quad x_i=-\frac1{a_{i, i}-\lambda}\sum_{k\neq i}a_{i, k}x_k
\end{align*}
On considère un indice $i$ tel que $|x_i|$ est maximal et donc \[
    |a_{i, i}-\lambda|=|\sum_{k\neq i}a_{i, k}\frac{x_k}{x_i}\leq \sum_{k\neq i}|a_{i, k}|=R_i
\]
et donc \[
    \lambda\in\mathcal D(a_{i, i}, R_i) \qquad \text{ reste vrai si } \lambda=a_{i, i}
\]

On a donc \[
    \mathrm{Sp}_{\mathbb C}(A)\subset \bigcup_{i=1}^n \underbrace{\mathcal D(a_{i, i}, R_i)}_{\text{disques de Gershgorin}}
\]

\section{Arithmétique des polynômes}

\subsection{Division euclidienne}

\begin{exo}
    Effectuer la division euclidienne de $X^n-1$ par $X^m-1$, pour $n, m\geq 1$
\end{exo}

\begin{proof}[Résolution]
    On note $(u_k)$ la suite définie par \[
        \begin{cases}
            u_0=n, u_1=m \\ u_{k+1} \quad \text{ reste de la division euclidienne de $u_{k-1}$ par $u_k$}
        \end{cases}
    \]
    tant que $u_k\neq 0$ et on note \[
        \begin{cases}
            U_0=X^n-1, U_1=X^m-1\\ U_{k+1}\quad \text{ reste de la division euclidienne de $U_{k-1}$ par $U_k$}
        \end{cases}
    \]
    tant que $U_k\neq 0$
    
    On vérifie par récurrence que $U_k=X^{u_k}-1$. Les cas $k=0, 1$ sont immédiats. On a $u_{k+1}+au_k=u_{k-1}$ donc par HR si $U_k\neq 0$ (ie $u_k\neq 0$), alors \[
        U_{k-1}=U^{u_{k+1}}X^{au_k}-1=\underbrace{X^{u_{k+1}}-1}_{\deg < \deg U_k}+\underbrace{X^{u_{k+1}}\left(X^{au_k}-1\right)}_{\text{divisible par } U_n}
    \]
    d'où le résultat. On en déduit que les deux suites s'arrêtent en même temps et donc \[
        U_0\land U_1=U_{k-1}=X^{m\land n}-1
    \]
\end{proof}

\subsection{Théorème chinois}

\begin{exo}
On se donne $P_1, \cdots, P_n$ deux à deux premiers entre eux et $R_1, \cdots, R_n\in\mathbb C[X]$. Montrer qu'il existe $Q_1, \cdots, Q_n$ tels que \[
    P_1Q_1+R_1=\cdots=P_nQ_n+R_n
\]
\end{exo}

\begin{proof}[Résolution]
    C'est une application directe du théorème chinois (dans un anneau principal): si $d_i=\deg P_i-1$ et $N=-1+\sum d_i$, alors \[
        \varphi: P\in \mathbb C_N[X]\longmapsto (P\mod P_1, \cdots, P\mod P_n)\in\mathbb C_{d_1}[X]\times \cdots \times \mathbb C_{d_n}[X]
    \]
    est un isomorphisme d'algèbres.
\end{proof}

\subsection{Théorèmes de Mason, Snyder, et Fermat}
\begin{lmm}[Snyder]
    Si $P\in\mathbb C[X]$ non nul, alors \[
        \deg P=\deg \left(P\land P'\right)+n_0(P)
    \]
    où $n_0(P)=\#Z_{\mathbb C}(P)$. 
\end{lmm}

\begin{proof}
    Le cas $P$ constant est immédiat. On suppose maintenant $P$ non constant: \[
        P(X)=c(X-a_1)^{\alpha_1}\cdots (X-a_p)^{\alpha_p}
    \]
    donc \[
        P\land P'=(X-a_1)^{\alpha_1-1}\cdots (X-a_p)^{\alpha_p-1}
    \]
\end{proof}

\begin{lmm}[Mason]
    Si $A, B, C$ sont des polynômes complexes deux à deux premiers entre eux avec $A+B=C$ et $(A', B', C')\neq (0, 0, 0)$, alors \[
        \deg C\leq n_0(ABC)-1
    \]
\end{lmm}

\begin{proof}
Comme $A=C-B$ et $B=C-A$ les rôles de $A, B, C$ sont symétriques. On peut supposer $A$ non constant. On a alors \[
    \left|\begin{matrix}A&C\\A'&C'\end{matrix}\right|=\left|\begin{matrix}A&B\\A'&B'\end{matrix}\right|
\]
donc $AC'-CA'=AB'-BA'$ donc \[
    A\land A', \quad B\land B', \quad C\land C'\quad\Big|\quad AC'-A'C=AB'-BA'
\]
Si $AC'-A'C=0$ alors $AC'=A'C$ et $C\;|\;C'$ donc $C=0$, et de même $B=0$ donc $A=0$ absurde. Les trois polynômes $A, B, C$ sont premiers entre eux donc les trois PGCD le sont aussi et \[
    A\land A' \quad \times \quad B\land B'\quad \times \quad C\land C'\quad \;\Big|\; AC'-A'C\neq 0
\]
donc \[
    \deg A\land A' + \deg B\land B'+\deg C\land C'\leq \deg(AC'-A'C)
\]
et par le résultat précédent \[
    \deg A+\deg B+\deg C \leq \underbrace{AB'-A'B}_{\leq \deg A+\deg B-1}+ \underbrace{n_0(A)+n_0(B)+n_0(C)}_{n_0(ABC)}
\]
\end{proof}

\begin{thm}[Fermat]
    Si $A, B, C\in\mathbb C[X]$ non constants, $A$ et $B$ premiers entre eux et $A^n+B^n=C^n$ alors $n\leq 2$
\end{thm}

\begin{proof}
On note $p=\max (\deg A, \deg B, \deg C)$ et \[
    np=\max(\deg A^n, \deg B^n, \deg C^n)< n_0(A^nB^nC^n)=n_0(ABC)\leq 3p \implies n<3
\]
\end{proof}

\subsection{Résultant}

On note $P=a_pX^p+\cdots + a_0$, $Q=b_qX^q+\cdots + b_0$ de degrés respectifs $p, q$.

On considère l'application linéaire \[
    \varphi: (U, V)\in\mathbb K_{q-1}[X]\times \mathbb K_{p-1}[X]\longmapsto UP+VQ
\]
Si $P\land Q=1$, alors pour $U, V\in\mathrm{Ker}\varphi$, on a $Q\;|\;U$ et $P\;|\;V$ donc $U=V=0$ en comparant les degrés. Sinon, si $P\land Q=\Delta$ non constant, alors $\varphi(Q/\Delta, -P/\Delta)=0$. Ainsi, $\varphi$ est un isomorphisme si et seulement si $P\land Q=1$

La matrice de $\varphi$ dans la base $(X^{q-1}, 0), \cdots, (X, 0), (1, 0), (0, X^{p-1}), \cdots, (0, 1)$ est \[
% Attention pas la même base que les notes du prof ! Comme ça le copier/collé de wikipedia marche bien :-)
\begin{pmatrix}a_{p}&0&\cdots &0&b_{q}&0&\cdots &0\\a_{p-1}&a_{p}&\ddots &\vdots &\vdots &b_{q}&\ddots &\vdots \\\vdots &a_{p-1}&\ddots &0&\vdots &&\ddots &0\\\vdots &\vdots &\ddots &a_{p}&b_{1}&&&b_{q}\\a_{0}&&&a_{p-1}&b_{0}&\ddots &\vdots &\vdots \\0&\ddots &&\vdots &0&\ddots &b_{1}&\vdots \\\vdots &\ddots &a_{0}&\vdots &\vdots &\ddots &b_{0}&b_{1}\\0&\cdots &0&a_{0}&0&\cdots &0&b_{0}\\\end{pmatrix}=:R(P, Q)
\]
On appelle résultant de $P$ et $Q$ le déterminant de cette matrice. Ainsi, \[
    P\leq Q\neq 1 \iff \det R(P, Q)=0
\]

\section{Polynômes irréductibles}

\subsection{Contenu de Gauss}

On note $P(X)=a_nX^n+\cdots + a_0\in\mathbb Z[X]$ de degré $n\geq 1$. On appelle contenu de $P$ le nombre \[
    a_n\land\cdots\land a_0
\]
On va montrer que si $P, Q\in\mathbb Z[X]$ non constants alors \[
    C(PQ)=C(P)C(Q)
\]

On commence par supposer que $C(P)=C(Q)=1$ et on suppose par l'absurde $C(PQ)\neq 1$. On note $p$ premier diviseur de $C(PQ)$. On note $i_0$ (resp $j_0$) le plus petit indice tel que $p$ ne divise pas $a_{i_0}$ (resp. $b_0$). \[
    P(X)=a_nX^n+\cdots +a_{i_0}X^{i_0}+\underbrace{\cdots}_{\text{divisible par } p} \]\[
    Q(X)=b_nX^n+\cdots +b_{j_0}X^{j_0}+\underbrace{\cdots}_{\text{divisible par } p}
\]
Alors, \[
    [PQ(X)]_{i_0+j_0}=a_{i_0}b_{j_0}+pk \quad \text{non divisible par $p$}
\]
Or c'est un coef de $PQ$ donc divisible par $P$, absurde.

Dans le cas général, \[
    C\left(\frac{PQ}{C(P)C(Q)}\right)=1=\frac{C(PQ)}{C(P)C(Q)}
\]

\begin{res}
    $P\in\mathbb Z[X]$ est irréductible dans $\mathbb Z[X]$ si et seulement s'il l'est dans $\mathbb Q[X]$
\end{res}
\begin{proof}
Si le polynôme est irréductible dans $Q[X]$ alors il l'est dans $\mathbb Z[X]$. On suppose $P$ irréductible dans $\mathbb Z[X]$ et $P=AB$ dans $Q[X]$. Il existe $p, q$ minimaux positifs tels que $pqP=pAqB$, $pA$ et $pQ$ dans $\mathbb Z[X]$ et \[
    1=C(pqP)=pqC(P) \implies pq=1 \implies p=q=1
\]
absurde.
\end{proof}

\subsection{Critère d'Einsenstein} 

\begin{res}[Critère d'Eisenstein]
On note $P=p_nX^n+\cdots + p_0\in\mathbb Z[X]$ non constant de degré $n$. Si $p$ est premier tel que \[
    \begin{cases}
    p\;|\;p_0,\cdots, p_{n-1}\\ p\;\not|\;p_n\\p^2\;\not|\;p_0
    \end{cases}
\]
alors $P$ est irréductible dans $\mathbb Q[X]$
\end{res}
\begin{proof}
On suppose par l'absurde que $P$ n'est pas irréductible dans $\mathbb Q[X]$ donc dans $\mathbb Z[X]$. On écrit alors $P=AB$ avec $A$ et $B$ non constants à coefficients entiers. On a \[
    p\;|\;p_0=a_0b_0
\]
donc par exemple $p$ divise $a_0$ mais pas $b_0$ (il ne peut pas diviser les deux par hypothèse). Si $p$ divise tous les $a_i$, alors $p\;|\;\dom A\times \dom B=p_n$ absurde. On note alors $i_0$ le plus petit $i$ tel que $p\;\not |\;a_i$. $B$ est non constant donc $i_0<n$. \[
    p\;|\; p_{i_0}=a_{i_0}b_0+\underbrace{a_{i_0-1}b_1+\cdots +a_0b_{i_0}}_{\text{divisible par }p} \qquad \text{non divisible par }p
\]
absurde d'où la conclusion.
\end{proof}

\begin{csq}
Pour $p$ premier, on considère \[
    \mu_p(X)=X^{p-1}+\cdots+X+1
\]
Alors, \[
    \mu_p\text{ irréductible }\iff \mu_p(X+1)\text{ irréductible } \iff \frac{(X+1)^p-1}{(X+1)-1}=\sum_{k=1}^p\binom pk X^{k-1}\text{ irréductible}
\]
vrai par le critère d'Eisenstein
\end{csq}

\begin{csq}
Pour tout $n\geq 1$, le polynôme \[
    S_n(X)=1+X+\frac{X}{2!}+\cdots +\frac{X^n}{n!}
\]
est irréductible. En effet, c'est équivalent à vérifier que \[
    n!S_n(X)=X^n+nX^{n-1}+\cdots+n!
\]
est irréductible dans $\mathbb Z[X]$.

Si $n=2m$, alors il existe $p$ premier tel que $m<p<2m$ donc $p<n<2p$ et on conclut avec Einsentein. Sinon, si $n=2m+1$ alors il existe $p$ premier tel que $m<p<2m$ et $p<2m<n$ et $2m\leq 2p-1$ donc $n\leq 2p-1<2p$ et on conclut de même.
\end{csq}

\subsection{$\mathbb Z[X]$ vs $\mathbb Z_p[X]$}

On va montrer que $X^4+1$ est irréductible dans $\mathbb Z[X]$ et n'est jamais irréductible dans $\mathbb Z_p[X]$. Dans $\mathbb C[X]$, \[
    X^4+1=\left(X-e^{i\frac\pi4}\right)\left(X-ie^{i\frac\pi4}\right)\left(X+e^{i\frac\pi4}\right)\left(X+ie^{i\frac\pi4}\right)
\]
Aucun des facteurs ni des produits n'est dans $\mathbb Q[X]$ donc $X^4+1$ est irréductible dans $\mathbb Q[X]$ donc dans $\mathbb Z[X]$.

Pour $p=2$, $(X^4+1)=(X+1)^4$ n'est pas irréductible. On suppose donc $p$ premier impair et on observe que \begin{align*}
    X^4+1&=(X^2)^2-(-1)\\&=(X^2+1)^2-2X^2\\&=(X^2-1)^2-(-2X^2)
\end{align*}
Il est donc suffisant que $-1, 2$ ou $-2$ soit un carré de $\mathbb Z_p$. Si $-1$ et $2$ ne sont pas des carrés, alors $-2$ est un carré. En effet, si on note $\mathcal C^\star$ l'ensemble des carrés non nuls, alors \[
    \varphi: x\in\mathbb Z_p^\star \longmapsto x^2\in\mathcal C^\star
\]
est un morphisme surjectif de noyeau $\ker \varphi=\{\pm1\}$ donc $\#\mathcal C^\star=\frac{p-1}2$. \[
    X^{\frac{p-1}2}-1
\]
a au plus $\frac{p-1}2$ racines ($\mathbb Z_p$ est un corps) et s'annule sur $\mathcal C^\star$ donc $\mathcal C^\star$ est exactement l'ensemble des racines de ce polynôme (le petit théorème de fermat donne $x^{p-1}=1$ pour tout $x$ non nul). Si $-1$ et $2$ ne sont pas carrés, alors \[
    (-2)^{\frac{p-1}2}=(-1)^{\frac{p-1}2}\times (2)^{\frac{p-1}2}=-1\times -1=1
\]
donc $-2$ carré.

\section{Polynômes cyclotomiques (HP)}

\begin{dfn}
Soit $n\geq 1$. \begin{itemize}
    \item On appelle racine primitive $n$-ième de l'unité tout générateur de $\mathbb U_n$. On note $\mathbb P_n$ ces racines.
    \item Il y a $\varphi(n)$ générateurs de $\mathbb U_n$: les $w^l$ avec $l\land n=1$ et $w$ un générateur.
    \item On appelle $n$-ième polynôme cyclotomique le polynôme \[
        \mu_n(X)=\prod_{u\in\mathbb P_n}(X-u)
    \]
\end{itemize}
\end{dfn}

\begin{prop}
Pour tout $n\in\mathbb N^\star$, \[
    X^n-1=\prod_{d\;|\;n}\mu_d(X)
\]
\end{prop}

\begin{proof}
Pour $d\;|\;n$, on note $E_d=\{u\in\mathbb U_n, \ord u=d\}$, donc \[
    \mathbb U_n =\bigsqcup_{d\;|\;n}E_d
\]
ainsi \[
    X^n-1=\prod_{d\;|\; n}\prod_{u\in E_d}(X-u)=\prod_{d\;|\;n}\mu_d(X)
\]
car $E_d=\mathbb P_d$ (les éléments d'ordre $d$ sont au nombre de $\varphi(d)$ et engendrent tous $\mathbb U_d$).
\end{proof}

\begin{rem}
Si $n=p$ est premier alors \[
    \mu_n(X)=\frac{X^p-1}{X-1}=X^{p-1}+\cdots +1
\]
\end{rem}

\begin{prop}
Pour tout $n\geq 1$, $\mu_n$ est unitaire à coefficients entiers, de degré $\varphi(n)$.
\end{prop}

\begin{proof}
On raisonne par récurrence sur $n$ pour le caractère entier, le reste est immédiat.
\begin{itemize}
    \item $n=1$ immédiat
    \item On se donne $n\geq 1$ et on suppose la propriété vraie jusqu'au rang $n$. Si $n+1$ est premier, la conclusion est immédiate. Sinon, \[
        X^{n+1}-1=\prod_{d\;|\;n+1}\mu_d(X)=\mu_{n+1}(X)\times
        \underbrace{\prod_{\substack{d\;|\;n+1\\ d\neq n+1}}\mu_d(X)}_{\in\mathbb Z[X], \text{ unitaire }}
    \]
    Donc $\mu_{n+1}$ est le quotient dans la division euclidienne de $X^{n+1}-1$ par un polynôme \textbf{unitaire} de $\mathbb Z[X]$, d'où la conclusion.
\end{itemize}
\end{proof}

\subsection{Expression de $\mu_n$}

On introduit la fonction arithmétique de Möbius \[
    \mu(n)=\begin{cases}
        1 &\text{si } n=1 \\
        (-1)^k &\text{si }n\text{ produit de $k$ premiers distincts}\\
        0 &\text{sinon}
    \end{cases}
\]
Cette fonction est multiplicative: pour $a, b$ premiers entre eux, $\mu(ab)=\mu(a)\mu(b)$. Puis, pour $n\geq 1$, \[
    \sum_{d\;|\;n}\mu(d)=\begin{cases}
        1&\text{si }n=1 \\
        0&\text{sinon}
    \end{cases}
\]
En effet, si $n=p_1^{\alpha_1}\cdots p_k^{\alpha_k}$, \[
    \sum_{d\;|\;n}\mu(d)=\sum_{\substack{d\;|\; n\\\mu(d)\neq 0}}=\sum_{I\subset \llbracket 1, k\rrbracket}(-1)^{\#I}=\sum_{i=0}^k\binom ki(-1)^i=(1-1)^k=0
\]

\begin{res}
\[
    \mu_n(X)=\prod_{d\;|\;n}\left(X^d-1\right)^{\mu\left(\frac nd\right)}
\]
\end{res}

\begin{proof}
\begin{align*}
    \prod_{d\;|\; n}\left(X^d-1\right)^{\mu\left(\frac nd\right)}&=\prod_{d\;|\;n}\;\prod_{d'\;|\;d}\mu_{d'}(X)^{\mu\left(\frac nd\right)} \\
    &= \prod_{d'\;|\; n}\;\prod_{u\;|\;\frac n{d'}}\mu_{d'}(X)^{\mu\left(\frac n{d'u}\right)} \\
    &= \prod_{d'\;|\;n}\mu_{d'}(X)^{\left(\displaystyle\sum_{u|\frac n{d'}}\mu\left(\frac n{d'u}\right)\right)}\\
    &=\mu_n(X)
\end{align*}
car \[
\sum_{u\;|\;\frac n{d'}}\mu\left(\frac n{d'u}\right)=\sum_{u\;|\;\frac n{d'}}\mu(u)=\begin{cases}
    1& \text{si }\frac n{d'}=1\iff d'=n \\
    0& \text{sinon}
\end{cases}
\]
\end{proof}

\begin{rem}
Cela donne une autre preuve de $\mu_n(X)\in\mathbb Z[X]$
\end{rem}

\subsection{Suites arithmétiques}

On note $p$ un nombre premier et $m\geq 1$ un entier non divisible par $p$. Supposons que $a$ est un entier et que \[
    \mu_m(a)= 0\pmod p
\]
On a dans ce cas \[
    p\;|\;\mu_m(a)\;|\;a^m-1 \text{ premier avec }a \implies p\;\not|\;a
\]

Si $d=\ord a$ alors $d\;|\; m$ donc $X^d-1|X^m-1$. Par l'absurde, si $d\neq n$ alors $X^d-1\land \mu_m=1$ (aucune racine commune dans $\mathbb C$) donc (Gauss) $(X^d-1)\times \mu_m(X)|X^m-1$. On en déduit ($d$ est l'ordre de $a$) \[
    (a^d-1)\mu_m(a)\;|\;a^m-1 \implies a^m-1=0\pmod {p^2}
\]
et comme $a+p$ vérifie les mêmes hypothèses ($a+p\equiv a\pmod p$ donc $\mu_m(a+p)=0\pmod {p}$ et l'ordre est le même donc $\neq m$), on a $(a+p)^m-1= 0\pmod {p^2}$, d'où \[
    p^2\;|\; (a+p)^m-a^m=p\left(a^{m-1}+\underbrace{\cdots}_{\text{multiple de }p}\right) \implies p\;|\; a
\]
c'est absurde, donc $\ord a = m$. Puis $m=\ord a\;|\;p-1$ donc $p$ est un terme de la suite $(1+km)_{k\in\mathbb N}$. Si cette suite n'a qu'un nombre fini de nombres premiers $p_1, \cdots, p_s$ alors pour $k$ assez grand, $\mu_m(kp_1\cdots p_s)>2$. On a aussi $\mu_m(0)=\pm 1$ (entier de module $1$) donc si $p$ est un premier qui divise $\mu_m(kp_1\cdots p_s)$ et $p$ est parmi les $p_i$ alors $p\;|\; \mu_m(0)$ absurde, donc $p$ n'est pas parmi les $p_i$ et la suite a une infinité de termes premiers.

\subsection{Irréductibilité}

Nous allons montrer que les polynômes cyclotomiques sont irréductibles par un argument de Landau (1929).

On note $\alpha$ une racine primirive $n$-ième de l'unité et $P$ son polynôme minimal. La suite $(P(\alpha^k))_{k\in\mathbb N}$ est $n-périodique$ à valeurs dasn $\mathbb Z[\alpha]$ et chacun des termes s'écrit  de manière unique $Q_k(\alpha)$ pour un polynôme $Q_k$ de degré $<\deg P$.

On note $A$ le plus grand coefficient de tous les $Q_k$ en valeurs absolues. Dans $\mathbb Z_p[X]$, $P(X^p)=P(X)^p$ (Frobenius) donc il existe $U\in\mathbb Z[X]$ tel que \[
    P(X^p)=P(X)^p+pU(X)
\]
Pour $p>A$, \[
    P(\alpha^p)=P(\alpha)^p+pU(\alpha)=pU(\alpha)=pV(\alpha) \quad \text{ avec }\deg V<\deg P
\]
donc \[
    Q_p(\alpha)=pV(\alpha) \qquad \text{ et }\qquad P\;|Q_p-pV
\]
et donc \[
    Q_p=pV
\]
d'où $p\;|\;$ les coefs de $Q_p$ qui sont de module $\leq A<p$ donc $Q_p=0$ et $P(\alpha^p)=0$. Ici, on s'est seulement servis du fait que $\alpha$ est une racine de l'unité, donc si $p_1, \cdots, p_r$ sont $>A$ alors $P(\alpha^{p_1\cdots p_r})=0$.

On note $m$ premier avec $n$, $N$ le produit des premiers $\leq A$. On a $m+nN=m\pmod n$ et si $p$ divise $m+nN$ alors $p>A$ donc $P(\alpha^{m+nN})=0=P(\alpha^m)$

Ainsi les $(\alpha^{m})$ avec $m\land n=1$ sont racines de $P$ donc $\mu_n(X) \;|\; P(X)$. $\mu_n(\alpha)=0$ donc $P$ divise $\mu_m$, donc $P=\mu_m$. C'est un polynôme minimal donc irréductible.

\section{Nombres algébriques}

Si $K\subset L$ sont deux corps alors on dit que $a\in L$ est algébrique sur $K$ s'il existe $P\in\mathbb K[X]$ non nul tel que $P(a)=0$.

Dans $\mathbb C$, on appelle nombre algébrique les nombres algébriques sur $\mathbb Q$ et on note $A$ l'ensemble de ces nombres.

\begin{rem}
Si $a$ est algébrique sur un corps $K$ alors \[
    I_a=\{P\in\mathbb K[X]\;/\; P(a)=0\}
\]
etst un idéal non réduit à $\{0\}$ de $K[X]$ euclidien donc principal, donc il existe un unique polynôme unitaire $\mu_a(X)$ tel que $I_a=\mu_a K[X]$, que l'on appelle polynôme minimal de $a$. Dans ce cas \[
    \Vect_Kr((a^k)_{k\in\mathbb N})=\Vect_K(1, a, \cdots, a^{\deg \mu_a-1})=\{P(a), P\in K[X]\}
\]
\end{rem}

\subsection{Structure de corps}

\begin{res}
$A$ est un corps
\end{res}

\begin{proof}
\begin{itemize}
    \item $A\subset \mathbb C$
    \item $1\in A$ en tant que racine de $X-1$
    \item Soient $x, y\in A$, $p=\deg \mu_x, q=\deg \mu_y$. Pour tout $n\in\mathbb N$, \[
        \begin{cases}
            x^n\in\Vect_{\mathbb Q}(1, x, \cdots, x^{p-1}) \\ y^n\in\Vect_{\mathbb Q}(1, y, \cdots, y^{q-1}) 
        \end{cases}\quad\text{ donc }\quad  x^ny^n, (x+y)^n\in\Vect_{\mathbb Q}(x^iy^j, i<q, j<q)
    \]
    
    Les familles $((x-y)^n)_n$ et $((xy)^n)_n$ sont liées donc $x-y, xy\in A$.
    \item Si $x\in A$ non nul et si on note $E=\Vect_{\mathbb Q}(1, x, \cdots, x^{p-1})$ alors \[
        \varphi: z\in E\longmapsto xz\in E
    \]
    est un endomorphisme injectif en dimension finie donc c'est un isomorphisme et $\varphi^{-1}(1)$ donne $x^{-1}\in E\subset A$
\end{itemize}
\end{proof}

\subsection{Règle de multiplicativité des degrés}

On note $a$ algébrique sur $K$ de degré $d$. \[
    \varphi: P\in K[X] \longmapsto P(a)
\]
est un morphisme de $K$-algèbres donc $\img\varphi=\Vect_K(1, \cdots, a^{p-1})$ est une $K$-algèbre, notée $K[a]$.

$K[d]$ est une algèbre de dimension finie $d$ sur $K$ et pour $x\in K[a]\setminus \{0\}$, \[
    \psi: y\in K[a]\longmapsto xy\in K[a]
\]
est un endomorphime injectif en dimension finie donc c'est un automorphisme et $x$ a un inverse dans $K[a]$, qui est donc un corps.

\begin{dfn}
Si $K\subset L$ alors $L$ est un $K$-ev et si $L$ est de dimension finie sur $K$ alors on note \[
    [L:K]=\dim_K(L)
\]
\end{dfn}

\begin{prop}
\begin{enumerate}
    \item Soit $K\subset L\subset N$ trois corps. Si $\dim_L N, \dim_K L<+\infty$ alors $\dim_K N<+\infty$ et \[
        [N:K]=[N:L]\times [L:K]
    \]
    \item Si $a\in\mathbb C$ est algébrique et si $b\in K$ est algébrique sur $\mathbb Q[a]$ alors $b$ est algébrique sur $\mathbb Q$.
\end{enumerate}
\end{prop}

\begin{proof} ~
\begin{enumerate}
    \item $(n_1, \cdots, n_p)$ base de $N$ sur $L$, $(\ell_1, \cdots, \ell_q)$ base de $L$ sur $K$. Si $n\in\mathbb N$ alors il existe $\lambda_1, \cdots, \lambda_p\in L$ tel que $n=\lambda_1n_1+\cdots+\lambda_p n_p$ et chaque $\lambda_i\in\Vect_{\mathbb Q}(\ell_1, \cdots, \ell_p)$ donc $n\in\Vect_{\mathbb Q}(\ell_in_j)$ (famille génératrice finie) donc \[
        [N:K]<+\infty
    \]
    Vérifions que $(\ell_in_j)_{\substack{1\leq i\leq q\\ 1\leq j\leq p}}$ est libre sur $\mathbb Q$. \[
        \sum_{i, j}\lambda_{i, j}\ell_in_j=0\implies \sum_{j=1}^pn_j\sum_{i=1}^q\lambda_{i, j}\ell_i=0 \implies \forall j\in\llbracket 1, p\rrbracket \sum_{i=1}^q\lambda_{i, j}\ell_i=0\implies \forall i, j, \lambda_{i, j}=0
    \]
    \item C'est une conséquence directe de $1.$
\end{enumerate}
\end{proof}

\begin{rem}
Si $b$ est racine d'une équation polynômiale à coefficients dans $\mathbb Q[a]$, il l'est aussi d'une équation polynômiale à coefficients dans $\mathbb Q$.
\end{rem}

\section{Formules d'interpolation}

\subsection{Rappels}

On considère $x_1, \cdots, x_n\in\mathbb K$ deux à deux distincts, et on note \[
    \Delta_i(X)=\prod_{j\neq i}\frac{X-x_j}{x_i-x_j}
\]
de telle manière que \[
    \Delta_i(x_j)=\delta_{i, j}
\]

\begin{thm}
    On considère $x_1, \cdots, x_n, y_1, \cdots, y_n\in\mathbb K$, les $x_i$ deux à deux distincts. Dans ce cas, il existe un unique polynôme $P\in\mathbb K_{n-1}[X]$ tel que \[\forall i\in\llbracket 1, n\rrbracket, \quad P(x_i)=y_i\]
    De plus, \[
        P(X)=\sum_{i=1}^ny_i\Delta_i(X)
    \]
\end{thm}

\begin{proof}
Le polynôme donné convient. Si $P$ et $Q$ conviennent, $P-Q$ a $n$ racines et est de degré au plus $n-1$ donc $P=Q$.
\end{proof}

\begin{rem}
Si on note \[
    \omega (x)=\prod_{j=1}^n(x-x_j)
\]
alors \[
    \prod_{j\neq i}(x_i-x_j)=\lim_{x\to x_i}\frac{\omega(x)-\omega(x_i)}{x-x_i}=w'(x_i)
\]
et \[
    \Delta_i(x)=\frac{\omega (x)}{\omega(x_i)(x-x_i)}
\]
donc \[
    P(x)=\underbrace{\left(\sum_{i=1}^n\frac{y_i}{\omega'(x_i)(x-x_i)}\right)}_{\text{DSE de }\frac{P(x)}{\omega(x)}}\omega(x)
\]
\end{rem}

\subsection{Formules d'ordre supérieur}

On se donne $n_1, \cdots, n_r\in\mathbb N^\star$, $x_1, \cdots, x_r\in\mathbb K$ deux à deux distincts et $n=n_1+\cdots+n_r$. \[
    \varphi: P\in\mathbb K_{n-1}[X]\longmapsto(P(x_1), \cdots, P^{(n_1-1)}(x_1), \cdot, P(x_r), \cdots, P^{(n_r-1)})\in\mathbb K^n
\]
est une application linéaire injective, il y a égalité des dimensions donc c'est un isomorphisme et on en déduit que pour $(y_{i, k})_{i\leq r, j\leq n_i}$ donné, il existe un unique polynôme de degré au plus $n-1$ tel que \[
    \forall i, k, P^{(k)}(x_i)=y_{i, k}
\]

\subsection{Analyse de l'erreur}

On se donne $f\in\mathbb C^n([a, b], \mathbb R) ,x_1, \cdots, x_n\in [a, b]$ deux à deux distincts. \[
    P(x)=\sum_{i=1}^nf (x_i)\Delta_i(x)
\]
l'interpolation de Lagrange de $f$. On veut estimer $f(x)-P(x)$. On note $x\in [a, b]\setminus\{x_1, \cdots, x_r\}$ et on note \[
    g(t)=f(t)-P(t)-\lambda_x\omega(t)
\]
où $\lambda_x$ tel que $g(x)=0$ (ajout d'une racine supplémentaire). Rolle itéré donne:

\begin{itemize}
    \item 
$g$ s'annule en $x_1, \cdots, x_n, x$ ($n+1$ points)
\item $\cdots$
\item 
$g^{(n)}$ s'annule au moins une fois en $c_n$
\end{itemize}

\[
    0=f^{(n)}(c_n)-P^{(n)}(c_n)-\lambda_x n!
\]
donc $\lambda_x = \dfrac{f^{(n)}(c_n)}{n!}$ et donc \[
    |f(x)-P(x)|=|\lambda_x\omega (x)|\leq |\omega(x)|\sup_{[a, b]}\frac{|f^{(n)}|}{n!}
\]
cette inégalité est encore vraie pour $x\in\{x_1, \cdots, x_n\}$

\begin{rem}
On constate que l'erreur est meilleure si \[
    |\omega(x)|=|(x-x_1)\cdots (x-x_n)|
\]
est petit. On peut donc s'intéresser aux choix des $x_i$ minimisant $\|\omega\|_\infty$ sur $[a, b]$. C'est un problème difficile, mais le choix de $x_1, \cdots, x_n$ équirépartis est mauvais.
\end{rem}

\section{Suites de polynômes}

\begin{thm}[Weierstrass]
    Si $f$ est continue de $[a, b]$ dans $\mathbb C$ alors il existe une suite $(P_n)$ de fonctions polynômiales telle que \[
        P_n\xrightarrow[{[a, b]}]{CVU} f
    \]
\end{thm}

\begin{proof}[Idée de la preuve]
    Bernstein puis convolution
\end{proof}

\begin{rem}
Le résultat est faux sur un intervalle autre qu'un segment.
\end{rem}

\subsection{Inégalité de Bernstein}

\begin{lmm}
Si $z_1, \cdots, z_n$ sont les racines du polynôme $z^n+1$ et $P$ polynôme réel de degré $n$, alors pour tout $t\in\mathbb C$, \[
    tP'(t)=\frac n2P(t)+\frac 1n\sum_{k=1}^nP(-z_k)\frac{2z_k}{(z_k-1)^2}
\]
\end{lmm}

\begin{proof}
On note $\varphi(z, t)=\frac{P(tz)-P(t)}{z-1}$ pour $z\neq 1$. C'est une fonction polynomiale de degré $\leq n-1$ pour $t$ fixé.

D'après la formule d'interpolation de Lagrange en $z_1, \cdots, z_n$, \begin{align*}
    \varphi(z, t)&=\sum_{k=1}^n\varphi(z_k, t)\frac{z^n+1}{z-z_k} \times \underbrace{\frac1{(z_k-z_1)\cdots \widehat{(z_k-z_k)}\cdots (z_k-z_n)}}_{=\lim_{z\to z_k}\limits \frac{z-z_k}{z^n+1}=\frac1{nz_k^{n-1}}=-\frac{z_k}n} \\ &= \sum_{k=1}^n-\varphi(z_k, t)\times \frac{z^n+1}{z-z_k}\frac{z_k}n
\end{align*}

On fait tendre $z$ vers $1$. \[
    tP'(t)=\sum_{k=1}^n\varphi(z_k, t)\frac{2}{z_k-1}\frac{z_k}n= 
    \frac1n\sum_{k=1}^nP(-tz_k)\frac{2z_k}{(z_k-1)^2}-\frac{P(t)}{n}\sum_{k=1}^n\frac{2z_k}{(z_k-1)^2}
\]
Il nous reste à évaluer ce dernier terme. On reprend la formule précédente avec $P(t)=t^n$. \[
    nt^n=\frac1n\sum_{k=1}^n2t^n\frac{z_k^{n+1}}{(z_k-1)^2}-\frac{t^n}{n}\sum_{k=1}^n\frac{2z_k}{(z_k-1)^2}\quad \underset{z_k^{n+1}=-z_k}{=}\quad -\frac{2t^n}n\sum_{k=1}^n\frac{2z_k}{(z_k-1)^2}
\]
Et alors,
\[
\sum_{k=1}^n\frac{2z_k}{(z_k-1)^2}=-\frac{n^2}2
\]
ce qui termine la preuve de la formule.
\end{proof}

\begin{res}
On note pour $P\in\mathbb R[X]$, $\|P\|=\sup_{|z|=1}\limits|P(z)|$. Si $P$ est de degré $n$ alors \[
    \|P'\|\leq n\|P\|
\]
\end{res}

\begin{proof}
On prend $t$ de module $1$, \[
    |tP'(t)|=|P'(t)|\leq \frac n2|P(t)|+\frac1n\sum_{k=1}^n\left|\frac{2z_k}{(z_k-1)^2}\right|\times \|P\|
\]
Et $z_k\in\mathbb U$ donc s'écrit $z=e^{i\theta}$ et donc ($\theta\in ]0, 2\pi[$) \[
    \frac{2z_k}{(z_k-1)^2}=\frac{2}{(e^{i\frac \theta 2}-e^{-i\frac \theta2})^2}=\frac 2{-4\sin^2\frac\theta 2}<0
\]
et \[
    \sum_{k=1}^n\left|\frac{2z_k}{(z_k-1)^2}\right|=-\sum_{k=1}^n\frac{2z_k}{(z_k-1)^2}=\frac{n^2}2
\]
ce qui termine la preuve.
\end{proof}

\subsection{Théorème d'oscillation de Tchebychev}

On note $C_n$ le $n$-ième polynôme de Tchebychev et $T_n=\dfrac{C_n}{2^{n-1}}$ de sorte que $T_n$ est unitaire.

\begin{res}
Pour tout $P$ unitaire de degré $n$, \[
    \frac 1{2^{n-1}}=\max_{[-1, 1]}|T_n|\leq \max_{[-1, 1]}|P|
\]
\end{res}

\begin{proof}
\[
    T_n(\cos \theta)=\frac1{2^{n-1}}\cos (n\theta) \implies \max_{[-1, 1]}|T_n|=\frac1{2^{n-1}}
\]
Supposons par l'absurde que $\displaystyle\max_{[-1, 1]}|P|<\frac1{2^{n-1}}$ et posons \[
    Q(X)=T_n(X)-P(X)
\]
Pour $x_k=\cos\left(\frac{k\pi}n\right), k\in\llbracket 0, n\rrbracket$, on a \[
    T_n(x_k)=\frac{(-1)^k}{2^{n-1}}
\]
d'où on tire le tableau de signes
\begin{center}
    \begin{tikzpicture}
        \tkzTabInit[espcl=2]{$x$ / 1 ,$Q(x)$ /1 }{$x_0$ , $x_1$, $x_2$, $\cdots$ , $x_n$ }%
        \tkzTabLine{+,0,-,0,+,0,\cdots ,0,(-1)^n\;}
    \end{tikzpicture}
\end{center}
On en déduit que $Q$ s'annule au moins $n$ fois et $\deg Q\leq n-1$ donc $Q=0$, donc $P=T_n$ absurde.
\end{proof}

\todo{exo fermés}

\section{Aspects topologiques}

\subsection{Le théorème de Baire}

\begin{res}
$(E, \|\;\|)$ un e.v.n tel que toute suite de Cauchy converge.Si $(O_n)_n$ est une suite d'ouverts denses de $E$ alors $\cap O_n$ est dense.
\end{res}

\begin{proof}
On note $U$ ouvert de $E$.
\begin{itemize}
    \item 
$U\cap O_0$ est un ouvert non vide donc il existe $x_0\in U\cap O_0$ et $2a_0>0$ tel que $\mathcal B_o(x_o, 2a_o)\subset U\cap O_0$ et donc $\mathcal B_f(x_0, a_0)\subset U\cap O_0$. On note $r_0=\min(a_0, 1)$.
\item $\mathcal B_o(a_0, r_0)\cap O_1$ ouvert non vide donc il existe $x_1$ et $a_1>0$ tels que $\mathcal B_f(x_1, a_1)\subset \mathcal B_0(x_0, r_0)\cap O_1$ et on note $r_1=\min\left(a_1, \frac12\right)$

\item $\cdots$
\end{itemize}
On construit ainsi deux suites $(x_n)_n\in E^{\mathbb N}$, $(r_n)\in\mathbb {R_+^\star}^{\mathbb N}$ telles que \[
    \begin{cases}
        \mathcal B_f(x_n, r_n)\subset \mathcal B_f(x_{n-1}, r_{n-1})\subset \cdots \subset \mathcal B_f(x_0, r_0) \\
        \mathcal B_f(x_n, r_n)\subset O_n, \quad r_n\leq \dfrac 1n
    \end{cases}
\]
Pour $m>n$, $x_m\in\mathcal B_f(x_m, r_m)\subset\mathcal B_f(x_n, r_n)$ donc \[
    \|x_m-x_n\|\leq \frac 1n
\]
donc $(x_n)$ de Cauchy donc convergente vers $x$.

Pour $m$ fixé, APCR $x_n\in\mathcal B_f(x_m, r_m)$ fermé donc $x\in\mathcal B_f(x_m, r_m)\subset O_m$ donc $x\in O_m$ pour tout $m$. Ainsi, \[
    x\in\bigcap_{n\in\mathbb N}O_n
\]
et $x\in\mathcal B_f(x_0, r_0)\subset U$ donc $x\in U$ et donc \[
    U\cap\left(\bigcap_{n\in\mathbb N}O_n\right)
\]
pour tout ouvert $U$.

\begin{cor}
Si $(F_n)_n$ est une suite de fermés d'intérieurs vides alors \[
    \bigcup_{n\in\mathbb N}F_n
\] est d'intérieur vide.
\end{cor}

% je me charge de la fin du poly (comme ça je pige en même temps ce que le prof écrit)
\end{proof}

\subsection{Incomplétude de $\mathbb R[X]$}

On supose qu'il existe une norme $\|\cdot \|$ telle que $(\mathbb R[X], \|\cdot \|)$ est complet (i.e. toutes les suites de Cauchy convergent)

\begin{itemize}
    \item $E_n=\Vect(1, \cdots, X^n$ est un fermé (s.e.v. de dimension finie)
    \item \[
    \forall P\in E_n, \forall \varepsilon>0, \quad P+\frac{X^{n+1}}k\xrightarrow[k\to+\infty]{} P
    \] donc APCR \[
        P+\frac{X^{n+1}}k \in \mathcal B_0(P, \varepsilon)
    \]
    donc $\mathcal B_o(P, \varepsilon)\not\subset E_n$ et $E_n$ d'intérieur vide.
    \item Par le théorème de Baire, \[
        \underbrace{\bigcap_{n\in\mathbb N} E_n}_{\text{intérieur vide}}=\underbrace{\mathbb R[X]}_{\text{intérieur non vide}}
    \]
    absurde.
\end{itemize}

Il n'existe donc aucune norme pour laquelle $\mathbb R[X]$ est complet.

En particulier, $(X_k)_k$ est une base qu'on peut orthonormaliser par le procédé de Gram-Schmidt en une base $(P_k)_k$ qui est totale, sans que $\mathbb R[X]$ soit complet.

\end{document}
