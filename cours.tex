\documentclass{report}
\newif\ifsolo

\solofalse

\usepackage[french]{babel}
\usepackage[a4paper, margin=2cm]{geometry}
\usepackage{amsmath}
\usepackage{bbm}
\usepackage{amsthm}
\usepackage{amssymb}
\usepackage{enumitem}
\usepackage{stmaryrd}
\usepackage{xcolor}
\usepackage{etoolbox}
\usepackage{parskip}
\usepackage{tikz,tkz-tab}
\usepackage{pgfplots}
\usepackage{titleps}
\ifsolo
\else
\usepackage{lastpage}
\usepackage[french]{minitoc}
\fi
\usepackage{tocloft}
% \usepackage{lmodern}
\usepackage{enumitem}
\usepackage{needspace}
\usepackage{imakeidx}
\usepackage[normalem]{ulem}
\usepackage[hidelinks]{hyperref}

\makeindex[intoc,columns=2,options={-s indexstyle.ist -o build/\jobname.ind}]
\makeatletter
\def\@idxitem{\par\hangindent 0pt}
\makeatother

\AtBeginDocument{\def\labelitemi{$\bullet$}}
\setlist[itemize]{leftmargin=.5cm}
\setlist[enumerate]{leftmargin=.7cm}

\theoremstyle{definition}
\newtheorem*{thm}{Théorème}
\newtheorem*{thmdef}{Théorème -- Définition}
\newtheorem*{dfn}{Définition}
\newtheorem*{defprop}{Définition -- Proposition}
\newtheorem*{lmm}{Lemme}
\newtheorem*{rem}{Remarque}
\newtheorem*{ex}{Exemple}
\newtheorem*{res}{Résultat}
\newtheorem*{prop}{Proposition}
\newtheorem{csq}{Conséquence}[subsection]
\renewcommand{\thecsq}{\arabic{csq}}
\newtheorem*{cor}{Corollaire}
\newtheorem*{exo}{Exercice}
\newtheorem*{notation}{Notation}

\AtBeginEnvironment{thmdef}{\Needspace{5\baselineskip}}
\AtBeginEnvironment{defprop}{\Needspace{5\baselineskip}}

\renewcommand{\thesection}{\arabic{section}}
\renewcommand{\thesubsection}{\emph{\alph{subsection}})}


\setlength{\cftsecnumwidth}{0.6cm}
\renewcommand{\cftsecpresnum}{}
\renewcommand{\cftsecaftersnum}{}


\ifsolo
\addto\captionsfrench{
  \renewcommand{\contentsname}%
    {Plan du cours}%
}
\else
\renewcommand{\thechapter}{\Roman{chapter}}
\mtcsetoffset{minitoc}{-2em}
\renewcommand{\cftsecfont}{\small\bfseries}
\setcounter{tocdepth}{1}
\setlength{\cftchapnumwidth}{1.4cm}
\fi

\newpagestyle{main}{
  \setheadrule{1pt}
  \ifsolo\else \setfootrule{1pt}\fi
  \sethead{\thesection ~--~\textsc{\sectiontitle}}{}{Page \thepage\ifsolo\else~/~\pageref{LastPage}\fi}
  \setfoot{\ifsolo\else\textbf{Chapitre \thechapter} : \textsc{\chaptertitle}\fi}{}{}
}

\DeclareMathOperator{\dom}{dom}
\newcommand{\ord}{\mathrm{ord}}
\newcommand{\Vect}{\mathrm{Vect}}
\newcommand{\diff}{\mathrm{d}}
\newcommand{\nop}{\|\hspace{-1pt}|}
\newcommand{\defeq}{\,\widehat=\,}
\newcommand{\1}{\mathbbm{1}}
\renewcommand{\mathring}[1]{\overset\circ{#1}}
\newcommand{\img}{\mathrm{Im}}
\newcommand{\Ker}{\mathrm{Ker}}
\newcommand{\Tr} {\mathrm{Tr}}
\newcommand{\Com}{\mathrm{Com}}

\makeatletter
\newcommand*{\xnrightarrow}[2][]{%
  \ext@arrow 0359\nrightarrowfill@{#1}{#2}%
}
\newcommand*{\nrightarrowfill@}{%
    \narrowfill@\relbar\relbar\rightarrow{\not \relbar}
}
\newcommand*{\narrowfill@}[5]{%
  $\m@th\thickmuskip0mu\medmuskip\thickmuskip\thinmuskip\thickmuskip
  \relax#5#1\mkern-8mu%
  \cleaders\hbox{$#5\mkern-2mu#2\mkern-2mu$}\hfill
  \mkern-5mu %
  #4%
  \mkern-5mu %
  \cleaders\hbox{$#5\mkern-2mu#2\mkern-2mu$}\hfill
  \mkern-7mu#3$%
}
\makeatother

\newcommand{\todo}[1]{{\color{red}À faire: #1}}

\newcommand*\circled[1]{\tikz[baseline=(char.base)]{
            \node[shape=circle,draw,inner sep=2pt] (char) {#1};}~}

\newcommand{\hyp}{\circled{H}}
\newcommand{\conc}{\circled{C}}

\newcommand{\Hyp}{~\\[.1cm] \circled{H}\hspace{1em}}
\newcommand{\Conc}{~\\[.1cm] \circled{C}\hspace{1em}}
\newenvironment{concenum}{
~\\[0.1cm]\begin{tabular}{@{}lp{15cm}} \conc & \vspace{-.45cm}
\begin{enumerate}[leftmargin=.5cm]
}
{
\end{enumerate}
\end{tabular}
}

\newcommand*{\colorboxed}{}
\def\colorboxed#1#{%
  \colorboxedAux{#1}%
}
\newcommand*{\colorboxedAux}[3]{%
  % #1: optional argument for color model
  % #2: color specification
  % #3: formula
  \begingroup
    \colorlet{cb@saved}{.}%
    \color#1{#2}%
    \boxed{%
      \color{cb@saved}%
      #3%
    }%
  \endgroup
}

\tocloftpagestyle{empty}
\pagestyle{main}

\pgfplotsset{compat=1.16,
  colormap={whitered}{color(0cm)=(white); color(1cm)=(orange!75!red)}
}
\pgfkeys{/pgf/plot/gnuplot call={cd build && gnuplot}}


\begin{document}

\pagestyle{empty}
\begin{center}

    ~ \vfill

    \textsc{\LARGE Lycée Faidherbe}\\[1.5cm]
    \textsc{\Large Classe de MP$^\star$}\\[0.5cm]

    \rule{\linewidth}{0.5mm} \\[0.4cm]
    {\huge \bfseries Cours de Mathématiques }\\[0.1cm]
    \rule{\linewidth}{0.5mm} \\[0.4cm]
    \textsc{\large Programme officiel et compléments}\\[1.5cm]

    % \begin{minipage}{0.4\textwidth}
    %     \begin{flushleft} \large
    %         \emph{Professeur}\\
    %         Serge \textsc{Varjabédian}
    %     \end{flushleft}
    %     \end{minipage}\hspace{1cm}\begin{minipage}{0.4\textwidth}
    %     \begin{flushright} \large
    %         \emph{Prise de notes} \\
    %         Théophile \textsc{Cailliau}
    %     \end{flushright}
    % \end{minipage}\\[2cm]
    \vfill
    \includegraphics[width=5cm]{faidherbe.pdf}\\[1cm]
    \vfill
    {\large Année 2019-2020}\\
\end{center}

\dominitoc
\tableofcontents
\faketableofcontents

\begin{minipage}{\textwidth}
Dans tout le cours, les éléments explicitement au programme sont précédés par \textbf{Théorème}, \textbf{Définition}, \textbf{Propriété}, \textbf{Proposition}. Les remarques, corollaires et conséquences sont généralement importants. Les \textbf{Résultats} ou éléments démontrés sans indication quant à leur nature sont des compléments de cours (hors programme). Tous les compléments de cours ont leur intérêt, mais il n'est pas nécessaire de tous les connaître.

Les démonstrations manquantes sont jugées suffisamment facile à retrouver rapidement. Lorsqu'une démonstration est admise, elle apparaît toujours sous la forme :

\begin{proof}
    Admis(e).
\end{proof}
\end{minipage}

\newpage

\pagestyle{main}

\ifsolo
    ~

    \vspace{1cm}

    \begin{center}
        \textbf{\LARGE Analyse Asymptotique} \\[1em]
    \end{center}
    \tableofcontents
\else
    \minitoc
\fi
\thispagestyle{empty}

\ifsolo \newpage \setcounter{page}{1} \fi
\section{Développements limités -- Rappels}

Cette section ne contient que des rappels de sup.

\begin{dfn}[Développement limité]\index{développement limité}
    Un développement limité de $f:V\longrightarrow \mathbb R$ à l'ordre $n$ au voisinage $V$ de $a$ (qu'on abrègera $\mathrm{DL}_n(a)$ de $f$) est une écriture du type: \[
        f(x)=\underbrace{a_0+a_1(x-a)+\cdots +a_n(x-a)^n}_{\text{partie polynômiale}}+o_a((x-a)^n)
    \]
\end{dfn}

\begin{rem}
    Lorsqu'un $\mathrm{DL}_n(a)$ existe, il y a unicité de la partie polynômiale.
\end{rem}

\begin{rem}
    Les DL usuels sont à connaître.

\begin{align*}
    e^x &= 1 + x + \frac{x^2}{2} +\ldots+ \frac{x^n}{n!} &+ o_0(x^n) \\
%              = \sum_{k=0}^n \frac{x^n}{n !} + x^{n+1}\varepsilon(x) \\
    \sin x &= x - \frac{x^3}{3!}+\frac{x^5}{5!} +\ldots+(-1)^p\frac{x^{2p+1}}{(2p+1)!}
           & +o_0(x^{2p+1} ) \\
%              = \sum_{k=0}^p (-1)^p\frac{x^{2p+1}}{(2p+1)!} + x^{2p+2}\varepsilon(x) \\
    \cos x &= 1 - \frac{x^2}{2!}+\frac{x^4}{4!} +\ldots+(-1)^p\frac{x^{2p}}{(2p)!}
           & +o_0(x^{2p}) \\
    \frac{1}{1-x} &= 1 + x + x^2 + + x^3 + \ldots + x^n
        & + o_0(x^n) \\
    \ln (1-x) &= -x - \frac{x^2}{2} - \frac{x^3}{3} - \frac{x^4}{4} - \ldots - \frac{x^n}{n}
        & + o_0(x^n) \\
    (1+x)^\alpha &= 1 + \alpha x + \alpha (\alpha-1) \frac{x^2}{2!} + \ldots \\ 
        & \qquad\qquad +\alpha(\alpha-1)\ldots(\alpha-n+1)\frac{x^n}{n!}
        & + o_0(x^n) \\
\end{align*}
Les règles de calcul sont aussi à connaître (vues en sup: composition, troncature, multiplication)
\end{rem}

\begin{dfn}[Voisinage\index{voisinage}]
    Un voisinage de $a\in\mathbb R$ est un intervalle du type $]a-\varepsilon, a+\varepsilon[$. Si $V$ est un voisinage de $a$, on note $V\in\mathcal V(a)$.
\end{dfn}

Si $V\in\mathcal V(0)$ et si pour $x\in V\setminus\{0\}$, $f(x)=a_0+a_1x +a_px^p+o_p(x^p)$, $a_p\neq 0$ et $p\geq 2$ alors \begin{itemize}
    \item $f$ est prolongeable en $0$ et $f(0)=a_0$
    \item $f$ est dérivable en $0$ et $f'(0)=a_1$
    \item $y=a_0+a_1x$ est l'équation cartésienne de la tangente au graphe de $f$ en $0$
    \item $f(x)-a_0-a_1x$ a le même signe que $a_px^p$ au voisinage de $0$.
\end{itemize}

\begin{rem}
    $f$ admet un $\mathrm{DL}_1(a)$ si et seulement si $f$ est dérivable en $a$. Ça n'est pas vrai pour les ordres supérieurs de dérivation.
\end{rem}

\begin{ex}
    On pose \[
        f: x\in]-\pi;\pi[\setminus\{0\} \longmapsto \frac 1x-\frac1{\sin x}
    \]
    et on a (en utilisant les DL usuels et les règles de calcul) \[
        \forall x\in D_f,\quad f(x)=\frac1x-\frac1{x-\frac{x^3}3+o_0(x^3)}=\frac1x \left( 1- \left( 1+\frac {x^2}6+o_0(x^2) \right)\right)=-\frac16x+o_0(x)
    \]
    donc $f$ est prolongeable en $0$ avec $f(0)=0$, $f'(0)=-\frac 16$.

    La fonction $f$ est $\mathcal C^1$ sur son domaine de définition, l'est-elle en $0$ ? On a \[
        \forall x\in D_f, \qquad f'(x)=-\frac1{x^2}+\frac{\cos x}{\sin^2x}=\frac1{x^2} \left( -1+\frac{1-\frac{x^2}2+o_0(x^2)}{(1-\frac{x^2}6+o_0(x^2))^2} \right)=-\frac16+o_0(1)\xrightarrow[x\to 0]{}-\frac16
    \]
    donc $f$ est $\mathcal C^1$ en $0$
\end{ex}

\section{Taylor-Young}

\begin{thm}[Taylor-Young\index{Taylor-Young (théorème de -- )}]
    \Hyp $f\in\mathcal C^n(V\in\mathcal V(0), \mathbb R)$
    \Conc \[
            \forall x\in V, \qquad f(x)=\sum_{k=0}^n\frac{f^{(k)}(0)}{k!}x^k+o_0(x^n)
    \]
\end{thm}

\begin{rem}
    On n'a en fait besoin que de $f\in\mathcal C^{n-1}(V, \mathbb R)$ et $f^{(n-1)}$ dérivable en $0$
\end{rem}

\begin{rem}
    On ne peut en général pas dériver termes à termes. Si $f$ est $\mathcal C^n$, on peut dériver termes à termes pour obtenir le $\mathrm{DL}_{n-1}(0)$ de $f'$.
\end{rem}

\begin{ex}[$a$]
    Pour obtenir un $\mathrm{DL}_3(1)$ de $\arctan$, on note $f=\arctan$ et \[
        f(1)=\frac\pi4\qquad f'(1)=\frac12\qquad f''(1)=-\frac12\qquad f^{(3)}(1)=\frac1{2}
    \]
    donc \[
        \forall x\in \left]-\frac\pi2;\frac\pi2  \right[, \qquad f(x)=\frac\pi4+\frac{x-1}2-\frac{(x-1)^2}4+\frac{(x-1)^3}{12}+o_1((x-1)^3)
    \]
\end{ex}

\begin{ex}[$b$]
    On note $f\in\mathcal C^{\infty}([-1; 1], \mathbb R)$ telle que $\forall n\in\mathbb N^\star$, \begin{align*}
        f \left( \frac1n \right) &= \frac{\sqrt n}{\sqrt{n+1}+\sqrt n}\\
                                 &= \frac1{1+\sqrt{1+\frac1n}}\\
                                 &= \frac{\sqrt{1+\frac1n}-1}{\frac 1n}\\
                                 &= n \left( \sum_{k=1}^{2020}\binom {1/2}{k}\frac1{n^k}+o_{+\infty}(\frac{1}{n^{2020}}) \right) \\
                                 &= \sum_{k=0}^{2019}\frac{f^{(k)}(0)}{k!}\frac1{n^k}+o_{+\infty}(\frac1{n^{2019}})
    \end{align*}
    Puis $n\to +\infty$ donne \[
        \binom{1/2}1=f(0)
    \]
    et $nf(1/n)$ avec $n\to+\infty$ donne \[
        \binom{1/2}2=\frac{f'(0)}{1!}
    \]
    et en itérant on obtient \[
        \binom{1/2}{2020}=\frac{f^{(2019)}(0)}{2019!}
    \]
\end{ex}

\section{Utilisation standard des développements limités}

\subsection{Calcul de limite}

On va calculer \[
    \lim_{x\to 0^+}\frac{x^{\sh x}-(\sh x)^x}{(\sin x)^x-x^{\sin x}}
\]
Et on a \[
    \sh x=x+\frac{x^3}6+o_0(x^3)\qquad\qquad \sin x=x-\frac{x^3}6+o_0(x^3)
\]
donc, après calculs, \[
    \frac{x^{\sh x}-(\sh x)^x}{(\sin x)^x-x^{\sin x}} = \frac{\frac{x^3}6\ln x+o_0(x^3\ln x)-\frac{x^3}6+o_0(x^3)}{\frac{-x^3}6+o_0(x^3)+\frac{x^3}6\ln x+o_0(x^3\ln x)}\underset{0+}\sim -1
\]



\ifsolo
    ~

    \vspace{1cm}

    \begin{center}
        \textbf{\LARGE Suites et séries numériques} \\[1em]
    \end{center}
    \tableofcontents
\else
    \minitoc
\fi
\thispagestyle{empty}

\ifsolo \newpage \setcounter{page}{1} \fi

\section{Rappels sur les séries numériques}

\begin{dfn}
    Soit $u\in\mathbb R^{\mathbb N}$. On appelle \textbf{série de terme général} \index{série numérique}$u_n$ la suite $(S_n(u))_n$ définie par \[
        S_n(u)\defeq \sum_{k=1}^nu_k
    \]
    On dira que cette série converge si $(S_n(u))$ converge, et on écrira \[
        \lim_{n\to+\infty}S_n(u)=\sum_{k=1}^{+\infty}u_k
    \]
    Dans ce cas, on pose \[
        R_n(u)\defeq\sum_{k=1}^{+\infty}u_k-S_n(u)
    \]
    qu'on appelle reste d'ordre $n$ de la série. La série de terme général $u_k$ sera notée $\sum u_k$
\end{dfn}


\begin{rem}
    Le terme général d'une séries convergente tend vers $0$. Si $u_n\longrightarrow \ell\in\bar{ \mathbb R}\setminus\{0\}$ alors la série diverge grossièrement. Si $\sum u_n$ converge alors $R_n(u)\longrightarrow0$ et on peut noter \[
        R_n(u)=\sum_{k=n+1}^{+\infty}u_k
    \]
\end{rem}

\begin{rem}
    On peut remplacer $\mathbb R$ par $\mathbb C$ ou n'importe quel espace vectoriel normé (traité dans un chapitre ultérieur)
\end{rem}

\begin{ex}
    Si $u_n=a^n$ pour un complexe $a$, alors \[
        S_n(u)= \begin{cases}
            n+1 &\text{ si }a=1\\[1em]
            \dfrac{1-a^{n+1}}{1-a}&\text{ sinon}
        \end{cases}
    \]
    d'où \[
        \sum u_n\text{ CV }\iff |a|<1
    \]
    et \[
        \sum_{k=0}^{+\infty}u_k=\frac1{1-a}
    \]
\end{ex}

\begin{ex}
    On a \[
        \ln \left( 1+\frac2{n(n+3)} \right)=\ln(n+1)+\ln(n+2)-\ln(n)-\ln(n+3)
    \]
    donc \[
        \sum_{k=1}^n\ln \left( 1+\frac2{k(k+3)} \right) =\ln3+\ln \left( \frac{n+1}{n+3} \right)\xrightarrow[n\to+\infty]{}\ln 3
    \]
\end{ex}

\needspace{5cm}
\section{Opérations}

\begin{prop}
    \Hyp $\sum u_n, \sum v_n$ deux séries numériques, $\lambda\in\mathbb C$
    \begin{concenum}
    \item $\lambda\sum u_n+\sum v_n=\sum(\lambda u_n+v_n)$
    \item Si les séries convergent, alors $\sum(\lambda u_n+v_n)$ converge (+ égalité des limites)
    \end{concenum}
\end{prop}

\begin{rem}
    On en déduit
    \begin{center}
    \begin{tabular}{ccccc}
        CV & + & CV & = & CV\\
        DV & + & CV & = & DV\\
        DV & + & DV & = & ?
    \end{tabular}
    \end{center}
\end{rem}

\begin{prop}
    \Hyp $u\in\mathbb C^{\mathbb N}$
    \Conc \[
        (u_n)\text{ CV }\iff \sum (u_{n+1}-u_n)\text{ CV }
    \]
\end{prop}

\begin{ex}[Séries de Riemann\index{série de Riemann}]
    On veut étudier $\sum\frac1{n^\alpha}$.
    \begin{itemize}
        \item Si $\alpha\leq0$ alors la série diverge grossièrement
        \item Si $\alpha> 0$ alors \[
                \frac1{(n+1)^\alpha}-\frac1{n^\alpha}=\frac1{n^\alpha} \left( -\frac\alpha n+O \left( \frac1n \right) \right)\sim -\frac\alpha{n^\alpha}
            \]
            or $(\frac1{n^{\alpha}})_n$ converge vers $0$ donc $\sum \frac1{(n+1)^\alpha}-\frac1{n^\alpha}$ aussi et donc, par équivalence\footnote{c'est une propriété qui sera vue dans la prochaine partie}, $\sum\frac1{n^{\alpha+1}}$ aussi.

        \item Pour $\alpha<0$, on trouve alors que $\sum\frac1{n^{\alpha+1}}$ diverge
    \end{itemize}
    Il ne reste que le cas $\alpha=1$ à traiter: la série diverge. On a donc \[
    \sum\frac1{n^\alpha}\text{ CV }\iff \alpha>1
\]
\end{ex}

On va maintenant s'intéresser à la série harmonique\index{série harmonique} \[
    H_n=\sum_{k=1}^n\frac1k.
\]
On a \[
    \ln(n+1)-\ln (n)\sim \frac1n \implies \sum\frac1k \text{ DV }
\]
et par équivalence des sommes partielles\footnote{vu plus tard} (ou bien par comparaison série-intégrale), \[
    H_n\sim\ln(n).
\]
On note $u_n=H_n-\ln n$. On a \[
    u_{n+1}-u_n=\frac1{n+1}-\ln \left( 1+\frac1n \right)=\frac1n\times \frac1{1+\frac1n}-\frac1n-\frac1{2n^2}+o \left( \frac1{n^2} \right)=-\frac2{n^2}+o \left( \frac1{n^2} \right)
\]
donc $\sum u_{n+1}-u_n$ converge donc $(u_n)$ converge vers un réel $\gamma$. On note \[
    v_n=H_n-\ln n-\gamma
\]
et \[
    v_{n+1}-v_n=-\frac1{2n^2}+o \left( \frac1{n^2} \right)
\]
donc $(v_n)$ converge. Par équivalence des restes (série CV), \[
    \sum_{k=n}^{+\infty}v_{k+1}-v_k\sim\sum_{k=n} -\frac12 \left( \frac1n-\frac1{n+1} \right)
\]
d'où $v_n\sim \frac1{2n}$.
On pose \[
    \omega_n=H_n-\ln n-\gamma-\frac1{2n}
\]
Après calculs (vérifiez-le), \[
    \omega_{n+1}-\omega_n=\frac1{6n^3}+o \left( \frac1{n^3} \right)\sim\frac1{6n^3}\sim\frac1{12}\cdot \left( \frac 1{n^2}-\frac1{(n+1)^2} \right)
\]
d'où $\omega_n\sim\frac1{12n^2}$

\textbf{Bilan.} (Question X) \[
    H_n=\ln n+\gamma+\frac1{2n}-\frac1{12n^2}+o \left( \frac1{n^2} \right)
\]

\begin{exo}[Mines-Ponts]
    On note \[
        k_n=\min\{k\in\mathbb N, \quad H_k>n\}
    \]
    Calculer \[
        \lim_{n\to+\infty}\frac{k_{n+1}}{k_n}
    \]
\end{exo}

\section{Séries à termes positifs}

\begin{prop}
    \Hyp $u$ est une suite positive
    \Conc $\sum u_n$ converge si et seulement si $(S_n(u))$ est majorée
\end{prop}

\begin{proof}
    \Hyp $u, v$ des suites positives
    \begin{concenum}
    \item Si $u_n\leq v_n$ à partir d'un certain rang et si $\sum v_n$ converge alors $\sum u_n$ aussi
    \item Si $u_n=O(v_n)$ et $\sum v_n$ converge alors $\sum u_n$ aussi
    \item Si $u_n\sim v_n$ alors $\sum u_n$ et $\sum v_n$ ont la même nature
    \end{concenum}
\end{proof}

\begin{proof}~
    \begin{enumerate}
        \item Si $u_n\leq v_n$ à partir du rang $N$, \[
                S_n(u)\leq \sum_{k=0}^Nu_k+\sum_{k=N+1}^{+\infty}v_k
            \]
            donc $(S_n(u))$ croissante est majorée donc converge.
        \item Idem à constante multiplicative près
        \item $u_n\sim v_n$ donne $u_n=O(v_n)$ et $v_n=O(u_n)$ et on applique $2$.
    \end{enumerate}
\end{proof}

\begin{exo}
    On note $(a_n)_{n\geq 1}$ une suite positive telle que $\sum a_n$ converge. Donner la nature de \[
        \sum a_n\sin(a_n), \qquad \sum\frac{a_n}{1+a_n^2}, \qquad \sum\frac{\sqrt{a_n}}{n}
    \]
\end{exo}

\begin{proof}[Résolution]
    À partir d'un certain rang, $0\leq a_n\sin(a_n)\leq a_n$ puis \[
        \frac{a_n}{1+a_n^2}\sim a_n
    \]
    et enfin (AM-GM) \[
        0\leq \frac{\sqrt{a_n}}n\leq\frac12 \left( a_n+\frac1{n^2} \right)
    \]
    donc les trois séries convergent.
\end{proof}

\begin{exo}[X]
    Soit $x$ une suite positive. Montrer que si $\sum x_n$ converge, alors \[
        \sum x_n^{\frac n{n+1}}
    \]
    aussi
\end{exo}

\begin{proof}[Résolution]~
    Idée: On fait un découpage. On pose \[
        I=\{n\in\mathbb N, \quad x_n\geq 0 \text{ et }x_n^{-\frac1{n+1}}>2\}
    \]
    Si $n\in I$ alors \[
        \frac12>x_n^{\frac1{n+1}}\iff \frac1{2^{n+1}}>x_n\implies \frac1{2^n}>y_n
    \]
    Sinon, \[
        x_n=0\implies y_n=0\leq 2x_n
    \]
    et \[
        x_n^{-\frac1{n+1}}\leq 2\implies y_n\leq 2x_n
    \]
    donc dans tous les cas, \[
        0\leq y_n\leq \frac1{2^n}+2x_n
    \]
\end{proof}

\section{Absolue convergence, semi-convergence}

\begin{dfn}
    Soit $a$ une suite réelle. \begin{enumerate}
        \item On dira que $\sum a_n$ est absolument convergnete\index{absolue convergence} si $\sum |a_n|$ converge
        \item On dira que $\sum a_n$ est \index{semi-convergence}semi convergente si elle est convergente mais pas absolument convergente
    \end{enumerate}
\end{dfn}

\begin{thm}
    L'absolue convergence entraîne la convergence
\end{thm}

\begin{proof}
    On écrit $a_n=a_n^+-a_n^-$ avec $a_n^+=\max(a_n, 0)$ et $a_n^-=\max(-a_n, 0)$ de sorte que $|a_n|=a_n^++a_n^-$
\end{proof}

\begin{rem}
    \begin{center}
    \begin{tabular}{ccccc}
        ACV & + & ACV & = & ACV\\
        ACV & + & SCV & = & SCV\\
        SCV & + & SCV & = & ?
    \end{tabular}
    \end{center}
\end{rem}

\section{Comparaison logarithmique}

\begin{prop}
    \Hyp $u, v$ deux suites strictement positives telles que APCR $N$ \[
    \frac{u_{n+1}}{u_n}\leq \frac{v_{n+1}}{v_n}
    \]
    \Conc $u_n=O(v_n)$
\end{prop}

\begin{proof}
    \[
        \forall n>N, \prod_{k=N}^{n-1}\frac{u_{k+1}}{u_k}=\frac{u_n}{u_N}\leq \frac{v_n}{v_N}\implies u_n\leq v_n\frac{u_N}{v_N}=O(v_n)
    \]
\end{proof}

\begin{thm}[Critère de D'Alembert\index{D'Alembert (critère de -- )}]
    \Hyp $a$ une suite positive telle que \[
        \frac{a_{n+1}}{a_n}\xrightarrow[n\to+\infty]{}\ell\in\bar{\mathbb R}
    \]
    \begin{concenum}
    \item Si $0\leq \ell <1$ alors $\sum a_n$ converge
    \item Si $\ell>1$ alors $\sum a_n$ diverge
    \end{concenum}
\end{thm}

\begin{proof}~
    \begin{itemize}
        \item $\exists \epsilon>0, \quad 0<\ell'=\ell+\epsilon<1$ et on applique la proposition précédente à $u=a$, $v=(\ell'^n)_n$.
        \item $\exists \epsilon > 0, \quad 1<\ell'=\ell-\epsilon$ et on fait pareil.
    \end{itemize}
\end{proof}

\begin{rem}
    Il faut faire attention à avoir $(a_n)$ jamais nulle à partir d'un certain rang. Le critère est intéressant lorsque $\frac{a_{n+1}}{a_n}$ a une expression simple.
\end{rem}

\ifsolo
    ~

    \vspace{1cm}

    \begin{center}
        \textbf{\LARGE Familles sommables} \\[1em]
    \end{center}
    \tableofcontents
\else
    \chapter{Familles sommables}

    \minitoc
\fi
\thispagestyle{empty}

\ifsolo \newpage \setcounter{page}{1} \fi

\section{Dénombrabilité}

\begin{dfn}
    Un ensemble est dénombrable si et seulement si il est en bijection avec $\N$
\end{dfn}

\begin{ex}
    $\N, 2\N, \N^\star, \Z$
\end{ex}

\begin{prop}
    \begin{enumerate}
        \item Toute partie infinie de $\N$ est dénombrable
        \item Toute partie de $\N$ est finie ou dénombrable
    \end{enumerate}
\end{prop}

\begin{proof}
    Soit $A$ une partie de $\N$ infinie. On note $\varphi(0)=\min A$ et \[
        \varphi(n)=\min (A\setminus \llbracket 0, \varphi(n-1)\rrbracket)
    \]
    de sorte que $\varphi$ est croissante donc injective, et surjective car si $a\in A$ n'a pas d'antécédent, $I=\{k, \quad a<\varphi(k)\}$ est non vide et a un minimum $p$, et $\varphi(0)<a$ donc $p\geq 1$ d'où $\varphi(p-1)< a<\varphi(p)$ absurde.
\end{proof}

\begin{prop}
    \begin{enumerate}
        \item Si $f:A\to B$ est injective et $B$ dénombrable alors $A$ est fini ou dénombrable
        \item Si $f:A\to B$ est surjective et $A$ dénombrable alors $B$ est fini ou dénombrable.
    \end{enumerate}
\end{prop}

\begin{proof}~
    \begin{enumerate}
        \item $B$ est en bijection avec $\N$ donc on peut supposer $B=\N$. Si $A$ est fini alors ok, sinon $f(A)$ est une partie infinie de $\N$ en bijection avec $A$.
        \item On note $\psi(b)$ un élément de $f^{-1}(\{b\})$. $f\circ \psi=\id_B$ injective donc $\psi$ est injective et on utilise 1.
    \end{enumerate}
\end{proof}

\begin{prop}
    \begin{enumerate}
        \item Si $I$ est non-vide, fini ou dénombrable, et $(A_i)_{i\in I}$ est une famille d'ensembles finis ou dénombrables, alors \[
                \bigcup_{i\in I}A_i
            \]
            est fini ou dénombrable
        \item $E_1,\cdots, E_n$ des ensembles finis ou dénombrables, non vides. $E_1\times \cdots \times E_n$ est fini ou dénombrable et $E_1\times \cdots \times E_n$ est dénombrable si et seulement si au moins l'un des $E_i$ est dénombrable.
    \end{enumerate}
\end{prop}

\begin{proof}~
    \begin{enumerate}
        \item On note $\varphi:\N\to I$ surjective et pour chaque $i$ on note $f_i:\N\to A_i$ bijective de sorte que \[
        \psi: (p, q)\in\N^2\longmapsto f_{\varphi(q)}(p)\in\bigcup_{i\in I}A_i
            \]
            est surjective, et $\N^2$ est dénombrable.
        \item Chaque $E_i$ est en bijection avec une partie de $\N$, on peut supposer $E_i\subseteq \N$. On note $p_1, \cdots, p_n$ des premiers disjoints. \[
                f: E_1\times \cdots \times E_n\longrightarrow \N, \quad (a_1, \cdots, a_n)\longmapsto p_1^{a_1}\cdots p_n^{a_n}
            \]
            est injective et $\N$ dénombrable. Si tous les $E_i$ sont finis alors le produit est fini et si l'un est infini alors le produit est infini.
    \end{enumerate}
\end{proof}

\section{Exemples d'ensembles classiques}

\begin{ex}
    $\Q$ est dénombrable car $\Z\times \N^\star$ est dénombrable et $(p, q)\in\Z\times \N^\star\longmapsto \frac pq$ est surjective
\end{ex}

\begin{ex}
    L'ensemble des nombres algébriques est dénombrable: \[
        \mathcal A=\bigcup_{n\in\N^\star}\bigcup_{P\in\Z_n[X]}\mathcal Z_{\C}(P)
    \]
\end{ex}

\begin{ex}[Diagonale de Cantor]
    $\R$ \index{Cantor!argument diagonal}n'est pas dénombrable. Si $\R$ est dénombrable alors $[0, 1[$ est dénombrable, et on note $(r_n)_{n\in\N}$ une énumération des éléments de $[0, 1[$. On construit le réel $x\in [0, 1[$ de la manière suivante: si la $n$-ième décimale de $r_n$ vaut $1$ alors la $n$-ième décimale de $x$ vaut $0$, sinon elle vaut $1$. Le réel ainsi construit ne peut pas être un $r_n$ car sa $n$-ième décimale est différente de celle de $r_n$. On a donc trouvé un réel qui n'est pas dans l'énumération, c'est absurde.
\end{ex}

\section{Discontinuités d'une fonction monotone}

On note $f:[a, b]\in\R$ croissante. On va montrer que l'ensemble des points de discontinuité de $f$ est fini ou dénombrable.

On note pour $x\in ]a, b[$, \[
    \delta(x)=\lim_{\substack{t\to x\\t>x}}f(t)-\lim_{\substack{t\to x\\t<x}}f(t)
\]

\begin{itemize}
    \item $f$ est croissante donc $\delta$ est positive et $f$ est continue en $a$ ssi $\delta(a)=0$.
    \item Pour tout $n\in\N^\star$, on note \[
            I_n= \left\{ x\in ]a, b[, \qquad \delta(x)\geq \frac{f(b)-f(a)+1}n \right\}
        \]
    \item Si $I_n$ est de cardinal $\geq n+1$ alors on peut trouver $x_1< \cdots< x_{n+1}$ dans $I_n$, et on se donne $a<t_1<x_1<\cdots <x_{n+1}<t_{n+2}<b$ pour avoir \[
            f(b)-f(a)=\underbrace{f(b)-f(t_{n+2})}_{\geq 0}+\underbrace{f(t_{n+2})-f(b_{n+1})}_{\geq \delta(x_{n+1})} +\cdots +\underbrace{f(t_2)-f(t_1)}_{\geq \delta(x_1)}+\underbrace{f(t_1)-f(b)}_{\geq 0}
        \]
        donc \[
            f(b)-f(a)\geq \delta(x_{n+1})+\cdots +\delta(x_1)\geq n\frac{f(b)-f(a)+1}n
        \]
        absurde donc $I_n$ est fini. $x\in ]a, b[$ est un point de discontinuité si et seulement si $\delta(x)>0$ ssi $\exists n\in\N, x\in I_n$ ssi \[
            x\in \bigcup_{n\in\N^\star}I_n
        \]
\end{itemize}

\section{Familles sommables de réels positifs}

Pour un ensemble dénombrable $I$, on note $\mathcal P_f(I)$ l'ensemble des parties finies de $I$

\begin{dfn}
    Soit $(a_i)_{i\in I}\in (\R_+)^I$. On dira que la famille $(a_i)_{i\in I}$ est \index{famille sommable}\textbf{sommable} si \[
        \left\{ \sum_{j\in J}a_j, \quad J\in\mathcal P_f(I) \right\}
    \]
    est majoré. On note alors \[
        \sum_{i\in I}a_i=\sup\left\{ \sum_{j\in J}a_j, \quad J\in\mathcal P_f(I) \right\}
    \]
    Si la famille n'est pas sommable on peut parfois écrire \[
        \sum_{i\in I}a_i=+\infty
    \]
\end{dfn}

\begin{prop}
    \Hyp $(a_i)_{i\in I}$ famille dénombrable de réels positive et $S\in \R$
    \Conc $(a_i)_{i\in I}$ est sommable et $S=\sum_{i\in I}a_i$ si et seulement si les deux conditions suivantes sont réunies: \begin{itemize}[left=2cm]
        \item $\forall J\in\mathcal P_f(I), \quad \displaystyle \sum_{j\in J}a_j\leq S$
        \item $\displaystyle \forall \epsilon>0, \exists J\in\mathcal P_f(J), \quad S-\epsilon\leq \sum_{j\in J}a_j$
    \end{itemize}
\end{prop}

\begin{proof}
    C'est une caractérisation de la borne supérieure.
\end{proof}

\begin{rem}
    Dans le cas $I=\N$ et $(a_i)_{i\in\N}$ positive, la sommabilité équivaut à la convergence de $\sum a_i$. Si la famille est sommable, $(S_n)$ est croissante majorée donc la série converge, et si la série converge, alors les sommes sur un segment sont majorées par la somme de la série.
\end{rem}

\begin{prop}
    \Hyp $(a_i)_{i\in I}$, $(b_i)_{i\in I}$ des familles de réels positifs
    \begin{concenum}
    \item Si $0\leq a_i\leq b_i$ pour tout $i\in I$ et $(b_i)_i$ est sommable, alors $(a_i)_i$ est sommable et \[
            \sum_{i\in I}a_i\leq \sum_{i\in I}b_i
        \]
    \item Pour $\lambda\in\R_+$, si $(a_i)_i$ et $(b_i)_i$ sont sommables, $(\lambda a_i+b_i)_i$ aussi et \[
            \lambda\sum_{i\in I}a_i+\sum_{i\in I}b_i=\sum_{i\in I}(\lambda a_i+b_i)
        \]
    \end{concenum}
\end{prop}

\begin{proof}~
    \begin{enumerate}
        \item Ok
        \item Pour tout $J\in \mathcal P_f(I)$, \[
                \sum_{j\in J}(\lambda a_j+b_j)=\lambda\sum_{j\in J}a_j+\sum_{j\in J}b_j\leq \lambda \sum_{i\in I}a_i+\sum_{i\in I}b_i\defeq M
            \]
            Soit $\epsilon>0$. Il existe $J_1, J_2\in\mathcal P_f(I)$ tels que \[
                \sum_{j\in I}a_j-\epsilon'\leq \sum_{j\in J_1}a_j\qquad \text{ et }\qquad \sum_{j\in I}b_j-\epsilon'\leq \sum_{j\in J_2}b_j
            \]
            donc pour $J=J_1\cup J_2$, \[
                M-\lambda\epsilon'-\epsilon'=M-\epsilon\leq \sum_{j\in J}(\lambda a_j+b_j)
            \]
            donc \conc
    \end{enumerate}
\end{proof}

\section{Familles sommables de complexes}

\begin{defprop}
    \Hyp $(a_k)_{k\in I}$ une famille dénombrable de complexes, $\alpha_k=\Re(a_k)$ et $\beta_k=\Im(a_k)$
    \begin{concenum}
    \item $(a_k)_{k\in I}$ est sommable\index{famille sommable} ssi $(|a_k|)_{k\in I}$ est sommable.
    \item Dans ce cas, $(\alpha_k^+)$, $(\alpha_k^-)$, $(\beta_k^+)$ et $(\beta_k^-)$ sont sommables\footnote{$u_n^+=\max(u_n, 0), \quad u_n^-=\max(-u_n, 0)$ pour écrire $u_n=u_n^+-u_n^-$} et on pose \[
            \sum_{i\in I}a_i\defeq\sum_{i\in I}\alpha_i^+-\sum_{i\in I}\alpha_i^-+i\sum_{i\in I}\beta_i^+-i\sum_{i\in I}\beta_i^-
        \]
    \item Si $(a_i)_i$ et $(b_i)_i$ sont sommables et $\lambda\in\C$ alors $(\lambda a_i+b_i)_{i\in I}$ est sommable et \[
            \lambda\sum_{i\in I}a_i+\sum_{i\in I}b_i=\sum_{i\in I}(\lambda a_i+b_i)
        \]
    \end{concenum}
\end{defprop}

\begin{proof}~
    \begin{enumerate}
        \item Définition
        \item $0\leq \alpha_k^\pm\leq |\alpha_k|\leq |a_k|$ pour la sommabilité
        \item L'inégalité triangulaire donne la sommabilité. La linéarité est admise car la démo est lourde.
    \end{enumerate}
\end{proof}

\begin{thm}[Sommation par paquets\index{sommation par paquets}]
    \Hyp $I$ dénombrable, $(I_n)_{n\in\N}$ une partition de $I$, $(a_i)_{i\in I}\in \C^I$
    \Conc Il y a équivalence entre \begin{enumerate}[left=1.1cm,label=\alph{enumi}.]
        \item la famille $(a_i)_i$ est sommable
        \item Pour tout $n\in\N$, $(a_i)_{i\in I_n}$ est sommable et \[
                \sum_{i\in I_n}a_i
            \]
            est le terme général d'une série absolument convergente. Dans ce cas, \[
                \sum_{i\in I}a_i=\sum_{n\in\N}\sum_{i\in I_n}a_i
            \]
    \end{enumerate}
\end{thm}

\begin{proof} La démonstration n'est pas au programme. On la fait dans le cas $(a_i)_i$ réelle positive.

    \begin{itemize}
        \item $(b\implies a)$ On prend $J$ une partie finie de $I$, \[
                I=\bigcup_{n\in\N}I_n
            \]
            donc il existe $n_0$ tel que \[
                J\subset \bigcup_{n\leq n_0}I_n
            \]
            On a \[
                0\leq \sum_{j\in J}a_j\leq \sum_{k=0}^{n_0}\sum{k\in I_n}a_k\leq \sum_{n\in\N}\sum_{k\in I_n}a_k
            \]
            donc $(a_i)_i$ sommable.
        \item $(a\implies b)$ Pour tout $J$ fini inclus dans $I_n$, $J\subset I$ donc \[
                0\leq \sum_{j\in J}a_j\leq \sum_{j\in I}a_j
            \]
            donc $(a_i)_{i\in I_n}$ est sommable.
            Soit $n_0\in\N$ et $J_0, \cdots, J_{n_0}$ finis tels que $J_k\subset I_k$.
            $J=J_0\cup \cdots \cup J_{n_0}$ est fini et \[
                \sum_{j\in J}a_j=\sum_{i\in J_0}a_i+\cdots +\sum_{i\in J_{n_0}}a_i\leq \sum_{j\in I}a_j
            \]
            On a alors \[
                \sum_{j\in J_0}a_j\leq \sum_{i\in I}a_i-\sum_{j\in J_1}a_j-\cdots -\sum_{j\in J_{n_0}}a_j
            \]
            et le membre de droite est indépendant de $J_0$. On passe à la borne supérieure par rapport à $J_0$: \[
                \sum_{i\in I_0}a_i=\sup_{\substack{J_0\subset I_0\\J_0\text{ fini}}}\sum_{j\in J_0}a_j\leq \sum_{i\in I}a_i-\sum_{j\in J_1}a_j-\cdots -\sum_{j\in J_{n_0}}a_j
            \]
            puis on recommence pour $J_1, \cdots, J_{n_0}$. On obtient ainsi \[
                \sum_{i\in I_0}a_i+\cdots +\sum_{i\in I_{n_0}}a_i\leq \sum_{j\in I}a_j.
            \]
            Les sommes partielles sont croissantes et majorées donc \[
                \sum \left( \sum_{k\in I_n}a_k \right)
            \]
            est absolument convergente (convergente + termes positifs). Puis, $n_0\to+\infty$ donne \[
                \sum_{n=0}^{+\infty}\sum_{k\in I_n}a_k\leq \sum_{i\in I}a_i
            \]
            Enfin, \[
                \sum_{j\in J}a_j\leq \sum_{k=0}^{n_0}\sum_{j\in I_k}a_j\leq \sum_{k=0}^{+\infty}\sum_{j\in I_k}a_j
            \]
            puis on passe au sup pour $J$ et on trouve la conclusion.
    \end{itemize}
    Pour $(a_i)$ complexe, on utilise la linéarité.
\end{proof}

\begin{cor}[Fubini discret\index{Fubini (théorème discret de -- )}]
    \Hyp $(a_{n, m})_{(n, m)\in\N^2}\in\C^{\N^2}$
    \Conc Il y a équivalence entre: \begin{enumerate}[left=1.2cm,label=\alph{enumi}.]
        \item $(a_{n,m})_{(n,m)\in\N^2}$ est sommable
        \item $\forall n\in\N, (a_{n, m})_{m\in\N}$ est sommable et $\sum_{m\in\N}|a_{n,m}|$ est le terme général d'une série convergente. Dans ce cas, \[
                \sum_{(n, m)\in\N^2}a_{n, m}=\sum_{n\in\N}\sum_{m\in\N}a_{n,m}=\sum_{m\in\N}\sum_{n\in\N}a_{n, m}
            \]
    \end{enumerate}
\end{cor}

\begin{ex}
    \[
        \sum_{n\geq 2}(\zeta(n)-1)
    \]
    On a $\zeta(n)-1=\sum_{k\geq 2}\frac1{k^n}$. La famille $\displaystyle \left( \frac1{k^n} \right)_{n, k\geq 2}$ est-elle sommable ?

    Pour $k\geq 2$, $\left(\frac1{k^n}\right)_{n\geq 2}$ est sommable (géométrique) et \[
        \sum_{n\geq 2}\frac1{k^n}=\frac1{k^2}\frac1{1-\frac1k}=\frac1{k(k-1)}
    \] est le terme général d'une série convergente, donc la famille est sommable et \[
    \sum_{m\geq 2}\sum_{n\geq 2}\frac1{k^n}=\sum_{k\geq 2}\frac1{k-1}-\frac1k=1=\sum_{n\geq 2}\sum_{k\geq 2}\frac1{k^n}=\sum_{n\geq 2}(\zeta(n)-1)
\]
donc \[
    \sum_{n\geq 2}(\zeta(n)-1)=1
\]
\end{ex}

\begin{rem}
    Une suite réelle est sommable si et seulement si elle est le terme général d'une série absolument convergente. Dans ce cas, \[
        \sum_{n\in\N}a_n=\sum_{n=0}^{+\infty}a_n
    \]
\end{rem}

\begin{prop}[Convergence cumulative]
    \Hyp $(a_n)_{n\in\N}$ est une suite complexe
    \Conc Il y a équivalence entre \begin{enumerate}[left=1.2cm,label=\alph{enumi}.]
        \item $(a_n)$ est sommable
        \item $\sum a_n$ est absolument convergente
        \item $\forall \sigma\in\mathfrak S(\N), (a_{\sigma(n)})$ est sommable
    \end{enumerate}
\end{prop}

\begin{proof}~
    \begin{itemize}
        \item $(a\iff b)$ OK
        \item $(b\implies c)$ Soit $\sigma\in\mathfrak S(\N)$ et $J$ fini. \[
                \sum_{k\in J}|a_{\sigma(k)}|=\sum_{k\in\sigma(J)}|a_k|\leq \sum_{k\in\N}|a_k|
            \]
        \item $(c\implies a)$ $\sigma=\id$
    \end{itemize}
\end{proof}

\begin{thm}[Produit de Cauchy]
    \Hyp $(a_p)_{p\in\N}$, $(b_q)_{q\in\N}$ des familles complexes sommables et \[
        c_n=\sum_{p+q=n}a_pb_q=\sum_{k=0}^na_kb_{n-k}=\sum_{k=0}^na_{n-k}b_k
    \]
    \begin{concenum}
    \item $(c_n)_{n\in\N}$ est sommable
    \item \[
            \sum_{n=0}^{+\infty}c_n= \left( \sum_{p=0}^{+\infty} a_p \right)\cdot \left( \sum_{q=0}^{+\infty}b_q \right)
        \]
    \end{concenum}
\end{thm}

\begin{proof}
    On note \[
        I_n=\{(p, q)\in\N^2, \quad p+q=n\}
    \]
    de sorte que $(I_n)$ est une partition de $\N^2$. On note \[
        J_n=\{n\}\times \N
    \]
    de sorte que $(J_n)$ est aussi une partition de $\N^2$. La famille $(a_nb_q)_{q\in\N}$ est sommable et \[
        \sum_{q\in\N}|a_nb_q|=|a_n|\underbrace{\sum_{q\in\N}|b_q|}_{\text{fixé}}
    \]
    est le terme général d'une série convergente ($(a_n)$ sommable).
    On a donc $(a_pb_q)_{(p, q)\in\N^2}$ sommable et on somme par paquets sur les $I_n$:\[
        \sum_{(p, q)\in\N^2}a_pb_q=\sum_{n=0}^{+\infty}\sum_{p+q=n}a_pb_q=\sum_{n\in\N}c_n
    \]
\end{proof}

\begin{rem}
    On peut généraliser avec \[
        \sum_{i_1, \cdots, i_r}a_{1,i_1}\cdots a_{r,i_r}=\sum_{n=0}^{+\infty}\sum_{i_1+\cdots+i_r=n}a_{1,i_1}\cdots a_{r, i_r}
    \]
\end{rem}

\section{Exemples}

\subsection{Théorème de réarrangement de Steinitz}

\index{Steinitz (théorème de réarrangement de -- )}

On note $(a_n)_n$ une suite réelle telle que $\sum a_n$ est semi-convergente. On va montrer que pour tout $\ell\in\bar{\R}$, il existe $\sigma\in\mathfrak S(\N)$ tel que \[
    \sum_{n=0}^{+\infty}a_{\sigma(n)}=\ell
\]
On note $a_n=a_n^+-a_n^-$ donc $\sum a_n^+$ et $\sum a_n^-$ divergent car $\sum a_n$ semi-convergente. On fixe $\alpha\in\R$, \[I^+=\{n\in\N, a_n\geq 0\}\qquad I^-=\{n\in\N, a_n<0\}\]
On note $(i_n)_n$ (resp. $(j_n)_n$) la suite croissante de tous les éléments de $I^+$ (resp. $I^-$).
On note $k_1$ le plus petit indice $k\geq 0$ tel que \[
    -\alpha+a_{i_1}+\cdots +a_{i_{k_1}}>0
\]
et $\rho_1$ le plus petit indice tel que \[
    -\alpha+a_{i_1}+\cdots +a_{i_{k_1}}+a_{j_1}+\cdots +a_{j_{\rho_1}}<0
\]
On itère le processus pour construire $\sigma=(i_1, \cdots, i_{k_1}, j_1, \cdots j_{\rho_1}, i_{k_1+1}, \cdots, i_{k_2}, \cdots)\in\mathfrak S(n)$ (c'est bijectif car $I^+\sqcup I^-=\N$).

Puis, \[
    \left| -\alpha+\sum_{k=0}^na_{\sigma(k)} \right|\leq |a_{p_n}|+|a_{q_n}|\xrightarrow[n\to+\infty]{}0
\]
avec $p_n$ le plus grand entier tel que $\rho_{p_n}\leq \sigma(n)$ et $q_n$ le plus petit entier tel que $\sigma(n)\leq \rho_{q_n}$
Pour $\ell=\pm\infty$ c'est le même principe.

\subsection{Inégalité de Carleman}

\index{Carleman (inégalité de -- )}

On suppose que $\sum a_n$ converge pour $(a_n)$ positive. On va montrer que \[
    \sum_{k=1}^{+\infty}(a_1\cdots a_k)^{\frac1k}\leq e\sum_{k=1}^{+\infty}a_n
\]
On note $(c_n)_n$ une suite à préciser. \[
    (a_1\cdots a_n)^{\frac1n}=(c_1a_1\cdots c_na_n)^{\frac1n}\cdot (c_1\cdots c_n)^{-\frac1n}
\]
On va prendre $c_k=\dfrac{(k+1)^k}{k^{k-1}}$ de sorte que $(c_1\cdots c_n)^{\frac1n}=n+1$. On a alors \begin{align*}
    (a_1\cdots a_n)^{\frac 1n}&\leq \frac1{n(n+1)}\sum_{k=1}^nc_ka_k \tag{AM-GM}\\
                              &\leq \frac e{n(n+1)}\sum_{k=1}^nka_k=\sum_{k=1}^n\frac{eka_k}{n(n+1)}
\end{align*}
car $c_k=k \left( 1+\frac1k \right)^k\leq ek $. On pose \[
    u_{k, n}\defeq \begin{cases}
        \dfrac{eka_k}{n(n+1)}&\text{ si }k\leq n\\[1em] 0 & \text{ sinon }
    \end{cases}
\]
Pour $k\in\N^\star,$ \[
    \sum_{n=1}^{+\infty}u_{k, n}=\sum_{n=k}^{+\infty}\frac{eka_k}{n(n+1)}=eka_k\sum_{n=k}^{+\infty} \left( \frac1k-\frac1{k+1} \right)=ea_k
\]
qui est le tg d'une série convergente, donc $(u_{k, n})_{k, n}$ est sommable, puis (Fubini) \[
    \sum (a_1\cdots a_n)^{\frac1n} \text{ CV }
\]
et \[
    \sum_{n=1}^{+\infty}(a_1\cdots a_n)^{\frac1n}\leq \sum_{k=1}^{+\infty}\sum_{n=1}^{+\infty}u_{k,n}=e\sum_{k=1}^{+\infty}a_k
\]

\begin{rem}
    La constante $e$ est optimale.
\end{rem}

Pour le voir, on pose \[
    a_k= \begin{cases}
        \frac1k&\text{ si }k\leq n\\ 0&\text{ sinon }
    \end{cases}
\]
et $c$ une autre constante qui convient. \[
    \sum_{k=1}^n(a_1\cdots a_k)^{\frac 1k}=\sum_{k=1}^n \left( \frac1{k!} \right)^{\frac1k}\leq c\sum_{k=1}^n\frac1k
\]
et \[
    \left( \frac1{k!} \right)^{\frac1k}=\exp \left( -\frac1k\sum_{i=1}^k\ln i \right)
\]
avec \[
    \ln i\sim i\ln i-(i-1)\ln (i-1) \implies \sum_{i=1}^k\ln i\sim k\ln k
\]
Puis on pose $\displaystyle u_k=\sum_{i=1}^k\ln i-k\ln k$ de sorte que \[
    u_{k+1}-u_k=\ln(k+1)-(k+1)\ln(k+1)+k\ln k\sim -1
\]
C'est le terme général d'une série divergente (signe constant APCR) donc $u_k\sim -k$ et \[
    -\frac1k\sum_{i=1}^k\ln i=-\ln k+1+o(1)
\]
donc \[
    \left( \frac1{k!} \right)^{\frac1k}\sim \frac ek.
\]
C'est le terme général d'une série divergente donc \[
    \sum_{k=1}^n (a_1\cdots a_k)^{\frac1k}\sim e\sum_{k=1}^n\frac1k\sim e\ln n
\]
donc $e\leq C$

\section{Produits infinis}

Soit $(a_n)$ une suite numérique\index{produit infini} et \[
    p_n=\prod_{k=0}^n(1+a_k)
\]
On dira que le produit $\prod(1+a_k)$ converge si et seulement si $p_n$ a une limite réelle \textbf{non nulle}

\begin{res}
    Si $\sum a_k$ est absolument convergente et $(a_n)$ ne vaut jamais $-1$, alors $\prod(1+a_k)$ converge
\end{res}

\begin{proof}
    On le montre dans le cas réel. On a $a_n\longrightarrow 0$ donc à partir d'un certain rang $n_0$, $(1+a_n)$ est strictement positive. Pour $N>n_0$, \[
        p_n=\underbrace{\prod_{k=0}^{n_0}(1+a_k)}_{\text{fixe } \neq 0}\times \prod_{k=n+1}^{N}\underbrace{(1+a_k)}_{>0}
    \]
    Or pour $k\geq n_0$, $|\ln(1+a_k)|\sim |a_k|$ donc $\sum \ln(1+a_k)$ ACV et sa somme vaut $\ell\in\R$. On a alors \[
        p_n=e^{\ln p_n}\xrightarrow[n\to+\infty]{}e^\ell \prod_{k=0}^{n_0}(1+a_k) \neq 0
    \]
\end{proof}

\begin{rem}
    Le resultat est vrai avec une suite complexe.
\end{rem}

\begin{ex}
    On note $f:\N^\star\longrightarrow \R_+$ complètement multiplicative\footnote{$\forall x,y, f(xy)=f(x)f(y)$} (le résultat que l'on va montrer est toujours vrai pour $f$ multiplicative\footnote{$\forall x,y, x\land y=1\implies f(xy)=f(x)f(y)$})

    On suppose que $\sum f(n)$ est absolument convergente. On va montrer que \[
        \prod_{p\text{ premier }} \frac1{1-f(p)}=\sum_{n=1}^{+\infty}f(n)
    \]
    On note $(p_i)_i$ la suite croissante des nombres premiers. Si il existe un premier $p$ tel que $f(p)=1$ alors $f(p^n)=1$ et $f(n)\xnrightarrow{}0$ ce qui est absurde, donc pour tout $p$ premier, $f(p)\neq 1$.
    On a alors $\sum f(p_i)$ ACV donc et $f(p_i)\neq 1$ donc $\prod (1-f(p_i))$ converge donc $\prod \frac1{1-f(p_i)}$ aussi. Puis, \[
        P_N=\prod_{i=1}^N\frac1{1-f(p_i)}=\prod_{i=1}^N \sum_{k=0}^{+\infty}f(p_i)^k\underset{\Pi\text{ de Cauchy}}=\sum_{k_1, \cdots, k_N}f(p_1^{k_1}\cdots p_N^{k_N})
    \]
    d'où \[
        \sum_{n=0}^{M_N}f(n)\leq P_N\leq\sum_{n=0}^{+\infty}f(n)
    \]
    avec $M_N=\max \{k, \quad \llbracket 1, k\rrbracket\text{ se décompose avec } p_1, \cdots p_N\}$. Puis, $N\to+\infty$ donne le résultat.
\end{ex}

Avec $s>1$ et $f:n\longmapsto \frac1{n^s}$, on trouve l'identité d'Euler\index{Euler!identité}: \[
    \prod_{p\in\mathbb P}\frac1{1-\frac1{p^s}}=\zeta(s)
\]

\endchapter

\ifsolo
    ~

    \vspace{1cm}

    \begin{center}
        \textbf{\LARGE Intégration sur un intervalle quelconque} \\[1em]
    \end{center}
    \tableofcontents
\else
    \minitoc
\fi
\thispagestyle{empty}

\ifsolo \newpage \setcounter{page}{1} \fi

\section{Rappels}

L'opération d'intégration sur $\CPM([a, b], \mathbb R)$ est linéaire positive croissante. On a \[
    \left| \int_a^bf \right|\leq \int_a^b|f|\leq (b-a)\sup_{[a,b]} |f|
\]
et similairement \[
    \left| \int_a^bfg \right|\leq \sup_{[a,b]}|f|\int_a^b|g|
\]
Si $f$ est continue alors \[
    \left| \int_a^bf \right|=\int_a^b|f|\iff f\text{ de signe constant }
\]
et si $g$ est continue positive, \[
    \inf_{[a,b]}f\int_a^bg\leq \int_a^bfg\leq \sup_{[a,b]}f\int_a^bg
\]
et si $g$ n'est pas identiquement nulle, \[
    \int_a^bg>0\implies \dfrac{\int_a^bfg}{\int_a^bg}\in[\inf f,\sup f]=f([a,b])\implies \exists c, \quad \int_a^bfg=f(c)\int_a^bg.
\]
Le résultat reste vrai si $g$ est identiquement nulle. Si $f$ est continue positive sur $[a, b]$ et $\int_a^bf=0$ alors $f=0$.
Si $f$ et $g$ sont de classe $\mathcal C^1$ alors \[
    \int_a^bfg'=[fg]_a^b-\int_a^bf'g
\]
et si $\varphi:[\alpha,\beta]\to[a,b]$ est $\mathcal C^1$, $f$ continue sur $\varphi([\alpha,\beta])$ alors \[
    \int_{\varphi(\alpha)}^{\varphi(\beta)}f(u)\diff u=\int_\alpha^\beta f(\varphi(t))\varphi'(t)\diff t
\]
Si $f$ est continue par morceaux sur $[a, b]$ alors \[
    \frac1{b-a}\sum_{k=0}^{n-1}f \left( a+k\frac{b-a}n \right)\xrightarrow[n\to+\infty]{}\int_a^bf
\]

\begin{thm}[Taylor avec reste intégral\index{Taylor!formule avec reste intégral}]
    \Hyp $f$ de classe $\mathcal C^n$ sur $[a, b]$
    \Conc \[
        f(b)=\sum_{k=0}^n\frac{f^{(k)}(a)}{k!}(b-a)^k+\int_a^b\frac{(b-t)^n}{n!}f^{(n+1)}(t)\diff t
    \]
\end{thm}

\begin{proof}
    Vu en sup.
\end{proof}

\begin{rem}
    On peut échanger $a$ et $b$ dans la formule.
\end{rem}

\begin{ex}[Irrationnalité de $\cos 1$]
    Supposons $\cos 1=\frac pq\in\mathbb Q$ de sorte que \[
        \cos 1=\sum_{k=0}^q\frac{\cos^{(k)}(0)}{k!}+\int_0^1\frac{(1-t)^q}{q!}\cos^{(q+1)}(t)\diff t\implies0\leq \underbrace{q! \left| \cos 1-\sum_{k=0}^q\frac{\cos^{(k)}(0)}{k!} \right|}_{\in\mathbb N}= \left| \int_0^1(1-t)^q\cos^{(q+1)}(t)\diff t \right|<1
    \]
    donc \[
        \cos 1=\sum_{k=0}^q\frac{\cos^{(k)}(0)}{k!}
    \]
    d'où \[
        \int_a^b\underbrace{(1-t)^q\cos^{(q+1)}(t)}_{\mathcal C^0\text{ de signe constant }}\diff t=0\implies (1-t)^q\cos^{(q+1)}(t)\equiv 0
    \]
    ce qui est aburde.
\end{ex}

\section{Intégrales de Wallis}

On appelle intégrales de Wallis les intégrales \[
    W_n\defeq\int_0^{\frac\pi2}\cos^n\underset{\;u=\frac\pi2-t\;}=\int_0^{\frac\pi2}\sin^n
\]
On a \[
    W_0=\frac\pi2\qquad \qquad \qquad W_1=[\sin t]_0^{\frac\pi2}=1
\]
Pour $n\geq 2$, \[
    W_n=\int_0^{\frac\pi2}\cos \cdot \cos^{n-1}=[\sin\cos^{n-1}]_0^{\frac\pi2}+\int_0^{\frac\pi2}\underbrace{\sin^2}_{1-\cos^2}\cdot~(n-1)\cos^{n-2}=(n-1)W_{n-2}-(n-1)W_n
\]
donc \[
W_n=\frac{n-1}nW_{n-2}
\]
de sorte que \[
    W_{2p}=\frac{(2p)!}{2^{2p}(p!)^2}\frac\pi2\qquad\qquad\qquad W_{2p+1}=\frac{2^{2p}p!^2}{(2p+1)!}.
\]
Pour $t\in [0, \frac\pi2]$, on a $\cos^{n+1}t\leq \cos^nt$ donc $W_{n+1}\leq W_n$. Puis \[
    W_{n+2}=\frac{n+1}{n+1}W_n\leq W_{n+1}\leq W_n \implies \frac{W_{n+1}}{W_n}\xrightarrow[n\to+\infty]{}W_{n+1}\sim W_n.
\]
La suite $((n+1)W_nW_{n+1})_n$ est constante égale à $W_0W_1=W_0$ donc \[
    \frac\pi2(n+1)W_{n+1}W_n\sim nW_n^2\implies W_n\sim\sqrt{\frac{\pi}{2n}}
\]
or \[
    W_{2p}\sim\frac{C\sqrt{2p}(2p)^{2p}e^{-2p}}{2^{2p}C^2(\sqrt pp^pe^{-p})^2}\frac\pi2\sim \frac{\sqrt2}{C\sqrt p}\sqrt{\frac\pi2}\underset{\text{ aussi }}\sim\sqrt{\frac{\pi}{4p}}
\]
donc \[
    C=\sqrt {2\pi}
\]

\section{Fonctions continues par morceaux}

\begin{dfn}[Rappel]\index{continuité par morceaux}
    $f:[a, b]\longrightarrow\mathbb R$ (ou $\mathbb C$) est dite continue par morceaux s'il existe une subdivision finie $\sigma=(a_1, \cdots, a_n)$ ($a_1=a$, $a_n=b$) de $[a, b]$ telle que $f$ est continue sur $]a_n, a_{n+1}[$, et admet une limite à droite et à gauche (ou d'un seul côté aux bords de l'intervalle) en chacun des points de la subdivision.

    L'ensemble de ces fonction est noté $\CPM([a, b], \mathbb R)$
\end{dfn}

\begin{dfn}
    Si $I$ est un intervalle non trivial de $\mathbb R$, on dira que $f$ est continue par morceaux sur $I$ si $f$ est continue par morceaux sur tous segments inclus dans $I$.
\end{dfn}

\begin{ex}
    La partie entière est $\CPM$ sur $\mathbb R$. L'application \[
        x\longmapsto \floor{\frac1x}
    \]
    est $\CPM$ sur $]0, 1]$
\end{ex}

\begin{prop}
    \Hyp $I$ est un intervalle réel non trivial, $f,g\in \CPM(I, \mathbb R)$
    \begin{concenum}
    \item Pour tout $\lambda\in\mathbb R$ (ou $\mathbb C$), $\lambda f+g\in\CPM(I, \mathbb R)$
    \item $fg$ est $\CPM$
    \item $|f|$ est $\CPM$. En particulier, $\max (f,g)$ et $\min (f,g)$ aussi
    \end{concenum}
\end{prop}

\begin{proof}
    On se ramène à un segment de $I$ et c'est vu en sup. \[
        \max(f,g)=\frac{f+g+|f-g|}2
    \]
\end{proof}

\begin{thm}[Rappel]
    \Hyp $f$ continue sur $[a,b]$
   \begin{concenum}
   \item  On note $E(I, \mathbb R)$ l'ensemble des fonctions en escaliers de $I$ dans $\mathbb R$. \[
           \forall \varepsilon>0, \exists \varphi\in E([a, b], \mathbb R), \quad \sup_{[a, b]}|f-\varphi|\leq \varepsilon
       \]
   \item \[
        \exists (\varphi_n)_n\in E([a,b]\in\mathbb R)^{\mathbb R}, \forall n\in\mathbb N^\star, \quad \sup_{[a,b]}|f-\varphi_n|\leq\frac1n
       \]
   \end{concenum}
\end{thm}

\begin{proof}[Idée de la preuve]
    $f$ est uniformément continue (Heine) donc avec une subdivision régulière assez petite on a bien l'approximation.
\end{proof}

\section{Intégration sur un intervalle quelconque}

On note $I$ un intervalle non trivial de $\mathbb R$ et $f\in\CPM(I, \mathbb R)$.

\begin{dfn}
    \begin{enumerate}
        \item Si $I=[a, b[$ avec $b\in\mathbb R\cup \{+\infty\}$, alors \[
                \int_If=\int_a^bf\defeq\lim_{x\to b}\int_a^xf
            \]
            lorsque cette limite existe (on dira alors que l'intégrale est convergente).
        \item Si $I=]a, b[$ alors de même, lorsque la limite existe on définit \[
                \int_If=\int_a^bf\defeq \lim_{x\to a}\lim_{y\to b}\int_x^yf
            \]
            On peut inverser les deux limites car pour $c\in I$, \[
                \int_x^yf=\int_x^cf+\int_c^yf
            \]
    \end{enumerate}
\end{dfn}

\begin{ex}
    L'application $t\longmapsto e^{-t}$ est $\CPM$ sur $\mathbb R_+$ et \[
        \int_0^x e^{-t}\diff t=1-e^x\xrightarrow[x\to+\infty]{}1\implies \int_0^{+\infty}e^{-t}\diff t=1
    \]
\end{ex}

\subsection{Lien avec les primitives}

Si $f$ est continue sur $[a, b[$ et $F$ est une primitive de $f$, alors pour $x\in[a, b[$, \[
    \int_a^xf=F(x)-F(a)
\]
donc $\int_a^b f$ CV $\iff F$ a une limite finie en $b$.

\begin{ex}
    \[
        \int_0^1\frac{\diff t}{t^\alpha}\text{ CV }\iff t\longmapsto \begin{cases}
            \dfrac{t^{-\alpha+1}}{1-\alpha} &\text{ si }\alpha\neq 1\\\ln|t|&\text{ sinon }
        \end{cases}
        \qquad \text{ a une limite finie en $0^+$ }
        \iff \alpha<1
    \]
    \[
        \int_1^{+\infty}\frac{\diff t}{t^\alpha}\text{ CV }\iff \alpha>1
    \]
\end{ex}

\ifsolo
    ~

    \vspace{1cm}

    \begin{center}
        \textbf{\LARGE Suites et Séries de fonctions} \\[1em]
    \end{center}
    \tableofcontents
\else
    \chapter{Suites et Séries de fonctions}

    \minitoc
\fi
\thispagestyle{empty}

Dans tout le chapitre, $I$ est un intervalle non trivial, $A$ est une partie non vide de $\R$ et $K=\R$ ou $\C$. Plus tard, on pourra remplacer $A$ par une partie d'un e.v.n (de dimension finie généralement) et $K$ un e.v.n.

\section{Différents types de convergence}

\begin{dfn}
    On note $(f_n)_n$ une suite de fonctions de $I$ dans $K$. \begin{enumerate}
        \item On dit que $(f_n)_n$ \emph{converge simplement}\index{convergence simple} (CVS) s'il existe $f:I\to K$ telle que $\forall x\in I, f_n(x)\xrightarrow[n\to+\infty]{} f(x)$
        \item On dit que $(f_n)_n$ \emph{converge uniformément}\index{convergence uniforme} (CVU) s'il existe $f:I\to K$ telle que \begin{itemize}
            \item $(f_n-g)$ est bornée à partir d'un certain rang
            \item $\sup_I|f_n-g|\xrightarrow[n\to+\infty]{}0$
        \end{itemize}
    \item On dit que $(f_n)$ converge uniformément sur tout segment (CVUTS) si pour tout segment $J\subset I$, on a $f_n\xrightarrow[J]{\mathrm{CVU}}f$
    \end{enumerate}
    Dans chacun des cas, on peut remplacer $I$ par $A$.
\end{dfn}

\begin{rem}
    \begin{itemize}
        \item CVU $\implies$ CVUTS $\implies$ CVS
        \item CVS $\centernot\implies$ CVUTS $\centernot \implies$ CVU
    \end{itemize}
\end{rem}

\begin{rem}
    La limite uniforme d'une suite uniformément convergente est identique à la limite simple de cette suite. En particulier, on a unicité de la limite pour les trois convergences.
\end{rem}

\section{Opérations simples sur les convergences}

\begin{prop}
    \Hyp $(u_n)_n,(v_n)_n$ sont des suites de fonctions de $I$ dans $K$, $u,v:I\to K$, $\lambda\in K$.
    \Conc Si $u_n\xrightarrow{}u$, $v_n\xrightarrow v$ avec de la convergence simple (resp. CVUTS, CVU) alors \[
        (\lambda u_n+v_n)\xrightarrow{}\lambda u+v
    \]
    avec la convergence simple (resp. CVUTS, CVU).
\end{prop}

\begin{proof}
    Évident.
\end{proof}

\begin{rem}
Pour la convergence simple, on a $u_nv_n \longrightarrow uv$, mais pas pour la  CVU ou la CVUTS. Par exemple, \[
\begin{matrix}
    u:& \R_+ & \longrightarrow &\R  \\
    & x & \longmapsto & e^x+\frac{1}{n}
\end{matrix}
\] 
On a \[
    u_n \xrightarrow[\R_+]{\text{CVU}} u
\] 
On a aussi  \[
    u_n^2 \xrightarrow[\R_+]{\text{CVS}} u^2
\]
donc si $u_n^2$ converge uniformément, \[
    u_n^2(x)-u^2(x) = \frac{2e^x}{n}+\frac{1}{n^2} \xrightarrow[n\to+\infty]{}0
\] 
ce qui est absurde
\end{rem}

\begin{prop}[Propriétées héritées par CVS]
    \Hyp $u_n: I\to \K$, $u_n \xrightarrow[]{\text{CVS}} u$ sur $I$
    \Conc La monotonie, la convexité, le caractère $k$-lipschitzien, la positivité sont hérités.
\end{prop}

\begin{proof}
On écrit les inégalités et on passe à la limite.
\end{proof}

\section{Propriétés héritées par CVU}

\begin{thm}[Double limite\index{double limite (théorème)}]
    \Hyp $u_n \xrightarrow[]{\text{CVU}} u$ sur $A$,  $a$ point adhérent de  $A$,  $u_n(x) \xrightarrow[x\to a]{}\ell _n \in  \K$
    \begin{concenum}
    \item $(\ell _n)$ converge vers $\ell  \in \K$
    \item La fonction $u$ a pour limite  $\ell $ en $a$, i.e.  \[
            \lim_{\substack{x \to  a\\x \in  A}} \lim_{x \to  +\infty} u_n(x)=\lim_{n \to  +\infty}\lim_{\substack{x \to  a\\ x \in  A}} u_n(x)
    \] 
    \end{concenum}
\end{thm}

\begin{proof}~
\begin{enumerate}
    \item (admis) On va construire $(n_k)$ strictement croissante telle que  $ \forall  n\geq n_k, |\ell _n-\ell _{n_k}| \leq  \dfrac{1}{2^k}$. \begin{itemize}
        \item $\forall \epsilon>0, \exists  N \in  \N, \forall  n\geq N, \forall  x \in  A, |u_n(x)-u(x)|\leq \epsilon'$
        \item $\forall  n, p\geq N, |u_n(x)-u_p(x)|\leq |u_n(x)-u(x)|+|u(x)-u_p(x)|\leq 2\epsilon'=\epsilon$ et $x \longrightarrow a$ donne  $|\ell _n-\ell _p|\leq \epsilon$
        \item Pour $\epsilon=1$, il existe  $n_0 \in  \N$ tel que $\forall n\geq n_0, |\ell _n-\ell _{n_0}|\leq 1$
        \item Pour $k \in  \N$, si on suppose $n_0<\cdots <n_k$ construits, pour $\epsilon=\frac{1}{2^{k+1}}$, il existe $N \in  \N$ tel que $\forall  p \geq N, |\ell  _n-\ell _p|\leq \epsilon$. On prend $n_{k+1}=\max(N, n_0,\cdots , n_k+1)$.
    \end{itemize}
    Puis $|\ell _{n_{k+1}}-\ell _{n_k}|\leq \frac{1}{2^k}$, c'est le terme général d'une série convergente donc $(\ell _{n_k})$ converge.

     Soit $\epsilon>0$. Il existe $K$ tel que  $k\geq K$, $\frac{1}{2^k}\leq \epsilon'$ et $|\ell  -\ell _{n_k}|\leq \epsilon'$ donc pour $n\geq n_k$, \[
    |\ell _n-\ell |\leq |\ell _n-\ell _{n_k}|+|\ell _{n_k}-\ell |\leq 2\epsilon'=\epsilon
    \] 
    donc $\ell _n \longrightarrow \ell $ 
\item On suppose $a \in  \R$. Soit $\epsilon>0$.  \begin{itemize}
    \item $\exists N \in  \N, \forall  n \geq  N, \forall  x \in  A, \quad  |u_n(x)-u(x)|\leq \epsilon'$
    \item $\exists N_1 \in  \N, \forall  n\geq N_1, \quad  |\ell _n-\ell |\leq \epsilon'$
    \item $M=\max(N, N_1). \qquad  \exists \delta>0, \forall  x \in  ]a-\delta, a+\delta[\cap A, |u_M(x)-\ell _M|\leq \epsilon'$.
\end{itemize}
Donc $|u(x)-\ell |\leq |u(x)-u_M(x)|+|u_M(x)-\ell _M| + |\ell _M-\ell |\leq 3\epsilon'=\epsilon$
\end{enumerate}
\end{proof}

\begin{rem}
    $a$ est un point adhérent  si et seulement si $ \exists  (a_n) \in  A^{\N}, a_n \longrightarrow a$
\end{rem}

\begin{cor}
    \Hyp $(u_n)$ une suite de fonctions continues sur  $A$ qui converge uniformément sur  $A$ vers  $u$
    \Conc  $u$ est continue sur  $A$
\end{cor}

\begin{proof}
    $a \in  A$ donc $a$ est un point adhérent de  $A$,  $u_n \xrightarrow[A]{\text{CVU}}u$ et $\forall  n \in  \N$, \[
        \lim_{\substack{x \to  a\\x \in  A}}u_n(x)=u_n(a) \quad (=\ell _n)
    \] 
    et TDL: \[
        \lim_{\substack{x \to  a\\x \in  A}}u(x)=\lim_{n \to  +\infty}u_n(x)=u(a)
    \] 
\end{proof}

\section{Convergence uniforme, dérivation, intégration}

\begin{thm}
    \Hyp $u_n:I\to K$, $u:I\to K$ et $u_n \xrightarrow[I]{\text{CVUTS}}u$, $a \in  I$, $u_n$ et  $u$ sont $\CPM(I, \K)$. On note $U_n:x \in  I \longmapsto \displaystyle\int_a^xu_n$ et $U:x \in  I \longmapsto \displaystyle \int_a^x u$
    \Conc $U_n \xrightarrow[I]{\text{CVUTS}} U$ et en particulier $ \forall \alpha, \beta \in  I, \displaystyle \int_{\alpha}^\beta u_n \xrightarrow[n\to+\infty]{}\int_\alpha^\beta u$
\end{thm}

\begin{proof}
    $[\alpha, \beta] \subset I, J$ plus petit segment qui contient  $a, \alpha, \beta$. On a sur $J$, avec $\mu(J)=\int_J\diff x$ (constant): \[
        \|U_n-U\|_{\infty}\leq \mu(J) \|u_n-u\|_\infty \xrightarrow[n\to+\infty]{}0
    \] 
    donc $U_n \xrightarrow[J]{\text{CVU}}U$ d'où la conclusion sur la CVUTS. Puis, \[
        \int_\alpha^\beta u_n=U_n(\beta)-U_n(\alpha) \xrightarrow[n\to+\infty]{}U(\beta)-U(\alpha)
    \] 
\end{proof}

\begin{ex}
    $u_n:x \in  [0, 1] \longmapsto n^2x^n(1-x)$ converge simplement vers la fonction nulle. Puis \[
        \int_0^1 u_n = n^2 \left( \frac{1}{n+1}-\frac{1}{n+2} \right) \xrightarrow[n\to+\infty]{}1\neq \int_0^10=0
    \] 
    donc il n'y a pas CVU.
\end{ex}

\begin{thm}[Dérivation]
    \Hyp $(u_n)$ suite de fonctions $\mathcal  C^1$ sur  $I$, telles que \[
        u_n \xrightarrow[I]{\text{CVS}}u \qquad  \text{ et }\qquad  u_n' \xrightarrow[I]{\text{CVUTS}}v
    \] 
    \begin{concenum}
    \item  $u_n \xrightarrow[I]{\text{CVUTS}}u$
    \item $v$ est continue et  $u'=v$
    \end{concenum}
\end{thm}

\begin{proof}
    On se donne  $a \in  I$ fixé, on écrit \[
        u_n(x)=u_n(a)+\int_a^xu_n'
    \] 
    et $n\to +\infty$ donne \[
        u(x)=u(a)+\int_a^xv
    \] 
    donc $u$ est  $\mathcal C^1$ et $u'=v$. Puis,  \[
        \int_a^xu_n \xrightarrow[I]{\text{CVUTS}}\int_a^x u'
    \] 
    donc \[
        u_n \xrightarrow[I]{\text{CVUTS}}u
    \] 
\end{proof}

\begin{thm}[Dérivation itérée]
    \Hyp \index{dérivation itérée (suite de fonctions)} $(u_n)$ fonctions  $\mathcal  C^p$, \[
        \forall  k \in  \llbracket 0, p-1 \rrbracket , \qquad  u_n^{(i)}\xrightarrow[I]{\text{CVS}}\varphi_i
    \] 
    et  \[
        u_n^{(p)}\xrightarrow[I]{\text{CVUTS}}\varphi_p
    \] 
    \begin{concenum}
    \item Il y a CVUTS à la place de la CVS pour toutes les dérivées d'ordre  $\leq p$
    \item $\varphi_0$ est  $\mathcal  C^p$ et $\varphi_0^{(k)}=\varphi_k$
    \end{concenum}
\end{thm}

\begin{proof}
    Admis. (Récurrence)
\end{proof}

\begin{thm}[Convergence dominée\index{convergence dominée (suites)}]
    \Hyp $(u_n), u$ des fonctions  $\CPM(I, \K)$, $u_n \xrightarrow[I]{\text{CVS}}u$\\ $ \exists \varphi \in  \mathcal  L^1(I, \R_+), \quad  \forall  x \in  I, \forall  n \in  \N, \quad  |u_n(x)|\leq \varphi(x)$
    \begin{concenum}
    \item $u_n$ et $u$ intégrable sur $I$
    \item  $\displaystyle \lim_{n\to +\infty}\int_Iu_n=\int_Iu$
    \end{concenum}
\end{thm}

\begin{proof}
Admis
\end{proof}

\begin{rem}
On note \[
\begin{matrix}
    f_n:& \R_+ & \longrightarrow &\R  \\
    & x & \longmapsto & \dfrac{x^ne^{-x}}{n!}
\end{matrix}
\] 
Les $f_n$ sont  $\mathcal  C^1$ et \[
    f_n'(x)=\frac{e^{-x}}{n!}\left( -x^n+nx^{n-1} \right) =\frac{x^{n-1}e^x(n-x)}{n!}
\] 
Le calcul donne \[
    \sup_{\R_+}|f_n| \xrightarrow[n\to+\infty]{}0 \qquad  \text{ donc } \qquad  f_n \xrightarrow[\R_+]{\text{CVU}}0
\]
Les $f_n$ sont intégrables (ce sont des $o(\frac{1}{x^2})$) et \[
    \int_0^{+\infty}f_n= \underbrace{\left[ -\frac{x^ne^{-x}}{n!} \right]_0^{+\infty}}_{=0}+\int_0^{+\infty} \frac{x^{n-1}}{(n-1)!}e^{-x}\diff x=\cdots =\int_0^{+\infty}e^{-x}\diff x=1
\]
Ainsi  \[
    \lim_{n\to +\infty} \int_0^{+\infty} \frac{x^ne^{-x}}{n!}\diff x=1 \neq  \int_0^{+\infty} \left( \lim_{n\to +\infty}f_n(x) \right) \diff x=0
\] 
Le théorème ne s'applique pas sans hypothèse de domination
\end{rem}

\begin{ex}
\[
\Gamma:x\longmapsto \int_{{0}}^{{+\infty}} {e^{-t}t^{x-1}}\;\diff t
\] 
Cette fonction est bien définie car l'intégrande est intégrable. On note \[
    f_n:t\longmapsto \1_{[0,n]}(t) \left( t-\frac{t}{n} \right) ^nt^{x-1}
\]
Ce sont des fonctions $\CPM$ sur  $\R_+^\star$. On a \[
    f_n \xrightarrow[\R_+^\star]{\text{CVS}}e^{-t}t^{x-1}
\]
Donc $\forall  n \in  \N^\star,$ \[ \forall  t >0, \qquad  0\leq f_n(t)\leq \1_{[0, n]}\exp\left(n \ln \left(1-\frac{t}{n}\right)\right)\leq e^{-t}t^{x-1} \]
et le membre de droite est $\CPM$ intégrable. Par convergence dominée,  \[
    \int_{0}^{+\infty} f_n \xrightarrow[t\to +\infty]{}\int_0^{+\infty}f=\Gamma(x) 
\] 
Puis (calculer) \[
    \int_0^{+\infty}f_n=\frac{n^xn!}{x(x+1)\cdots (x+n)}
\] 
donc \[
    \frac{n^xn!}{x(x+1)\cdots (x+n)} \xrightarrow[n\to+\infty]{}\Gamma(x)
\] 
\end{ex}

\section{Méthodes pour la convergence uniforme}

On suppose que l'on a une suite $(u_n)_n$ de fonctions de  $I$ dans  $\C$ et $u$ de  $I$ dans  $\C$ telle que \[
    u_n \xrightarrow[I]{\text{CVS}}u
\]
Comment établir $u_n \xrightarrow[I]{\text{CVS}}u$ ?
\begin{itemize}
    \item Étude de $|u_n-u|$ (ou $u_n-u$ pour les fonctions réelles)
    \item On trouve $(a_n)$ tel que  $\forall  x \in  I, \quad  |u_n(x)-u(x)|\leq a_n$ et $a_n\longrightarrow 0$
    \item Autre (il faut réfléchir)
\end{itemize}

Comment montrer qu'il n'y a pas CVU ?
\begin{itemize}
    \item Étudier $\sup_I|u_n-u|$ et vérifier que ça ne tend pas vers  $0$
    \item Trouver  $(x_n)$ tel que  $(u_n-u)(x_n)$ ne tend pas vers  $0$
    \item Trouver un segment  $I$ tel que  $\int_I u_n$ ne tend pas vers  $\int_Iu$
    \item Régularité de la limite simple
    \item Autre (réfléchir)
\end{itemize}

\subsection{Un premier exemple}

\[
    f_n(x)= \begin{cases}
        \dfrac{\sin(n^2x)^2}{n\sin x} &\text{ si } x \in  ]0, \frac{\pi}{2}]\\
        0 &\text{si }x=0
    \end{cases}
\] 
On a \[
    f_n \xrightarrow[]{\text{CVS}}0
\] 
mais \[
    f_n \left( \frac{1}{n} \right) \underset{}\sim \sin(n)^2 \xnrightarrow{}0
\] 
donc il n'y a pas CVU.


\ifsolo
~

\vspace{1cm}

\begin{center}
    \textbf{\LARGE Algèbre générale} \\[1em]
\end{center}
\tableofcontents
\else
\minitoc
\fi
\thispagestyle{empty}

\ifsolo \newpage \setcounter{page}{1} \fi
\section{Groupes}

\textbf{Rappel.} $(G, \star)$ est un groupe\index{groupe} ssi \begin{itemize}
    \item $\star$ est une loi de composition interne associative
    \item $\star$ a un neutre dans $G$
    \item Tous les éléments de $G$ ont un inverse dans $G$ pour $\star$
\end{itemize}

\begin{notation}
    Si $\star$ est commutative, on la notera en général $+$ et l'inverse sera alors noté $-x$. Sinon, on notera souvent la loi $\cdot$ et l'inverse $x^{-1}$. Le neutre de $G$ se note $e_G$ (parfois $0_G$ ou $0$ dans le cas d'une loi notée additivement)
\end{notation}

Si $G$ est de cardinal fini, alors on note $|G|$ ou $\#G$ son cardinal.

\section{Constructeurs de groupes}

\begin{prop}~
    \begin{enumerate}
        \item Si $(G, \cdot)$ est un groupe et $X$ un ensemble non vide alors on muni l'ensemble de fonction de $X$ dans $G$ de la loi $\star$ en posant \[
                \forall f, g: X\longrightarrow G, \quad f\star g:x\longmapsto f(x)\cdot g(x)
            \]
            $(\mathcal F(X, G), \star)$ est alors un groupe.
        \item Soit $(G, \land)$, $(G', \star)$ deux groupes. On munit $G\times G'$ de la loi $\cdot$ en posant \[
                (x, x')\cdot (y, y')=(x\land x', y\star y')
            \]
            $(G\times G', \cdot)$ est alors un groupe.
    \end{enumerate}
\end{prop}

\begin{proof}
    Facile mais pénible.
\end{proof}

\section{Sous-groupes}

\begin{thmdef}
    \Hyp $(G, \cdot)$ un groupe, $H\subset G$
    \Conc On dira que $H$ est un sous-groupe\index{sous-groupe} de $G$ si $H\neq \emptyset$, $\cdot$ est une l.c.i. sur $H$ et $(H, \cdot)$ est un groupe. Dans ce cas, $e_H=e_G$. Il y a équivalence entre: \begin{enumerate}[label=(\alph*)]
        \item $H$ est un sous-groupe de $G$
        \item $H\subset G$, $H\neq \emptyset$ et $\forall x, y\in H, x\cdot y^{-1}\in H$
    \end{enumerate}
\end{thmdef}

\begin{notation}
    $(G, \cdot)$ groupe, $H\subset G$, $a\in G$. \[
        aH=\{a\cdot h, \quad h\in H\}
    \]
    \[
        H^{-1}={h^{-1}, \quad h\in H}
    \]
    et si $A\subset G$,
    \[
        AH = \{a\cdot h, \quad a\in A, h\in H\}
    \]
\end{notation}

Ainsi $H\subseteq G$ est un sous-groupe de $G$ si et seulement si $H\neq \emptyset$ et $HH^{-1}\subseteq H$

\begin{exo}
    Soient $H, K$ des sous-groupes de $G$. Montret que $HK$ est un sous groupe si et seulement si $HK=KH$.
\end{exo}

\begin{proof}[Résolution]
    Si $HK$ sous-groupe de $G$ alors $HK=(HK)^{-1}=K^{-1}H^{-1}=KH$. Si $HK=KH$ alors $HK(HK)^{-1}=HKK^{-1}H=HKKH=HHKK\subseteq HK$
\end{proof}

\subsection{Centre d'un groupe}

\begin{dfn}
    Le centre d'un groupe\index{centre (d'un groupe)}\index{sous-groupe!centre} $G$ est le groupe \[
        Z(G)=\{x\in G, \quad \forall y\in G, xy=yx\}
    \]
\end{dfn}

\begin{rem}
    C'est un sous-groupe de $G$, on peut le vérifier manuellement.
\end{rem}

\subsection{Normalisateur}

On note $H$ un sous-groupe de $G$. On appelle \textbf{normalisateur}\index{normalisateur} de $H$ l'ensemble \[
    N(H)=\{x\in G, \quad xHx^{-1}=H\}
\]
On peut vérifier que c'est un sous-groupe de $G$. On dira que $H$ est un sous-groupe distingué \index{sous-groupe!distingué} de $G$ si $N(H)=G$.

\begin{rem}
    On note $\mathcal R$ la relation \[
        x\mathcal Ry\iff y^{-1}x\in H
    \]
    C'est bien une relation d'équivalence (réfléxive symétrique transitive) et si $a\in H$, $x\in \bar a \iff x\in aH$. De même, $x\mathcal R'y\iff xy^{-1}\in H$ définit une autre relation d'équivalence dont les classes sont les $Ha$. Lorsque $H$ est distingué (on notera $H\triangleleft G$), $aH=Ha$
\end{rem}

\subsection{Sous-groupe de torsion}

Soit $G$ un groupe \emph{abélien}. On note \[
    \tau(G)=\{g\in G, \quad \exists n\in\mathbb N^\star, g^n=e_G\}
\]
C'est un sous-groupe\footnote{on a besoin de l'hypothèse de commutativité de $G$, ça n'est pas vrai sinon} de $G$ appelé sous-groupe de torsion\index{sous-groupe!de torsion}.

\subsection{Sous-groupes de $\mathbb Z$ et $\mathbb R$}

\ifsolo
    ~

    \vspace{1cm}

    \begin{center}
        \textbf{\LARGE Arithmétique} \\[1em]
    \end{center}
    \tableofcontents
\else
    \minitoc
\fi
\thispagestyle{empty}

\ifsolo \newpage \setcounter{page}{1} \fi
\section{Arithmétique classique}

\subsection{Rappels}

On se place dans l'anneau $(\mathbb Z, +, \times)$, et on note $\mathbb P$ l'ensemble des nombres premiers.

\begin{thm}[Division Euclidienne]
    \index{division euclidienne} Pour $a, b\in\mathbb Z$, $b$ non nul, il existe des uniques entiers $q, r$ tels que \[
        a=qb+r\qquad \qquad \text{ et } 0\leq r<|b|
    \]
\end{thm}

\begin{rem}
    L'existence d'un tel $r$ montre que $\mathbb Z$ est euclidien pour le stathme $|\cdot|$
\end{rem}

\begin{thm}[Théorème fondamental de l'arithmétique]
    Tout entier relatif $n$ non nul s'écrit de manière\index{théorème fondamental de l'arithmétique} unique (à l'ordre près des facteurs et à une constante inversible près) comme produit de puissances de premiers: \[
        \forall n\in\mathbb Z^\star, \exists! p_1, \cdots, p_r\in\mathbb P, \exists! \alpha_1, \cdots, \alpha_r\in\mathbb N^\star, \exists \epsilon\in\mathcal U_{\mathbb Z}=\{-1, 1\}, \qquad n=\epsilon p_1^{\alpha_1}\cdots p_r^{\alpha_r}
    \]
\end{thm}

\begin{thmdef}[Bézout, PGCD]
    Pour $a, b\in\mathbb Z$, il existe un unique entier positif $d$ tel que $a\mathbb Z+b\mathbb Z=d\mathbb Z$. On appelle cet entier le PGCD de $a$ et $b$, et on le note $d=a\land b=(a, b)=\mathrm{PGCD}(a, b)$. Si $d=a\land b$ alors il existe $u, v\in \mathbb Z$ tels que $au+bv=d$ (c'est la  propriété de Bézout)\index{Bézout (théorème de -- )}
\end{thmdef}

\begin{rem}
    Cette définition du PGCD coïncide avec la définition usuelle.
\end{rem}

\begin{thmdef}[Identité de Bézout, Coprimalité]
    Deux entiers $a$ et $b$ sont dits premiers entre eux si les ensembles de leurs diviseurs stricts sont disjoints. C'est équivalent à \[
        \exists u, v\in\mathbb Z, \qquad au+bv=1
    \]
\end{thmdef}

\begin{thm}[Lemme de Gauss]
    \index{Gauss!lemme} Si $a, b, c\in \mathbb Z$ et $a\land c=1$ alors \[
        a\;|\;bc\implies a\;|\; b
    \]
\end{thm}

\subsection{Utilisation d'un Vandermonde}

On note $a_1, \cdots, a_n\in\mathbb Z$. On veut montrer que \[
    \prod_{1\leq i<j\leq n}(j-i)\;\Big|\; \prod_{1\leq i<j\leq n}(a_j-a_i)
\]

On introduit \[
    \Delta(X_1, \cdots, X_n)= \begin{vmatrix}
        \binom{X_1}0 & \cdots & \binom{X_1}{n-1} \\
        \vdots & \ddots & \vdots \\
        \binom{X_n}0 & \cdots & \binom{X_n}{n-1}
    \end{vmatrix}\qquad \text{ avec }\binom Xk=\frac{X(X-1)\cdots (X-k+1)}{k!}
\]
On a \[
    \binom X{n-1}=\frac{X^{n-1}}{(n-1)!}+\underbrace{c_{n-2}X^{n-2}+\cdots +c_0}_{\in \Vect \left( \binom X0, \cdots, \binom X{n-2} \right)}
\]
de sorte que dans le déterminant on peut remplacer la dernière colonne par \[
    \begin{pmatrix}
        \frac{X_1^{n-1}}{(n-1)!} \\
        \vdots \\
        \frac{X_n^{n-1}}{(n-1)!}
    \end{pmatrix}
\]
et en itérant sur les colonnes de droite à gauche, \[
    \Delta(X_1, \cdots, X_n)= \begin{vmatrix}
        \;\;\;1\;\; & \dfrac{X_1}1 & \cdots & \dfrac{X_1^{n-1}}{(n-1)!} \\
        \;\vdots & \ddots & \ddots & \vdots \\
        \;\vdots & \ddots & \ddots & \vdots \\
        \;\;\;1\;\; & \cdots & \cdots & \dfrac{X_n^{n-1}}{(n-1)!}
    \end{vmatrix}
\]
et \[
    \Delta(a_1, \cdots, a_n)=\frac1{1!2!\cdots (n-1)!}V(a_1, \cdots, a_n)=\frac1{\prod_{1\leq i<j\leq n}(j-i)}\prod_{1\leq i<j\leq n}(a_j-a_i)
\]
or $\Delta(a_1, \cdots, a_n)$ est bien un entier car les $\binom{a_i}k$ sont des entiers, ce qui conclut.

\subsection{Méthode de la descente infinie}

On note $p$ premier et pour $n\geq 3$, on considère l'équation $x^n+py^n=p^2z^n$.
\begin{itemize}
    \item $(0, 0, 0)$ est solution. On suppose qu'il y en a d'autres et on note $(x, y, z)\neq (0, 0, 0)$ une solution minimale pour $|x|+|y|+|z|$.
    \item On a $p\;|\;p^2z^n-py^n=x^n$ donc $p$ divise $x$: on écrit $x=px'$. On a \[
            p^{n-1}x'^n+y^n=pz^n
        \]
        donc $p$ divise $y$ qu'on écrit $y=py'$ et donc $p$ divise $z$ qui s'écrit $z=pz'$. On a ainsi trouvé une nouvelle solution $(x', y', z')$ strictement inférieure à la précédente, ce qui est absurde. Ainsi, la seule solution est la solution nulle.
\end{itemize}

\subsection{Formule de Legendre}

\begin{dfn}
    \index{valuation $p$-adique} Pour tout premier $p$, on définit la \textbf{valuation $\bm p$-adique} $v_p$ par : \[
        v_p: n\in\mathbb Z^\star \longmapsto \max\{k\in\mathbb N, \quad p^k\;|\; n\}
    \]
\end{dfn}

On va montrer la formule de Legendre\index{Legendre!formule}: \[
    v_p(n!)=\sum_{i=1}^{+\infty}\floor{\frac n{p^i}}
\]
Pour cela, on remarque \[
    v_p(n!)=\sum_{m=1}^nv_p(m)=\sum_{m=1}^n\sum_{k=1}^{v_p(m)}1=\sum_{k=1}^{+\infty}\sum_{\substack{k\leq v_p(m)\\ 1\leq m\leq n}}1=\sum_{k=1}^{+\infty}\sum_{d\leq m=p^ku\leq n}1=\sum_{k=1}^{+\infty}\floor{\frac n{p^k}}
\]

\begin{exo}[X]
     Pour $m, n\in\mathbb N$, montrer \[
        \frac{(2m)!(2n)!}{(m+n)!n!m!}\in\mathbb N
     \]
\end{exo}

\subsection{Coefficients du binôme}

\todo{Retrouver ça}

\section{L'anneau $\mathbb Z_n$}

Soit $A$ un anneau commutatif et $I$ un idéal de $A$. On définit la relation d'équivalence suivante: $x\mathcal Ry\iff x-y\in I$. On note $A/I$ l'ensemble des classes d'équivalences: c'est un anneau commutatif avec les lois $\bar{x+y}=\bar x+\bar y$ et $\bar{x\times y}=\bar x\times \bar y$.

Pour $A=\mathbb Z$, $I=n\mathbb Z$, il y a $n$ classes d'équivalence dans $\mathbb Z_n\defeq\mathbb Z/n\mathbb Z$, ce sont les $\bar 0, \cdots , \bar{n-1}$.

\section{Groupe $\mathbb Z_n^\star$ des inversibles de $\mathbb Z_n$}

\begin{prop}
    \begin{itemize}
        \item La classe de $a\in\mathbb Z$ est inversible dans $\mathbb Z_n$ si et seulement si $a\land n=1$
        \item $\#\mathbb Z_n^\star=\varphi(n)$
        \item (Euler\index{Euler!théorème (arithmétique modulaire)}) Pour $\bar a$ inversible, $\bar a^{\varphi(n)}\equiv \bar 1\pmod n$
    \end{itemize}
\end{prop}

\begin{proof}~
    \begin{enumerate}
        \item $\bar a\in\mathbb Z_n^\star\iff \exists \bar u\in\mathbb Z_n, \bar a\times \bar u\equiv 1\pmod n\iff \exists u, v\in\mathbb Z, au=1+nv \iff a\land n=1$
        \item C'est évident vu 1.
        \item $\ord(\bar a)\;|\;\#\mathbb Z_n^\star=\varphi(n)$
    \end{enumerate}
\end{proof}

\begin{cor}[Petit théorème de Fermat]
    \index{Fermat!petit théorème} Pour tout $a\in\mathbb Z, \quad \bar a^p\equiv \bar a\pmod p$ pour $p$ premier
\end{cor}

\section{Caractérisations des corps}

\begin{prop}
    $\mathbb Z_n$ est un corps si et seulement si $n$ est premier
\end{prop}

\begin{proof}
    $\varphi(n)=n-1\iff n\in \mathbb P$
\end{proof}

\subsection{Théorème de Wilson}

\begin{res}[Théorème de Wilson]
    $p\geq 3$ est premier si et seulement si $(p-1)!\equiv -1\pmod p$\index{Wilson (théorème de -- )}
\end{res}

\begin{proof}~
    \begin{itemize}
        \item $(\implies)$ Dans $\mathbb Z_p$, $x^2\equiv 1$ n'a que deux solutions: $1$ et $-1$ (car on est dans un corps donc un anneau intègre). Ce sont donc les deux seuls éléments qui sont leurs propres inverses, les autres sont couplés deux à deux dans l'écriture de $(p-1)!$. On a donc \[
                (p-1)!\equiv 1\times (2\times \cdots \times (p-2))\times (p-1)\equiv 1\times (p-1)\equiv -1 \pmod p
            \]
        \item $(\impliedby)$ On note $n\geq 3$ non premier, il s'écrit donc $n=ab$ avec $a\land b=1$ et $a, b\geq 2$. Si $a\neq b$ alors $ab\;|\;(n-1)!$. Sinon, $n=a^2$ et $a\geq 2$ donc $a^2\;|\;(n-1)!$ car $a$ et $2a$ apparaissent dans l'écriture de la factorielle. Dans les deux cas, $(n-1)!\equiv 0\pmod n$
    \end{itemize}
\end{proof}

\subsection{RSA}

On va voir (rapidement) le principe du chiffrement RSA. On choisit $p, q$ des nombres premiers distincts, et on note $n=pq$. On note $a, b$ tels que $ab\equiv 1\pmod{\varphi(n)}$. Les nombres $n$ et $a$ sont publics, et $b$ (qu'on appelle clé privée) n'est connu que de Bob, qui veut recevoir un message chiffré d'alice

Alice veut transmettre $x\in\mathbb Z_n^\star$. Elle calcule $x^n\equiv y\pmod n$, et $y$ est le message chiffré.

Bob calcule $y^b\equiv x^{ab}\equiv x\times x^{\varphi(n)k}\equiv x\pmod n$

\subsection{$\mathbb Z_{p^\alpha}^\star$ pour $p$ premier impair}

\subsubsection{$\mathbb Z_p^\star$ est cyclique}

On va d'abord montrer les résultats suivants, vrais dans tous les groupes abéliens finis: \begin{itemize}
    \item L'ensemble des ordres est stable par ppcm
    \item L'exposant d'un groupe est l'ordre d'un élément
\end{itemize}
On utilisera ensuite la structure de corps de $\mathbb Z/p\mathbb Z$ pour conclure.

Soit $(G, \cdot)$ un groupe abélien fini. On va montrer que l'ensemble des ordres est stable par ppcm. Soient $a, b$ deux éléments d'ordres respectifs $m, n$. On cherche un élément d'ordre $m\lor n$. On note $d=m\land n$ de telle sorte que $m=dm'$, $n=dn'$ et $m\land n'=1$. De cette manière, on a $m\lor n=mn'$
\[
    \ord \left(a\right)=m\qquad \qquad \ord\left(b^d\right)=n'
\]
et donc $c=ab^d$ est un élément d'ordre $mn'=m\lor n$, ce qui conclut.

On définit l'exposant\index{groupe!exposant} $\lambda$ du groupe $G$ comme l'ordre maximal des éléments de $G$. L'ensemble des ordres étant stable par ppcm, et le ppcm de tous les ordres majorant l'ensemble, le maximum des ordres $\lambda$ est le ppcm des ordres des éléments. Puis, la stabilité garantit qu'il existe un élément d'ordre $\lambda$.

On va maintenant montrer que $\mathbb Z_p^\star$ est cyclique. $\mathbb Z_p$ est un corps (déjà vu). Ainsi, un polynôme de $\mathbb Z_p[X]$ admet au plus autant de racines que son degré. On pose $P=X^\lambda-1$. Ce polynôme admet au moins $p-1$ racines distinctes (les éléments non nuls du corps) par définition de $\lambda$. On a donc $\lambda\geq p-1$. Puis, $\lambda$ est un ordre donc divise $\#\mathbb Z_p^\star=p-1$ (Lagrange). Ainsi, $\lambda\mid p-1$ et $\lambda=p-1$.

L'exposant du groupe vaut le cardinal du groupe, ce groupe est donc cyclique (un élément d'ordre $\lambda$ est automatiquement un générateur du groupe).

\subsubsection{$\mathbb Z_{p^\alpha}^\star$ est cyclique}

On va d'abord montrer par récurrence \[
    \forall n\in\mathbb N, \quad (1+p)^{p^n}\equiv 1+p^{n+1}\pmod {p^{n+2}}.
\]

\begin{itemize}
    \item Le cas $n=0$ est direct.
    \item On suppose le résultat vrai au rang $n$: \[
        (1+p)^{p^n}=1+p^{n+1}+p^{n+2}q.
    \]
    On a alors \begin{align*}
        (1+p)^{p^{n+1}}&\equiv(1+p^{n+1}+p^{n+2}q)^p\\
                       &\equiv 1+\sum_{k=1}^{p}\binom pk p^{(n+1)k}(1+pq)^k\\ 
                       &\equiv 1 + p^{n+1}(1+pq)p \\
                       &\equiv 1+p^{n+2}&\pmod{p^{n+3}}.
    \end{align*}
    Donc l'égalité est vraie pour tout entier naturel $n$.
\end{itemize}
Il est suffisant de trouver un élément de $(\mathbb Z/p^\alpha\mathbb Z)^\star$ d'ordre $\varphi(p^\alpha)=(p-1)p^{\alpha-1}$ (cette égalité s'obtient via la définition de l'indicatrice d'Euler), et vu la stabilité des ordres par ppcm vue précédemment, il est suffisant de trouver un élément d'ordre $p-1$ et un élément d'ordre $p^{\alpha-1}$.

On va d'abord exhiber un élément d'ordre $p^{\alpha-1}$. On a 
\[
(1+p)^{p^{\alpha-2}} 
\equiv 1+p^{\alpha-1}\pmod {p^\alpha}    \]
et \[
    (1+p)^{p^{\alpha-1}}\equiv 1\pmod {p^\alpha}.
\]
Donc $\ord(1+p)\mid p^{\alpha-1}$ mais $\ord(1+p)\nmid p^{\alpha-2}$ donc $\ord(1+p)=p^{\alpha-1}$.

Il reste à trouver un élément d'ordre $p-1$. On sait que $\mathbb Z_p^\star$ est cyclique, on note $g$ un générateur. On a:
\[
    \forall k<p-1, \quad p\;\nmid\;g^k-1\quad \text{ie}\quad p^\alpha\;\nmid\; g^k-1
\]
et $p\mid g^{p-1}-1$ donc $\ord g=k(p-1)$ est $g^k$ et d'ordre $p-1$.

\section{Théorème chinois}

\begin{thm}
    \Hyp $m, n\in\mathbb N^\star$, $m\land n=1$
    \begin{concenum}
    \item La classe de $\bar x\in\mathbb Z_{mn}$ dans $\mathbb Z_n$ ne dépend pas du choix de représentant dans $\mathbb Z_{mn}$
    \item L'application \[
            \begin{matrix}
                \varphi: & \mathbb Z_{m, n} & \longrightarrow & \mathbb Z_n\times \mathbb Z_m \\
                         & \bar x &\longmapsto & (\bar x, \bar x)
            \end{matrix}
        \]
        est un isomorphisme d'anneaux
    \item $\varphi$ est une bijection de $\mathbb Z_{m, n}^\star$ dans $\mathbb Z_n^\star \times \mathbb Z_m^\star$
    \item Pour $a, b\in\mathbb Z$ fixés, le système \[
            \begin{cases}
                x=a\pmod n\\
                x=b\pmod m
            \end{cases}
        \]
        admet une unique solution modulo $mn$
    \end{concenum}
\end{thm}

\begin{proof}~
    \begin{enumerate}
        \item $a, b$ de représentants de $\bar x$ dans $\mathbb Z_{m, n}$. $a\equiv b\pmod mn\implies a\equiv b\pmod n$ donc $\bar a=\bar b$ dans $\mathbb Z_n$
        \item C'est bien un morphisme, il est injectif car $\Ker\varphi=\{\bar 0\}$ donc c'est un isomorphisme car les ensembles d'antécédents et d'images sont équipotents finis.
        \item C'est clair ($a\land n\neq 1\implies a\land mn\neq 1$ d'où on déduit l'injectivité).
        \item C'est 2.
    \end{enumerate}
\end{proof}

\begin{rem}
    Le point 3. donne la multiplicativité de l'indicatrice d'euler: pour $m, n$ premiers entre eux, $\varphi(mn)=\varphi(m)\varphi(n)$
\end{rem}

\begin{rem}
    On peut généraliser le résultat à $n_1, \cdots, n_r$ deux à deux premiers entre eux
\end{rem}

\begin{rem}
    Si $un+vm=1$ alors $x=avm+bun$ est une solution
\end{rem}

\subsection{Puissances parfaites}

On va montrer qu'il existe $n$ entiers consécutifs qui ne sont pas des puissances parfaites.

On note $p_1, \cdots, p_n$ des premiers deux à deux distincts. Le système \[
    \begin{cases}
        x\equiv p_1-1\pmod {p_1^2}\\
        \hspace{1cm}\vdots\\
        x\equiv p_n-n\pmod{p_n^2}
    \end{cases}
\]
possède une solution $x\in\mathbb N$ et $v_{p_i}(x+i)=1$ donc $x+i$ n'est pas une puissance parfaite (car $x+i\neq p_i$ quitte à translater $x$ de $p_1^2\cdots p_n^2$)

\subsection{Points visibles}

\begin{center}
    \includegraphics{src/figures/arithmetique-points-visibles.pdf}
\end{center}

On considère l'ensemble des points (qu'on appelle \emph{points visibles}) de $\mathbb Z^2$ tels que pour un point $(x, y)$, l'ensemble $\{(1-t)(0, 0)+t(x, y), \quad t\in ]0, 1[\}$ ne contient pas de point de $\mathbb Z^2$. On veut montrer qu'il existe des rectangles pleins arbitrairement grands dans cet ensemble, c'est à dire que pour tout $n\in\mathbb N$, il existe $(a, b)$ tel que $\forall i, j\leq n, (a+i, a+j)$ est invisible.

On vérifie facilement que cet ensemble est l'ensemble $\{(a, b)\in\mathbb Z^2, \quad a\land b=1\}$. On note $A$ une matrice $k\times k$ dont les coefficients sont des nombres premiers deux à deux distincts. On note $m_i$ le produit de la $i$-ième ligne, $M_i$ le produit de la $i$-ième colonne. Le théorème chinois donne l'existence de $a$, $b$ tels que \[
    \begin{cases}
        a\equiv -1\pmod {m_1} \\
        \hspace{.4cm}\vdots \\
        a\equiv -k\pmod{m_k}
    \end{cases}
    \qquad \begin{cases}
        b\equiv -1\pmod {M_1}\\
        \hspace{.4cm}\vdots \\
        b\equiv -k\pmod{M_k}
    \end{cases}
\]
de sorte que $(a+i)\land (b+j)$ est divisible par $m_i\land M_j\neq 1$.

\subsection{Les congruences de Lucas}

On note $p\in\mathbb P$ et \[
    n\defeq\sum_{k=0}^Na_kp^k\geq \sum_{k=0}^Nb_kp^k\defeq m
\]
On va montrer que \[
    \binom nm\equiv \binom{a_0}{b_0}\cdots \binom{a_N}{b_N}\pmod p
\]

On a $(1+X)^p=1+X^p$ dans $\mathbb Z_p[X]$. Puis, \[
    (1+X)^n=\prod_{i=0}^N((1+X)^p)^{a_i}=\prod_{i=0}^n\sum_{k=0}^{a_i}\binom{a_i}{k}X^{kp^i}
\]
et il suffit d'identifier le coefficient de $X^m$

\section{Principe d'inclusion-exclusion}

Pour $E$ fini, $A_1, \cdots, A_N\in\mathcal P(E)$, \[
    \#A_1\cup\cdots \cup A_N=\sum_{i_1=1}^N\#A_{i_1}-\sum_{i_1<i_2}\#A_{i_1}\cap A_{i_2}+\cdots +(-1)^{N-1}\sum_{i_1<\cdots <i_N}\#A_1\cap \cdots \cap A_N
\]

Pour le voir, on remarque: \[
    \sum_{x\in E}\1_{A}(x)=\#A \qquad\quad \1_{A\cap B}=\1_A \cdot \1_B \qquad \quad \1_{A^c}=1-\1_A
\]
et \[
    (A_1\cup \cdots \cup A_N)^c=A_1^c\cap \cdots \cap A_N^c
\]
donc \[
    \mathbbm1_{(A_1\cup \cdots \cup A_N)^c}=\prod_{i=1}^N(1-\1_{A_i^c})
\]
puis on développe le produit et on somme la fonction prise sur les éléments de $E$


\begin{ex}[Application aux permutations sans points fixes]
    On note $d_n$ le nombre de permutations de $\mathfrak S_n$ sans points fixes.

    Il y a $\displaystyle \binom ni d_{n-i}$ permutations qui ont $i$ points fixes donc \[
        n!=\sum_{i=0}^n\binom nid_{n-i}=\sum_{i=0}^n\binom nid_i.
    \]
    Matriciellement, \[
        \begin{pmatrix}
            n!\\[1em]
            \vdots\\[1em]
            \vdots\\[1em]
            0!
            \end{pmatrix}= \underbrace{\begin{pmatrix}
        \displaystyle\binom nn & \cdots & \cdots & \displaystyle\binom n0 \\[1em]
        0 & \displaystyle\binom{n-1}{n-1} & \ddots & \vdots \\[1em]
        \vdots & \ddots & \ddots & \vdots \\[1em]
        0 & \cdots & 0 & \displaystyle\binom 00
    \end{pmatrix}}_{\defeq A_n\in \mathrm{GL}_n(\mathbb R)} \begin{pmatrix}
        d_n\\[1em]
        \vdots \\[1em]
        d_1\\[1em]
        d_0
        \end{pmatrix}
    \]
    On a $A_n^\top=\mathcal {M}_{\text{base canonique}}(f)$ avec $f:P\in\mathbb R_n[X]\longmapsto P(X+1)$. Puis, $(A_n^\top)^{-1}=(A_n^{-1})^\top=\mathcal M(f^{-1})$ or $f^{-1}:P\longmapsto P(X-1)$ d'où \[
        (A_n^{-1})^\top = \begin{pmatrix}
            \displaystyle\binom nn & 0 & \cdots & 0 \\[1em]
            \displaystyle -\binom n{n-1} & \ddots & \ddots & \vdots \\[1em]
            \vdots &  & \ddots & 0 \\[1em]
            (-1)^n\displaystyle\binom n0 & \;\;\;\cdots\;\;\; & \;\;\;\cdots\;\;\; & \displaystyle\binom 00
        \end{pmatrix}
    \]
    d'où \[
        d_n=\sum_{i=0}^n\binom n{n-i}(n-i)!(-1)^i=\sum_{i=0}^n\frac{(-1)^in!}{i!}\sim \frac{n!}e
    \]
\end{ex}

\section{Principe des tiroirs}

\begin{prop}[Principe des tiroirs]
    Si on place $n$ éléments dans $k$ tiroirs, alors l'un des tiroirs contient au moins $\ceil{\frac nk}$ éléments
\end{prop}

\begin{proof}
    Par l'absurde.
\end{proof}

\subsection{Problème d'Erdős}

On choisit $n+1$ éléments distincts de $\llbracket 1, 2n\rrbracket$. On va montrer que l'un divise un autre.

Tous les entiers s'écrivent $2^n(2s+1)$, $2s+1\in\llbracket 1, n\rrbracket$. On en a choisit $n+1$ (parmi $n$ possibilités) donc deux entiers ont la même partie impaire ce qui conclut.

\subsection{Existence d'univers parallèles (moyennant quelques hypothèses physiques)}

On définit un volume de Hubble: une boule de $4\cdot 10^{26}$m de rayon qui contient $10^{115}$ cases contenant ou non un proton.

L'univers est en expansion, donc la lumière se fait rare et il fait noir. Mais, il ne fait pas noir. Donc il y a des volumes de Hubble qui apparaissent.

Il n'y a qu'un nombre fini de volumes de Hubble possibles, ainsi, il y a deux volumes de Hubble identiques dans l'univers.

\subsection{Tiroirs de Dirichlet}

On se donne $\alpha\in\mathbb R\setminus \mathbb Q$. On va montrer qu'il existe une infinité de $(h, k)$ tels que \[
    k>0,\qquad \quad \left| \alpha-\frac hk \right|\leq \frac1{k^2}
\]
On note $\{p\alpha\}=p\alpha-\floor{p\alpha}$ la partie décimale de $p\alpha$ pour $p\in\llbracket 0, n\rrbracket$. Il y a $n+1$ valeurs possibles, donc il existe $k\in\llbracket 0, n-1\rrbracket$ tel que \[
    \{p\alpha\}, \{q\alpha\}\in \left[ \frac kn;\frac {k+1}n \right[,\qquad \text{ et }\qquad p<q
\]
Puis, \[
    |\{p\alpha\}-\{q\alpha\}|=|(q-p)\alpha-(\floor{q\alpha}-\floor{p\alpha})|<\frac1n
\]
donc \[
    \Big|\alpha-\underbrace{\frac{\floor{q\alpha}-\floor{p\alpha}}{q-p}}_{=\frac hk\quad \text{convient}}\Big|<\frac1{n(q-p)}\leq \frac1{(q-p)^2}
\]

S'il y a un nombre fini de $(h, k)$ solutions alors l'un d'eux réalise le minimum ($>0$), donc il existe $n$ tel que \[
    \frac1n<\Big|\alpha-\frac hk\Big|
\]
et vu ce qui précède il existe $p\neq q$ tel que \[
    \left| \alpha-\frac{\floor{q\alpha}-\floor{p\alpha}}{q-p} \right|<\frac1{n|q-p|}\leq \left| \alpha-\frac hk \right|
\]
absurde

\section{Dénombrement de partitions}

Soient $n, p\in\mathbb N^\star$. On appelle partition de $n$ en $p$ parties un $p$-uplet $(x_1, \cdots, x_p)\in\llbracket 0, n\rrbracket ^p$ tel que $x_1+\cdots +x_p=n$

Pour compter les partitions de $n$ en $p$-uplets, on compte le nombre de manière de positionner $p-1$ signes $+$ parmi $n+p-1$ emplacements (le nombre d'emplacements entre le $i$-ième signe $+$ et le $i+1$-ième sera $x_{i+1}$). Le nombre de $p$-partitions de $n$ est donc \[
    \binom{n+p-1}{p-1}
\]

Si on impose que les $x_i$ soient non nuls, alors il n'y a plus que $\displaystyle\binom{n-1}{p-1}$ possibilités.

\section{Une congruence subtile}

On se donne $a, b\in\mathbb N$, $a\geq b$. On va montrer que pour $p$ premier \[
    \binom{pa}{pb}\equiv \binom ab\pmod {p^2}
\]
On considère l'ensemble $E$ des matrices de $\mathcal M_{a,p}(\mathbb Z_2)$ dont tous les coefficients sont nuls sauf $pb$ d'entre eux. On munit $E$ de la relation d'équivalence \[
    A\mathcal RB\iff \exists c_1, \cdots, c_a \text{ permutations circulaires de }\llbracket 1, p\rrbracket, \quad (a_{i, c_i(j)})_{\substack{1\leq i\leq a\\ 1\leq j\leq p}}=A
\]
Par exemple, \[
    \begin{pmatrix}
        {\color{gray}0}&{\color{gray}0}&{\color{gray}0}\\1&{\color{gray}0}&{\color{gray}0}\\{\color{gray}0}&{\color{gray}0}&{\color{gray}0}\\1&{\color{gray}0}&1
    \end{pmatrix}\mathcal R
    \begin{pmatrix}
        {\color{gray}0}&{\color{gray}0}&{\color{gray}0}\\1&{\color{gray}0}&{\color{gray}0}\\{\color{gray}0}&{\color{gray}0}&{\color{gray}0}\\{\color{gray}0}&1&1
    \end{pmatrix}\mathcal R
    \begin{pmatrix}
        {\color{gray}0}&{\color{gray}0}&{\color{gray}0}\\{\color{gray}0}&1&{\color{gray}0}\\{\color{gray}0}&{\color{gray}0}&{\color{gray}0}\\1&1&{\color{gray}0}
    \end{pmatrix}\mathcal R
    \begin{pmatrix}
        {\color{gray}0}&{\color{gray}0}&{\color{gray}0}\\{\color{gray}0}&1&{\color{gray}0}\\{\color{gray}0}&{\color{gray}0}&{\color{gray}0}\\{\color{gray}0}&1&1
    \end{pmatrix}
\]
Les classes d'équivalences sont de cardinal $1$ si et seulement si les lignes ont chacune un seul des coefficients $1$ ou $0$ ($p$ est premier). Il y a $b$ parmi $a$ telles classes. Puis, si une ligne contient deux coefficients distincts alors il y a au moins une autre telle ligne (car on a choisi $pb$ coefficients $1$). Ainsi, si une classe d'équivalence n'est pas un singleton, son cardinal est divisible au moins par $p^2$. On a donc \[
    \binom{pa}{pb}-\binom ab=\sum_{\#\bar x\geq 2}\#\bar x=p^2k
\]
d'où le résultat.

\section{Méthodes combinatoires}

On note $E$ un ensemble fini et $A_1, \cdots, A_k\in\mathcal P(E)$. On va compter les $k$-uplets d'ensembles qui vérifient certaines propriétés

\subsection{Suite croissante pour l'inclusion}

On va démonbrer le nombre de $A_1, \cdots, A_k$ tels que $A_1\subset A_2\subset \cdots\subset A_n$
\[
    \colorboxed{black}{\text{Poubelle}\defeq A_0} \qquad\qquad \colorboxed{black}{A_1}\quad\subset\quad \colorboxed{black}{A_2}\quad\subset\quad \cdots \quad\subset\quad \colorboxed{black}{A_k}
\]
On associe à chaque $x\in E$ le plus petit $i\in\llbracket 1, k\rrbracket$ tel que $x\in A_i$ ($0$ sinon), on note $f:C\to \mathfrak F(E, \llbracket 0, k\rrbracket)$ cette fonction, avec $C$ l'ensemble des $k$-uplets qui satisfont la condition.

Cette fonction est clairement bijective donc $\#C=(k+1)^{\#E}$

\subsection{Union disjointe}

On veux dénombrer le nombre de $k$-uplets tels que $A_i\cap A_j=\emptyset$ pour $i\neq j$
C'est le même principe: on note \[
    f: (A_1, \cdots, A_k)\in C\longmapsto (x\longmapsto \text{ unique }i\text{ tel que }x\in A_i)
\]
avec $A_0$ l'ensemble des éléments qui n'apparaissent dans aucun des $A_i$. Cette application est aussi clairement bijective ce qui permet de conclure.
\endchapter

\ifsolo
    ~

    \vspace{1cm}

    \begin{center}
        \textbf{\LARGE Réduction des endomorphismes} \\[1em]
    \end{center}
    \tableofcontents
\else
    \minitoc
\fi
\thispagestyle{empty}

\ifsolo \newpage \setcounter{page}{1} \fi

Dans tout le chapitre, on note $\mathbb K$ un corps.

\section{Rappels sur les polynômes}

\begin{thm}
    \Hyp $A, B\in \mathbb K[X]$, $B\neq 0$
    \Conc Il existe un unique couple de polynômes $(Q, R)$ dans $\mathbb K[X]$ tels que \[
        A=QB+R
    \]
    et $\deg R<\deg B$
\end{thm}

\begin{thmdef}
    $a\in\mathbb K$ est une racine de $P$ ssi \[
    P(a)=0\iff (X-a)\;|\;P
\]
et $a$ est racine de multiplicité $\alpha \geq 1$ ssi \[
    \begin{cases}
        P(a)=P'(a)=\cdots=P^{(\alpha - 1)}(a)=0 \\
        P^{(\alpha)}(a)\neq 0
    \end{cases}
    \iff \begin{cases}
        (X-a)^\alpha \; |\; P\\
        (X-a)^{\alpha+1}\;\not|\;P
    \end{cases}
\]
Si $a$ n'est pas racine de $P$, on convient que $a$ est racine de multiplicité $0$.
\end{thmdef}

\begin{thm}[Interpolation de Lagrange]
    \index{Lagrange!interpolation} Pour $a_1, \cdots, a_n\in\mathbb R$ des abscisses distinctes et $b_1, \cdots, b_n\in\mathbb R$ des ordonnées distinctes, il existe un unique polynôme de degré au plus $n-1$ tel que \[
        \forall i\in\llbracket 1, n\rrbracket, \qquad P(a_i)=b_i
    \]
\end{thm}
\begin{proof}
    \[
        \varphi:P\in\mathbb K_{n-1}[X]\longmapsto (P(a_1), \cdots, P(a_n))\in\mathbb R^n
    \]
    est un isomorphisme.
\end{proof}

\section{Idéaux de {$\mathbb K[X]$} et arithmétique}

\begin{thm}
    $\mathbb K[X]$ est principal
\end{thm}
\begin{proof}
    C'est un anneau euclidien donc principal (cf. chapitre \textbf{Algèbre générale})
\end{proof}

\begin{thmdef}
    \begin{enumerate}
        \item Il existe un unique polynôme unitaire $Q$ tel que $P\mathbb K[X]+Q\mathbb K[X]=Q\mathbb K[X]$. C'est le PGCD de $A$ et $B$. Deux polynômes sont premiers entre eux si leur PGCD vaut $1$.
        \item Il existe un unique polynôme unitaire $Q$ tel que $P\mathbb K[X]\cap Q\mathbb K[X]=Q\mathbb K[X]$. C'est le PPCM de $A$ et $B$.
        \item (Gauss) \index{Gauss!lemme (polynômial)} Si $A$ et $B$ sont premiers entre eux et $A\;|\;BC$ alors $A\;|\;C$
        \item (Bézout) \index{Bézout (théorème de -- )} $A$ et $B$ sont premiers entre eux si et seulement si il existe $U, V$ tels que $AU+BV=1$.
    \end{enumerate}
\end{thmdef}

\begin{ex}
    $(X^n-1)\land (X^m-1)=X^{n\land m}-1$
\end{ex}

\section{Décomposition en facteurs irréductibles}

\begin{dfn}
    Un anneau $A$ est factoriel si \begin{enumerate}
        \item Pour tout $a\in A$, il existe des éléments irréductibles $p_1, \cdots, p_n$ tels que $a=p_1\cdots p_n$
        \item Cette décomposition est unique à l'ordre des facteurs près et à association près (deux éléments $a$ et $b$ sont associés si il existe un inversible $u$ tel que $a=ub$)
    \end{enumerate}
\end{dfn}

\begin{thm}
    $\mathbb K[X]$ est un anneau factoriel
\end{thm}

\begin{ex}
    Dans $\mathbb C[X]$, \begin{itemize}
        \item \[
                X^n-1=\prod_{\omega\in\mathbb U_n}(X-\omega)
            \]
        \item \[
                X^{2m}-1=(X+1)(X-1)\prod_{k=1}^{n-1} \left( X-\exp \left( \frac{ik\pi}n \right) \right) \left( X-\exp \left( -\frac{ik\pi}n \right) \right)
            \]
    \end{itemize}
\end{ex}

\begin{exo} Montrer que
    \begin{itemize}
        \item $P(\mathbb R)\subset \mathbb R_+ \iff \exists A, B\in\mathbb R[X], P=A^2+B^2$
        \item $P(\mathbb R_+)\subset \mathbb R_+\iff \exists A, B\in\mathbb R[X], P=A^2+XB^2$
    \end{itemize}
\end{exo}

\begin{proof}[Résolution]
    Éléments de résolution pour le premier point: l'ensemble des $A^2+B^2$ est stable par produit, et en décomposant $P\geq 0$ en facteurs irréductibles, chacun des facteurs est somme de deux carrés.
\end{proof}

\begin{exo}
    $P$ unitaire de degré $n$ dans $\mathbb R[X]$. Montrer \[
        P\text{ scindé dans }\mathbb R[X] \iff \forall z\in\mathbb C, \quad |P(z)|\geq |\mathfrak{Im}(z)|^n
    \]
\end{exo}

\begin{proof}~
    \begin{itemize}
        \item $(\impliedby)$ $P$ scindé dans $\mathbb C$ de racines $(a_i)$. $0=|P(a_i)|\geq |\mathfrak{Im}(z)|^n$ donc $a_i\in\mathbb R$
        \item $(\implies)$ \[
                \left| P(x+iy) \right|= \left| \prod_{j=1}^n(x-a_j+iy) \right|=\prod_{j=1}^n|x-a_j+iy|\geq \|\mathfrak{Im}(z)|^n
            \]
    \end{itemize}
\end{proof}


\section{Polynômes d'endomorphismes}

\begin{defprop}
    \Hyp On note $E$ un $\mathbb K$-e.v., $u\in\mathcal L(E)$ et \[
        P=\sum_{k=0}^{+\infty}a_kX^k\in\mathbb K[X]
    \]
    \begin{concenum}
    \item ~ \[
            P(u)\defeq\sum_{k=0}^{+\infty} a_ku^k
        \]
    \item $\varphi: P\in\mathbb K[X]\longmapsto P(u)\in\mathcal L(E)$ est un morphisme d'algèbres.
    \end{concenum}
\end{defprop}

\begin{proof}
    On justifie $(PQ)(u)=P(u)\circ Q(u)$. On considère \[
        \psi: (P, Q)\in\mathbb K[X]^2\longmapsto (PQ)(u)-P(u)\circ Q(u)\in\mathbb L(u)
    \]
    \begin{itemize}
        \item $\psi$ est bilinéaire
        \item À $Q$ fixé, $P\longmapsto \psi(P, Q)$ AL nulle ssi $\forall j\in\mathbb N, \psi(X^j, Q)=0$
        \item Pour chaque $j$, $Q\longmapsto \psi(X^j, Q)$ nulle ssi $\forall k\in\mathbb N, \psi(X^j, X^k)=0$ qui est vrai.
    \end{itemize}
\end{proof}

\begin{dfn}
    Un polynôme $P$ est dit \textbf{polynôme annulateur\index{polynôme!annulateur}} de $u\in\mathcal L(E)$ si $P(u)$ est l'application nulle
\end{dfn}

\begin{ex}
    $p$ est un projecteur si et seulement si $X^2-X$ est annulateur. $s$ est une symétrie si et seulement si $X^2-1$ est annulateur
\end{ex}

\begin{defprop}
    L'ensemble des polynômes annulateurs d'un endomorphisme $u$ est un idéal. Il possède un unique générateur unitaire, qu'on appelle \textbf{polynôme minimal\index{polynôme!minimal}}, qu'on notera $\mu_u$ ou $\Pi_u$
\end{defprop}

\begin{proof}
    $(\id, u, \cdots, u^{n^2})$ famille liée de $\mathcal L(E)$ donc l'idéal n'est pas trivial
\end{proof}

\begin{prop}
    $\mathbb K[u]\defeq \{P(u), P\in\mathbb K[X]\}$ est une $\mathbb K$-algèbre de dimension $\deg \mu_u$ de base $(1, u, \cdots, u^{\deg \mu_u-1})$
\end{prop}

\begin{ex}
    $P=X^2-3X+2$ polynôme minimal \[
        A= \begin{pmatrix}
            -2 & -6 \\ 2 & 5
        \end{pmatrix}
    \]
    Puis $X^n=\mu_A\times Q+\alpha_n X+\beta_n$. En injectant $1$ et $2$ (les racines de $\mu_A$) on peut exprimer $\alpha_n$ et $\beta_n$ et $A^n=\alpha_nA+\beta_n$
\end{ex}

\section{Théorème de décomposition des noyaux}

\begin{thm}
    \Hyp $E$ un $\mathbb K$-e.v., $u\in\mathcal L(E)$
    \begin{concenum}
    \item Si $P, Q\in\mathbb K[X]$ sont premiers entre eux, alors $\Ker (PQ)(u)=\Ker P(u)\oplus \Ker Q(u)$. En particulier, si $PQ$ est annulateur alors $\Ker(PQ)(u)=E$
    \item Si $P_1, \cdots, P_q\in\mathbb K[X]$ deux à deux premiers entre eux et si $P=P_1\cdots P_q$ alors \[
            \Ker P(u)=\Ker P_1(u)\oplus\cdots\oplus \Ker P_q(u)
        \]
    \end{concenum}
\end{thm}

% \input{Analyse réelle et fonctions convexes}
% \input{Espaces vectoriels normés}
% \input{Limite -- Continuité -- Fonctions vectorielles}
% \input{Compacité -- Connexité}
\ifsolo
    ~

    \vspace{1cm}

    \begin{center}
        \textbf{\LARGE Séries entières} \\[1em]
    \end{center}
    \tableofcontents
\else
    \chapter{Séries entières}

    \minitoc
\fi
\thispagestyle{empty}

\section{Définitions}

\begin{dfn}
    Soit $(a_n)_n$ une suite complexe. La série de fonctions de terme général $(a_nz^n)_n$ est appelée série entière\index{série entière} et se note $\sum a_nz^n$. Lorsqu'elle converge, on note sa somme \[
        \sum_{n=0}^{+\infty}a_nz^n
    \]
\end{dfn}

\begin{ex}
    Le domaine de convergence de $\sum z^n$ est $\mathcal D_0(0, 1)$.

    Le somaine de convergence $\mathcal D$ de $\sum \frac{z^n}n$ vérifie $\mathcal D_o(0, 1)\subsetneq\mathcal D\subsetneq \mathcal D_f(0, 1)$
\end{ex}

\section{Rayon de convergence}
\begin{thmdef}
    \Hyp $(a_n)_{n\in\mathbb N}\in\mathbb C^{\mathbb N}$
    \begin{concenum}
    \item Les cinq nombres suivants sont égaux {\begin{align*}
                R_1 &= \sup\{|z|, \quad z\in\mathbb C, (a_nz^n)_n\text{ bornée }\} \\
                R_2 &= \sup\{|z|, \quad z\in\mathbb C, a_nz^n\longrightarrow 0\}\\
                R_3 &= \sup\{|z|, \quad z\in\mathbb C, \sum a_nz^n\text{ converge }\} \\
                R_4 &= \sup\{|z|, \quad z\in\mathbb C, \sum a_nz^n\text{ converge absolument }\} \\
                R_5 &= \inf\{|z|, \quad z\in\mathbb C, (a_nz^n)_n\text{ non bornée}\}
    \end{align*}
avec la convention $\inf\emptyset=+\infty$.}
    On appelle ce nomber \emph{rayon de convergence\index{rayon de convergence}} de $\sum a_nz^n$ et on le note $R(a_nz^n)$
\item Si $\sum a_nz^n$ admet pour rayon de convergence $R$, alors le domaine de convergence est tel que \[
        \mathcal D_0(0, R)\subseteq \mathcal D\subseteq \mathcal D_f(0, R)
    \]
    \end{concenum}
\end{thmdef}

\begin{proof}
    \begin{enumerate}
        \item $R_1\geq R_2\geq R_3\geq R_4$. Par l'absurde, on suppose $R_4<R_5$. Alors, on prend $z\in\mathbb C$ tel que $R_5>|z|>R_4$. $(a_nz^n)_n$ est bornée donc si on note $z'$ tel que $|z|>|z'|>R_4$, alors \[
                a_nz'^n=\underbrace{a_nz^n}_{\text{borné}} \underbrace{\left( \frac{z'}{z}  \right)^n}_{|\;|<1} \text{ tg série ACV absurde car }|z'|>R_4
            \]
            Donc $R_4\geq R_5$. Par l'absurde, $R_1>R_5$. On se donne $z\in\mathbb C$ tel que $R_1>|z|>R_5$ et il existe $z', z''\in\mathbb C$ tel que $R_1\geq |z''|>|z|>|z'|\geq R_5$, $(a_nz'^n)$ non bornée et $(a_nz''^n)$ bornée. \[
                \underbrace{a_nz'^n}_{\lnot \text{ borné}}=a_nz''^n \left( \frac{z'}{z''} \right)^n\xrightarrow{}_{n\to+\infty}0
            \]
            absurde donc $R_1=R_2=R_3=R_4=R_5$.
        \item Immédiat vu $1$.
    \end{enumerate}
\end{proof}
% 
\begin{rem}[Lemme d'Abel]
    Si $a\in\mathbb C^{\mathbb N}$ et $z_0\in\mathbb C^\star$. Si la suite $(a_nz_0^n)$ est bornée alors $\forall z\in\mathcal D_o(0, |z_0|), \sum a_nz^n$ ACV
\end{rem}

\section{Opération algébrique sur les séries entières}

\begin{prop}
    \Hyp $\sum a_nz^n$ de rayon $R_a$ et $\sum b_nz^n$ de rayon $R_b$
    \begin{concenum}
    \item Pour $\lambda\in\mathbb C^\star$, $R(\sum \lambda a_nz^n)=R_a$
    \item Pour $\lambda\in\mathbb C^\star$, $R(\sum(\lambda a_n+b_n)z^n)\geq \min(R_a, R_b)$ et si $R_a\neq R_b$, il y a égalité.
    \item Le produit de Cauchy des séries $\sum a_nz^n$ et $\sum b_nz^n$ est la série $\sum c_nz^n$ avec \[
            c_n=\sum_{p+q=n}a_pb_q
        \]
        et $R(\sum c_nz^n)\geq \min(R_a, R_b)$.
    \end{concenum}
\end{prop}

\begin{proof}
    \begin{enumerate}
        \item "Trivial"
        \item $\geq 0$ OK et $R_a\neq R_b\implies =$ OK
        \item Si $|z|<\min(R_a, R_b)$ alors $\sum a_nz^n, \sum b_nz^n$ ACV donc produit de Cauchy ie \conc
    \end{enumerate}
\end{proof}

\section{Régularité des séries entières}

\begin{prop}
    \Hyp $\sum a_nz^n$ de RCV $R>0$ (ou $+\infty$)
    \begin{concenum}
    \item $\forall a\in ]0, R[, \quad \sum a_nz^n$ CVN sur $\mathcal D_f(0, a)$.
    \item $z\in\mathcal D_o(0, R)\longmapsto\displaystyle \sum_{k=0}^{+\infty}a_nz^n$ est continue
    \end{concenum}
\end{prop}

\begin{proof}
    $a\in]0, R[$. \[
        \sum_{z\in\mathcal D_f(0, a)}|a_nz^n|=|a_na^n|\text{ tg série CV car } a<R \text{ donc il y a CVN sur }\mathcal D_f(0, a)
    \]
    d'où la continuité sur tous les $\mathcal D_f(0, a)$ donc sur $\mathcal D_o(0, a)$.
\end{proof}

\begin{thm}
    \Hyp $f(z)=\sum_{n\geq 0}a_nz^n$ série entière de rayon $R>0$
    \begin{concenum}
    \item On appelle série dérivée la série $\sum_{n\geq 1}na_nz^{n-1}$
        \begin{enumerate}
            \item Cette série a le même rayon de convergence que $f$
            \item $f$ est dérivable sur $]-R, R[$ et $\forall t\in ]-R, R[,$ \[
                    f'(t)=\sum_{n\geq 1}a_nnt^{n-1}
                \]
        \end{enumerate}
        \item La série $\sum_{n\geq 0}\frac{a_n}{n+1}z^{n+1}$ a même rayon de convergence que $f$ et pour $t\in ]-R, R[$, \[
            \sum_{n\geq 0}\frac{a_n}{n+1}z^{n+1}=\int_0^tf(u)\diff u
            \]
    \end{concenum}
\end{thm}

\begin{proof}
    \begin{enumerate}
        \item $|na_nz^{n-1}|=|a_na^n|\times \left| (\frac za)^{n-1} \right|\xrightarrow{}0$ si $|z|<R$.
    \end{enumerate}
\end{proof}

% \input{Variables aléatoires discrètes}
% \input{Espérance -- Variance}
% \input{Les intégrales à paramètres}
% \input{Les équations différentielles}
% \input{Espaces préhilbertiens réels}
% \input{Fonctions de plusieurs variables}
\ifsolo
~

\vspace{1cm}

\begin{center}
    \textbf{\LARGE Calcul différentiel} \\[1em]
\end{center}
\tableofcontents
\else
\minitoc
\fi
\thispagestyle{empty}

\ifsolo \newpage \setcounter{page}{1} \fi
\section{Notions, Définitions}

$E$ est un $\mathbb R$-e.v de dimension $n$ (donc isomorphe à $\mathbb R^n$), $F$ un sev de $E$ et $\Omega$ un ouvert de $E$.

\begin{rem}
    Pour $a\in\Omega, v\in E, \exists \epsilon>0, \forall t\in ]-\epsilon, \epsilon[, a+tv\in\Omega$ (par définition d'un ouvert)
\end{rem}

\begin{dfn}
    On note $f:\Omega \longrightarrow F$. On dira que $f$ possède une dérivée\index{dérivée partielle} en $a\in\Omega$ selon $v\in E$ si \[
        \lim_{\substack{t\to 0\\ t\neq 0}}\frac{f(a+tv)-f(a)}t
    \] existe. On note cette limite $D_vf(a)$.

    Si $\mathcal B=(e_1,\cdots, e_n)$ base de $E$ et $f$ a une dérivée en $a\in\Omega$ selon $e_{i_0}$ alors on appelle $i_0$-ième dérivée partielle $D_{e_{i_0}}f(a)$, notée \[
        \frac{\partial f}{\partial x_{i_0}}(a)\qquad \text{ ou }\qquad \partial_{i_0}f(a)
    \]
\end{dfn}

\begin{rem}
    Pour $\mathcal B=(e_1, \cdots, e_n)$ on note $x\underset{\mathcal B}{\longleftrightarrow}(x_1, \cdots, x_n)$ et on a $f(x)=f(x_1e_1+\cdots + x_ne_n)$ qu'on note abusivement $f(x_1, \cdots, x_n)$. \[
        \frac{f(x+te_1)-f(x)}{t}=\frac{f(x_1+t, x_2, \cdots, x_n)-f(x_1, \cdots, x_n)}{t}
    \]
    Pour calculer $\partial_if(a)$, on fait comme si les autres variables étaient constantes et on dérive par rapport à $x_i$.
\end{rem}

\begin{rem}
    $D_v$ est linéaire
\end{rem}

\begin{rem}
    Avoir une dérivée en $a$ n'implique pas la continuité en $a$. Contre-exemple: \[
        f(x, y)=\begin{cases}
            \frac{y^2}x &\text{ si }x\neq 0\\ y & \text{ sinon}
        \end{cases}
    \]
\end{rem}

\section{Différentielle}

\begin{defprop}
    \Hyp $f:\Omega\longrightarrow F$, $a\in \Omega$. \begin{concenum}
    \item On dira que $f$ est différentiable\index{différentielle} en $a$ s'il existe $\varphi\in\mathcal L(E, F), \epsilon>0$ tels que \[
            \forall h\in\mathcal B_o(0, \epsilon), f(a+h)=f(a)+\varphi(h)+o_0(h)
        \]
        Dans ce cas $\varphi$ est unique et se note $\diff f_a$
    \item Si $f$ est différentiable en $a$ alors $f$ est continue en $a$
    \item Si $f$ est différentiable en $a$ alors $f$ a une dérivée en $a$ selon tout $v$ et \[
            D_vf(a)=df_a(v)
        \]
    \end{concenum}
\end{defprop}

\begin{proof} ~
    \begin{enumerate}
        \item Si $\varphi, \psi$ conviennent, $\varphi(tv)-\psi(tv)=o_0(tv)$ ie \[
                \varphi(v)-\psi(v)=o_0(v)\xrightarrow[t\to 0]{}0
            \]
            d'où $\varphi\equiv \psi$.
        \item $f(a+h)=f(a)+\varphi(h)+o_0(h)\xrightarrow[h\to 0]{}f(a)$
        \item \[
                \frac{f(a+tv)-f(a)}{t}=\diff f_a(v)+o_0(v)\xrightarrow[\substack{t\to 0\\t\neq 0}]{}\diff f_a(v)
            \]
    \end{enumerate}
\end{proof}

\begin{thmdef}
    ~ \begin{enumerate}
        \item Soit $f\in\mathcal L(E, F)$ et $a\in E$. La fonction $f$ est différentiable en $a$ et $df_a=f$.
        \item Si $\mathcal B=(e_1, \cdots, e_n)$ est une base de $E$ et si $\pi_i$ est l'application $i$-ième coordonnée alors $\forall a\in E, \diff \pi_i=\pi_i$ se dépend pas de $a$. On note cette application $\diff x_i$.
        \item Si $\mathcal B$ est une base de $E$ et $f$ est différentiable en $a$ alors \[
                \diff f_a=\sum_{i=1}^n\frac{\partial f}{\partial x_i}(a)\diff x_i
            \]
    \end{enumerate}
\end{thmdef}

\begin{proof} ~
    \begin{enumerate}
        \item $f(a+h)=f(a)+f(h)$
        \item Ok
        \item $h=h_1e_1+\cdots +h_ne_n$, \[
                \diff f_a(h)=h_1\diff f_a(e_1)+\cdots +h_n\diff f_a(e_n)=\sum_{i=1}^n\frac{\partial f}{\partial x_i}(a)\diff x_i
            \]
    \end{enumerate}
\end{proof}

\begin{rem}
    Pour une fonction $f:\mathbb R\longrightarrow F$, $f'(a)=\diff f_a(1)$
\end{rem}

\section{Calculs de différentielles}

\subsection{Quelques exemples}
\begin{ex} Étude en $0$ de
    \[
        f(x, y)=\begin{cases}
            \frac{y^2}x &\text{ si }x\neq 0\\ y & \text{ sinon}
        \end{cases}
    \]
\end{ex}

\[
    \frac{\partial f}{\partial x}(0, 0)=0 \qquad\qquad \frac{\partial f}{\partial y}(0, 0)=f(0, 1)=1
\]
donc \[
    \frac{\partial f}{\partial x}(0, 0) \diff x+ \frac{\partial f}{\partial y}(0, 0)\diff y
\]
existe mais $f$ est non différentiable car non $\mathcal C^0$

\begin{ex} Calcul de la différentielle de
    {$f:A\in\mathcal M_n(\mathbb R)\longmapsto A^2$}
\end{ex}

\[
    f(A+H)=f(A)+AH+HA+H^2
\]
et pour une norme sous-multiplicative, $\|H^2\|\leq \|H\|^2=o_0(H)$ donc \[
    \diff f_A(H)=AH+HA
\]

\begin{ex} Calcul de la différentielle de
    {$f:M\in GL_n(\mathbb R)\longmapsto M^{-1}$}
\end{ex}

On fixe $M\in GL_n(\mathbb R)$. Alors $\exists \epsilon >0 / \mathcal B_o(M, \epsilon)\subset GL_n(\mathbb R)$. Pour $H$ dans $\mathcal B_o(0, \epsilon)$, \[
    M+H\in\mathcal B_o(M, \epsilon)
\]
et \[
    \left((M+H)^{-1}-M^{-1}\right)(M+H)=I_n-M^{-1}(M+H)=-M^{-1}H
\] donc
\begin{align*}
    (M+H)^{-1}-M^{-1}&=-M^{-1}H(M+H)^{-1}\\&=-M^{-1}HM^{-1}+M^{-1}Ho_0(1)\\&=\underbrace{-M^{-1}HM^{-1}}_{\diff f_M(H)}+o_0(H)
\end{align*}

\begin{exo} Calcul de la différentielle de
    {$f:A\in\mathcal M_n(\mathbb R)\longmapsto A^\intercal A$}
\end{exo}

\needspace{5\baselineskip}
\subsection{Interprétation matricielle}

\begin{thmdef}
    On note $f:\Omega\to \mathbb R^m$, $\Omega\subset \mathbb R^n$, on écrit $f=(f_1, \cdots, f_m)$. On suppose que $f$ est différentiable en $a\in\Omega$ et on note $\mathcal C_n$ (resp. $\mathcal C_m$) la base canonique de $\mathbb R^n$ (resp. $\mathbb R^m$). Alors: \begin{enumerate}
        \item On appelle matrice jacobienne\index{jacobienne (matrice -- )} de $f$ en $a$ la matrice \[
                J_f(a)=\left(\frac{\partial f_i}{\partial x_j}(a)\right)_{\substack{1\leq i\leq n\\ 1\leq j\leq n}}
            \]
        \item Si $x\underset{\mathcal C_n}{\longleftrightarrow}X$ alors \[
                \diff f_a(x)\underset{\mathcal C_m}\longleftrightarrow J_f(a)X
            \]
    \end{enumerate}
\end{thmdef}

\begin{proof}
    \[
        \diff f_a(x)=\sum_{i=1}^n\underbrace{\frac{\partial f}{\partial x_i}(x)}_{\displaystyle \left(\frac{\partial f_1}{\partial x_i}(x), \cdots, \frac{\partial f_m}{\partial x_i}(x)\right)}\overbrace{\diff x_i(x)}^{=x_i} \qquad \underset{\mathcal C_n}{\longleftrightarrow}\qquad \begin{pmatrix}
            \sum_{i=1}^n\limits \frac{\partial f_1}{\partial x_i}(a)x_i \\
            \vdots \\
            \sum_{i=1}^n\limits \frac{\partial f_m}{\partial x_i}(a)x_i
        \end{pmatrix}=J_f(a)X
    \]
\end{proof}

\section{Opérations sur les différentielles}

\begin{prop}
    Si $f,g:\omega\subset E\to F$ sont différentiables en $a$, alors pour $\lambda\in\mathbb R$, $\lambda f+g$ est différentiable en $a$ et $\diff (\lambda f+g)_a=\lambda df_a+\diff g_a$ \index{différentielle!opérations}
\end{prop}

\begin{prop}
    \Hyp $B:E_1\times E_2\to F$ bilinéaire, $f_1:\Omega_1\to E_1$ différentiable en $a_1$, $f_2:\Omega_2\to E_2$ différentiable en $a_2$.
    \Conc $f=B(f_1, f_2)$ est différentiable en $a=(a_1, a_2)$ et \[\diff f_a(h_1, h_2)=B(\diff {f_1}_{a_1}(h_1), f_2(a_2))+B(f_1(a_1), \diff {f_2}_{a_2}(h_2)) \]
\end{prop}

\begin{proof}
    On développe $f(a+h)=B(f_1(a_1+h_1), f_2(a_2+h_2))$
\end{proof}

\begin{ex} ~
    \begin{itemize}
        \item $(E, (\; |\; ))$ euclidien, $\varphi:(x, y)\longmapsto (x|y)$ \[
                \diff \varphi_{(x, y)}(h, k)=(h|y)+(x|k)
            \]
        \item $\varphi = \det $. On note \[
                A = (C_1 \cdots C_n) \quad H=(H_1 \cdots H_n)
            \]
            alors \[
                \det(A+H)=\det(A)+\det(H_1, C_2, \cdots, C_n)+\cdots + \det(C_1, \cdots, C_{n-1}, H_n)+o_0(H)
            \]
            On en déduit \begin{align*}
                \diff \varphi_A(H) &= \sum_{j=1}^n \det(C_1, \cdots, H_i, \cdots, C_n)\\
                                   &= \sum_{j=1}^n\sum_{i=1}^n(-1)^{i+j}h_{i, j}\Delta_{i, j}(A) \\
                                   &= \Tr(\Com(A)^\intercal H)
            \end{align*}
    \end{itemize}
\end{ex}

\section{Composition des fonctions différentielles}

\begin{prop}
    \Hyp $f : \Omega \subset E \to F$ différentiable en $a$ telle que $f(\Omega)\subset \Omega'$, $g : \Omega'\subset F \to G$ différentiable en $f(a)$
    \Conc $g\circ f$ est différentiable en $a$ et $\diff (g\circ f)_a=\diff g_{f(a)}\circ \diff f_a$ \index{différentielle!composition}
\end{prop}

\begin{proof}
    On a \[
        g(f(a)+k)=g(f(a))+\diff g_{f(a)}(k)+o(k)
    \]
    et \[
        f(a+h)=f(a)+\underbrace{\diff f_a(h)+o(h)}_{k}
    \]
    donc \[
        g\circ f(a+h)=g(f(a))+\diff g_{f(a)}(\diff f_a(h)+o_0(h)) + \underbrace{o(\underbrace{\diff f_a(h)+o_0(h)}_{o(h)})}_{o(h)}
    \]
    d'où \conc
\end{proof}

\begin{rem}
    $E=\mathbb R^n$, $F=\mathbb R^m$, $G=\mathbb R^p$, $f=(f_1, \cdots, f_m):E\to F$, $g=(g_1, \cdots, g_p):F\to G$, $\mathcal C_k$ base canonique de $\mathbb R^k$.

    Si $x \underset{\mathcal C_n}\longleftrightarrow X$ alors $\diff f_a(x)\underset{\mathcal C_m}\longleftrightarrow J_f(a)X$ et $\diff (g\circ f)_a(x)\underset{\mathcal C_p}\longleftrightarrow J_{g\circ f}(a)X\underset{\mathcal C_p}\longleftrightarrow J_g(f(a))J_f(a)X$. C'est vrai pour tout $X$ donc \[
        \underbrace{J_{g\circ f}(a)}_{\in\mathcal M_{p,n}} = \underbrace{J_g(f(a))}_{\in\mathcal M_{p,m}}\underbrace{J_f(a)}_{\in \mathcal M_{m,n}}
    \]
    On va identifier le coefficient $i, j$. \begin{align*}
        & \left[ J_{g\circ f}(a) \right]_{i, j}=\sum_{k=1}^m \left[ J_g(f(a)) \right]_{i, k} \left[ J_f(a) \right]_{k, j} \\
        \iff & \frac{\partial (g_i\circ f)}{\partial x_j} (a)=\sum_{k=1}^m \frac{\partial g_i}{\partial y_k}(f(a)) \frac{\partial f_k}{\partial x_j} (a) \\
        \iff & \colorboxed{red}{\partial_j (g_i\circ f)(a)=\sum_{k=1}^m\partial_kg_i(f(a))\partial_jf_k(a)} \quad \text{ \emph{(règle de la chaîne)} }
    \end{align*}

    En particulier, si $p = n = 1$, $\varphi:t\in \mathbb R\to g(f_1(t), \cdots, f_m(t))$ alors \[
        (g\circ f)'(a)=\sum_{k=1}^m \frac{\partial g}{\partial y_k}(f(a))f'_k(a)
    \]
\end{rem}

\begin{rem}
    Si $f:\Omega\subset E\to \mathbb R$ est différentiable en $a$ et $f(a)\neq 0$ alors $\frac 1f$ est différentiable en $a$ car $x\longmapsto \frac 1x$ est différentiable en $f(a)$
\end{rem}

\begin{rem}
    Si $f:\Omega\subset E\to F$ est $a\in\Omega$, $v\in E$ alors $\varphi:t \longmapsto f(a+tv)$ est définie au $\mathcal V(0)$ et $\varphi'(t)=\diff f_{a+tv}(v)$
\end{rem}

\needspace{5cm}
\section{Applications de classe \texorpdfstring{$\mathcal C^1$}{C1}}

\begin{thmdef}
    \Hyp $f:\Omega\subset E\longrightarrow F$
    \begin{concenum}
    \item On dira que $f$ est $\mathcal C^1$ sur $\Omega$ si $f$ est différentiable sur $\Omega$ et si $a\in\Omega \longmapsto \diff f_a\in\mathcal L(E, F)$ est $\mathcal C^0$
    \item Une fonction $\mathcal C^1$ est continue
    \item On note $\mathcal B=(e_1, \cdots, e_n)$ une base de $E$, $\mathcal C=(u_1, \cdots, u_m)$ une base de $F$ et $f=(f_1, \cdots, f_m)$. Il y a équivalence entre: \begin{enumerate}
        \item $f$ est $\mathcal C^1$ sur $\Omega$
        \item Les $f_i$ ont toutes leurs dérivées partielles sur $\Omega$ et les $\partial_jf_i$ sont $\mathcal C^0$ sur $\Omega$
    \end{enumerate}
\item On note $\mathcal C^1(\Omega, F)$ l'ensemble de ces fonctions
\end{concenum}
\end{thmdef}

\begin{proof}~
    \begin{enumerate}
        \setcounter{enumi}{1}
    \item Les fonctions $\mathcal C^1$ sont différentiables donc continues.
    \item $(a) \implies (b)$: \[
            J_f(a)=\mathcal M_{\mathcal B, \mathcal C}(\diff f_a)= \left( \frac{\partial f_i}{\partial x_j} (a) \right)_{\substack{1\leq i\leq m\\1\leq j\leq n}}
        \]
        est continue par rapport à $a$ par hypothèse.

        $(b) \implies (a)$: Il suffit de montrer que $f$ est différentiable sur $\Omega$. On commence par le cas $F=\mathbb R$. \begin{align*}
            f(a+h)-f(a) = \phantom{+} & f(a_1+h_1, \cdots, a_n+h_n)-f(a_1, a_2+h_2, \cdots, a_n+h_n) \\
            + & f(a_1, a_2+h_2, \cdots, a_n+h_n)-f(a_1, \cdots, a_i, a_{i+1}+h_{i+1}, \cdots, a_n+h_n) \\
            + & f(a_1, \cdots, a_i, a_{i+1}+h_{i+1}, \cdots, a_n+h_n) + \cdots \\
            + & f(a_1, \cdots, a_{n-1}, a_n+h_n)-f(a_1, \cdots, a_n)
        \end{align*}
        et chaque ligne est fonction d'une seule variable, donc le théorème des accroissements finis donne
        \begin{align*}
            \exists c_{i,h}\in]a_i,a_i+h_i[, \quad f(a+h)-f(a)=\phantom+ &
            h_1\frac{\partial f}{\partial x_1}(c_{1,h}, a_2+h_2, \cdots, a_n+h_n)\\
            + & h_2 \frac{\partial f}{\partial x_2}(a_1, c_{2,h},a_3+h_3,\cdots, a_n+h_n)\\
            + & \hspace{2cm}\vdots \\
            + & \underbrace{h_n \frac{\partial f}{\partial x_n} (a_1, \cdots, a_{n-1}, c_{n,h})}_{\longrightarrow h_n \frac{\partial f}{\partial x_n}(a) \text{ donc } =h_n \frac{\partial f}{\partial x_n}(a)+o(1) } \\
            =\phantom + & \underbrace{h_1 \frac{\partial f}{\partial x_1}(a)+\cdots + h_n \frac{\partial f}{\partial x_n} (a) }_{\text{AL}}  + \underbrace{h_1o(1)+\cdots h_no(1)}_{o(h)}
        \end{align*}
        donc $f$ est différentiable sur $\Omega$. Si $F\neq R$ on raisonne coordonnée par coordonnée.
\end{enumerate}
\end{proof}

\begin{prop} ~
    \begin{enumerate}
        \item $\mathcal C^1(\Omega, F)$ est un $\mathbb R$-e.v.
        \item Si $\mathcal C$ base de $F$ est $f:\Omega\longrightarrow F$ donnée par $f=(f_1, \cdots, f_n)$ (dans $\mathcal C$) alors il y a équivalence entre \begin{enumerate}
            \item $f$ est $\mathcal C^1$
            \item Les $f_i$ sont toutes $\mathcal C^1$
        \end{enumerate}
    \item Si $f:\Omega\longrightarrow F$ et $g: \Omega'\supset f(\Omega)\longrightarrow G$ sont $\mathcal C^1$ alors $g\circ f$ aussi
\end{enumerate}
\end{prop}

\begin{proof}
    1. et 2. sont faciles, 3. est déjà vu.
\end{proof}

\begin{thm}
    \Hyp $f:\Omega\subset E\longrightarrow \mathbb R$, $a,b\in\Omega$ tels que $[a, b]\subset \Omega$
    \begin{concenum}
    \item \[ f(b)-f(a)=\int_0^1\mathrm df_{(1-t)a+tb}(b-a)\diff t\]
    \item Si $\nop \diff f_x\nop\leq M$ pour $x\in[a, b]$ alors $|f(b)-f(a)|\leq M\|b-a\|$
    \end{concenum}
\end{thm}

\begin{proof}
    \begin{enumerate}
        \item $\varphi:t\longmapsto f(a+t(b-a))$ est telle que $\varphi'(t)=df_{(1-t)a+tb}(b-a)$ et \[
                f(b)-f(a)=\varphi(1)-\varphi(0)=\int_0^1\varphi'
            \]
        \item \[
                |f(b)-f(a)|\leq \int_0^1\|\diff f_{(1-t)a+tb}(b-a)\|\diff t\leq M\int_0^1\|b-a\|\diff t
            \]
    \end{enumerate}
\end{proof}

\section{Étude de régularité}

\subsection{\texorpdfstring{$f:(x, y)\longmapsto \min(x^2, y^2)$}{f(x, y)=min(x², y²)}}

D'abord, \[
    f(x, y)=\frac{x^2+y^2-|x^2-y^2|}2
\]
donc $f$ est $\mathcal C^0$. On note $\Delta$ le fermé $\{(x, y)\in\mathbb R^2, x^2=y^2\}$, $\mathbb R^2\setminus \Delta$ est ouvert de $f\in \mathcal C^1(\mathbb R^2\setminus \Delta, \mathbb R)$ comme composée de fonctions $\mathcal C^1$.

Il reste à savoir si $f$ est différentiable sur $\Delta$ et si éventuellement la différentielle est continue.

\begin{align*}
    f(x+h, x+k)-f(x,x) &= \frac{(x+h)^2+(x+k)^2-|(x+h)^2-(x+k)^2|}{2}-x^2 \\
                       &= x(h+k)+\frac{h^2+k^2}2 - \left|x(h+k)+ \frac{h^2-k^2}{2}  \right| \\
\end{align*}
Pour $k=0$, \[
    \frac{f(x+h, x)-f(x, x)}h=x+\frac h2-\frac{\left|xh+\frac{h^2}2\right|}h\xrightarrow[h\to0\pm]{}x\pm|x|
\]

Donc $f$ n'a pas de dérivée partielle par rapport à $x$ en $(x, x)$ pour $x\in\Delta\setminus\{0\}\defeq\Delta^\star$ (idem en $(x, -x)$) donc n'est pas différentiable sur les points de $\Delta$.

En $(0, 0)$, $f$ est différentiable car $0\leq f(x, y)\leq x^2+y^2=o(\|(x, y)\|_2)$ donc $f(x, y)=f(0, 0)+0+o((x, y))$ et $\diff f_{(0, 0)}=0$

Ainsi $(x, y)\longmapsto \diff f_{(x, y)}$ est définie sur $\mathbb R^2\setminus \Delta^\star$. Elle est continue sur $\mathbb R^2\setminus \Delta$, puis on va montrer qu'elle est continue en $(0, 0)$. On note $(x_0, y_0)\in\mathbb R^2\setminus \Delta$ tels que $x_0^2>y_0^2$ donc au voisinage de $(x_0, y_0)$, $f(x, y)=y^2$ et $\diff f_{(x_0, y_0)}=2y_0\diff y$ puis sinon $\diff f_{(x_0, y_0)}=2x_0\diff x$. Dans tous les cas, $\diff f_{(x, y)}\xrightarrow[(x, y)\to 0]{}0=\diff f_{(0, 0)}$ donc la différentielle est continue sur son domaine de définition.

\subsection{Domaine de continuité}

On pose \[
    f:(x, y)\longmapsto \begin{cases}
        \dfrac{x^2y^2}{x^2+y^2} &\text{ si } (x, y)\neq (0, 0) \\
        0 &\text{ sinon }
    \end{cases}
\]

$f$ est clairement continue sur $\mathbb R^2\setminus\{(0, 0)\}$ puis $x^2y^2\leq (x^2+y^2)^2$ donc $0\leq f(x)\leq x^2+y^2=o((x, y))$ donc $f$ continue en $0$

\subsection{Une fonction différentiable non \texorpdfstring{$\mathcal C^1$}{C1}}

On pose \[
    f:(x, y)\longmapsto \begin{cases}
        \displaystyle (x^2+y^2)\sin \left( \frac{1}{\sqrt{x^2+y^2}}  \right) &\text{ si } (x, y)\neq (0, 0) \\
        0 &\text{ sinon }
    \end{cases}
\]
La fonction est clairement $\mathcal C^1$ sur $\mathbb R^2\setminus\{0\}$ puis $f(x, y)=o_0((x, y))$ donc $\diff f_0=0$ et $f$ est différentiable. Il reste à voir si la différentielle est continue en $(0, 0)$.

Puis, \[
    \frac{\partial f}{\partial x} (x, y)= \begin{cases}
        \displaystyle 2x\sin \left( \frac{1}{\sqrt{x^2+y^2}} \right)-\frac x{\sqrt{x^2+y^2}}\cos \left( \frac{1}{\sqrt{x^2+y^2}}  \right) &\text{ si } (x, y)\neq (0, 0) \\ 0 &\text{ sinon }
    \end{cases}
\]
et \[
    \frac{\partial f}{\partial x} (x, 0)\xnrightarrow[x\to 0]{} 0
\]
donc $f$ n'est pas $\mathcal C^1$

\subsection{Expression intégrale}

On considère $g:\mathbb R^2\to \mathbb R$ de classe $\mathcal C^1$ et \[
    F:x\longmapsto \int_0^xg(x, t)\diff t
\]

On va montrer que $F$ est $\mathcal C^1$ et on va calculer $F'$.
On a \[
    F(x) \underset{t=ux}=\int_0^1g(x, ux)\diff u
\]
On note donc $h: (x, u) \longmapsto g(x, ux)$ et $F$ est $\mathcal C^1$ par théorème de dérivation sous le signe somme puis \begin{align*}
    F'(x)&=\int_0^1 g(x, ux)\diff u+x\int_0^1\frac{\partial g}{\partial x}(x, ux)\diff u+x\int_0^1\frac{\partial g}{\partial t}(x, ux)\diff u \\
         &= \int_0^1\frac{\partial}{\partial t}(ug(x, ux))+x\int_0^1 \frac{\partial g}{\partial x} (x, ux)\diff u \\ &= g(x, x)+\int_0^x \frac{\partial g}{\partial x}(x, t)\diff t
\end{align*}

\section{Caractérisation des fonctions constantes}

On se place dans $E$ et on note $\Omega$ un ouvert connexe par arcs de $E$, et $f:\Omega \to F$. On note $\mathcal B=(e_1, \cdots, e_n)$ une base de $E$. On va montrer que $f$ est constante sur $\Omega$ si et seulement si $f$ est $\mathcal C^0$ et a toutes ses dérivées partielles nulles.

\begin{itemize}
    \item ($\implies$) Direct
    \item ($\impliedby$) Par hypothèse, $\partial_if =0$ est continue donc $f$ est $\mathcal C^1$ donc pour $[a, b]\subset \Omega$, \[
            f(b)-f(a)=\int_0^1\diff f_{(1-t)a+tb}(b-a)\diff t=0\implies f(b)=f(a)
        \]

        Vu ce qui précède, $f$ est constante sur toutes les boules invertes incluses dans $\Omega$. On prend $a\in\Omega$ et on note \[
            \mathcal E=\{x\in\Omega, f(x)=f(a)\}.
        \]
        Cet ensemble est un fermé en tant que préimage par une application continue ($f$) par le fermé $\{f(a)\}$. C'est aussi un ouvert car pour $x_0\in\mathcal E$, $f$ est constante sur une boule ouverte centrée en $x_0$ donc par définition de $\mathcal E$, cette boule est dans $\mathcal E$.

        On note $h=\1_{\mathcal E}$ de sorte que $h^{-1}(\{0\})=\Omega\setminus \mathcal E$, $h^{-1}(\{1\})=\Omega$, $h^{-1}(\emptyset)=\emptyset$. Ainsi, $h$ est continue car la préimage d'un fermé est un fermé (induit). Ainsi, $h(\Omega)$ est connexe par arcs donc $h(\Omega)=\{1\}$ car $\mathcal E$ non vide et $\Omega=\mathcal E$
\end{itemize}


\textbf{Autre méthode:} Il suffit de montrer que $\Omega$ est connexe par arcs polygonaux. On fixe $a\in\Omega$ et on note $\mathcal E=\{x\in\Omega, \; \text{il existe un arc polygonal entre }a \text{ et } x\}$. C'est un ouvert (même justification que dans la méthode d'avant). Si $(\alpha_n)$ est une suite de $\mathcal E$ convergente vers $\alpha\in\Omega$, alors APCR $\alpha_n$ est dans une boule incluse dans $\Omega$ centrée en $\alpha$, puis il suffit de relier $\alpha_n$ et $\alpha$ dans $\Omega$ pour conclure que $\alpha\in\mathcal E$. Ainsi c'est un ouvert fermé de $\Omega$ donc c'est $\Omega$ (déjà fait)

\section{Extrema -- Gradient}

Dans cette partie, $f:\Omega\to\mathbb R$. On suppose $(E, (\;|\;))$ euclidien.

\begin{thmdef}
    \Hyp $f:\Omega\to\mathbb R$ différentiable en $a\in\Omega$
    \begin{concenum}
    \item $\diff f_a$ est une forme linéaire
    \item Il existe un unique $w\in E$ tel que \[
            \forall x\in E, \qquad \diff f_a(x)=(w|x)
        \]
        Ce vecteur est appelé gradient\index{gradient} de $f$ en $a$ et se note $\nabla f(a), \overrightarrow{\mathrm{grad}} f(a), \dots$
    \item Si $\mathcal B=(e_1, \cdots, e_n)$ est une base ON de $E$ alors \[
            \nabla f(a)=\sum_{i=1}^n\partial_i f(a)e_i
        \]
    \end{concenum}
\end{thmdef}

\begin{proof}~
    \begin{enumerate}
        \item Trivial
        \item C'est une application du théorème de représentation de Riesz\index{Riesz (théorème de représentation de -- )}
        \item \[
                \diff f_a(x)=\sum_{i=1}^n\partial_if(a)\underbrace{\diff x_i(x)}_{(e_i|x)}
            \]
    \end{enumerate}
\end{proof}

\begin{rem}[Interprétation géométrique]
    On note $u\in E$ unitaire. \[ f(a+tu)=f(a)+\diff f_a(tu)+o_0(t)=f(a)+t (\nabla f(a)|u) +o_0(t).\] Pour $a$ fixé tel que $\nabla f(a)\neq 0$, $(\nabla f(a)|u)$ est maximal pour $u=\dfrac{\nabla f(a)}{\|\nabla f(a)\|}$ et dans ce cas $(\nabla f(a)|u)=\|\nabla f(a)\|$
\end{rem}

\begin{rem}[Égalité des accroissements finis\index{accroissements finis!égalité}]
    On note $f:\Omega\to E$ différentiable, et $[a, b]\subset \Omega$. On note \[ g:t\in[0, 1]\longmapsto f((1-t)a+tb)\in\mathbb R \]
    \begin{itemize}
        \item $g$ est continue sur $[0, 1]$, dérivable sur $]0, 1[$ (car dérivable sur $[0, 1]$)
        \item On applique l'égalité des accroissements finis à $g$: \[
                \exists t_0\in ]0, 1[, \quad g(1)-g(0)=g'(t_0)
            \]
            soit encore \[
                f(b)-f(a)=\diff f_{(1-t_0)a+t_0b}(b-a)=(\nabla f(\underbrace{(1-t_0)a+t_0b}_{\defeq c})\;|\;b-a)
            \]
    \end{itemize}
    Ainsi, il existe $c\in ]a, b[$ tel que $f(b)-f(a)=\diff f_c(b-a)$
\end{rem}

\begin{thmdef}
    \Hyp $f:\Omega\subset E\to \mathbb R, a\in \Omega$, $f$ différentiable en $a$.
    \begin{concenum}
    \item On dira que $f$ a un point critique\index{point critique} en $a$ si $\diff f_a=0$ (i.e. si $\nabla f(a)=0$ avec un produit scalaire)
    \item Si $a$ est un extremum local de $f$ alors $a$ est un point critique de $f$.
    \end{concenum}
\end{thmdef}

\begin{proof}
    \begin{enumerate}
        \setcounter{enumi}{1}
    \item On note $u\in E$, $\varphi: t\longmapsto f(a+tu)$ est définie sur un intervalle du type $]-\varepsilon, \varepsilon[$, et $\varphi$ atteint un extremum local en $0$. Ainsi, $\varphi'(0)=0=\diff f_a(u)$ vrai pour tout $u$ donc $\diff f_a=0$
\end{enumerate}
\end{proof}

\begin{ex}
    On pose \[
        f:(x, y)\longmapsto (x^2+y^2)\exp(x^2-y^2)
    \]
    Cette fonction est $\mathcal C^1$ et si $(x, y)$ est un extremum alors \[
        \begin{cases}
            \displaystyle \frac{\partial f}{\partial x}(x, y)=\exp(x^2-y^2)2x(1+(x^2+y^2)) =0 \\[1em]
            \displaystyle \frac{\partial f}{\partial y}(x, y)=\exp(x^2-y^2)2y(1-(x^2+y^2)) =0 \\
            \end{cases} \iff \begin{cases}
            x=0 \\ y(1-y^2)=0
            \end{cases} \iff \begin{cases}
            x=0\\
            y\in \llbracket -1, 1\rrbracket
        \end{cases}
    \]
    Il reste à vérifier si les trois points sont bien des extrema. \begin{itemize}
        \item $(0, 0)$ est un minimum global
        \item On fait l'étude en $(0, 1)$. \[
                \underbrace{f(h, 1+k)-f(0, 1)}_{\Delta(h, k)}=2e^{-1}(h^2-k^2)+o_0(\|(h, k)\|)
            \]
            et $\Delta(h, 0)>0$ pour $h>0$ assez petit, $\Delta(0, k)<0$ pour $k>0$ assez petit donc $\Delta$ change de signe au voisinage de $(0, 1)$ donc $(0, 1)$ n'est pas un extremum (c'est un point selle).
        \item De même, $(0, -1)$ n'est pas un extremum.
    \end{itemize}

    \begin{center}
        \begin{tikzpicture}
            \begin{axis}[
                colormap name=viridis,
                % 3d box,
                width=15cm,
                view={25}{20},
                axis equal image,
                % enlargelimits=false,
                grid=major,
                domain=-0.5:0.5,
                samples=15,
                yticklabels={,$-1$,,$0$,,$1$},
                xticklabels={,,,$0$},
                zticklabels={,,},
                xlabel=$x$,
                ylabel=$y$,
                zlabel={$f(x, y)$},
                ]
                \addplot3 [y domain = -1.2:1.2, surf]
                    {(x*x+y*y)*exp(x*x-y*y)};
            \end{axis}
        \end{tikzpicture}
    \end{center}

\end{ex}

\ifsolo
~

\vspace{1cm}

\begin{center}
\textbf{\LARGE Compléments sur les polynômes} \\[1em]
\end{center}
\fi


\ifsolo
\tableofcontents
\else
\chapter{Compléments sur les polynômes}
\fi

Tous les théorèmes ou propositions présents dans ce chapitre sont soit déjà connus (donc au programme) soit nouveaux et sont alors des compléments de cours (hors programme). Le chapitre dans sa totalité n'est pas au programme, seuls quelques rappels du programme y sont présents car pertinents et/ou utiles au développement d'autres compléments.
\ifsolo
\else
\minitoc
\fi
\thispagestyle{empty}

\ifsolo \newpage \setcounter{page}{1} \fi
\section{Rappels}

$\mathbb K$ désigne un corps (donc commutatif). Un polynôme de $\mathbb K[X]$ s'écrit $P$ ou $P(X)$, l'évaluation en $a\in\mathbb K$ de $P$ se note $P(a)$. L'application \[
    P\in\mathbb K[X] \longmapsto (a\longmapsto P(a)\in\mathbb K)
\]
est un morphisme d'algèbres qui est injectif si $\mathbb K$ est infini. Pour $P\in\mathbb K[X]\setminus \{0\}$, $a\in\mathbb K$ est une racine de $P$ ssi \[
    P(a)=0\iff (X-a)\;|\;P
\]
et $a$ est racine de multiplicité $\alpha \geq 1$ ssi \[
    \begin{cases}
        P(a)=P'(a)=\cdots=P^{(\alpha - 1)}(a)=0 \\
        P^{(\alpha)}(a)\neq 0
    \end{cases}
    \iff \begin{cases}
        (X-a)^\alpha \; |\; P\\
        (X-a)^{\alpha+1}\;\not|\;P
    \end{cases}
\]
Si $a$ n'est pas racine de $P$, on convient que $a$ est racine de multiplicité $0$.

\begin{notation} ~
    \begin{itemize}
        \item
            On note $Z_\mathbb K(P)$ l'ensemble des racines de $P$ dans $\mathbb K$
        \item On note $\mathrm{mult}(P, a)$ la multiplicité de la racine $a$ dans $P$
    \end{itemize}
\end{notation}

\begin{thm}
    Soit $P\in \mathbb K[X]\setminus \{0\}$. Le polynôme $P$ a au plus $\deg P$ racines dans $\mathbb K$.
    En particulier, le polynôme nul est le seul polynôme qui a une infinité de racines.
\end{thm}

\begin{thm}
    Soit $P\in\mathbb K[X]$.
    \begin{enumerate}
        \item \[
                P(X)=\sum_{k=0}^{+\infty} \frac{P^{(k)}(0)}{k!}X^k
            \]
        \item \[
                \forall a\in\mathbb K, \quad P(X)=\sum_{k=0}^{+\infty}\frac{P^{(k)}(a)}{k!}(X-a)^k
            \]
    \end{enumerate}
\end{thm}
\begin{proof} ~
    \begin{enumerate}
        \item L'application \[
                \psi: P\longmapsto P(X)-\sum_{k=0}^{+\infty}\frac{P^{(k)}(a)}{k!}(X-a)^k
            \]
            est linéaire nulle sur les vecteurs de la base canonique.
    \end{enumerate}
\end{proof}

\begin{rem}
    On montre de même que \[
        P(a)=\sum_{k=0}^{+\infty}\frac{P^{(k)}(X)}{k!}(a-X)^k
    \]
\end{rem}

\section{Factorisation des polynômes}

\subsection{Rappels}

\begin{thm}
    Soit $P\in\mathbb K[X]$, $a_1, \cdots, a_p$ deux à deux distincts et $n_1, \cdots, n_p\in\mathbb N^\star$. Si $a_1, \cdots, a_p$ sont des racines de multiplicités $n_1, \cdots, n_p$ de $P$, alors \[
        (X-a)^{n_1}\cdots (X-a_p)^{n_p}\;|\;P
    \]
    En particulier, si $\deg P=\sum_i n_i$ alors il existe $\lambda\in\mathbb K^\star$ tel que \[
        P(X)=\lambda(X-a_1)^{n_1}\cdots (X-a_p)^{n_p}
    \]
\end{thm}

\subsection{Théorème de d'Alembert-Gauss}

\begin{thm}[D'Alembert-Gauss] \index{D'Alembert-Gauss (théorème de -- )}
    Tout polynôme non constant de $\mathbb C[X]$ admet au moins une racine
\end{thm}

\begin{proof}[Preuve de Gauss]
    \begin{itemize}
        \item $P\in \mathbb C[X]$ non constant a même racine que $\frac1{\deg P}P$ donc on suppose $P$ unitaire.
        \item Si $P(X)=a_0+\cdots+ a_{n-1}X^{n-1}+X^n$, et $M=\max |a_i|$, alors pour $|z|\geq M+2$, \begin{align*}
                |P(z)|&\geq |z|^n-M\frac{|z|^n-1}{|z|-1}=\frac{|z|^{n+1}-|z|^n-M|z|^n+M}{|z|-1}\\
                      &\geq \frac{|z|^n}{|z|-1} \left( |z|-1-M \right)\geq |z|^n\left(1-\frac M{M+1}\right)\geq \frac{(M+2)^n}{M+1}>M
            \end{align*}
        \item \begin{align*}
                \inf_{\mathbb C}|P|&=\min\left(\underbrace{\inf_{|z|>M+2}|P(z)|}_{>M}, \underbrace{\inf_{|z|\leq M+2}|P(z)|}_{\leq |P(0)|\leq M}\right) \\&=\inf _{|z|\leq M+2}|P(z)|
            \end{align*}
            Par compacité il existe $z_0\in\mathcal B_f(0, M+2)$ tel que $\inf_{\mathbb C}|P|=|P(z_0)|$.
        \item Si $P(z_0)\neq 0$ alors il existe $k\geq 1$ tel que \begin{align*}
                P(z_0+h)&=P(z_0)+\frac{P^{(k)}(z_0)}{k!}h^k+o(h^k), \quad P(z_0)\neq 0 \\
                        &=P(z_0)\left(1+h^k\frac{P^{(k)}(z_0)}{P(z_0)k!}+o(h^k) \right)
            \end{align*}
        \item En écrivant \[
                \frac{P^{(k)}(z_0)}{P(z_0)k!}=\rho e^{i\theta}, \quad h=\epsilon e^{-i\frac{\theta+\pi}k}
            \]
            on obtient \[
                P(z_0+\epsilon e^{-i\frac\theta k})=\underbrace{P(z_0)\underbrace{\left(1 - \epsilon^k \rho + o(\epsilon^k)\right)}_{|\;\;|<1}}_{|\;\;|<|P(z_0)|}
            \]
            absurde.
    \end{itemize}
\end{proof}

\section{Polynômes sous contraintes}

\subsection{Contraintes sur les zéros}

\begin{exo}
    Si $Z_{\mathbb C}(P)=Z_{\mathbb C}(Q)$ et $Z_{\mathbb C}(P+1)=Z_{\mathbb C}(Q+1)$ pour $P,Q$ non constants de $\mathbb C[X]$, alors $P=Q$
\end{exo}

On suppose par l'absurde que $P\neq Q$ et par exemple $n=\deg P\geq \deg Q=m$. On note $u_1, \cdots, u_r$ les racines de $P$ (et donc de $Q$), $v_1, \cdots, v_s$ les racines de $P+1$ (donc de $Q+1$). Les $u_i$ et les $v_j$ sont tous distincts, donc \[
    \deg P=n\geq \deg(P-Q)\geq r+s \quad\text{car }P-Q\text{ s'annule en $u_i$ et $v_j$ et }P-Q\neq 0
\]
On a $P'=(P+1)'$ donc $P'$ a $n-r$ racines dans les $u_i$ avec multiplicité, et $n-s$ dans les $n_j$ donc \[
    n-r+n-s\leq \deg P'=n-1\iff n+1\leq r + s
\]
Or $r+s\leq n$ absurde et donc $P = Q$

\subsection{Coefficients unitaires}

\begin{exo}
    Déterminer les $P\in\mathbb R[X]$ non constants avec les coefficients $\pm1$ et dont toutes les racines sont réelles.
\end{exo}

\begin{proof}[Résolution]
    On note $P$ un polynôme de degré $n\geq 2$ qui convient et quitte à changer $P$ en $-P$, $P$ est unitaire.
    \begin{itemize}
        \item $P(X)=X^n+p_{n-1}X^{n-1}+\cdots +p_0$
        \item $r_1, \cdots, r_n$ les racines (non nulles car $p_0\neq 0$) donne \[
                r_1^2+\cdots +r_n^2=p_{n-1}^2-2p_{n-2}>0 \implies p_{n-2}=-1 \text{ et }r_1^2+\cdots+r_n^2=3
            \]
        \item \[
                p_0 X^n P \left( \frac1X \right)=X^n+p_0p_1X^{n-1}+p_0p_2X^{n-2}+\cdots
            \]
            vérifie les mêmes propriétés donc \[
                \frac 1{r_1^2} +\cdots +\frac1{r_n^2}=3
            \]

        \item \[
                \sum_{i=1}^n\underbrace{\left(r_i^2+\frac1{r_i^2}\right)}_{\geq 2}=6 \implies n\leq 3
            \]
    \end{itemize}
    On doit étudier $12$ polynômes
    \begin{center}
        \begin{tabular}{lll}
            \sout{$X^2+X+1$} & \sout{$X^3+X^2+X+1$} & \sout{$X^3-X^2+X+1$} \\
            $X^2+X-1$ & \sout{$X^3+X^2+X-1$} & \sout{$X^3-X^2+X-1$} \\
            \sout{$X^2-X+1$} & \sout{$X^3+X^2-X+1$} & $X^3-X^2-X+1$ \\
            $X^2-X-1$ & $X^3+X^2-X-1$ & \sout{$X^3-X^2-X-1$}
        \end{tabular}
    \end{center}
\end{proof}

\subsection{Équation fonctionnelle}

\begin{exo}
    Déterminer les polynômes $P\in\mathbb C[X]$ tels que $P(X)P(X+2)+P(X^2)=0$
\end{exo}

\begin{proof}[Résolution] $-1$ et $0$ sont les deux polynômes constants solutions.

    Soit $P$ non constant solution et $a$ une racine. $P(a)=0\implies P(a^2)=0$ donc $P$ admet une infinité de racines sauf si $|a|=1$ ou $a=0$.

    Si $a=0$ est racine alors $P(-2)P(0)+P(4)=0$ donc $4$ est racine ce qui est absurde. Donc $|a|=1$.

    Si $a$ est racine alors $(a-2)^2$ racine donc $|a-2|=1=|a|$ donc $a=1$ (intersection de deux cercles), donc $1$ seule racine de $P$ et donc pour un $k\geq 1$, \[
        P=\lambda(X-1)^k
    \]
    En calculant, on trouve $\lambda^2+\lambda=0$ et $\lambda \neq 0$ donc $\lambda = -1$
\end{proof}

\begin{exo}
    Trouver les polynômes $P\in\mathbb R[X]$ tels que \[
        (X+1)P(X-1)-(X-1)P(X)
    \]
    est un polynôme constant
\end{exo}

\begin{proof}[Résolution]
    On suppose que $P$ convient et $(X+1)P(X-1)-(X-1)P(X)=\lambda$. \begin{itemize}
        \item $2P(-1)=\lambda$ et $2P(0)=\lambda$ donc $P(0)=P(-1)$
        \item $Q(X)=P(X)-\frac\lambda 2$ s'annule en $-1$ et $0$ donc $Q(X)=X(X+1)A(X)$ i.e.  $P(X)=X(X+1)A(X)+\frac\lambda 2$.
        \item \[
                (X+1)((X-1)X A(X-1)+\frac\lambda2)-(X-1)X(X+1)A(X)-(X-1)\frac\lambda2=\lambda
            \]
            donc \[
                X(X-1)(X+1)(A(X-1)-A(X))=0 \implies A(X)=A(X-1)
            \]
            et donc $A$ est un polynôme constant. Donc $P$ est de la forme $P(X)=aX(X+1)+\mu, \quad a, \mu\in\mathbb R$
    \end{itemize}
\end{proof}

\section{Liens coefficients et racines}

\subsection{Relations de Viète}

\index{Viète!relations}

Pour $x_1, \cdots, x_n\in\mathbb C$ on note: \[
    \sigma_i(x_1, \cdots, x_n)=\sum_{\substack{I\subset \llbracket 1, n\rrbracket\\|I|=i}}\prod_{l\in I}x_l
\]
Soit $P(X)=a_nX^n+\cdots+a_0$ un polynôme de degré $n$ et $x_1, \cdots, x_n$ ses racines dans $\mathbb C$. \begin{itemize}
    \item \begin{align*}
            P(X)=a_n(X-x_1)\cdots (X-x_n)=a_n(X^n-\sigma_1 X^{n-1}+ \sigma_2 X^{n-2}+\cdots (-1)^n\sigma_nX^{n-n})
        \end{align*}
    \item Ainsi, \[
            \forall i\in\llbracket 1, n\rrbracket, \quad (-1)^i\sigma_i=\frac{a_{n-i}}{a_n}
        \]
\end{itemize}
En particulier, on en déduit \[
    x_1+\cdots+x_n = -\frac{a_{n-1}}{a_n}
\]
et \[
    x_1\times \cdots\times x_n=(-1)^n\frac{a_0}{a_n}
\]

\subsection{Somme de Newton}

On note $P(X)=a_nX^n+\cdots+a_0$ un polynôme de $\mathbb C[X]$, $x_1, \cdots, x_n$ ses racines. On note $S_k=x_1^k+\cdots + x_n^k$ appelée $k-$ième somme de Newton \index{Newton!sommes}.

Pour $k\geq n$, \begin{align*}
    a_nS_k+a_{n-1}S_{k-1}+\cdots + a_0S_{k-n}&=a_nx_1^k+\cdots +a_0x_1^{k-n}+\cdots +a_nx_n^k+\cdots +a_0x_n^{k-n} \\ &=x_1^{k-n}P(x_1)+\cdots + x_n^{k-n}P(x_n)=0
\end{align*}

Avec les relations de Viète, pour $k\geq n$, \[
    \underbrace{\sigma_0}_{=\; 0 \text{ convention}} S_k-\sigma_1S_{k-1}+\cdots + (-1)^nS_{k-n}=0
\]

Il est possible de calculer $S_1, \cdots, S_{n-1}$ par des formules de récurrence. \begin{itemize}
    \item $S_0=n$ clair. On va montrer que \[
            k\sigma_k=\sum_{i=1}^k(-1)^{i-1}\sigma_{k-i}S_i
        \]
    \item On note $P$ le polynôme $P/\dom P$\[
            P(x)=\prod_{i=1}^n(x-x_i)=\frac1{a_n}\sum_{k=0}^na_kx^k=\sum_{k=0}^n\sigma_{n-k}(-1)^{n-k}x^k
        \]
        d'où \[
            x^nP\left(\frac1x\right)=\underbrace{\prod_{k=1}^n(1-xx_k)}_{Q(x)}=\sum_{k=0}^n\sigma_{n-k}(-1)^{n-k}x^{n-k}=\sum_{k=0}^n\sigma_k(-1)^kx^k
        \]
    \item \begin{align*}
            \sum_{k=0}^n k\sigma_k(-1)^kx^k&= x\sum_{k=1}^n \frac{Q(x)}{1-xxk}(-x_k) \\
                                           &=-\sum_{k=1}^n\frac{xx_k}{1-xx_k}Q(x) \\
                                           &=-\left(\sum_{k=1}^n\sum_{i\geq 0}(xx_k)^{i+1}\right)Q(x) \\
                                           &=-\sum_{i\geq 0}x^{i+1}S_{i+1}\sum_{l=0}^n(-1)^k\sigma_kx^k
        \end{align*}

        En identifiant le coefficient de $x^k$, \begin{align*}
            (-1)^kk\sigma_k=-\sum_{l=0}^{k-1}(-1)^l\sigma_lS_{k-l} \iff k\sigma_k &=-\sum_{l=0}^{k-1}(-1)^{k-l}\sigma_lS_{k-l} \\
                                                                                  & =\sum_{l=1}^k(-1)^{l-1}\sigma_{k-l}S_l
        \end{align*}
\end{itemize}

\section{Polynômes scindés sur $\mathbb R$}

\subsection{Une CNS}

\begin{exo}
    Montrer que $P\in\mathbb R[X]$ de degré $n$ est scindé sur $\mathbb R$ ssi \[
        \forall z\in\mathbb C, \quad |P(z)|\geq |\dom(P)|\times |\Im(z)|^n
    \]
\end{exo}

\begin{proof}[Résolution]
    \emph{($\Longrightarrow$)} clair, les racines complexes de $P$ ont une partie imaginaire nulle.

     \emph{($\implies$)} On écrit $P(X)=\dom(P)(X-a_1)\cdots (X-a_n), \quad a_i\in\mathbb R$. En injectant $z=a+ib$, \[
        |P(z)|=|\dom P|\times |a-a_1+ib| \cdots |a-a_n+ib|\geq |\dom P| |b|^n
    \]
\end{proof}

\subsection{Opérations stabilisant $\mathcal S$}

On note $\mathcal S$ l'ensemble des polynômes scindés de $\mathbb R[X]$. On note \[
    P(X)=\lambda (X-a_i)^{\alpha_1}\cdots (X-a_p)^{\alpha_p}, \quad a_1<\cdots <a_p
\]
et on a les propriétés de stabilité suivantes.

\vspace{.2cm}

 Rolle donne que $P'$ s'annule au moins une fois sur $]a_i, a_{i+1}[$ pour $1\geq i < p$. Il y a alors au total $(\alpha_1-1)+\cdots+(\alpha_p-1)+p-1=n-1$ racines réelles (avec multiplicité) et $\deg P'=n-1$ donc $P'$ est scindé et $\mathcal S$ est stable par dérivation.

\vspace{.2cm}

 Pour $a\in\mathbb R$ fixé, on note $g(x)=e^{ax}P(x)$, $g'(x)=(P'(x)+aP(x))e^{ax}$. $P'+aP$ s'annule en $a_i$ avec multiplicité $\geq \alpha_i-1$. Rolle donne que $g'$ s'annule au moins une fois sur $]a_i, a_{i+1}[$ donc $P'+aP$ aussi. Ce polynôme a $n-1$ racines réelles et est de degré $n$, donc par division la dernière racine est réelle. Donc $P'+aP\in\mathcal S$

\vspace{.2cm}

 Plus généralement, si $D$ est l'opérateur de dérivation polynomiale, pour $Q\in\mathcal S$ alors
\begin{itemize}
    \item
\[
    Q(X)=\lambda (X-b_1)\cdots (X-b_p)
\]
\item \[
        Q(D)=\lambda(D-b_1\mathrm{id})\circ \cdots \circ (D-b_p \mathrm{id})
    \]
\end{itemize}
et pour $P\in\mathcal S$, $Q(D)(P)\in\mathcal S$.

\subsection{Faisceau de polynômes à racines entrelacées (ENS)}

\begin{exo}
    On note $P, Q\in\mathbb R[X]$ deux polynômes scindés à racines simples distinctes tels qu'entre deux racines de l'un, il y en aie au moins une de l'autre. Montrer que $\lambda P+\mu Q$ est scindé sur $\mathbb R$
\end{exo}

\begin{proof}[Résolution]
    On suppose que $P$ a la plus petite racine, on note $x_1 <\cdots<x_n$ les racines de $P$. Sur $]x_1, x_2[$, il y a au moins une racine de $Q$ et au plus une (sinon $x_1$ et $x_2$ ne sont pas consécutives) donc les $n-1$ premières racines de $Q$ satisfont \[
        x_1<y_1<\cdots <x_{n-1}<y_{n-1}<x_n
    \]
    Il y a lors deux cas: $Q$ possède une racine $y_n>x_n$, ou $Q$ a $n-1$ racines réelles. On se place dans le premier cas (l'autre se traite similairement).

    Comme $\lambda, \mu\in\mathbb R$ sont quelconques, on peut supposer $P$ et $Q$ unitaires donc \[
        P(X)=(X-x_1)\cdots (X-x_n) \qquad Q(X)=(X-y_1)\cdots (X-y_n)
    \]
    Pour $\lambda, \mu\neq 0$, les racines réelles de $\lambda P+\mu Q$ sont différentes des $x_i, y_i$, donc sont les mêmes que celles de $\frac{\lambda P+\mu Q}{PQ}$ sur $\mathbb R\setminus \{x_i, y_i\}$. \[
        \varphi(x)=\frac{\lambda P(x)+\mu Q(x)}{P(x)Q(x)}=\frac\lambda{Q(x)}+\frac\mu{P(x)}
    \]
    Si $\lambda, \mu>0$, alors $\varphi$ s'annule $n$ fois car il y a changement de signe entre $]x_i, y_i[$ (les polynômes sont unitaires donc on connaît exactement les signes de $P$ et $Q$). Si $\lambda>0>\mu$, on trouve $n-1$ racines réelles de la même manière, et par division la dernière est aussi réelle. Les autres cas se traitent de la même manière.
\end{proof}

\section{Valeurs des polynômes}

\subsection{Polynôme à valeurs entières}

On va caractériser les polynômes de $\mathbb C[X]$ tels que $P(\mathbb Z)\subset \mathbb Z$

On définit \[
    \binom X0=1\qquad \text{ et }\qquad \forall n\geq 1,\quad \binom Xn=\frac1{n!}X(X-1)\cdots(X-n+1)
\]

\begin{lmm}
    Pour tout $n\in\mathbb N$ et pour tout $x\in\mathbb Z$, \[
        \binom xn\in\mathbb Z
    \]
\end{lmm}

\begin{proof}
    Pour $0\leq x\leq n-1$, $\binom xn=0$. Pour $x\geq n$, $\binom xn$ est un coef du binôme. Pour $x<0$, $(-1)^n\binom xn$ est un coef du binôme.
\end{proof}

 On introduit \[
    \Delta : P\in\mathbb K[X]\longmapsto P(X+1)-P(X)
\]
Pour $n\geq 1$, \[
    \Delta\binom Xn=\binom X{n-1}\qquad\text{ et }\qquad \Delta\binom X0=0
\]
et $\displaystyle \left(\binom Xn\right)_{n\geq 0}$ est une base de $\mathbb C[X]$ (degré échelonné). On suppose que $P\in\mathbb C[X]$ vérifie $P(\mathbb Z)\subset \mathbb Z$. On note $a_0, \cdots, a_n$ les coefficients de la décomposition de $P$ dans cette base. Alors \[
    P(0)=a_0\in\mathbb Z, \quad a_1=\Delta P(0)\in\mathbb Z, \cdots, \quad a_n=\Delta^nP(0)\in\mathbb Z
\]

Ainsi, $a_0, \cdots, a_n\in\mathbb Z$ et la réciproque est évidente. On a ainsi établi \[
    P(\mathbb Z)\subset \mathbb Z\iff \exists n\in\mathbb N, \exists, a_0, \cdots, a_n\in\mathbb Z, P(X)=\sum_{k=0}^na_k\binom Xk
\]

\begin{rem}
    On a aussi établi la formule de Gregory\index{Gregory (formule de -- )} \[
        P(X)=\sum_{k\geq 0}\Delta^kP(0)\binom Xk
    \]
\end{rem}

\begin{exo}[ENS]
    On note $P\in\mathbb R[X]$ un polynôme de degré $n$ tel que $P(0), P(1), P(4), \cdots, P(n^2)\in\mathbb Z$. Montrer que $\forall a\in\mathbb Z, P(a^2)\in \mathbb Z$.
\end{exo}

\begin{proof}[Résolution]
    $Q(X)=P((X-n)^2)$ de degré $2n$ envoie $2n+1$ valeurs de $\mathbb Z$ dans $\mathbb Z$ donc a des coefficients entiers dans la base explicitée précédemment donc est à valeurs dans $\mathbb Z$ ce qui conclut.
\end{proof}

\subsection{Polynôme stabilisant le cercle unité}

\begin{exo}
    Déterminer les polynômes $P\in\mathbb C[X]$ tels que $\mathbb U$ est stable par $P$.
\end{exo}

\begin{proof}[Résolution]
    Le cas des polynômes constants est immédiat. On note $P(z)=a_0+\cdots +a_nz^n$. On a \[
        P(e^{i\theta})\overline{P(e^{i\theta})}=1\implies P(e^{i\theta})\overline P(e^{-i\theta})=1 \implies P(e^{i\theta})\overline P(e^{-i\theta})e^{in\theta}=e^{in\theta}
    \]
    Si on note $Q(X)=\overline P\left(\frac1X\right)X^n=\bar a_0X^n+\cdots +\bar a_n$ alors $PQ-X^n$ s'annule sur $\mathbb U$ donc est nul donc $P(X)\;|\;X^n$, $\deg P=n$ et $P=\lambda X^n$, où $\lambda=P(1)\in\mathbb U$. La réciproque est claire.
\end{proof}

\begin{exo}
    Variantes \begin{itemize}
        \item $P(\mathbb C)\subset \mathbb R$. Solution: Formule de Gregory, interpolation de Lagrange, polynôme conjugué, ...
        \item $P(\mathbb Q)\subset \mathbb Q$. Solution: Formule de Gregory, interpolation de lagrange.
        \item $P(\mathbb Q)\subset \mathbb R\setminus \mathbb Q$ et $P(\mathbb R\setminus \mathbb Q)\subset \mathbb Q$
        \item $P(\mathbb Q)=\mathbb Q$
    \end{itemize}
\end{exo}

\begin{proof}[Résolution]
    Résolution pour $P(\mathbb Q)=\mathbb Q$. Déjà $P\in\mathbb Q[X]$ (par la $2$-ième variante), donc il existe $d$ entier positif minimal tel que $Q=dP\in\mathbb Z[X]$. On se donne $v$ premier avec les coefs $a_i$ de $P$ et $d$ et $v$ nombre premier. Il existe $\frac ab$, $a\land b=1$, tel que $P(\frac ab)=\frac dv$ donc \[
        v(a_na^n+\cdots +a_0b^n)=db^n \implies v\;|\; b^n\implies v\;|\; b
    \]
    et si $n\geq 2$ alors $v^2\;|\;db^n$ et donc $v|a_na^n+\cdots$ donc $v\;|\;a$ absurde. Par suite, $n=1$, et la réciproque est évidente.
\end{proof}

\subsection{Polynômes positifs}

On va caractériser les polynômes tq $P(\mathbb R_+)\subset \mathbb R_+$.
On pose \[
    \mathcal E=\{P\in\mathbb R[X] \;/\; \exists A, B\in\mathbb R[X], P=A^2+XB^2\}
\]
et on note $\mathcal S$ l'ensemble des polynômes qui conviennent. On a clairement $\mathcal E\subset \mathcal S$, puis $\mathcal E$ est stable par produit car \[
    (A^2+XB^2)(A_1^2+XB_1^2)=(AA_1+XBB_1)^2+X(BA_1+AB_1)^2
\]
Soit $P\in\mathcal S$. Dans la décomposition de $P$ en irréductibles de $\mathbb R[X]$, si on note $\alpha_1, \cdots, \alpha_p$ les puissances des facteurs de degré $1$, on a $(X-a_i)$ change de signe si $\alpha_i$ impair et dans ce cas $a_i<0$ et $X-a_i=\sqrt{-a_i}^2+X1^2\in\mathcal E$. Les autres termes sont dans $\mathcal E$ et donc $\mathcal S=\mathcal E$.

\section{Localisation des racines}

\subsection{Théorème de Gauss-Lucas}

\index{Gauss-Lucas (théorème de -- )}

On note $P\in\mathbb C[X]$ avec $n=\deg P\geq 2$. On va montrer que si $z_1, \cdots, z_n$ sont les racines de $P$ (répétées avec multiplicités) alors les racines de $P'$ sont dans l'enveloppe convexe de $z_1, \cdots, z_n$. On a:
\[
    \frac{P'}P(z)=\frac{\alpha_1}{z-z_1}+\cdots +\frac{\alpha_p}{z-z_p}
\]
donc si $z$ est une racine de $P'$ différente des $z_1, \cdots, z_n$, alors \[
    \alpha_1\frac{z-z_1}{|z-z_1|^2}+\cdots+\alpha_p\frac{z-z_p}{|z-z_p|^2}=0 \iff z \left( \frac{\alpha_1}{|z-z_1|^2}+\cdots+\frac{\alpha_p}{|z-z_p|^2} \right)=\frac{\alpha_1}{|z-z_1|^2}z_1+\cdots+\frac{\alpha_p}{|z-z_p|^2}z_p
\]
Donc $z$ est bien un barycentre de $z_1, \cdots, z_n$.

\subsection{Borne de Cauchy}

\index{Cauchy!borne}

On note $P(z)=z^n+a_{n-1}z^{n-1}+\cdots+a_0\in\mathbb C[X]$ et on pose $M=\max_{i\leq n-1}\limits |a_i|$. On note $z\in\mathbb C$ tel que $|z|>M+1$. Par l'absurde si $P(z)=0$ alors \begin{align*}
    |z^n|=|a_{n-1}z^{n-1}+\cdots +a_0|\leq M\frac{|z|^n-1}{|z|-1} &\implies |z|^{n+1}-|z|^n\leq M|z|^n-M \\ &\implies |z| |z|^n\leq (M+1)|z|^n \\ &\implies |z|\leq M+1\quad\text{ absurde }
\end{align*}
Bilan les racines de $P$ sont dans le disque $D(0, M+1)$

\subsection{Enestrom-Kakeya}

\begin{res}[Cauchy]
    \index{Cauchy!localisation de racines}
    Si $P(x)=x^n-b_{n-1}x^{n-1}-\cdots -b_0$ est un polynôme tel que les $b_i$ sont positifs non tous nuls alors $P$ a une seule racine $a$ dans $\mathbb R_+^\star$, $a$ est racine simple et $\mathbb Z_{\mathbb C}(P)\subset \mathcal B_f(0, a)$.
\end{res}

\begin{proof}
    Pour $x>0$, $f(x)=-\frac{P(x)}{x^n}=\frac{b_0}{x^n}+\cdots + \frac{b_{n-1}}{x}-1$ est décroissante.

    \begin{center}
        \begin{tikzpicture}
            \tkzTabInit{$x$ / 1 , $f(x)$ / 1}{$0$, $a$, $+\infty$}
            \tkzTabVar{+/ $+\infty$, R /, -/ $-1$}
        \end{tikzpicture}
    \end{center}
    Donc $f$ a une unique racine $a>0$, puis \[
        f'(a)=-n\frac{b_0}{x^{n+1}}-\cdots-\frac{b_{n-1}}{x^2}<0 \implies P'(a)\neq 0
    \]
    donc $a$ est racine simple de $P$. Si $z_0$ est une racine de $P$ dans $\mathbb C$ alors $P(z_0)=0$ donc \[
        |z_0^n|=|b_{n-1}z_0^{n-1}\cdots b_0|\leq b_{n-1}|z_0^n|+b_0
    \]
    donc $P(|z_0|)\leq 0$ et $f(|z_0|)\geq 0$ donc $|z_0|\leq a$
\end{proof}

\begin{res}[Enestrom-Kakeya]
    \index{Enestrom-Kakeya (théorème de -- )}
    Si $P(x)=a_{n-1}x^{n-1}+\cdots +a_0$ est un polynôme à coefficients strictement positifs, alors si $e$ est une racine complexe de $P$, \[
        \delta=\min_{i\leq n-1}\frac{a_{i-1}}{a_i} \leq |z|\leq \max_{i\leq n-1}\frac{a_{i-1}}{a_i}=\gamma
    \]
\end{res}

\begin{proof}
    On considère le polynôme \begin{align*}
        Q(x)=(x-\gamma)P(x)&=a_{n-1}x^{n}+(a_{n-2}-\gamma a_{n-1})x^{n-1}+\cdots +(a_0-\gamma a_1)x -\gamma a_0\\ &= a_{n-1}\underbrace{\left(x^n-\frac{\gamma a_{n-1}-a_{n-2}}{a_{n-1}}x^{n-1} -\cdots -\frac{\gamma a_1-a_0}{a_{n-1}}x-\frac{\gamma a_0}{a_{n-1}}\right)}_{\text{vérifie les hypothèses du résultat précédent donc a une unique racine réelle $>0$, $\gamma$}}
    \end{align*}
    d'où la seconde inégalité. La première s'obtient en remarquant que $z$ racine de $P$ entraîne $z$ non nul et $\frac 1z$ racine de $X^{n-1}P\left(\frac1X\right)$ auquel on applique le même raisonnement.
\end{proof}

\subsection{Disques de Gershgorin}

\index{Gershgorin (disques -- )}

On se donne $A\in\mathcal M_n(\mathbb C)$ et $\lambda\neq a_{1, 1}, \cdots, a_{n, n}$. On note $D=\mathrm{diag} (a_{1, 1}, \cdots, a_{n, n})$ et $E=A-D$. Si $\lambda$ est valeur propre, alors \[
    A_\lambda=\underbrace{A-\lambda I_n}_{\text{non inversible}}=\underbrace{D-\lambda I_n}_{\text{inversible}}+E
\]
Puis $\exists X\in\mathcal M_{n, 1}(\mathbb C), X\neq 0 / A_\lambda X=0$ donc \begin{align*}
    (D-\lambda I_n)X+EX=0&\iff X=-(D-\lambda I_n)^{-1}EX \\&\iff \forall i\in\llbracket 1, n\rrbracket, \quad x_i=-\frac1{a_{i, i}-\lambda}\sum_{k\neq i}a_{i, k}x_k
\end{align*}
On considère un indice $i$ tel que $|x_i|$ est maximal et donc \[
    |a_{i, i}-\lambda|=|\sum_{k\neq i}a_{i, k}\frac{x_k}{x_i}\leq \sum_{k\neq i}|a_{i, k}|=R_i
\]
et donc \[
    \lambda\in\mathcal D(a_{i, i}, R_i) \qquad \text{ reste vrai si } \lambda=a_{i, i}
\]

On a donc \[
    \mathrm{Sp}_{\mathbb C}(A)\subset \bigcup_{i=1}^n \underbrace{\mathcal D(a_{i, i}, R_i)}_{\text{disques de Gershgorin}}
\]

\section{Arithmétique des polynômes}

\subsection{Division euclidienne}

\index{division euclidienne}

\begin{exo}
    Effectuer la division euclidienne de $X^n-1$ par $X^m-1$, pour $n, m\geq 1$
\end{exo}

\begin{proof}[Résolution]
    On note $(u_k)$ la suite définie par \[
        \begin{cases}
            u_0=n, u_1=m \\ u_{k+1} \quad \text{ reste de la division euclidienne de $u_{k-1}$ par $u_k$}
        \end{cases}
    \]
    tant que $u_k\neq 0$ et on note \[
        \begin{cases}
            U_0=X^n-1, U_1=X^m-1\\ U_{k+1}\quad \text{ reste de la division euclidienne de $U_{k-1}$ par $U_k$}
        \end{cases}
    \]
    tant que $U_k\neq 0$

    On vérifie par récurrence que $U_k=X^{u_k}-1$. Les cas $k=0, 1$ sont immédiats. On a $u_{k+1}+au_k=u_{k-1}$ donc par HR si $U_k\neq 0$ (ie $u_k\neq 0$), alors \[
        U_{k-1}=U^{u_{k+1}}X^{au_k}-1=\underbrace{X^{u_{k+1}}-1}_{\deg < \deg U_k}+\underbrace{X^{u_{k+1}}\left(X^{au_k}-1\right)}_{\text{divisible par } U_n}
    \]
    d'où le résultat. On en déduit que les deux suites s'arrêtent en même temps et donc \[
        U_0\land U_1=U_{k-1}=X^{m\land n}-1
    \]
\end{proof}

\subsection{Théorème chinois}

\index{chinois (théorème)}

\begin{exo}
On se donne $P_1, \cdots, P_n$ deux à deux premiers entre eux et $R_1, \cdots, R_n\in\mathbb C[X]$. Montrer qu'il existe $Q_1, \cdots, Q_n$ tels que \[
    P_1Q_1+R_1=\cdots=P_nQ_n+R_n
\]
\end{exo}

\begin{proof}[Résolution]
    C'est une application directe du théorème chinois (dans un anneau principal): si $d_i=\deg P_i-1$ et $N=-1+\sum d_i$, alors \[
        \varphi: P\in \mathbb C_N[X]\longmapsto (P\mod P_1, \cdots, P\mod P_n)\in\mathbb C_{d_1}[X]\times \cdots \times \mathbb C_{d_n}[X]
    \]
    est un isomorphisme d'algèbres.
\end{proof}

\subsection{Théorèmes de Mason, Snyder, et Fermat}
On va montrer le théorème de Mason (ou Mason-Stothers) avec une démonstration de
Snyder\footnote{\url{https://doi.org/10.1007/s000170050074}}

\begin{lmm}
    % \index{Snyder (théorème de -- )}
    Si $P\in\mathbb C[X]$ non nul, alors \[
        \deg P=\deg \left(P\land P'\right)+n_0(P)
    \]
    où $n_0(P)=\#Z_{\mathbb C}(P)$.
\end{lmm}

\begin{proof}
    Le cas $P$ constant est immédiat. On suppose maintenant $P$ non constant: \[
        P(X)=c(X-a_1)^{\alpha_1}\cdots (X-a_p)^{\alpha_p}
    \]
    donc \[
        P\land P'=(X-a_1)^{\alpha_1-1}\cdots (X-a_p)^{\alpha_p-1}
    \]
\end{proof}

\begin{thm}[Mason]
    \index{Mason (théorème de -- )}
    Si $A, B, C$ sont des polynômes complexes deux à deux premiers entre eux avec $A+B=C$ et $(A', B', C')\neq (0, 0, 0)$, alors \[
        \deg C\leq n_0(ABC)-1
    \]
\end{thm}

\begin{proof}
Comme $A=C-B$ et $B=C-A$ les rôles de $A, B, C$ sont symétriques. On peut supposer $A$ non constant. On a alors \[
    \left|\begin{matrix}A&C\\A'&C'\end{matrix}\right|=\left|\begin{matrix}A&B\\A'&B'\end{matrix}\right|
\]
donc $AC'-CA'=AB'-BA'$ donc \[
    A\land A', \quad B\land B', \quad C\land C'\quad\Big|\quad AC'-A'C=AB'-BA'
\]
Si $AC'-A'C=0$ alors $AC'=A'C$ et $C\;|\;C'$ donc $C=0$, et de même $B=0$ donc $A=0$ absurde. Les trois polynômes $A, B, C$ sont premiers entre eux donc les trois PGCD le sont aussi et \[
    A\land A' \quad \times \quad B\land B'\quad \times \quad C\land C'\quad \;\Big|\; AC'-A'C\neq 0
\]
donc \[
    \deg A\land A' + \deg B\land B'+\deg C\land C'\leq \deg(AC'-A'C)
\]
et par le résultat précédent \[
    \deg A+\deg B+\deg C \leq \underbrace{AB'-A'B}_{\leq \deg A+\deg B-1}+ \underbrace{n_0(A)+n_0(B)+n_0(C)}_{n_0(ABC)}
\]
\end{proof}

\begin{thm}[Fermat]
    \index{Fermat!grand théorème (version polynômiale)}
    Si $A, B, C\in\mathbb C[X]$ non constants, $A$ et $B$ premiers entre eux et $A^n+B^n=C^n$ alors $n\leq 2$
\end{thm}

\begin{proof}
On note $p=\max (\deg A, \deg B, \deg C)$ et \[
    np=\max(\deg A^n, \deg B^n, \deg C^n)< n_0(A^nB^nC^n)=n_0(ABC)\leq 3p \implies n<3
\]
\end{proof}

\subsection{Résultant}

\index{résultant}

On note $P=a_pX^p+\cdots + a_0$, $Q=b_qX^q+\cdots + b_0$ de degrés respectifs $p, q$.

On considère l'application linéaire \[
    \varphi: (U, V)\in\mathbb K_{q-1}[X]\times \mathbb K_{p-1}[X]\longmapsto UP+VQ
\]
Si $P\land Q=1$, alors pour $U, V\in\mathrm{Ker}\varphi$, on a $Q\;|\;U$ et $P\;|\;V$ donc $U=V=0$ en comparant les degrés. Sinon, si $P\land Q=\Delta$ non constant, alors $\varphi(Q/\Delta, -P/\Delta)=0$. Ainsi, $\varphi$ est un isomorphisme si et seulement si $P\land Q=1$

La matrice de $\varphi$ dans la base $(X^{q-1}, 0), \cdots, (X, 0), (1, 0), (0, X^{p-1}), \cdots, (0, 1)$ est \[
% Attention pas la même base que les notes du prof ! Comme ça le copier/collé de wikipedia marche bien :-)
\begin{pmatrix}a_{p}&0&\cdots &0&b_{q}&0&\cdots &0\\a_{p-1}&a_{p}&\ddots &\vdots &\vdots &b_{q}&\ddots &\vdots \\\vdots &a_{p-1}&\ddots &0&\vdots &&\ddots &0\\\vdots &\vdots &\ddots &a_{p}&b_{1}&&&b_{q}\\a_{0}&&&a_{p-1}&b_{0}&\ddots &\vdots &\vdots \\0&\ddots &&\vdots &0&\ddots &b_{1}&\vdots \\\vdots &\ddots &a_{0}&\vdots &\vdots &\ddots &b_{0}&b_{1}\\0&\cdots &0&a_{0}&0&\cdots &0&b_{0}\\\end{pmatrix}\defeq R(P, Q)
\]
On appelle résultant de $P$ et $Q$ le déterminant de cette matrice. Ainsi, \[
    P\land Q\neq 1 \iff \det R(P, Q)=0
\]

\section{Polynômes irréductibles}

\index{polynôme!irréductible}

\subsection{Contenu de Gauss}

\index{Gauss!contenu}

On note $P(X)=a_nX^n+\cdots + a_0\in\mathbb Z[X]$ de degré $n\geq 1$. On appelle contenu de $P$ le nombre \[
    a_n\land\cdots\land a_0
\]

\begin{dfn}
    $P\in\mathbb Z[X]$ est dit primitif\index{polynôme!primitif} si $C(P)=1$
\end{dfn}

On va montrer que si $P, Q\in\mathbb Z[X]$ non constants alors \[
    C(PQ)=C(P)C(Q)
\]

On commence par supposer que $P$ et $Q$ sont primitifs et on suppose par l'absurde $C(PQ)\neq 1$. On note $p$ premier diviseur de $C(PQ)$. On note $i_0$ (resp $j_0$) le plus petit indice tel que $p$ ne divise pas $a_{i_0}$ (resp. $b_0$). \[
    P(X)=a_nX^n+\cdots +a_{i_0}X^{i_0}+\underbrace{\cdots}_{\text{divisible par } p} \]\[
    Q(X)=b_nX^n+\cdots +b_{j_0}X^{j_0}+\underbrace{\cdots}_{\text{divisible par } p}
\]
Alors, \[
    [PQ(X)]_{i_0+j_0}=a_{i_0}b_{j_0}+pk \quad \text{non divisible par $p$}
\]
Or c'est un coef de $PQ$ donc divisible par $p$, absurde.

Dans le cas général, \[
    C\left(\frac{PQ}{C(P)C(Q)}\right)=1=\frac{C(PQ)}{C(P)C(Q)}
\]

\begin{res}
    $P\in\mathbb Z[X]$ primitif est irréductible dans $\mathbb Z[X]$ si et seulement s'il l'est dans $\mathbb Q[X]$
\end{res}
\begin{proof}
Si le polynôme est irréductible dans $\mathbb Q[X]$ alors il l'est dans $\mathbb Z[X]$. On suppose $P$ irréductible dans $\mathbb Z[X]$ et $P=AB$ dans $\mathbb Q[X]$. Il existe $p, q$ minimaux positifs tels que $pqP=pAqB$, $pA$ et $qB$ dans $\mathbb Z[X]$ et \[
    C(pqP)=pqC(P)=pq=C(pA)C(qB) \implies P=\frac{pA}{C(pA)}\times \frac{qB}{C(qB)}
\]
absurde (les deux facteurs sont dans $\mathbb Z[X]$).
\end{proof}

\subsection{Critère d'Eisenstein}

\index{Eisenstein (critère de -- )}

\begin{res}[Critère d'Eisenstein]
On note $P=p_nX^n+\cdots + p_0\in\mathbb Z[X]$ non constant de degré $n$. Si $p$ est premier tel que \[
    \begin{cases}
    p\;|\;p_0,\cdots, p_{n-1}\\ p\;\not|\;p_n\\p^2\;\not|\;p_0
    \end{cases}
\]
alors $P$ est irréductible dans $\mathbb Q[X]$
\end{res}
\begin{proof}
On suppose par l'absurde que $P$ n'est pas irréductible dans $\mathbb Q[X]$ donc dans $\mathbb Z[X]$. On écrit alors $P=AB$ avec $A$ et $B$ non constants à coefficients entiers. On a \[
    p\;|\;p_0=a_0b_0
\]
donc par exemple $p$ divise $a_0$ mais pas $b_0$ (il ne peut pas diviser les deux par hypothèse). Si $p$ divise tous les $a_i$, alors $p\;|\;\dom A\times \dom B=p_n$ absurde. On note alors $i_0$ le plus petit $i$ tel que $p\;\not |\;a_i$. $B$ est non constant donc $i_0<n$. \[
    p\;|\; p_{i_0}=a_{i_0}b_0+\underbrace{a_{i_0-1}b_1+\cdots +a_0b_{i_0}}_{\text{divisible par }p} \qquad \text{non divisible par }p
\]
absurde d'où la conclusion.
\end{proof}

\begin{csq}
Pour $p$ premier, on considère \[
    \mu_p(X)=X^{p-1}+\cdots+X+1
\]
Alors, \[
    \mu_p\text{ irréductible }\iff \mu_p(X+1)\text{ irréductible } \iff \frac{(X+1)^p-1}{(X+1)-1}=\sum_{k=1}^p\binom pk X^{k-1}\text{ irréductible}
\]
vrai par le critère d'Eisenstein
\end{csq}

\begin{csq}
Pour tout $n\geq 1$, le polynôme \[
    S_n(X)=1+X+\frac{X}{2!}+\cdots +\frac{X^n}{n!}
\]
est irréductible. En effet, c'est équivalent à vérifier que \[
    n!S_n(X)=X^n+nX^{n-1}+\cdots+n!
\]
est irréductible dans $\mathbb Z[X]$.

Si $n=2m$, alors il existe $p$ premier tel que $m<p<2m$ donc $p<n<2p$ et on conclut avec Einsentein. Sinon, si $n=2m+1$ alors il existe $p$ premier tel que $m<p<2m$ et $p<2m<n$ et $2m\leq 2p-1$ donc $n\leq 2p-1<2p$ et on conclut de même.
\end{csq}

\subsection{{$\mathbb Z[X]$} vs {$\mathbb Z_p[X]$}}

On va montrer que $X^4+1$ est irréductible dans $\mathbb Z[X]$ et n'est jamais irréductible dans $\mathbb Z_p[X]$. Dans $\mathbb C[X]$, \[
    X^4+1=\left(X-e^{i\frac\pi4}\right)\left(X-ie^{i\frac\pi4}\right)\left(X+e^{i\frac\pi4}\right)\left(X+ie^{i\frac\pi4}\right)
\]
Aucun des facteurs ni des produits n'est dans $\mathbb Q[X]$ donc $X^4+1$ est irréductible dans $\mathbb Q[X]$ donc dans $\mathbb Z[X]$.

Pour $p=2$, $(X^4+1)=(X+1)^4$ n'est pas irréductible. On suppose donc $p$ premier impair et on observe que \begin{align*}
    X^4+1&=(X^2)^2-(-1)\\&=(X^2+1)^2-2X^2\\&=(X^2-1)^2-(-2X^2)
\end{align*}
Il est donc suffisant que $-1, 2$ ou $-2$ soit un carré de $\mathbb Z_p$. Si $-1$ et $2$ ne sont pas des carrés, alors $-2$ est un carré. En effet, si on note $\mathcal C^\star$ l'ensemble des carrés non nuls, alors \[
    \varphi: x\in\mathbb Z_p^\star \longmapsto x^2\in\mathcal C^\star
\]
est un morphisme surjectif de noyeau $\ker \varphi=\{\pm1\}$ donc $\#\mathcal C^\star=\frac{p-1}2$. \[
    X^{\frac{p-1}2}-1
\]
a au plus $\frac{p-1}2$ racines ($\mathbb Z_p$ est un corps) et s'annule sur $\mathcal C^\star$ donc $\mathcal C^\star$ est exactement l'ensemble des racines de ce polynôme (le petit théorème de fermat donne $x^{p-1}=1$ pour tout $x$ non nul). Si $-1$ et $2$ ne sont pas carrés, alors \[
    (-2)^{\frac{p-1}2}=(-1)^{\frac{p-1}2}\times (2)^{\frac{p-1}2}=-1\times -1=1
\]
donc $-2$ carré.

\section{Polynômes cyclotomiques (HP)}

\index{polynôme!cyclotomique}

\begin{dfn}
Soit $n\geq 1$. \begin{itemize}
    \item On appelle racine primitive $n$-ième de l'unité tout générateur de $\mathbb U_n$. On note $\mathbb P_n$ ces racines.
    \item Il y a $\varphi(n)$ générateurs de $\mathbb U_n$: les $w^l$ avec $l\land n=1$ et $w$ un générateur.
    \item On appelle $n$-ième polynôme cyclotomique le polynôme \[
        \mu_n(X)=\prod_{u\in\mathbb P_n}(X-u)
    \]
\end{itemize}
\end{dfn}

\begin{prop}
Pour tout $n\in\mathbb N^\star$, \[
    X^n-1=\prod_{d\;|\;n}\mu_d(X)
\]
\end{prop}

\begin{proof}
Pour $d\;|\;n$, on note $E_d=\{u\in\mathbb U_n, \ord u=d\}$, donc \[
    \mathbb U_n =\bigsqcup_{d\;|\;n}E_d
\]
ainsi \[
    X^n-1=\prod_{d\;|\; n}\prod_{u\in E_d}(X-u)=\prod_{d\;|\;n}\mu_d(X)
\]
car $E_d=\mathbb P_d$ (les éléments d'ordre $d$ sont au nombre de $\varphi(d)$ et engendrent tous $\mathbb U_d$).
\end{proof}

\begin{rem}
Si $n=p$ est premier alors \[
    \mu_n(X)=\frac{X^p-1}{X-1}=X^{p-1}+\cdots +1
\]
\end{rem}

\begin{prop}
Pour tout $n\geq 1$, $\mu_n$ est unitaire à coefficients entiers, de degré $\varphi(n)$.
\end{prop}

\begin{proof}
On raisonne par récurrence sur $n$ pour le caractère entier, le reste est immédiat.
\begin{itemize}
    \item $n=1$ immédiat
    \item On se donne $n\geq 1$ et on suppose la propriété vraie jusqu'au rang $n$. Si $n+1$ est premier, la conclusion est immédiate. Sinon, \[
        X^{n+1}-1=\prod_{d\;|\;n+1}\mu_d(X)=\mu_{n+1}(X)\times
        \underbrace{\prod_{\substack{d\;|\;n+1\\ d\neq n+1}}\mu_d(X)}_{\in\mathbb Z[X], \text{ unitaire }}
    \]
    Donc $\mu_{n+1}$ est le quotient dans la division euclidienne de $X^{n+1}-1$ par un polynôme \textbf{unitaire} de $\mathbb Z[X]$, d'où la conclusion.
\end{itemize}
\end{proof}

\subsection{Expression de $\mu_n$}

On introduit la fonction arithmétique de Möbius\index{Möbius!fonction arithmétique} \[
    \mu(n)=\begin{cases}
        1 &\text{si } n=1 \\
        (-1)^k &\text{si }n\text{ produit de $k$ premiers distincts}\\
        0 &\text{sinon}
    \end{cases}
\]
Cette fonction est multiplicative: pour $a, b$ premiers entre eux, $\mu(ab)=\mu(a)\mu(b)$. Puis, pour $n\geq 1$, \[
    \sum_{d\;|\;n}\mu(d)=\begin{cases}
        1&\text{si }n=1 \\
        0&\text{sinon}
    \end{cases}
\]
En effet, si $n=p_1^{\alpha_1}\cdots p_k^{\alpha_k}$, \[
    \sum_{d\;|\;n}\mu(d)=\sum_{\substack{d\;|\; n\\\mu(d)\neq 0}}=\sum_{I\subset \llbracket 1, k\rrbracket}(-1)^{\#I}=\sum_{i=0}^k\binom ki(-1)^i=(1-1)^k=0
\]

\begin{res}
\[
    \mu_n(X)=\prod_{d\;|\;n}\left(X^d-1\right)^{\mu\left(\frac nd\right)}
\]
\end{res}

\begin{proof}
\begin{align*}
    \prod_{d\;|\; n}\left(X^d-1\right)^{\mu\left(\frac nd\right)}&=\prod_{d\;|\;n}\;\prod_{d'\;|\;d}\mu_{d'}(X)^{\mu\left(\frac nd\right)} \\
    &= \prod_{d'\;|\; n}\;\prod_{u\;|\;\frac n{d'}}\mu_{d'}(X)^{\mu\left(\frac n{d'u}\right)} \\
    &= \prod_{d'\;|\;n}\mu_{d'}(X)^{\left(\displaystyle\sum_{u|\frac n{d'}}\mu\left(\frac n{d'u}\right)\right)}\\
    &=\mu_n(X)
\end{align*}
car \[
\sum_{u\;|\;\frac n{d'}}\mu\left(\frac n{d'u}\right)=\sum_{u\;|\;\frac n{d'}}\mu(u)=\begin{cases}
    1& \text{si }\frac n{d'}=1\iff d'=n \\
    0& \text{sinon}
\end{cases}
\]
\end{proof}

\begin{rem}
Cela donne une autre preuve de $\mu_n(X)\in\mathbb Z[X]$
\end{rem}

\subsection{Suites arithmétiques}

On note $p$ un nombre premier et $m\geq 1$ un entier non divisible par $p$. Supposons que $a$ est un entier et que \[
    \mu_m(a)= 0\pmod p
\]
On a dans ce cas \[
    p\;|\;\mu_m(a)\;|\;a^m-1 \text{ premier avec }a \implies p\;\not|\;a
\]

Si $d=\ord a$ alors $d\;|\; m$ donc $X^d-1|X^m-1$. Par l'absurde, si $d\neq n$ alors $X^d-1\land \mu_m=1$ (aucune racine commune dans $\mathbb C$) donc (Gauss) $(X^d-1)\times \mu_m(X)|X^m-1$. On en déduit ($d$ est l'ordre de $a$) \[
    (a^d-1)\mu_m(a)\;|\;a^m-1 \implies a^m-1=0\pmod {p^2}
\]
et comme $a+p$ vérifie les mêmes hypothèses ($a+p\equiv a\pmod p$ donc $\mu_m(a+p)=0\pmod {p}$ et l'ordre est le même donc $\neq m$), on a $(a+p)^m-1= 0\pmod {p^2}$, d'où \[
    p^2\;|\; (a+p)^m-a^m=p\left(a^{m-1}+\underbrace{\cdots}_{\text{multiple de }p}\right) \implies p\;|\; a
\]
c'est absurde, donc $\ord a = m$. Puis $m=\ord a\;|\;p-1$ donc $p$ est un terme de la suite $(1+km)_{k\in\mathbb N}$. Si cette suite n'a qu'un nombre fini de nombres premiers $p_1, \cdots, p_s$ alors pour $k$ assez grand, $\mu_m(kp_1\cdots p_s)>2$. On a aussi $\mu_m(0)=\pm 1$ (entier de module $1$) donc si $p$ est un premier qui divise $\mu_m(kp_1\cdots p_s)$ et $p$ est parmi les $p_i$ alors $p\;|\; \mu_m(0)$ absurde, donc $p$ n'est pas parmi les $p_i$ et la suite a une infinité de termes premiers.

\subsection{Irréductibilité}

\index{polynôme!irréductible}

Nous allons montrer que les polynômes cyclotomiques sont irréductibles par un argument de Landau.

On note $\alpha$ une racine primirive $n$-ième de l'unité et $P$ son polynôme minimal. La suite $(P(\alpha^k))_{k\in\mathbb N}$ est $n$-périodique à valeurs dasn $\mathbb Z[\alpha]$ et chacun des termes s'écrit  de manière unique $Q_k(\alpha)$ pour un polynôme $Q_k$ de degré $<\deg P$.

On note $A$ le plus grand coefficient de tous les $Q_k$ en valeurs absolues. Dans $\mathbb Z_p[X]$, $P(X^p)=P(X)^p$ (Frobenius) donc il existe $U\in\mathbb Z[X]$ tel que \[
    P(X^p)=P(X)^p+pU(X)
\]
Pour $p>A$, \[
    P(\alpha^p)=P(\alpha)^p+pU(\alpha)=pU(\alpha)=pV(\alpha) \quad \text{ avec }\deg V<\deg P
\]
donc \[
    Q_p(\alpha)=pV(\alpha) \qquad \text{ et }\qquad P\;|Q_p-pV
\]
et donc \[
    Q_p=pV
\]
d'où $p\;|\;$ les coefs de $Q_p$ qui sont de module $\leq A<p$ donc $Q_p=0$ et $P(\alpha^p)=0$. Ici, on s'est seulement servis du fait que $\alpha$ est une racine de l'unité, donc si $p_1, \cdots, p_r$ sont $>A$ alors $P(\alpha^{p_1\cdots p_r})=0$.

On note $m$ premier avec $n$, $N$ le produit des premiers $\leq A$. On a $m+nN=m\pmod n$ et si $p$ divise $m+nN$ alors $p>A$ donc $P(\alpha^{m+nN})=0=P(\alpha^m)$

Ainsi les $(\alpha^{m})$ avec $m\land n=1$ sont racines de $P$ donc $\mu_n(X) \;|\; P(X)$. $\mu_n(\alpha)=0$ donc $P$ divise $\mu_m$, donc $P=\mu_m$. C'est un polynôme minimal donc irréductible.

\section{Nombres algébriques}
\index{nombres algébriques}

Si $K\subset L$ sont deux corps alors on dit que $a\in L$ est algébrique sur $K$ s'il existe $P\in\mathbb K[X]$ non nul tel que $P(a)=0$.

Dans $\mathbb C$, on appelle nombre algébrique les nombres algébriques sur $\mathbb Q$ et on note $A$ l'ensemble de ces nombres.

\begin{rem}
Si $a$ est algébrique sur un corps $K$ alors \[
    I_a=\{P\in\mathbb K[X]\;/\; P(a)=0\}
\]
est un idéal non réduit à $\{0\}$ de $K[X]$ euclidien donc principal, donc il existe un unique polynôme unitaire $\mu_a(X)$ tel que $I_a=\mu_a K[X]$, que l'on appelle polynôme minimal de $a$. Dans ce cas \[
    \Vect_Kr((a^k)_{k\in\mathbb N})=\Vect_K(1, a, \cdots, a^{\deg \mu_a-1})=\{P(a), P\in K[X]\}
\]
\end{rem}

\subsection{Structure de corps}
\index{nombres algébriques!structure}

\begin{res}
$A$ est un corps
\end{res}

\begin{proof}
\begin{itemize}
    \item $A\subset \mathbb C$
    \item $1\in A$ en tant que racine de $X-1$
    \item Soient $x, y\in A$, $p=\deg \mu_x, q=\deg \mu_y$. Pour tout $n\in\mathbb N$, \[
        \begin{cases}
            x^n\in\Vect_{\mathbb Q}(1, x, \cdots, x^{p-1}) \\ y^n\in\Vect_{\mathbb Q}(1, y, \cdots, y^{q-1})
        \end{cases}\quad\text{ donc }\quad  x^ny^n, (x+y)^n\in\Vect_{\mathbb Q}(x^iy^j, i<q, j<q)
    \]

    Les familles $((x-y)^n)_n$ et $((xy)^n)_n$ sont liées donc $x-y, xy\in A$.
    \item Si $x\in A$ non nul et si on note $E=\Vect_{\mathbb Q}(1, x, \cdots, x^{p-1})$ alors \[
        \varphi: z\in E\longmapsto xz\in E
    \]
    est un endomorphisme injectif en dimension finie donc c'est un isomorphisme et $\varphi^{-1}(1)$ donne $x^{-1}\in E\subset A$
\end{itemize}
\end{proof}

\subsection{Règle de multiplicativité des degrés}

\index{multiplicativité des degrés}

On note $a$ algébrique sur $K$ de degré $d$. \[
    \varphi: P\in K[X] \longmapsto P(a)
\]
est un morphisme de $K$-algèbres donc $\Img\varphi=\Vect_K(1, \cdots, a^{p-1})$ est une $K$-algèbre, notée $K[a]$.

$K[d]$ est une algèbre de dimension finie $d$ sur $K$ et pour $x\in K[a]\setminus \{0\}$, \[
    \psi: y\in K[a]\longmapsto xy\in K[a]
\]
est un endomorphime injectif en dimension finie donc c'est un automorphisme et $x$ a un inverse dans $K[a]$, qui est donc un corps.

\begin{dfn}
Si $K\subset L$ alors $L$ est un $K$-ev et si $L$ est de dimension finie sur $K$ alors on note \[
    [L:K]=\dim_K(L)
\]
\end{dfn}

\begin{prop}
    \index{nombres algébriques!transitivité}
\begin{enumerate}
    \item Soit $K\subset L\subset N$ trois corps. Si $\dim_L N, \dim_K L<+\infty$ alors $\dim_K N<+\infty$ et \[
        [N:K]=[N:L]\times [L:K]
    \]
    \item Si $a\in\mathbb C$ est algébrique et si $b\in K$ est algébrique sur $\mathbb Q[a]$ alors $b$ est algébrique sur $\mathbb Q$.
\end{enumerate}
\end{prop}

\begin{proof} ~
\begin{enumerate}
    \item $(n_1, \cdots, n_p)$ base de $N$ sur $L$, $(\ell_1, \cdots, \ell_q)$ base de $L$ sur $K$. Si $n\in\mathbb N$ alors il existe $\lambda_1, \cdots, \lambda_p\in L$ tel que $n=\lambda_1n_1+\cdots+\lambda_p n_p$ et chaque $\lambda_i\in\Vect_{\mathbb Q}(\ell_1, \cdots, \ell_p)$ donc $n\in\Vect_{\mathbb Q}(\ell_in_j)$ (famille génératrice finie) donc \[
        [N:K]<+\infty
    \]
    Vérifions que $(\ell_in_j)_{\substack{1\leq i\leq q\\ 1\leq j\leq p}}$ est libre sur $\mathbb Q$. \[
        \sum_{i, j}\lambda_{i, j}\ell_in_j=0\implies \sum_{j=1}^pn_j\sum_{i=1}^q\lambda_{i, j}\ell_i=0 \implies \forall j\in\llbracket 1, p\rrbracket \sum_{i=1}^q\lambda_{i, j}\ell_i=0\implies \forall i, j, \lambda_{i, j}=0
    \]
    \item C'est une conséquence directe de $1.$
\end{enumerate}
\end{proof}

\begin{rem}
Si $b$ est racine d'une équation polynômiale à coefficients dans $\mathbb Q[a]$, il l'est aussi d'une équation polynômiale à coefficients dans $\mathbb Q$.
\end{rem}

\section{Polynômes orthogonaux}
\index{polynômes orthogonaux}

On note $p: I\longrightarrow \mathbb R_+^\star$ continue telle que pour tout entier naturel $k$, $t\longmapsto t^kp(t)$ est intégrable sur $I$. Dans ce cas, \[
    (P|Q)=\int_Ip(t)P(t)Q(t)\;\mathrm dt
\]
est bien définit et est un produit scalaire sur $\mathbb R[X]$

\begin{prop}
Il existe une unique famille $(P_k)_k$ de polynômes unitaires tels que
\begin{itemize}
    \item $\forall k\in\mathbb N, \quad \deg P_k=k$
    \item les $P_k$ sont deux à deux orthogonaux
\end{itemize}
\end{prop}

\begin{proof}
    Le procédé de Gram-Schmidt appliqué à la base canonique donne l'existence. Si $(P_k)$ convient alors
\begin{itemize}
    \item $P_0=1$ (unitaire de degré $0$)
    \item Si $P_0, \cdots, P_k$ conviennent, alors \[
        P_{k+1}=\underbrace{X^{k+1}}_{unitaire} + \underbrace{\lambda_0P_0+\cdots +\lambda_k P_k}_{\deg \leq k \text{ se décompose sur }P_0, \cdots, P_k}
    \]
et l'orthogonalité détermine de manière unique les $\lambda_i$
\end{itemize}
\end{proof}

\subsection{Récurrence d'ordre 2}

\begin{prop}
La famille des $(P_n)_n$ vérifie pour $n\geq 2$ \[
    P_n(X)=-(X+\lambda_{n-1})P_{n-1}(X)-\mu_{n-2}P_{n-2}(X)=0
\]
avec \[
    \lambda_{n-1}=-\frac{(XP_{n-1}|P_{n-1})}{\|P_{n-1}\|^2}, \quad \mu_{n-2}=-\frac{\|P_{n-1}\|^2}{\|P_{n-2}\|^2}
\]
\end{prop}

\begin{proof} On note $E_i=\mathbb R_i[X]$.
Pour $Q\in\Vect(P_0, \cdots, P_{n-2})=\mathbb R_{n-2}[X]$, $(XP_n|Q)=(P_n|XQ)=0$ car $XQ\in\mathbb R_{n-1}[X]$. On a alors \[
    P_n(X)-XP_{n-1}(X)\in E_{n-3}^\bot
\]
or $\deg (P_n-XP_{n-1})\leq n-1$ (simplification du coef dominant) donc \[
    P_n(X)-XP_{n-1}(X)=\lambda_{n-1}P_{n-1}(X)+\cdots +\lambda_0P_0(X)
\]
et on a $P_n-XP_{n-1}\bot P_0, \cdots, P_{n-3}$ donc $\lambda_0=\cdots =\lambda_{n-3}=0$. On en déduit, \[
    \lambda_{n-1}(P_{n-1}|P_{n-1})=(P_n|P_{n-1})-(XP_{n-1}|P_{n-1})
\]
et \[
    \lambda_{n-2}(P_{n-2}|P_{n-2})=\underbrace{(P_n|P_{n-2})}_{=0}-(XP_{n-1}|P_{n-2})=-(P_{n-1}|\underbrace{XP_{n-2}}_{P_{n-1}+\in E_{n-2}})=-(P_{n-1}|P_{n-1})
\]
d'où la conclusion avec $\mu_{n-2}=\lambda_{n-2}$
\end{proof}

\subsection{Changements de signes}

On appelle changement de signe d'un polynôme un point $x_0$ tel que $(x-x_0)P(x)$ est de signe constant au voisinage de $x_0$. On note $N_I(P)$ le nombre de changements de signes de $P$ sur l'intervalle $I$

Supposons $N_I(P_n)=p<n$. On note $x_1, \cdots, x_p$ les changements de signes et \[
    (P_n | \underbrace{(X-x_1)\cdots (X-x_p)}_{\in E_{n-1}})=0=\int_Ip(t)\underbrace{P_n(t)(t-x_1)\cdots (t-x_p)}_{\text{signe constant}}\;\mathrm dt
\]
donc \[
    \forall t\in I, p(t)P_n(t)(t-x_1)\cdots (t-x_p)=0
\]
et $P_n$=0 absurde. Donc $N_I(P_n)=n$ et $P_n$ est scindé à racines simples, toutes dans $I$.

\subsection{Entrelacement des racines}

On note \[
    w_n(x)=\left|\begin{matrix}P_n(x) & P_{n+1}(x)\\ P_n'(x) & P_{n+1}'(x)\end{matrix}\right|=P_n(x)P_{n+1}'(x)-P_n'(x)P_{n+1}(x)
\]
On a $w_0\equiv 1$ et après calcul avec la relation de récurrence, \[
    w_n(x)=P_n(x)^2\underbrace{-\mu_{n-1}}_{>0}w_{n-1}(x)
\]
donc comme $w_0>0$, on en déduit que pour tout $x, n$, $w_n(x)>0$. Notons $\alpha < \beta$ deux racines consécutives de $P_{n+1}$. \[
    P_{n+1}'(\alpha)P_n(\alpha)>0\qquad \text{ et }\qquad P_{n+1}'(\beta)P_n(\beta)>0
\] or $P_{n+1}$ de signe constant sur $]\alpha, \beta[$ et donc $P_{n+1}'(\alpha)P_{n+1}'(\beta)<0$ (car sinon il y aurait changement de signe). Ainsi, $P_n(\alpha)P_n(\beta)<0$ donc $P_n$ s'annule sur $]\alpha, \beta[$. On a ainsi trouvé $n$ racines de $P_n$, une entre chaque racine de $P_{n+1}$: il y a entrelacement des racines

\section{Formules d'interpolation}

\index{Lagrange!interpolation}

\subsection{Rappels}

On considère $x_1, \cdots, x_n\in\mathbb K$ deux à deux distincts, et on note \[
    \Delta_i(X)=\prod_{j\neq i}\frac{X-x_j}{x_i-x_j}
\]
de telle manière que \[
    \Delta_i(x_j)=\delta_{i, j}
\]

\begin{thm}
    On considère $x_1, \cdots, x_n, y_1, \cdots, y_n\in\mathbb K$, les $x_i$ deux à deux distincts. Dans ce cas, il existe un unique polynôme $P\in\mathbb K_{n-1}[X]$ tel que \[\forall i\in\llbracket 1, n\rrbracket, \quad P(x_i)=y_i\]
    De plus, \[
        P(X)=\sum_{i=1}^ny_i\Delta_i(X)
    \]
\end{thm}

\begin{proof}
Le polynôme donné convient. Si $P$ et $Q$ conviennent, $P-Q$ a $n$ racines et est de degré au plus $n-1$ donc $P=Q$.
\end{proof}

\begin{rem}
Si on note \[
    \omega (x)=\prod_{j=1}^n(x-x_j)
\]
alors \[
    \prod_{j\neq i}(x_i-x_j)=\lim_{x\to x_i}\frac{\omega(x)-\omega(x_i)}{x-x_i}=w'(x_i)
\]
et \[
    \Delta_i(x)=\frac{\omega (x)}{\omega(x_i)(x-x_i)}
\]
donc \[
    P(x)=\underbrace{\left(\sum_{i=1}^n\frac{y_i}{\omega'(x_i)(x-x_i)}\right)}_{\text{DSE de }\frac{P(x)}{\omega(x)}}\omega(x)
\]
\end{rem}

\subsection{Formules d'ordre supérieur}

On se donne $n_1, \cdots, n_r\in\mathbb N^\star$, $x_1, \cdots, x_r\in\mathbb K$ deux à deux distincts et $n=n_1+\cdots+n_r$. \[
    \varphi: P\in\mathbb K_{n-1}[X]\longmapsto(P(x_1), \cdots, P^{(n_1-1)}(x_1), \cdot, P(x_r), \cdots, P^{(n_r-1)})\in\mathbb K^n
\]
est une application linéaire injective, il y a égalité des dimensions donc c'est un isomorphisme et on en déduit que pour $(y_{i, k})_{i\leq r, j\leq n_i}$ donné, il existe un unique polynôme de degré au plus $n-1$ tel que \[
    \forall i, k, P^{(k)}(x_i)=y_{i, k}
\]

\subsection{Analyse de l'erreur}

On se donne $f\in\mathbb C^n([a, b], \mathbb R) ,x_1, \cdots, x_n\in [a, b]$ deux à deux distincts. \[
    P(x)=\sum_{i=1}^nf (x_i)\Delta_i(x)
\]
l'interpolation de Lagrange de $f$. On veut estimer $f(x)-P(x)$. On note $x\in [a, b]\setminus\{x_1, \cdots, x_r\}$ et on note \[
    g(t)=f(t)-P(t)-\lambda_x\omega(t)
\]
où $\lambda_x$ tel que $g(x)=0$ (ajout d'une racine supplémentaire). Rolle itéré donne:

\begin{itemize}
    \item
$g$ s'annule en $x_1, \cdots, x_n, x$ ($n+1$ points)
\item $\cdots$
\item
$g^{(n)}$ s'annule au moins une fois en $c_n$
\end{itemize}

\[
    0=f^{(n)}(c_n)-P^{(n)}(c_n)-\lambda_x n!
\]
donc $\lambda_x = \dfrac{f^{(n)}(c_n)}{n!}$ et donc \[
    |f(x)-P(x)|=|\lambda_x\omega (x)|\leq |\omega(x)|\sup_{[a, b]}\frac{|f^{(n)}|}{n!}
\]
cette inégalité est encore vraie pour $x\in\{x_1, \cdots, x_n\}$

\begin{rem}
On constate que l'erreur est meilleure si \[
    |\omega(x)|=|(x-x_1)\cdots (x-x_n)|
\]
est petit. On peut donc s'intéresser aux choix des $x_i$ minimisant $\|\omega\|_\infty$ sur $[a, b]$. C'est un problème difficile, mais le choix de $x_1, \cdots, x_n$ équirépartis est mauvais.
\end{rem}

\section{Suites de polynômes}

\begin{thm}[Weierstrass]
    \index{Weierstrass (théorème de -- )}
    Si $f$ est continue de $[a, b]$ dans $\mathbb C$ alors il existe une suite $(P_n)$ de fonctions polynômiales telle que \[
        P_n\xrightarrow[{[a, b]}]{CVU} f
    \]
\end{thm}

\begin{proof}[Idée de la preuve]
    Bernstein puis convolution
\end{proof}

\begin{rem}
Le résultat est faux sur un intervalle autre qu'un segment.
\end{rem}

\subsection{Inégalité de Bernstein}

\index{Bernstein!inégalité}

\begin{lmm}
Si $z_1, \cdots, z_n$ sont les racines du polynôme $z^n+1$ et $P$ polynôme réel de degré $n$, alors pour tout $t\in\mathbb C$, \[
    tP'(t)=\frac n2P(t)+\frac 1n\sum_{k=1}^nP(-z_k)\frac{2z_k}{(z_k-1)^2}
\]
\end{lmm}

\begin{proof}
On note $\varphi(z, t)=\frac{P(tz)-P(t)}{z-1}$ pour $z\neq 1$. C'est une fonction polynomiale de degré $\leq n-1$ pour $t$ fixé.

D'après la formule d'interpolation de Lagrange en $z_1, \cdots, z_n$, \begin{align*}
    \varphi(z, t)&=\sum_{k=1}^n\varphi(z_k, t)\frac{z^n+1}{z-z_k} \times \underbrace{\frac1{(z_k-z_1)\cdots \widehat{(z_k-z_k)}\cdots (z_k-z_n)}}_{=\lim_{z\to z_k}\limits \frac{z-z_k}{z^n+1}=\frac1{nz_k^{n-1}}=-\frac{z_k}n} \\ &= \sum_{k=1}^n-\varphi(z_k, t)\times \frac{z^n+1}{z-z_k}\frac{z_k}n
\end{align*}

On fait tendre $z$ vers $1$. \[
    tP'(t)=\sum_{k=1}^n\varphi(z_k, t)\frac{2}{z_k-1}\frac{z_k}n=
    \frac1n\sum_{k=1}^nP(-tz_k)\frac{2z_k}{(z_k-1)^2}-\frac{P(t)}{n}\sum_{k=1}^n\frac{2z_k}{(z_k-1)^2}
\]
Il nous reste à évaluer ce dernier terme. On reprend la formule précédente avec $P(t)=t^n$. \[
    nt^n=\frac1n\sum_{k=1}^n2t^n\frac{z_k^{n+1}}{(z_k-1)^2}-\frac{t^n}{n}\sum_{k=1}^n\frac{2z_k}{(z_k-1)^2}\quad \underset{z_k^{n+1}=-z_k}{=}\quad -\frac{2t^n}n\sum_{k=1}^n\frac{2z_k}{(z_k-1)^2}
\]
Et alors,
\[
\sum_{k=1}^n\frac{2z_k}{(z_k-1)^2}=-\frac{n^2}2
\]
ce qui termine la preuve de la formule.
\end{proof}

\begin{res}
On note pour $P\in\mathbb R[X]$, $\|P\|=\sup_{|z|=1}\limits|P(z)|$. Si $P$ est de degré $n$ alors \[
    \|P'\|\leq n\|P\|
\]
\end{res}

\begin{proof}
On prend $t$ de module $1$, \[
    |tP'(t)|=|P'(t)|\leq \frac n2|P(t)|+\frac1n\sum_{k=1}^n\left|\frac{2z_k}{(z_k-1)^2}\right|\times \|P\|
\]
Et $z_k\in\mathbb U$ donc s'écrit $z=e^{i\theta}$ et donc ($\theta\in ]0, 2\pi[$) \[
    \frac{2z_k}{(z_k-1)^2}=\frac{2}{(e^{i\frac \theta 2}-e^{-i\frac \theta2})^2}=\frac 2{-4\sin^2\frac\theta 2}<0
\]
et \[
    \sum_{k=1}^n\left|\frac{2z_k}{(z_k-1)^2}\right|=-\sum_{k=1}^n\frac{2z_k}{(z_k-1)^2}=\frac{n^2}2
\]
ce qui termine la preuve.
\end{proof}

\subsection{Théorème d'oscillation de Tchebychev}

\index{Tchebychev!théorème d'oscillation}
\index{Tchebychev!polynômes}

On note $C_n$ le $n$-ième polynôme de Tchebychev et $T_n=\dfrac{C_n}{2^{n-1}}$ de sorte que $T_n$ est unitaire.

\begin{res}
Pour tout $P$ unitaire de degré $n$, \[
    \frac 1{2^{n-1}}=\max_{[-1, 1]}|T_n|\leq \max_{[-1, 1]}|P|
\]
\end{res}

\begin{proof}
\[
    T_n(\cos \theta)=\frac1{2^{n-1}}\cos (n\theta) \implies \max_{[-1, 1]}|T_n|=\frac1{2^{n-1}}
\]
Supposons par l'absurde que $\displaystyle\max_{[-1, 1]}|P|<\frac1{2^{n-1}}$ et posons \[
    Q(X)=T_n(X)-P(X)
\]
Pour $x_k=\cos\left(\frac{k\pi}n\right), k\in\llbracket 0, n\rrbracket$, on a \[
    T_n(x_k)=\frac{(-1)^k}{2^{n-1}}
\]
d'où on tire le tableau de signes
\begin{center}
    \begin{tikzpicture}
        \tkzTabInit[espcl=2]{$x$ / 1 ,$Q(x)$ /1 }{$x_0$ , $x_1$, $x_2$, $\cdots$ , $x_n$ }%
        \tkzTabLine{+,0,-,0,+,0,\cdots ,0,(-1)^n\;}
    \end{tikzpicture}
\end{center}
On en déduit que $Q$ s'annule au moins $n$ fois et $\deg Q\leq n-1$ donc $Q=0$, donc $P=T_n$ absurde.
\end{proof}

% \todo{exo fermés}

\section{Aspects topologiques}

\subsection{Le théorème de Baire}

\begin{res}
    \index{Baire (théorème de -- )}
    \index{Cauchy!suites}
    \index{espace complet}
    $(E, \|\;\|)$ un e.v.n tel que toute suite de Cauchy converge (complet). Si $(O_n)_n$ est une suite d'ouverts denses de $E$ alors $\cap O_n$ est dense.
\end{res}

\begin{proof}
On note $U$ ouvert de $E$.
\begin{itemize}
    \item
$U\cap O_0$ est un ouvert non vide donc il existe $x_0\in U\cap O_0$ et $2a_0>0$ tel que $\mathcal B_o(x_o, 2a_o)\subset U\cap O_0$ et donc $\mathcal B_f(x_0, a_0)\subset U\cap O_0$. On note $r_0=\min(a_0, 1)$.
\item $\mathcal B_o(a_0, r_0)\cap O_1$ ouvert non vide donc il existe $x_1$ et $a_1>0$ tels que $\mathcal B_f(x_1, a_1)\subset \mathcal B_0(x_0, r_0)\cap O_1$ et on note $r_1=\min\left(a_1, \frac12\right)$

\item $\cdots$
\end{itemize}
On construit ainsi deux suites $(x_n)_n\in E^{\mathbb N}$, $(r_n)\in\mathbb {R_+^\star}^{\mathbb N}$ telles que \[
    \begin{cases}
        \mathcal B_f(x_n, r_n)\subset \mathcal B_f(x_{n-1}, r_{n-1})\subset \cdots \subset \mathcal B_f(x_0, r_0) \\
        \mathcal B_f(x_n, r_n)\subset O_n, \quad r_n\leq \dfrac 1n
    \end{cases}
\]
Pour $m>n$, $x_m\in\mathcal B_f(x_m, r_m)\subset\mathcal B_f(x_n, r_n)$ donc \[
    \|x_m-x_n\|\leq \frac 1n
\]
donc $(x_n)$ de Cauchy donc convergente vers $x$.

Pour $m$ fixé, APCR $x_n\in\mathcal B_f(x_m, r_m)$ fermé donc $x\in\mathcal B_f(x_m, r_m)\subset O_m$ donc $x\in O_m$ pour tout $m$. Ainsi, \[
    x\in\bigcap_{n\in\mathbb N}O_n
\]
et $x\in\mathcal B_f(x_0, r_0)\subset U$ donc $x\in U$ et donc \[
    U\cap\left(\bigcap_{n\in\mathbb N}O_n\right)
\]
pour tout ouvert $U$.

\begin{cor}
Si $(F_n)_n$ est une suite de fermés d'intérieurs vides alors \[
    \bigcup_{n\in\mathbb N}F_n
\] est d'intérieur vide.
\end{cor}
\end{proof}

\subsection{Incomplétude de {$\mathbb R[X]$}}

On supose qu'il existe une norme $\|\cdot \|$ telle que $(\mathbb R[X], \|\cdot \|)$ est complet (i.e. toutes les suites de Cauchy convergent)

\begin{itemize}
    \item $E_n=\Vect(1, \cdots, X^n)$ est un fermé (s.e.v. de dimension finie)
    \item \[
    \forall P\in E_n, \forall \varepsilon>0, \quad P+\frac{X^{n+1}}k\xrightarrow[k\to+\infty]{} P
    \] donc APCR \[
        P+\frac{X^{n+1}}k \in \mathcal B_0(P, \varepsilon)
    \]
    donc $\mathcal B_o(P, \varepsilon)\not\subset E_n$ et $E_n$ d'intérieur vide.
    \item Par le théorème de Baire, \[
        \underbrace{\bigcap_{n\in\mathbb N} E_n}_{\text{intérieur vide}}=\underbrace{\mathbb R[X]}_{\text{intérieur non vide}}
    \]
    absurde.
\end{itemize}

Il n'existe donc aucune norme pour laquelle $\mathbb R[X]$ est complet.

En particulier, $(X_k)_k$ est une base qu'on peut orthonormaliser par le procédé de Gram-Schmidt en une base $(P_k)_k$ qui est totale, sans que $\mathbb R[X]$ soit complet.
\endchapter



\newpage
\pagestyle{empty}

\printindex

\end{document}
