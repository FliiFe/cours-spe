\ifsolo
    ~

    \vspace{1cm}

    \begin{center}
        \textbf{\LARGE Réduction des endomorphismes} \\[1em]
    \end{center}
    \tableofcontents
\else
    \minitoc
\fi
\thispagestyle{empty}

\ifsolo \newpage \setcounter{page}{1} \fi

Dans tout le chapitre, on note $\mathbb K$ un corps.

\section{Rappels sur les polynômes}

\begin{thm}
    \Hyp $A, B\in \mathbb K[X]$, $B\neq 0$
    \Conc Il existe un unique couple de polynômes $(Q, R)$ dans $\mathbb K[X]$ tels que \[
        A=QB+R
    \]
    et $\deg R<\deg B$
\end{thm}

\begin{thmdef}
    $a\in\mathbb K$ est une racine de $P$ ssi \[
    P(a)=0\iff (X-a)\;|\;P
\]
et $a$ est racine de multiplicité $\alpha \geq 1$ ssi \[
    \begin{cases}
        P(a)=P'(a)=\cdots=P^{(\alpha - 1)}(a)=0 \\
        P^{(\alpha)}(a)\neq 0
    \end{cases}
    \iff \begin{cases}
        (X-a)^\alpha \; |\; P\\
        (X-a)^{\alpha+1}\;\not|\;P
    \end{cases}
\]
Si $a$ n'est pas racine de $P$, on convient que $a$ est racine de multiplicité $0$.
\end{thmdef}

\begin{thm}[Interpolation de Lagrange]
    \index{Lagrange!interpolation} Pour $a_1, \cdots, a_n\in\mathbb R$ des abscisses distinctes et $b_1, \cdots, b_n\in\mathbb R$ des ordonnées distinctes, il existe un unique polynôme de degré au plus $n-1$ tel que \[
        \forall i\in\llbracket 1, n\rrbracket, \qquad P(a_i)=b_i
    \]
\end{thm}
\begin{proof}
    \[
        \varphi:P\in\mathbb K_{n-1}[X]\longmapsto (P(a_1), \cdots, P(a_n))\in\mathbb R^n
    \]
    est un isomorphisme.
\end{proof}

\section{Idéaux de {$\mathbb K[X]$} et arithmétique}

\begin{thm}
    $\mathbb K[X]$ est principal
\end{thm}
\begin{proof}
    C'est un anneau euclidien donc principal (cf. chapitre \textbf{Algèbre générale})
\end{proof}

\begin{thmdef}
    \begin{enumerate}
        \item Il existe un unique polynôme unitaire $Q$ tel que $P\mathbb K[X]+Q\mathbb K[X]=Q\mathbb K[X]$. C'est le PGCD de $A$ et $B$. Deux polynômes sont premiers entre eux si leur PGCD vaut $1$.
        \item Il existe un unique polynôme unitaire $Q$ tel que $P\mathbb K[X]\cap Q\mathbb K[X]=Q\mathbb K[X]$. C'est le PPCM de $A$ et $B$.
        \item (Gauss) \index{Gauss!lemme (polynômial)} Si $A$ et $B$ sont premiers entre eux et $A\;|\;BC$ alors $A\;|\;C$
        \item (Bézout) \index{Bézout (théorème de -- )} $A$ et $B$ sont premiers entre eux si et seulement si il existe $U, V$ tels que $AU+BV=1$.
    \end{enumerate}
\end{thmdef}

\begin{ex}
    $(X^n-1)\land (X^m-1)=X^{n\land m}-1$
\end{ex}

\section{Décomposition en facteurs irréductibles}

\begin{dfn}
    Un anneau $A$ est factoriel si \begin{enumerate}
        \item Pour tout $a\in A$, il existe des éléments irréductibles $p_1, \cdots, p_n$ tels que $a=p_1\cdots p_n$
        \item Cette décomposition est unique à l'ordre des facteurs près et à association près (deux éléments $a$ et $b$ sont associés si il existe un inversible $u$ tel que $a=ub$)
    \end{enumerate}
\end{dfn}

\begin{thm}
    $\mathbb K[X]$ est un anneau factoriel
\end{thm}

\begin{ex}
    Dans $\mathbb C[X]$, \begin{itemize}
        \item \[
                X^n-1=\prod_{\omega\in\mathbb U_n}(X-\omega)
            \]
        \item \[
                X^{2m}-1=(X+1)(X-1)\prod_{k=1}^{n-1} \left( X-\exp \left( \frac{ik\pi}n \right) \right) \left( X-\exp \left( -\frac{ik\pi}n \right) \right)
            \]
    \end{itemize}
\end{ex}

\begin{exo} Montrer que
    \begin{itemize}
        \item $P(\mathbb R)\subset \mathbb R_+ \iff \exists A, B\in\mathbb R[X], P=A^2+B^2$
        \item $P(\mathbb R_+)\subset \mathbb R_+\iff \exists A, B\in\mathbb R[X], P=A^2+XB^2$
    \end{itemize}
\end{exo}

\begin{proof}[Résolution]
    Éléments de résolution pour le premier point: l'ensemble des $A^2+B^2$ est stable par produit, et en décomposant $P\geq 0$ en facteurs irréductibles, chacun des facteurs est somme de deux carrés.
\end{proof}

\begin{exo}
    $P$ unitaire de degré $n$ dans $\mathbb R[X]$. Montrer \[
        P\text{ scindé dans }\mathbb R[X] \iff \forall z\in\mathbb C, \quad |P(z)|\geq |\mathfrak{Im}(z)|^n
    \]
\end{exo}

\begin{proof}~
    \begin{itemize}
        \item $(\impliedby)$ $P$ scindé dans $\mathbb C$ de racines $(a_i)$. $0=|P(a_i)|\geq |\mathfrak{Im}(z)|^n$ donc $a_i\in\mathbb R$
        \item $(\implies)$ \[
                \left| P(x+iy) \right|= \left| \prod_{j=1}^n(x-a_j+iy) \right|=\prod_{j=1}^n|x-a_j+iy|\geq \|\mathfrak{Im}(z)|^n
            \]
    \end{itemize}
\end{proof}


\section{Polynômes d'endomorphismes}

\begin{defprop}
    \Hyp On note $E$ un $\mathbb K$-e.v., $u\in\mathcal L(E)$ et \[
        P=\sum_{k=0}^{+\infty}a_kX^k\in\mathbb K[X]
    \]
    \begin{concenum}
    \item ~ \[
            P(u)\defeq\sum_{k=0}^{+\infty} a_ku^k
        \]
    \item $\varphi: P\in\mathbb K[X]\longmapsto P(u)\in\mathcal L(E)$ est un morphisme d'algèbres.
    \end{concenum}
\end{defprop}

\begin{proof}
    On justifie $(PQ)(u)=P(u)\circ Q(u)$. On considère \[
        \psi: (P, Q)\in\mathbb K[X]^2\longmapsto (PQ)(u)-P(u)\circ Q(u)\in\mathbb L(u)
    \]
    \begin{itemize}
        \item $\psi$ est bilinéaire
        \item À $Q$ fixé, $P\longmapsto \psi(P, Q)$ AL nulle ssi $\forall j\in\mathbb N, \psi(X^j, Q)=0$
        \item Pour chaque $j$, $Q\longmapsto \psi(X^j, Q)$ nulle ssi $\forall k\in\mathbb N, \psi(X^j, X^k)=0$ qui est vrai.
    \end{itemize}
\end{proof}

\begin{dfn}
    Un polynôme $P$ est dit \textbf{polynôme annulateur\index{polynôme!annulateur}} de $u\in\mathcal L(E)$ si $P(u)$ est l'application nulle
\end{dfn}

\begin{ex}
    $p$ est un projecteur si et seulement si $X^2-X$ est annulateur. $s$ est une symétrie si et seulement si $X^2-1$ est annulateur
\end{ex}

\begin{defprop}
    L'ensemble des polynômes annulateurs d'un endomorphisme $u$ est un idéal. Il possède un unique générateur unitaire, qu'on appelle \textbf{polynôme minimal\index{polynôme!minimal}}, qu'on notera $\mu_u$ ou $\Pi_u$
\end{defprop}

\begin{proof}
    $(\id, u, \cdots, u^{n^2})$ famille liée de $\mathcal L(E)$ donc l'idéal n'est pas trivial
\end{proof}

\begin{prop}
    $\mathbb K[u]\defeq \{P(u), P\in\mathbb K[X]\}$ est une $\mathbb K$-algèbre de dimension $\deg \mu_u$ de base $(1, u, \cdots, u^{\deg \mu_u-1})$
\end{prop}

\begin{ex}
    $P=X^2-3X+2$ polynôme minimal \[
        A= \begin{pmatrix}
            -2 & -6 \\ 2 & 5
        \end{pmatrix}
    \]
    Puis $X^n=\mu_A\times Q+\alpha_n X+\beta_n$. En injectant $1$ et $2$ (les racines de $\mu_A$) on peut exprimer $\alpha_n$ et $\beta_n$ et $A^n=\alpha_nA+\beta_n$
\end{ex}

\section{Théorème de décomposition des noyaux}

\begin{thm}[Décomposition des noyaux\index{décomposition des noyaux (théorème)}]
    \Hyp $E$ un $\mathbb K$-e.v., $u\in\mathcal L(E)$
    \begin{concenum}
    \item Si $P, Q\in\mathbb K[X]$ sont premiers entre eux, alors $\Ker (PQ)(u)=\Ker P(u)\oplus \Ker Q(u)$. En particulier, si $PQ$ est annulateur alors $\Ker(PQ)(u)=E$
    \item Si $P_1, \cdots, P_q\in\mathbb K[X]$ deux à deux premiers entre eux et si $P=P_1\cdots P_q$ alors \[
            \Ker P(u)=\Ker P_1(u)\oplus\cdots\oplus \Ker P_q(u)
        \]
    \item Dans ce cas, si $Q_i=\frac P{P_i}$, \begin{enumerate}
        \item Il existe $U_1, \cdots, U_q$ tels que $U_1Q_1+\cdots +U_qQ_q=1$
        \item $(U_iQ_i)(u)$ est la projection sur $\Ker P_i(u)$ associée à la décomposition
    \end{enumerate}
    \end{concenum}
\end{thm}

\begin{proof}~
    \begin{enumerate}
        \item On écrit $UP+QV=1$ et si $x\in\Ker P(u)\cap \Ker Q(u)$ alors $(UP+QV)(u)(x)=0=\id(x)=x$ donc la somme est directe

            Puis, la valeur de la somme est la bonne car $\Ker P(u),\Ker Q(u)\subseteq \Ker PQ(u)$ et pour $x\in \Ker PQ(u)$, $x=UP(u)(x)+VQ(u)(x)\in \Ker Q(u)+\Ker P(u)$
        \item Par récurrence, c'est vrai
        \item Les $Q_i$ sont premiers entre eux dans leur ensemble, donc Bézout donne (a). Puis, pour $x$ dans $\Ker P(u)$, \[
                x= \underbrace{U_1Q_1(u)(x)}_{\in\Ker P_1(u)}+\cdots +\underbrace{U_qQ_q(u)(x)}_{\in\Ker P_q(u)}
            \]
            et la somme est directe ce qui conclut.
    \end{enumerate}
\end{proof}

\begin{ex}
    Si $u^3+u=0$, alors $E=\Ker u\oplus \Ker(u^2+\id)$ or $\Ker(u^2+\id)\subset \img u$ donc $E=\Ker u+\img u$
\end{ex}

\begin{ex}
    $E$ de dimension finie, $u$ un endomorphisme de $E$ tel que $u\in\Vect(u^k, k\geq 2)$. Il y a donc un polynôme annulateur pour $u$. De même que précédemment, \[
        E=\Ker u\oplus \img u
    \]
\end{ex}

\begin{ex}
    On note $E$ l'ensemble des suites $(u_n)$ telles que $\forall n\in\mathbb N, u_{n+3}=4u_{n+2}-5u_{n+1}+2u_n$. On note $T$ l'endomorphisme $(u_n)_n\longmapsto (u_{n+1})_n$ de sorte que \[
        E=\Ker(T^3-4T^2+5T-2\id)=\Ker((T-\id)^2)\oplus \Ker(T-2\id)
    \]
    et $\Ker(T-2\id)=\{\lambda (2^n)_n, \lambda\in\mathbb R\}$, $\Ker(T-\id)=\{(\lambda)_n, \lambda\in\mathbb R\}$ puis \[
        u\in \Ker((T-\id)^2)\iff T(u)-u\in\Ker(T-\id)\iff (u_{n+1}-u_n)_n=(\lambda)_n
    \]
    donc $\Ker((T-\id)^2)=\{(\nu+n\mu)_n, \nu, \mu\in\mathbb R\}$
\end{ex}

\section{Valeurs propres, vecteurs propres}

\begin{dfn}
    Pour $E$ un $\mathbb K$-e.v. et $u$ endomorphisme de $E$, \begin{enumerate}
        \item On dira que $\lambda\in\mathbb K$ est \textbf{valeur propre\index{valeur propre}} de $u$ si $\Ker(u-\lambda\id)\neq \{0\}$
        \item On dira que $x\in E$ est un \textbf{vecteur propre\index{vecteur propre}} associé à la valeur propre $\lambda$ si $x\in\Ker(u-\lambda\id)\setminus\{0\}$
        \item L'ensemble des valeurs propres (appelé \textbf{spectre\index{spectre}}) est noté $\Sp_{\mathbb K}(u)$ ou simplement $\Sp(u)$.
        \item L'ensemble des vecteurs propres pour la valeur propre $\lambda$ (on compte le vecteur nul pour garder la structure d'e.v.) est noté $E_\lambda=\Ker(u-\lambda \id)$ et s'appelle \textbf{sous-espace propre\index{sous-espace propre}} de $u$ pour $\lambda$.
    \end{enumerate}
\end{dfn}

\subsection{Étude du spectre}

\begin{prop}
    \Hyp $u, v\in\mathcal L(E)$, $P\in\mathbb K[X]$ et $\lambda\in\mathbb K$
    \begin{concenum}
    \item Si $u$ et $v$ commutent, alors $\Ker u$ et $\img u$ sont stables par $v$. De même, $E_\lambda(u)$ est stable par $v$.
    \item Si $x\in E_\lambda(u)$, $P(u)(x)=P(\lambda)x$
    \item $P(\lambda)\in\Sp_{\mathbb K}(P(u))$ pour $\lambda\in \Sp_{\mathbb K}(u)$
    \item Si $P$ est annulateur alors $\Sp_{\mathbb K}(u)\subseteq \mathcal Z_{\mathbb K}(P)$
    \item $\Sp_{\mathbb K}(u)=\mathcal Z_{\mathbb K}(\mu_u)$
    \end{concenum}
\end{prop}

\begin{proof}~ \begin{enumerate}
    \item 2. 3. Ok
    \setcounter{enumi}{3}
    \item $P(u)(x)=P(\lambda)x$ pour $x$ vecteur propre $\implies P(\lambda) = 0$ pour $P$ annulateur
    \item On a une inclusion par 4.

        Si $\lambda$ racine de $\mu_u$ dans $\mathbb K$ alors \[ \lambda \not\in\Sp(u)\implies u-\lambda\id \in \mathrm{GL}(E)
        \]
        donc dans ce cas $\mu_u=(X-\lambda) Q$ et $Q(u)=0$ absurde par minimalité de $\mu_u$. Donc $\lambda\in\Sp(u)$
\end{enumerate}
\end{proof}

\subsection{Polynôme caractéristique}

\index{polynôme!caractéristique}

\begin{thmdef}
    \Hyp $u\in\mathcal L(E)$ avec $E$ un $\mathbb K$-e.v. de dimension $n$, muni d'une base $\mathcal B$. $\mathcal M_{\mathcal B}(u)\defeq A$
    \begin{concenum}
        \item Il y a équivalence entre \begin{enumerate}
            \item $\lambda\in\Sp_{\mathbb K}(u)$
            \item $\det(\lambda\id-u)=0$
            \item $\det(\lambda I_n-A)=0$
        \end{enumerate}
        \item Le polynôme $\chi_A=\chi_u$ donné par $\det(X\id -u)=\det(X I_n-A)$ est appelé \textbf{polynôme caractéristique} de $A$ (ou de $u$)
        \item $\chi_A$ est unitaire de degré $n$ et \[
                \chi_A=X^n-\Tr (A) X^{n-1}+\cdots +(-1)^n\det A
            \]
        \item  $\Sp_{\mathbb K}(u)=\mathcal Z_{\mathbb K}(\chi_u)$
    \end{concenum}
\end{thmdef}

\begin{proof}
    \begin{enumerate}
        \item 2. Ok
        \setcounter{enumi}{2}
    \item Le polynôme est unitaire de degré $n$ car la seule permutation qui donne un coefficient de degré $n$ dans la formule du déterminant est l'identité, et ce coefficient vaut $1$. Toutes les autres permutations ne donnent que des coefficients de degré $<n-1$ donc le coefficient de $X^{n-1}$ ne peut venir que du développement du produit de la diagonale ($\sigma=\id$), dont le coefficient en $X^{n-1}$ vaut bien la trace. Pour le terme constant, on injecte $0$ et par définition on obtient $\det(-A)=(-1)^n\det A$
    \item \[
            \lambda\in\Sp(u)\iff \det(\lambda \id -u)=0\iff \lambda\in\mathcal Z_{\mathbb K}(\chi_u)
        \]
    \end{enumerate}
\end{proof}

\begin{ex}[Cas d'une matrice compagnon] On note \[
    A= \begin{pmatrix}
0 & 0 & \dots & 0 & -a_0 \\
1 & 0 & \dots & 0 & -a_1 \\
0 & 1 & \dots & 0 & -a_2 \\
\vdots & \vdots & \ddots & \vdots & \vdots \\
0 & 0 & \dots & 1 & -a_{p} \\
\end{pmatrix}\in \mathcal M_{p+1}(\mathbb K)
\]
Le calcul de $\det(XI_n-A)$ donne $\chi_A=X^{p+1}+P(X)$ avec $P(X)=a_pX^p+\cdots +a_0$
\end{ex}

\section{Théorème de Cayley-Hamilton}

\begin{thm}[Cayley-Hamilton\index{Cayley-Hamilton (théorème de -- )}]
    \Hyp $E$ de dimension finie $n$ et $u\in\mathcal L(E)$
    \Conc $\chi_u$ est annulateur de $u$.
\end{thm}

\begin{proof} (Hors-Programme)

    On note $A$ canoniquement associée à $u$, et $B=XI_n-A$. On a \[
        B\Com(B)^\top = \det(B)I_n=\chi_A(X)I_n
    \]
    $\Com(B)^\top$ a des coefficients dans $\mathbb K_{n-1}[X]$ donc il existe des matrices $B_0, \cdots, B_{n-1}$ telles que \[
        \Com(B)^\top=X^{n-1}B_{n-1}+\cdots +B_0.
    \]
    Alors, \begin{align*}
        B\Com(B)^\top&=(XI_n-A)(X^{n-1}B_{n-1}+\cdots +B_0)\\
                     &=X^nB_{n-1}+X^{n-1}(B_{n-2}-AB_{n-1})+\cdots + X(B_0-AB_1)-AB_0\\
                     &=\chi_A(X)I_n
    \end{align*}
    On identifie: \[
        B_{n-1}=I_n\qquad \qquad \qquad \text{ et les autres }B_{i+1}-AB_i\text{ sont de la forme }\alpha_i I_n, \alpha_i\in\mathbb R
    \]
    Si on note $\chi_A=X^n+a_{n-1}X^{n-1}+\cdots +a_0$, on a \[
        B_{n-2}-AB_{n-1}=a_{n-1}I_n\quad , \quad \cdots \quad,\quad B_0-AB_1=a_1 I_n\quad,\quad  -AB_0=a_0I_n
    \]
    En remplaçant dans $\chi_A(A)$ tous les termes s'annulent.
\end{proof}

\begin{csq*}~
    \begin{itemize}
        \item Pour $A$ dans $\mathcal M_n(\mathbb K)$, $\mu_A\;|\;\chi_A$ donc $\deg \mu_A\leq n$
        \item Si \[
                \chi_A=\prod_{i=1}^r(X-\lambda_i)^{\alpha_i}
            \]
            avec des $\lambda_i$ deux à deux distincts, alors \[
                \mathcal M_n(\mathbb K)=\bigoplus_{i=1}^r\Ker\left((A-\lambda_iI_n)^{\alpha_i}\right)
            \]
        \item Si $A$ est nilpotente alors $A^n=0$ car $\deg \mu_A\leq n$
    \end{itemize}
\end{csq*}

\begin{exo}
    $A\in\mathcal M_n(\mathbb C)$ non inversible. Montrer qu'il existe $B\in\mathcal M_n(\mathbb C)$ non nulle telle que $\forall p\geq 1, (A+B)^p=A^p+B^p$
\end{exo}

\begin{proof}[Éléments de résolution]
    $\mu_A=XP(X)$, $P(A)\neq 0$ par minimalité de $\mu_A$ et $P(A)$ convient.
\end{proof}

\begin{exo}
    Montrer que $A^3+A^2+A+I_n=0\implies \Tr A\leq 0$
\end{exo}
\begin{proof}[Éléments de résolution]
    $P(X)=(X+1)(X^2+1)$ est annulateur donc $\Sp_{\mathbb C}(A)\subseteq\{-1, i, -i\}$ et $\chi_A=(X+1)^{\alpha}(X+i)^{\beta}(X-i)^{\gamma}$ polynôme réel donc $\beta=\gamma$ et la trace est le coeff en $X^{n-1}$
\end{proof}

