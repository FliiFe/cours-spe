\ifsolo
    ~

    \vspace{1cm}

    \begin{center}
        \textbf{\LARGE Compacité -- Connexité} \\[1em]
    \end{center}
    \tableofcontents
\else
    \chapter{Compacité -- Connexité}

    \minitoc
\fi
\thispagestyle{empty}

\section{Définitions}

\begin{dfn}
On note $E$ un  $\K$-e.v.n. \begin{enumerate}
    \item On dira que $(x_n)\in E^{\N}$ possède une \textbf{valeur d'adhérence}\index{valeur d'adhérence} $\ell$ s'il y a une suite extraite de $ (x_n)$ qui tend vers  $\ell$
    \item Pour  $x=(x_n)$, on notera  $V(x)$ l'ensemble des valeurs d'adhérence.
\end{enumerate}
\end{dfn}

\begin{ex}~
\begin{enumerate}
    \item $E=\R,\qquad x=((-1)^n),\qquad V(x)=\{-1,1\} $
    \item $E=\R,\qquad x_n \xrightarrow[n\to+\infty]{}\ell,\qquad  V(x)=\{\ell\} $
    \item $E=\R,\qquad (r_n)$ une énumération de $\Q$. $V(r)=\R$ (par densité)
\end{enumerate}
\end{ex}

\section{Compacité}

\begin{dfn}
On dira que $A\subset E$ est une partie compact si  $A=\emptyset$ ou si toute suite d'éléments de  $A$ possède une valeur d'adhérence dans  $A$
\end{dfn}

\subsection{Propriétés topologiques}

\begin{prop}
\Hyp $A\subset E$
 \begin{concenum}
 \item Si  $A$ est compacte, alors c'est une partie fermée bornée
 \item Un fermé inclus dans  $A$ est compact
\end{concenum}
\end{prop}

\begin{proof}
\begin{enumerate}
    \item Soit $(x_n)\in A^{\N}$ qui converge vers $\ell$. C'est la seule valeur d'adhérence de  $x$ donc elle est dans  $A$ et $A$ est fermé. Si  $A$ n'est pas bornée, la suite  $(x_n)$ définie par  \[
            \forall  n \in  \N,\qquad  x_n \in  A\setminus    \mathcal  B_f(0,n)
    \] 
    est bien définie et n'a aucune VA, ce qui est absurde.
\item C'est évident avec les définitions
\end{enumerate}
\end{proof}

\begin{prop}
\Hyp $K$ compact de $E$,  $u$ une suite de  $K^{\N}$ 
\Conc Il y a équivalence entre \begin{enumerate}
    \item $u$ converge
    \item  $u$ a une seule valeur d'adhérence
\end{enumerate}
\end{prop}

\begin{proof} ~

    $(1\implies 2)$ évident

    $(2\implies 1)$ On note $\ell$ la VA de  $u$, et on suppose par l'absurde  \[
        x_n \xnrightarrow[n\to+\infty]{}\ell
    \]
    Alors $\exists \epsilon>0,\forall  N \in  \N,\exists  n\geq N, |u_n-\ell|\geq \epsilon$. Pour un tel $\epsilon$, on construit la suite extraite suivante \begin{itemize}
        \item $N=1$, il existe  $n_1\geq 1$ tel que $|u_{n_1}-\ell|\geq \ell$
        \item $\vdots$
        \item  $N=n_{N-1}+1$, on construit $n_N$
    \end{itemize}
    On a ainsi construit une suite extraite qui n'admet pas $\ell$ comme VA, donc elle admet une autre VA  $\ell'$ dans  $K$ ce qui est absurde.
\end{proof}


\begin{exo}
On note $u$ une suite bornée réelle telle que  \[
    u_n+\frac{1}{2}u_{2n} \xrightarrow[n\to+\infty]{}0
\] 
Montrer que $u$ converge.
\end{exo}

 \begin{rem}
 Soit $u \in  E^{\N}$. On va montrer que \[
     V(u)=\bigcap_{n\geq 0} \overline{\{ u_k,\quad k\geq n\}}
 \]
 On note $U_n=\{u_k, k\geq n\}, F_n=\overline{U_n}$ et $ F$ l'intersection des  $F_n$. Si  $ \ell  \in V(u)$ alors il existe une extractrice $\varphi$ telle que  $u_{\varphi (k)} \xrightarrow[n\to+\infty]{}\ell $ donc $\ell  \in  \overline{ U_n }=F_n$ pour tout $n$, donc  $\ell  \in  F$.

 Si $\ell  \in  F$, alors on construit $(k_n)$ de la manière suivante:  \begin{itemize}
     \item $\mathcal  B_o(\ell ,1)\cap U_1$ est non vide ($ \ell  \in  F_1$) donc $ \exists  k_1\geq 1,\quad |u_{k_1}-\ell |\leq 1$
     \item $\mathcal  B_o(\ell ,\dfrac{1}{2}) \cap U_{k_1+1}$ est non vide ($\ell  \in  F_{k_1+1}$) donc $ \exists  k_2>k_1,\quad |u_{k_2}-\ell |\leq \dfrac{1}{2}$
     \item $\cdots $
 \end{itemize}
 Et on a ainsi construit une suite extraite qui tend vers $\ell $, ce qui conclut.
 \end{rem}

 \subsection{Produit cartésien}

 \begin{prop}
     \Hyp $(E_1, \|\;\|_1),\cdots ,(E_p,\|\;\|_p)$ des evn, $A_1\subset E_1, \cdots , A_p\subset E_p$ des compacts, et $\|\;\|=\sum_i \| \;\|_i$ 
     \Conc $A_1\times \cdots  \times A_p$ est un compact de $E=E_1\times \cdots \times E_p$ pour la norme $\|\;\|$
 \end{prop}

 \begin{proof}
     On note $(X_n)=(x_{n,1},\cdots ,x_{n,p})\in (A_1\times \cdots \times A_p)^{\N}$. Alors \begin{itemize}
         \item $(x_{n,1})$ est une suite d'éléments de $A_1$ qui est un compact donc il existe une extractrice  $\varphi_1$ qui fait converger la suite extraite vers $\ell_{1} \in  A_1$ pour $\|\;\|_1$
         \item  $(x_{\varphi_1(n),2})$ est une suite du compact $A_2$, on fait pareil. 
         \item $\cdots $
     \end{itemize}
     Et ainsi $(X_{\varphi_p\circ \cdots \circ \varphi_1(n)})$ converge vers $(l_1,\cdots ,l_p)\in  A_1\times \cdots \times A_p$.
 \end{proof}

 \begin{rem}
     $\R^p=\R\times \cdots \times \R$ munit de $\|\;\|_1$ (c'est la norme produit avec $|\;|$ sur  $\R$), $[-1,1]^p$ est compact pour  $\|\;\|_1$ donc pour $\|\;\|_\infty$ (équivalence des normes en dimension finie) et $\mathcal  B_f^\infty(0,1)$ est un fermé inclus dans $[-1,1]^p$ donc c'est un compact.

     Plus généralement,  $\mathcal  B_f^\infty(0,r)$ est compact donc toute partie finie bornée pour $\|\;\|_\infty$ est compacte.
 \end{rem}

 \subsection{Théorème des bornes atteintes}

 \begin{thm}
     \Hyp $(E, \|\;\|)$ un $\K$-evn, $A\subset E$ un compact et  $f:A\to \R$ continue.\index{bornes atteints (théorème des -- )}
\begin{concenum}
\item $f(A)$ est un compact de  $\R$
\item $ \exists  a,b\in A,\quad \quad f(a)=\sup_A\limits f$ et $f(b)=\inf_A\limits f$
\end{concenum}
 \end{thm}

 \begin{proof}
 
 \end{proof}
