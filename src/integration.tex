\ifsolo
    ~

    \vspace{1cm}

    \begin{center}
        \textbf{\LARGE Intégration sur un intervalle quelconque} \\[1em]
    \end{center}
    \tableofcontents
\else
    \minitoc
\fi
\thispagestyle{empty}

\ifsolo \newpage \setcounter{page}{1} \fi

\section{Rappels}

L'opération d'intégration sur $\mathcal C^0_{PM}([a, b], \mathbb R)$ est linéaire positive croissante. On a \[
    \left| \int_a^bf \right|\leq \int_a^b|f|\leq (b-a)\sup_{[a,b]} |f|
\]
et similairement \[
    \left| \int_a^bfg \right|\leq \sup_{[a,b]}|f|\int_a^b|g|
\]
Si $f$ est continue alors \[
    \left| \int_a^bf \right|=\int_a^b|f|\iff f\text{ de signe constant }
\]
et si $g$ est continue positive, \[
    \inf_{[a,b]}f\int_a^bg\leq \int_a^bfg\leq \sup_{[a,b]}f\int_a^bg
\]
et si $g$ n'est pas identiquement nulle, \[
    \int_a^bg>0\implies \dfrac{\int_a^bfg}{\int_a^bg}\in[\inf f,\sup f]=f([a,b])\implies \exists c, \quad \int_a^bfg=f(c)\int_a^bg.
\]
Le résultat reste vrai si $g$ est identiquement nulle. Si $f$ est continue positive sur $[a, b]$ et $\int_a^bf=0$ alors $f=0$.
