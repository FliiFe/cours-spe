\ifsolo
    ~

    \vspace{1cm}

    \begin{center}
        \textbf{\LARGE Variables aléatoires discrètes} \\[1em]
    \end{center}
    \tableofcontents
\else
    \chapter{Variables aléatoires discrètes}

    \minitoc
\fi
\thispagestyle{empty}

\section{Tribu}
\subsection{Définition}

Dans tout le chapitre, $\Omega$ désigne un ensemble non vide.

\begin{dfn}
    Un ensemble $\mathcal A$ de parties de $\Omega$ est une \textbf{tribu}\index{tribu} si \begin{itemize}
        \item $ \mathcal A$ est stable par passage au complémentaire
        \item $\emptyset \in \mathcal A$
        \item $ \mathcal A$ est stable par union dénombrable
    \end{itemize}
    On dit alors que $(\Omega, \mathcal A)$ est un \textbf{espace probabilisable}\index{espace probabilisable}  et les éléments de $\mathcal A$ sont appelés  \textbf{événements} \index{evenement@événement}
\end{dfn}


\subsection{Propriétés ensemblistes}

\begin{prop}
    Soit $(\Omega, \mathcal A)$ un espace probabilisable. Si $(A_n)\in \mathcal A^{\N}$ alors \[
        \bigcap_{n\in\N}A_n \in \mathcal A
    \]
\end{prop}

\begin{proof}
    C'est une union dénombrable de complémentaires
\end{proof}

\begin{notation}[HP]
    \[
        \underline{\lim}A_n=\bigcup_{n\in\N} \left( \bigcap_{m\geq n} A_m \right) \qquad \qquad \overline{\lim} A_n=\bigcap_{n\in\N} \left( \bigcup_{m\geq n}A_m \right) 
    \]
\end{notation}

\subsection{Opérations sur les tribus}

Soit $(\mathcal A_i)_{i\in I}$ une famille de tribus. L'ensemble \[
    \bigcap_{i\in I}\mathcal A_i
\]
est une tribu (facile).

\begin{rem}
    On définit ainsi la notion de tribu engendrée: la tribu engendrée par $\mathcal E\subset \mathcal P(\Omega)$ est l'intersection de toutes les tribus contenant $\mathcal E$.
\end{rem}

\begin{dfn}
    Soit $(\Omega, \mathcal A)$ un espace probabilisable. \begin{itemize}
        \item $A,B\in \mathcal A$ sont \textbf{incompatible} si $A \cap B=\emptyset$
        \item Pour $A\in \mathcal A$, on appelle  \textbf{événement contraire} l'événement $A^c$
        \item On dit que $(A_i)_{i\in I} \in  \mathcal A^I$ est un \textbf{système complet d'événements}\index{système complet d'événements, SCE} si et seulement si $(A_i)_{i\in I}$ partitionne $\Omega$.
    \end{itemize}
\end{dfn}

\section{Espace probabilisé}

\begin{dfn}
    Soit $(\Omega, \mathcal A)$ un espace probabilisable. On appelle \textbf{probabilité}\index{probabilité} sur $(\Omega, \mathcal A)$ une application $\P:\mathcal A \to  [0, 1]$ telle que \begin{itemize}
        \item $\P(\Omega)=1$
        \item $\P$ est $\sigma$-additive, c'est-à-dire que pour toute suite $(A_i)$ d'événements deux à deux incompatibles, \[
                \P \left( \bigcup_{n\in\N}A_n \right) =\sum_{n\geq 0}\P(A_n)
        \]
    \end{itemize}
    Si $\P$ convient, on dira que $(\Omega, \mathcal A, \P)$ est un \textbf{espace probabilisé}\index{espace probabilisé}.
\end{dfn}

\begin{rem}
    \begin{itemize}
        \item La suite constante égale à $\emptyset$ donne $\P(\emptyset)=0$.
        \item Si $A$ et $B$ sont incompatibles, alors $\P(A\cup B)=\P(A)+\P(B)$ 
        \item $\P(A^c)=\P(A\cup A^c)-\P(A)=1-\P(A)$
        \item $A\subset B \implies P(B)=P(A)+P(B\setminus A)\geq P(A)$
    \end{itemize}
\end{rem}

\begin{exo}[Inégalité d'Edith Kosmanek\index{Kosmanek (inégalité de -- )}]
    Soit $(\Omega, \mathcal A, \P)$ un espace probabilisé, $A, B$ des événements. Montrer que \[
        |\P(A\cap B)-\P(A)\P(B)|\leq \frac{1}{4}
    \]
\end{exo}

\begin{proof}[Résolution]
    $(A\cap B, A\cap B^c, A^c\cap B, A^c\cap B^c)$ est un SCE et un peu de calcul donne \[
        \P(A\cap B)-\P(A)\P(B)=\P(A\cap B)\P(A^c\cap B^c)-\P(A^c\cap B)\P(A\cap B^c)
    \]
    Puis si $a,b\geq 0$ sont tels que $a+b\leq 1$, on a $ab=\sqrt{ab}^2\leq \left( \frac{a+b}{2} \right) ^2\leq \frac{1}{4}$ donc les deux produits sont $\leq \frac{1}{4}$, ce qui conclut.
\end{proof}

\begin{exo}
    $(\Omega, \mathcal A, \P)$ probabilisé, $A, B \in \mathcal A$. Montrer que $|\P(A)-\P(B)| \leq \P(A\Delta B)$
\end{exo}

\section{Propriétés élémentaires des probabilités}

\begin{thm}[Continuité croissante\index{Continuité croissante (probabilités)}]
    \Hyp $(\Omega, \mathcal A, \P)$ probabilisé, $(A_n)_{n\in \N}$ suite croissante d'événements
    \Conc \[
        \P\left(\bigcup_{n\in \N}A_n \right)=\lim_{n \to \infty} \P(A_n).
    \]
\end{thm}

\begin{proof} Par $\sigma$-additivité, en posant $A_{-1}=\emptyset$, on a
    \[
        \P\left( \bigcup_{n\in N} A_n \right) =\sum_{n\geq 0}\P(A_n\setminus A_{n-1})= \lim_{n \to \infty} \sum_{k=0}^{n} \underbrace{\P(A_k\setminus A_{k-1})}_{=\P(A_k)-\P(A_{k-1})}=\lim_{n \to \infty} \P(A_n).
    \]
\end{proof}

\begin{thm}[Continuité décroissante]
    \Hyp $(\Omega, \mathcal A, \P)$ probabilisé, $(A_n)_{n\in \N}$ suite décroissante d'événements
    \Conc \[
        \P\left(\bigcap_{n\in \N}A_n \right)=\lim_{n \to \infty} \P(A_n).
    \]
\end{thm}

\begin{proof}
    C'est la continuité croissante en passant au complémentaire.
\end{proof}

\begin{thm}[Sous-additivité]
    \Hyp $(\Omega, \mathcal A, \P)$ espace probabilisé, $(A_n)$ suite d'événements
    \Conc Avec la convention qu'une somme de série divergente à terme général positif vaut $+\infty$, \[
        \P\left( \bigcup_{n \in  \N}A_n  \right) \leq \sum_{n=0}^{+\infty} \P(A_n).
    \] 
\end{thm}

\begin{proof}
    On note \[
    X_n=\bigcup_{k\in \N}A_k 
    \] 
    de sorte que $(X_n)$ est une suite croissante d'événements. Pour deux événements $A\subset B$, on a \[
        \P(A\cup B)=\P(A\cup(B\setminus A))=\P(A)+\P(B\setminus A)\leq \P(A)+\P(B)
    \] 
    et en itérant, on trouve \[
        \forall n \in  \N, \qquad \P(X_n)\leq \sum_{k=0}^{n} \P(A_i)
    \] 
    donc \[
        \lim_{n \to \infty} \P(X_n) = \P\left( \bigcup_{n\in \N} A_n \right) \leq \sum_{n=0}^{+\infty} \P(A_n)
    \] 
\end{proof}

\begin{rem}
    Un événement $A\in \mathcal A$ est dit \textbf{négligeable} si $\P(A)=0$. Si $(A_n)$ est une suite d'événements négligeables, alors \[
        \P(\bigcup_{n\in \N}A_n )\leq \sum_{n=0}^{+\infty} \P(A_n)=0
    \] 
\end{rem}

\subsection{Formule de Poincaré}

Soit $(\Omega, \mathcal A, \P)$ un espace probabilisé, $A_1, \cdots, A_n \in  \mathcal A$. On va montrer que \[
    \P(A_1\cup \cdots \cup A_n)=\sum_{1\leq i_1\leq n}P(A_{i_1})-\sum_{1\leq i_1<i_2\leq n}P(A_{i_1}\cap A_{i_2}) +\cdots +(-1)^{n-1}\sum_{1\leq i_1<\cdots <i_n\leq n}P(A_{i_1}\cap \cdots \cap A_{i_n})
\] 
On remarque pour cela \[
    \1_{(A_1\cup \cdots \cup A_n)^c}=1-\1_{A_1\cup \cdots \cup A_n}=\1_{A_1^c\cap \cdots \cap A_n^c}=\1_{A_1^c}\times \cdots \times \1_{A_n^c}=(1-\1_{A_1})\times \cdots \times (1-\1_{A_n})
\]
donc en développant, \[
    \1_{(A_1\cup \cdots \cup A_n)^c} = 1-\sum_{i=1}^n\1_{A_i}-\sum_{i_1<i_2}\1_{A_{i_1}\cap A_{i_2}}+\cdots +(-1)^{n-1}\sum_{i_1<\cdots <i_n}\1_{A_{i_1}\cap\cdots\cap A_{i_n}}
\]
Or (on le verra), $\E(\1_A)=\P(A)$ et $\E$ est linéaire, ce qui conclut.

\begin{ex}
    On place $r$ boules dans $n$ cases. Calculer la probabilité qu'aucune case ne soit vide. On note $A_i$ l'événement "La case n°$i$ est vide" et $B=A_1^c\cap \cdots \cap A_n^c$. On cherche $\P(B)$. On a \[
        \forall k \in  \llbracket 1, n\rrbracket, \qquad \P(A_{i_1}\cap \cdots \cap A_{i_k})= \left( 1-\frac{k}{n} \right) ^r.
    \] 
    Puis, \begin{align*}
        \P(B^c)=1-\P(B)&=\sum_{1\leq i_1\leq n} \left( 1-\frac{1}{n} \right) ^r-\sum_{i_1<i_2} \left( 1-\frac{2}{n} \right) ^r+\cdots +(-1)^{n-1} \sum_{i_1<\cdots <i_n} \left( 1-\frac{n}{n} \right) ^r\\
                       &=\sum_{k=0}^n(-1)^k\binom nk \left( 1-\frac{k}{n} \right) ^r
    \end{align*}
\end{ex}

\section{Probabilité conditionnelle}

\begin{dfn}
    Soit $(\Omega, \mathcal A, \P)$ un espace probabilisé et $B\in \mathcal A$ tel que $\P(B)>0$. On appelle probabilité de $A \in \mathcal A$ sachant $B$ le nombre  \[
        \P_B(A)=\P(A|B)\defeq \frac{\P(A\cap B)}{P(B)}
    \] 
\end{dfn}

\begin{prop}
    \Hyp $(\Omega, \mathcal A, \P)$ un espace probabilisé, $A_1, \cdots , A_n \in  \mathcal A$ tels que $\P(A_1 \cap \cdots \cap A_n)>0$
    \Conc \[
        \P(A_1\cap \cdots \cap A_n)=\P(A_1)\P(A_2|A_1)\P(A_3|A_1\cap A_2)\cdots \P(A_n|A_1\cap\cdots\cap A_{n-1})
    \] 
\end{prop}

\begin{proof}
    La seule difficulté est de montrer que les probabilités utilisées existent. C'est le cas car \[
        \forall i\leq n, \qquad 0<\P(A_1\cap\cdots\cap A_n)\leq \P(A_1\cap \cdots \cap A_i)
    \] 
\end{proof}
