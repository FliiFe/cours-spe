\ifsolo
    ~

    \vspace{1cm}

    \begin{center}
        \textbf{\LARGE Variables aléatoires discrètes} \\[1em]
    \end{center}
    \tableofcontents
\else
    \chapter{Variables aléatoires discrètes}

    \minitoc
\fi
\thispagestyle{empty}

\section{Tribu}
\subsection{Définition}

Dans tout le chapitre, $\Omega$ désigne un ensemble non vide.

\begin{dfn}
    Un ensemble $\mathcal A$ de parties de $\Omega$ est une \textbf{tribu}\index{tribu} si \begin{itemize}
        \item $ \mathcal A$ est stable par passage au complémentaire
        \item $\emptyset \in \mathcal A$
        \item $ \mathcal A$ est stable par union dénombrable
    \end{itemize}
    On dit alors que $(\Omega, \mathcal A)$ est un \textbf{espace probabilisable}\index{espace probabilisable}  et les éléments de $\mathcal A$ sont appelés  \textbf{événements} \index{événement}
\end{dfn}


\subsection{Propriétés ensemblistes}

\begin{prop}
    Soit $(\Omega, \mathcal A)$ un espace probabilisable. Si $(A_n)\in \mathcal A^{\mathbb N}$ alors \[
        \bigcap_{n\in\mathbb N}A_n \in \mathcal A
    \]
\end{prop}

\begin{proof}
    C'est une union dénombrable de complémentaires
\end{proof}

\begin{notation}[HP]
    \[
        \underline{\lim}A_n=\bigcup_{n\in\mathbb N} \left( \bigcap_{m\geq n} A_m \right) \qquad \qquad \overline{\lim} A_n=\bigcap_{n\in\mathbb N} \left( \bigcup_{m\geq n}A_m \right) 
    \]
\end{notation}

\subsection{Opérations sur les tribus}

Soit $(\mathcal A_i)_{i\in I}$ une famille de tribus. L'ensemble \[
    \bigcap_{i\in I}\mathcal A_i
\]
est une tribu (facile).

\begin{rem}
    On définit ainsi la notion de tribu engendrée: la tribu engendrée par $\mathcal E\subset \mathcal P(\Omega)$ est l'intersection de toutes les tribus contenant $\mathcal E$.
\end{rem}

\begin{dfn}
    Soit $(\Omega, \mathcal A)$ un espace probabilisable. \begin{itemize}
        \item $A,B\in \mathcal A$ sont \textbf{incompatible} si $A \cap B=\emptyset$
        \item Pour $A\in \mathcal A$, on appelle  \textbf{événement contraire} l'événement $A^c$
        \item On dit que $(A_i)_{i\in I} \in  \mathcal A^I$ est un \textbf{système complet d'événements}\index{système complet d'événements, SCE} si et seulement si $(A_i)_{i\in I}$ partitionne $\Omega$.
    \end{itemize}
\end{dfn}

\section{Espace probabilisé}

\begin{dfn}
    Soit $(\Omega, \mathcal A)$ un espace probabilisable. On appelle \textbf{probabilité}\index{probabilité} sur $(\Omega, \mathcal A)$ une application $\mathbb P:\mathcal A \to  [0, 1]$ telle que \begin{itemize}
        \item $\mathbb P(\Omega)=1$
        \item $\mathbb P$ est $\sigma$-additive, c'est-à-dire que pour toute suite $(A_i)$ d'événements deux à deux incompatibles, \[
                \mathbb P \left( \bigcup_{n\in\mathbb N}A_n \right) =\sum_{n\geq 0}\mathbb P(A_n)
        \]
    \end{itemize}
    Si $\mathbb P$ convient, on dira que $(\Omega, \mathcal A, \mathbb P)$ est un \textbf{espace probabilisé}\index{espace probabilisé}.
\end{dfn}

\begin{rem}
    \begin{itemize}
        \item La suite constante égale à $\emptyset$ donne $\mathbb P(\emptyset)=0$.
        \item Si $A$ et $B$ sont incompatibles, alors $\mathbb P(A\cup B)=\mathbb P(A)+\mathbb P(B)$ 
        \item $\mathbb P(A^c)=\mathbb P(A\cup A^c)-\mathbb P(A)=1-\mathbb P(A)$
        \item $A\subset B \implies P(B)=P(A)+P(B\setminus A)\geq P(A)$
    \end{itemize}
\end{rem}

\begin{exo}[Inégalité d'Edith Kosmanek]
    Soit $(\Omega, \mathcal A, \mathbb P)$ un espace probabilisé, $A, B$ des événements. Montrer que \[
        |\mathbb P(A\cap B)-\mathbb P(A)\mathbb P(B)|\leq \frac{1}{4}
    \]
\end{exo}

\begin{proof}[Résolution]
    $(A\cap B, A\cap B^c, A^c\cap B, A^c\cap B^c)$ est un SCE et un peu de calcul donne \[
        \mathbb P(A\cap B)-\mathbb P(A)\mathbb P(B)=\mathbb P(A\cap B)\mathbb P(A^c\cap B^c)-\mathbb P(A^c\cap B)\mathbb P(A\cap B^c)
    \]
    Puis si $a,b\geq 0$ sont tels que $a+b\leq 1$, on a $ab=\sqrt{ab}^2\leq \left( \frac{a+b}{2} \right) ^2\leq \frac{1}{4}$ donc les deux produits sont $\leq 14$, ce qui conclut.
\end{proof}

\begin{exo}
    $(\Omega, \mathcal A, \mathbb P)$ probabilisé, $A, B \in \mathcal A$. Montrer que $|\mathbb P(A)-\mathbb P(B)| \leq \mathbb P(A\Delta B)$
\end{exo}
