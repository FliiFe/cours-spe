\ifsolo
    ~

    \vspace{1cm}

    \begin{center}
        \textbf{\LARGE Espérance -- Variance} \\[1em]
    \end{center}
    \tableofcontents
\else
    \chapter{Espérance -- Variance}

    \minitoc
\fi
\thispagestyle{empty}

\section{Espérance}

\begin{dfn}
    On note $(\Omega, \mathcal  A, \P)$ un espace probabilisé, $X$ une variable aléatoire \textbf{réelle} discrète. On dira que $X$ est d'\textbf{espérance fini}\index{espérance} si  $(x\P(X=x))_{x\in X(\Omega)}$ est sommable. Dans ce cas, on appelle \textbf{espérance} la quantité \[
        \E(X)= \sum_{x \in  X(\Omega)}x\P(X=x)
    \] 
\end{dfn}

\begin{rem}[Rappel]
    Si $X(\Omega)=\{x_n,\quad n\in \N\} $, les $x_n$ deux à deux distincts, alors  \[
        \sum_{n\in \N}|x_n|\P(X=x_n) \text{ CV } \iff \left( x_n\P(X=x_n) \right)_{n \in  \N}\text{ sommable }
    \]
\end{rem}

\begin{ex}~
    \begin{itemize}
        \item $X\suit G(p)$ est d'espérance finie car  $(nq^{n-1}p)_n$ est sommable, et \[
                \E(X)=\sum_{n\geq 0}nq^{n-1}p=\frac{1}{p}
        \] 
    \item $X\suit \mathcal  P(\lambda)$ pour $\lambda>0$, on a  $n\P(X=n)=e^{-\lambda}\frac{n\lambda^n}{n!}$ et $ \E(X)=\lambda$
    \item $X=\1_A$ pour  $A\in \mathcal  A$, $\E(X)=\P(X=1)=\P(A)$
    \end{itemize}
\end{ex}

\section{Formule de transfert}

\begin{thm}
    \Hyp $(\Omega, \mathcal  A, \P)$ un espace probabilisé, $X$ une v.a.r.d et  $f:X(\Omega)\to \R$
    \begin{concenum}
    \item $f(X)$ est une v.a.r.d
    \item  $f(X)$ est d'espérance finie  si et seulement si $(f(x)\P(X=x))_{x\in X(\Omega)}$ et dans ce cas, \[
            \E(f(X))=\sum_{x\in X(\Omega)}f(x)\P(X=x)
    \] 
    \end{concenum}
\end{thm}

\begin{proof}~
    \begin{enumerate}
        \item Lemme des coalitions
        \item Pour $y\in f(X(\Omega))$, on note $I_y=f^{-1}(\{y\} )$. On a \[
                [f(X)=y]=\bigcup_{x\in I_y}[X=x] 
        \] 
        et cette union est disjointe donc \[
            \P(f(X)=y)=\sum_{x\in I_y}\P(X=x).
        \]
        La famille $(I_y)_{y\in f(X(\Omega))}$ forme une partition de $X(\Omega)$. Puis  \[
            \sum_{x \in  I_y}\P(X=x)f(x)=y\sum_{x\in I_y}\P(f(X)=y)=y\P(f(X)=y).
        \] Si la famille $(f(x)\P(X=x))_{x \in  X(\Omega)}$ est sommable, alors pour tout $y\in f(X(\Omega))$, $(f(x)\P(X=x))_{x\in I_y}$ est sommable et $(I_y)$ est une partition donc la famille  \[
        \left( \sum_{x \in  I_y} P(X=x)f(x)\right) _{y\in f(X(\Omega))}
        \] 
        est sommable et \[
            \sum_{x\in X(\Omega)}f(x)\P(X=x)=\sum_{y\in f(X(\Omega))}\sum_{x \in  I_y}f(x)\P(X=x)=\sum_{y\in f(X(\Omega))}y\P(f(X)=y)=\E(f(X)).
        \] 
        Si $f(X)$ est d'espérance finie, alors $(|y|\P(f(X)=y))_{y\in  f(X(\Omega))}$ est sommable et \[
            |y|\P(f(X)=y)=\sum_{x \in  I_y}|f(x)|\P(X=x)
        \] 
        donc $(|f(x)|P(X=x))_{x\in X(\Omega)}$ est sommable.
    \end{enumerate}
\end{proof}

\begin{ex}
    $X\suit G(p)$, $Y=X^2$. La variable $Y$ est-elle d'espérance finie ? La famille $(n^2q^{n-1}p)_n$ est sommable donc $Y$ a une espérance finie et \[
        \E(Y)=\sum_{n\geq 1}n^2q^{n-1}p=\frac{2-p}{p^2}
    \] 
    On peut faire de même avec $Y=\exp (-X)$, ou encore si $X\suit \mathcal  P(\lambda)$, $Y=\cos  X$
\end{ex}

\section{Conséquences du théorème de transfert}

\begin{prop}
    \Hyp $(\Omega, \mathcal  A, \P)$ un espace probabilisé, $X,Y$ deux v.a.r.d, $Y$ d'espérance finie
    \Conc Si  $|X|\leq Y$, alors  $X$ est d'espérance finie et $\E(|X|)\leq \E(Y)$
\end{prop}

\begin{proof}
    On note $Z=(X,Y)$ qui est une v.a.d et  $f:\R^2\to \R,(x,y)\longmapsto x$ de sorte que $X=f(Z)$. Alors  \begin{align*}
        X\text{ d'espérance finie } &\iff (f(x, y)\P(X=x,Y=y))_{(x,y)\in X(\Omega)\times Y(\Omega)}\text{ sommable } \\ &\iff  (x\P(X=x,Y=y))_{(x,y)\in X(\Omega)\times Y(\Omega)}\text{ sommable }
    \end{align*}
    On a \[
        |x\P(X=x,Y=y)|\leq y\P(X=x,Y=y)\tag{$\star$}
    \] 
    Or \[
        [Y=y]=\bigcup_{x \in  X(\Omega)}[X=x,Y=y] \qquad \therefore \qquad  \P(Y=y)=\sum_{x \in  X(\Omega)}\P(X=x,Y=y)
    \]
    donc $(y\P(X=x,Y=y))_{x \in  X(\Omega)}$ est sommable. $Y$ est d'espérance finie donc  \[
        (y\P(Y=y))_{y\in Y(\Omega)}=\left( \sum_{x\in X(\Omega)}y\P(X=x,Y=y) \right) _{y \in  Y(\Omega)}
    \] 
    est sommable, donc $(y\P(X=x,Y=y))_{(x,y)\in  X(\Omega)\times Y(\Omega)}$ est sommable et $(\star)$ conclut sur la sommabilité. Puis  \begin{align*}
        \E(|X|)=\E(|f(Z)|)&=\sum_{\substack{x \in  X(\Omega)\\y\in Y(\Omega)}}|x|\P(X=x,Y=y)\\&\leq \sum_{\substack{x \in  X(\Omega)\\y\in Y(\Omega)}}y\P(X=x,Y=y)\\&\leq \sum_{y\in Y(\Omega)}\sum_{x\in X(\Omega)}y\P(X=x,Y=y)=\E(Y)
    \end{align*}
\end{proof}

\begin{thm}
    \Hyp $(\Omega, \mathcal A, \P)$ espace probabilisé, $\lambda\in \R$ est $X,Y$ des v.a.r.d. d'espérance finie.
     \begin{concenum}
     \item $X+Y$ et $\lambda X$ sont des v.a.r.d d'espérance finie et  $\E(X+Y)=\E(X)+\E(Y)$ et $ \E(\lambda X)=\lambda \E(X)$
     \item $X\geq 0 \implies \E(X)\geq 0$
     \item $X\geq Y\implies\E(X)\geq \E(Y)$
     \item Si $X$ et $Y$ sont indépendantes, $\E(XY)=\E(X)\E(Y)$
    \end{concenum}
\end{thm}

\begin{proof}~
\begin{enumerate}
    \item L'homogénéité est évidente. On montre l'additivité. On pose $Z=(X,Y)$ et  $f:(x,y)\longmapsto x+y$ de sorte que $T=f(Z)$ est une v.a.r.d par le lemme des coalitions.  $T$ est d'espérance finie  si et seulement si $((x+y)\P(X=x,Y=y))_{x,y}$ est sommable.

        La famille $(|x|\P(X=x,Y=y))_{y}$ est sommable car \[
            [X=x]=\bigcup_{y \in  Y(\Omega)} [X=x,Y=y] \qquad \therefore \qquad  \sum_{y \in  Y(\Omega)}|x|\P(X=x,Y=y)=|x|\P(X=x)
        \]
        sommable pour $x$ donc $(|x|\P(X=x,Y=y))_{x,y}$ est sommable, idem pour $(|y|\P(X=x,Y=y))_{x,y}$ donc $X+Y$ est d'espérance finie. Finalement,  \[
            \E(T)=\sum_x \left( \sum_y x\P(X=x,Y=y) \right) +\sum_y \left( \sum_x y\P(X=x,Y=y) \right) =\E(X)+\E(Y)
        \] 
    \item On applique la définition
    \item On utilise $2$ et $1$.
    \item $XY$ d'espérance finie  si et seulement si $(xy\P(X=x,Y=y))_{x,y}$ est sommable soit encore si la famille  $(x\P(X=x)y\P(Y=y))_{x,y}$ est sommable. Le théorème de Fubini discret donne la sommabilité et \[
            \sum_{x,y}xy\P(X=x)\P(Y=y)=\E(X)\E(Y)
    \]
\end{enumerate}
\end{proof}

\begin{ex}
    On note $(E, (\;|\;))$ un espace euclidien,  $v_1,\cdots ,v_n$ des vecteurs unitaires. On veut montrer qu'il existe $\epsilon_1,\cdots ,\epsilon_n\in \{-1,1\} $ tels que \[
        \|\epsilon_1v_1+\cdots +\epsilon_nv_n\|\leq \sqrt{n}
    \] 
    On se donne $X_1,\cdots ,X_n \quad  \indep$ qui suivent une loi de Rademacher\index{loi de Rademacher}, c'est-à-dire $\P(X_i=-1)=\P(X_i=1)=\frac{1}{2}$. On note \[
        N=\|X_1v_1+\cdots +X_nv_n\|^2
    \] 
    On a \[
        \E(N)=\sum_{i=1}^n\underbrace{\E(X_i^2)\|v_i\|^2}_{=1}+\sum_{i\neq  j}\underbrace{\E(X_iX_j)}_{=\E(X_i)\E(X_j)=0}(v_i|v_j)=n
    \]
    Donc $\P(N\leq n)\neq 0$ ce qui conclut.
\end{ex}

\section{Variance}

\begin{thmdef}
    \Hyp $(\Omega, \mathcal  A, \P)$ un espace probabilisé, $X$ une v.a.r.d
     \begin{concenum}
     \item Si $X^2$ est d'espérance finie, alors  $X$ est d'espérance finie et  $E(X)^2\leq E(X^2)$.
     \item Dans ce cas, on appelle \textbf{variance}\index{variance} la quantité $V(X)=\E((X-\E(X))^2)$ et \textbf{écart-type} la quantité $\sigma(X)=\sqrt{V(X)}$ 
     \item $V(X)=\E(X^2)-\E(X)^2$ (König-Huygens\index{Konig-Huygens@König-Huygens})
     \item $V(aX+b)=a^2V(X)$
    \end{concenum}
\end{thmdef}

\begin{proof}~
\begin{enumerate}
    \item $|X|\leq \frac{X^2+1}{2}$ et $\E((X+t)^2)=\E(X^2)+2t\E(X)+t^2\geq 0$ donc $\Delta\leq 0$ ie \conc
        \stepcounter{enumi}
    \item $\E((X-\E(X))^2)=\E(X^2-2X\E(X)+E(X)^2)=\E(X^2)-2\E(X)\E(X)+\E(X)^2=\E(X^2)-\E(X)^2$
\end{enumerate}
\end{proof}

\begin{rem}
On note $X$ une vard de variance non nulle. La variable \textbf{centrée réduite} \[
    Y=\frac{X-\E(X)}{\sigma(X)}
\] 
est telle que $\E(Y)=0$ et $V(Y)=1$
\end{rem}

\begin{rem}
    Si la quantité $\E(X^n)$ existe, on l'appelle \textbf{moment d'ordre $\bm n$} de $X$ \index{moment d'ordre supérieur} 
\end{rem}

\begin{exo}
    Pour $X\suit G(p)$, montrer  $\E(X)=\frac{1}{p}$ et $V(X)=\frac{q}{p^2}$. Pour $X\suit \mathcal  P(\lambda)$, montrer $ \E(X)=\lambda, \quad  V(X)=\lambda$
\end{exo}

\begin{thmdef}
    \Hyp $(\Omega, \mathcal  A, \P)$ un espace probabilisé, $X,Y$ des vard avec variance.
    \begin{concenum}
    \item $X,Y$ sont d'espérances finies et  \[
            \E(XY)^2\leq \E(X^2)\E(Y^2)
    \] 
\item Dans ce cas, on appelle covariance de $X$ et $Y$ la quantité \[
        \Cov(X,Y)=\E((X-\E(X))(Y-\E(Y)))
\] 
et on a \[
    \Cov(X,Y)=\E(XY)-\E(X)\E(Y)
\] 
\item Lorsqu'il existe ($\sigma(X)\sigma(Y)\neq 0$), on appelle coefficient de corrélation le réel \[
        P(X,Y)=\frac{\Cov(X,Y)}{\sigma(X)\sigma(Y)}\in [-1, 1]
\]
    \end{concenum}
\end{thmdef}

\begin{proof}~
    \begin{enumerate}
        \item $|XY|\leq \frac{X^2+Y^2}{2}$ d'où l'existence de l'espérance puis la négativité du discriminant dans $\E((X+tY)^2)\geq 0$ conclut, sauf si $\E(Y^2)=0$, auquel cas $\E((X+tY)^2)$ est une constante, ce qui donne $E(XY)=0$.
        \item C'est la linéarité de l'espérance.
        \item $|\Cov(X,Y)|^2\leq \E((X-\E(X))^2)\E((Y-\E(Y))^2)=\sigma^2(X)\sigma^2(Y)$
    \end{enumerate}
\end{proof}
