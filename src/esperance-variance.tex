\ifsolo
    ~

    \vspace{1cm}

    \begin{center}
        \textbf{\LARGE Espérance -- Variance} \\[1em]
    \end{center}
    \tableofcontents
\else
    \chapter{Espérance -- Variance}

    \minitoc
\fi
\thispagestyle{empty}

\section{Espérance}

\begin{dfn}
    On note $(\Omega, \mathcal  A, \P)$ un espace probabilisé, $X$ une variable aléatoire \textbf{réelle} discrète. On dira que $X$ est d'\textbf{espérance fini}\index{espérance} si  $(x\P(X=x))_{x\in X(\Omega)}$ est sommable. Dans ce cas, on appelle \textbf{espérance} la quantité \[
        \E(X)= \sum_{x \in  X(\Omega)}x\P(X=x)
    \] 
\end{dfn}

\begin{rem}[Rappel]
    Si $X(\Omega)=\{x_n,\quad n\in \N\} $, les $x_n$ deux à deux distincts, alors  \[
        \sum_{n\in \N}|x_n|\P(X=x_n) \text{ CV } \iff \left( x_n\P(X=x_n) \right)_{n \in  \N}\text{ sommable }
    \]
\end{rem}

\begin{ex}~
    \begin{itemize}
        \item $X\suit G(p)$ est d'espérance finie car  $(nq^{n-1}p)_n$ est sommable, et \[
                \E(X)=\sum_{n\geq 0}nq^{n-1}p=\frac{1}{p}
        \] 
    \item $X\suit \mathcal  P(\lambda)$ pour $\lambda>0$, on a  $n\P(X=n)=e^{-\lambda}\frac{n\lambda^n}{n!}$ et $ \E(X)=\lambda$
    \item $X=\1_A$ pour  $A\in \mathcal  A$, $\E(X)=\P(X=1)=\P(A)$
    \end{itemize}
\end{ex}

\section{Formule de transfert}

\begin{thm}
    \Hyp $(\Omega, \mathcal  A, \P)$ un espace probabilisé, $X$ une v.a.r.d et  $f:X(\Omega)\to \R$
    \begin{concenum}
    \item $f(X)$ est une v.a.r.d
    \item  $f(X)$ est d'espérance finie  si et seulement si $(f(x)\P(X=x))_{x\in X(\Omega)}$ et dans ce cas, \[
            \E(f(X))=\sum_{x\in X(\Omega)}f(x)\P(X=x)
    \] 
    \end{concenum}
\end{thm}

\begin{proof}~
    \begin{enumerate}
        \item Lemme des coalitions
        \item Pour $y\in f(X(\Omega))$, on note $I_y=f^{-1}(\{y\} )$. On a \[
                [f(X)=y]=\bigcup_{x\in I_y}[X=x] 
        \] 
        et cette union est disjointe donc \[
            \P(f(X)=y)=\sum_{x\in I_y}\P(X=x).
        \]
        La famille $(I_y)_{y\in f(X(\Omega))}$ forme une partition de $X(\Omega)$. Puis  \[
            \sum_{x \in  I_y}\P(X=x)f(x)=y\sum_{x\in I_y}\P(f(X)=y)=y\P(f(X)=y).
        \] Si la famille $(f(x)\P(X=x))_{x \in  X(\Omega)}$ est sommable, alors pour tout $y\in f(X(\Omega))$, $(f(x)\P(X=x))_{x\in I_y}$ est sommable et $(I_y)$ est une partition donc la famille  \[
        \left( \sum_{x \in  I_y} P(X=x)f(x)\right) _{y\in f(X(\Omega))}
        \] 
        est sommable et \[
            \sum_{x\in X(\Omega)}f(x)\P(X=x)=\sum_{y\in f(X(\Omega))}\sum_{x \in  I_y}f(x)\P(X=x)=\sum_{y\in f(X(\Omega))}y\P(f(X)=y)=\E(f(X)).
        \] 
        Si $f(X)$ est d'espérance finie, alors $(|y|\P(f(X)=y))_{y\in  f(X(\Omega))}$ est sommable et \[
            |y|\P(f(X)=y)=\sum_{x \in  I_y}|f(x)|\P(X=x)
        \] 
        donc $(|f(x)|P(X=x))_{x\in X(\Omega)}$ est sommable.
    \end{enumerate}
\end{proof}

\begin{ex}
    $X\suit G(p)$, $Y=X^2$. La variable $Y$ est-elle d'espérance finie ? La famille $(n^2q^{n-1}p)_n$ est sommable donc $Y$ a une espérance finie et \[
        \E(Y)=\sum_{n\geq 1}n^2q^{n-1}p=\frac{2-p}{p^2}
    \] 
    On peut faire de même avec $Y=\exp (-X)$, ou encore si $X\suit \mathcal  P(\lambda)$, $Y=\cos  X$
\end{ex}

\section{Conséquences du théorème de transfert}

\begin{prop}
    \Hyp $(\Omega, \mathcal  A, \P)$ un espace probabilisé, $X,Y$ deux v.a.r.d, $Y$ d'espérance finie
    \Conc Si  $|X|\leq Y$, alors  $X$ est d'espérance finie et $\E(|X|)\leq \E(Y)$
\end{prop}
