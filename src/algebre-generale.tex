
\ifsolo
    ~

    \vspace{1cm}

    \begin{center}
        \textbf{\LARGE Algèbre générale} \\[1em]
    \end{center}
    \tableofcontents
\else
    \chapter{Algèbre générale}

    \minitoc
\fi
\thispagestyle{empty}

\ifsolo \newpage \setcounter{page}{1} \fi
\section{Groupes}

\textbf{Rappel.} $(G, \star)$ est un groupe\index{groupe} ssi \begin{itemize}
    \item $\star$ est une loi de composition interne associative
    \item $\star$ a un neutre dans $G$
    \item Tous les éléments de $G$ ont un inverse dans $G$ pour $\star$
\end{itemize}

\begin{notation}
    Si $\star$ est commutative, on la notera en général $+$ et l'inverse sera alors noté $-x$. Sinon, on notera souvent la loi $\cdot$ et l'inverse $x^{-1}$. Le neutre de $G$ se note $e_G$ (parfois $0_G$ ou $0$ dans le cas d'une loi notée additivement)
\end{notation}

Si $G$ est de cardinal fini, alors on note $|G|$ ou $\#G$ son cardinal.

\section{Constructeurs de groupes}

\begin{prop}~
    \begin{enumerate}
        \item Si $(G, \cdot)$ est un groupe et $X$ un ensemble non vide alors on muni l'ensemble de fonction de $X$ dans $G$ de la loi $\star$ en posant \[
                \forall f, g: X\longrightarrow G, \quad f\star g:x\longmapsto f(x)\cdot g(x)
            \]
            $(\mathcal F(X, G), \star)$ est alors un groupe.
        \item Soit $(G, \land)$, $(G', \star)$ deux groupes. On munit $G\times G'$ de la loi $\cdot$ en posant \[
                (x, x')\cdot (y, y')=(x\land x', y\star y')
            \]
            $(G\times G', \cdot)$ est alors un groupe.
    \end{enumerate}
\end{prop}

\begin{proof}
    Facile mais pénible.
\end{proof}

\section{Sous-groupes}

\begin{thmdef}
    \Hyp $(G, \cdot)$ un groupe, $H\subset G$
    \Conc On dira que $H$ est un sous-groupe\index{groupe!sous-groupe} de $G$ si $H\neq \emptyset$, $\cdot$ est une l.c.i. sur $H$ et $(H, \cdot)$ est un groupe. Dans ce cas, $e_H=e_G$. Il y a équivalence entre: \begin{enumerate}[label=(\alph*)]
        \item $H$ est un sous-groupe de $G$
        \item $H\subset G$, $H\neq \emptyset$ et $\forall x, y\in H, x\cdot y^{-1}\in H$
    \end{enumerate}
\end{thmdef}

\begin{notation}
    $(G, \cdot)$ groupe, $H\subset G$, $a\in G$. \[
        aH=\{a\cdot h, \quad h\in H\}
    \]
    \[
        H^{-1}={h^{-1}, \quad h\in H}
    \]
    et si $A\subset G$,
    \[
        AH = \{a\cdot h, \quad a\in A, h\in H\}
    \]
\end{notation}

Ainsi $H\subseteq G$ est un sous-groupe de $G$ si et seulement si $H\neq \emptyset$ et $HH^{-1}\subseteq H$

\begin{exo}
    Soient $H, K$ des sous-groupes de $G$. Montrer que $HK$ est un sous groupe si et seulement si $HK=KH$.
\end{exo}

\begin{proof}[Résolution]
    Si $HK$ sous-groupe de $G$ alors $HK=(HK)^{-1}=K^{-1}H^{-1}=KH$. Si $HK=KH$ alors $HK(HK)^{-1}=HKK^{-1}H=HKKH=HHKK\subseteq HK$
\end{proof}

\subsection{Centre d'un groupe}

\begin{dfn}
    Le centre d'un groupe\index{centre (d'un groupe)}\index{groupe!sous-groupe!centre} $G$ est le groupe \[
        Z(G)=\{x\in G, \quad \forall y\in G, xy=yx\}
    \]
\end{dfn}

\begin{rem}
    C'est un sous-groupe de $G$, on peut le vérifier manuellement.
\end{rem}

\subsection{Normalisateur}

On note $H$ un sous-groupe de $G$. On appelle \textbf{normalisateur}\index{normalisateur} de $H$ l'ensemble \[
    N(H)=\{x\in G, \quad xHx^{-1}=H\}
\]
On peut vérifier que c'est un sous-groupe de $G$. On dira que $H$ est un sous-groupe distingué \index{groupe!sous-groupe!distingué} de $G$ si $N(H)=G$.

\begin{rem}
    On note $\mathcal R$ la relation \[
        x\mathcal Ry\iff y^{-1}x\in H
    \]
    C'est bien une relation d'équivalence (réfléxive symétrique transitive) et si $a\in H$, $x\in \bar a \iff x\in aH$. De même, $x\mathcal R'y\iff xy^{-1}\in H$ définit une autre relation d'équivalence dont les classes sont les $Ha$. Lorsque $H$ est distingué (on notera $H\triangleleft G$), $aH=Ha$
\end{rem}

\subsection{Sous-groupe de torsion}

Soit $G$ un groupe \emph{abélien}. On note \[
    \tau(G)=\{g\in G, \quad \exists n\in\mathbb N^\star, g^n=e_G\}
\]
C'est un sous-groupe\footnote{on a besoin de l'hypothèse de commutativité de $G$, ça n'est pas vrai sinon} de $G$ appelé sous-groupe de torsion\index{groupe!sous-groupe!de torsion}.

\subsection{Sous-groupes de $\mathbb Z$ et $\mathbb R$}

On va montrer que les sous-groupe de $\mathbb Z$ sont de la forme $d\mathbb Z$, et $d$ est unique au signe près.
On va pour cela montrer la propriété plus générale suivante (hors-programme): si $A$ est un anneau euclidien, alors c'est un anneau principal.

\begin{dfn}[Hors-Programme]
    Un anneau $A$ commutatif est dit euclidien\index{anneau!euclidien} lorsqu'il existe une application $v:A\setminus\{0\} \to \mathbb N$ qui satisfait les propriétés suivantes:
    \begin{enumerate}
        \item \[
                \forall (a, b)\in A\times A\setminus\{0\}, \exists q, r\in A, \qquad a=qb+r\quad \text{ et } \quad r=0\text{ ou } v(r)<v(b)
            \]
        \item \[
                \forall a, b\in A\setminus\{0\}, \qquad v(b)\leq v(ab)
            \]
    \end{enumerate}
    Une telle application sera appelée \textbf{stathme euclidien}\index{anneau!euclidien!stathme, préstathme} sur $A$. Une application qui ne satisfait que la première propriété sera appelée \textbf{préstathme euclidien} sur $A$.

    Un anneau $A$ commutatif sera dit principal\index{anneau!principal} lorsque tous ses idéaux\footnote{notion abordée plus en détail dans la section \textbf{Idéal d'un anneau}} (sous-groupes $I$ de $(A, +)$ tels que pour tout $a\in A$, $aI\subset I$) sont principaux (i.e. de la forme $aA$)
\end{dfn}

L'anneau $\mathbb Z$ est euclidien (la valeur absolue est un stathme euclidien sur $\mathbb Z$). Si $H$ est un sous-groupe de $\mathbb Z$ alors pour tout $k\in\mathbb Z$, $kH=H+H+\cdots +H\subset H$ donc $H$ est un idéal de l'anneau $\mathbb Z$. Il sera donc suffisant de montrer que tout anneau euclidien est principal.

\begin{prop}[Hors-Programme]
    Un anneau euclidien est principal
\end{prop}

\begin{proof}
    Soit $A$ un anneau commutatif euclidien (de stathme $v$), et $I$ un idéal de $A$. On va montrer que $I$ s'écrit $aA$ pour un $a\in A$. Si $I=\{0\}$ alors $a=0$ convient.

    On suppose $I\neq \{0\}$. L'ensemble $v(I\setminus\{0\})$ admet un minimum, on note $a\in I\setminus\{0\}$ un élément minimal pour $v$. On a clairement $aA\subset I$ car $a\in I$ et $I$ idéal. On prend $b\in I$ et \[
        \exists q, r, \qquad b=qa+r\quad \text{ et }\quad r=0\text{ ou }v(r)<v(a).
    \]
    On a $r=b-qa\in I$ donc $v(r)<v(a)$ est impossible par définition de $a$, donc $r=0$ et $b=qa$. On a donc $I\subset aA$ d'où $aA=I$
\end{proof}

\begin{rem}
    Si $I$ est un idéal principal, il s'écrit $I=aA$. S'il s'écrit également $a'A$ alors $a\;|\;a'$ et $a'\;|\;a$ donc $a=\varepsilon a'$ pour $\varepsilon\in\mathcal U(A)$ (le groupe des inversibles de $A$). Dans $\mathbb Z$, les inversibles sont $-1$ et $1$, d'où l'unicité du générateur positif.
\end{rem}

\begin{rem}
    On aurait pu écrire la démonstration directement pour $A=\mathbb Z$, $v:x\mapsto |x|$
\end{rem}

On va maintenant montrer qu'un sous groupe $H$ non trivial de $\mathbb R$ est soit dense dans $\mathbb R$ soit de la forme $a\mathbb Z$. On note $\ell=\inf H\cap \mathbb R_+^\star$. Il y a deux cas: \begin{itemize}
    \item $\ell>0$. Si $l\not\in H$ alors il existe $a\in\mathbb R_+^\star$ tel que $a\in ]\ell, 2\ell[$ et $a\in H$, il existe $b\in H$ tel que $b\in ]\ell, a[$ donc $a-b\in H\cap \mathbb R_+^\star$ ce qui est absurde car $a-b>\ell$ par définition de $\ell$. Donc $\ell \in H$ et $\ell \mathbb Z\subset H$. Pour l'autre inclusion, fait comme pour $\mathbb Z$: on écrit $x=\lfloor x/\ell \rfloor \ell + r$ et $r=0$.
    \item $\ell=0$. Soient $0<x<y$. Il existe $\alpha\in H\cap \mathbb R_+^\star$ tel que $0<\alpha<y-x$ et \[
            x<\underbrace{\floor{\frac x\alpha}+\alpha}_{\in H}<x+y-x=y
        \]
        donc $H$ est dense dans $\mathbb R$ (les autres configurations de $x$ et $y$ se traitent similairement)
\end{itemize}

Par exemple, $a\mathbb Z+b\mathbb Z$ est dense dans $\mathbb R$ si et seulement si $\frac ab\not\in \mathbb Q$, de sorte que $ \cos(\mathbb N) $ est dense dans $\mathbb R$ car $\mathbb Z+2\pi\mathbb Z$ est dense dans $\mathbb R$

\section{Opérations sur les sous-groupes}

\begin{prop}
    \Hyp $(H_i)_{i\in I}$ famille de sous-groupes de $G$
    \Conc L'ensemble \[
        \bigcap_{i\in I}H_i
    \]
    est un sous-groupe de $G$
\end{prop}

\begin{rem}
    En général, l'union de deux sous-groupes n'est pas un sous-groupe
\end{rem}

\section{Sous-groupes engendrés}

\begin{thmdef}
    \Hyp $(G, \cdot)$ un groupe, $A\subset G$, $S_A=\{H\in\mathcal P(G), A\subset H, H\text{ sg de }G\}$
    \begin{concenum}
    \item On appelle sous-groupe engendré\index{groupe!engendré} par $A$ l'ensemble \[
            \langle A\rangle = \bigcap_{H\in S_A}H
        \]
    \item Si $H$ est un sous-groupe de $G$ contenant $A$ alors $\langle A\rangle \subseteq H$
    \item \[
            \langle A\rangle =\{e_G\}\cup \{x_1\cdots x_n, \quad n\in\mathbb N, x_1, \cdots, x_n\in A\cup A^{-1}\}
        \]
    \end{concenum}
\end{thmdef}

\begin{proof}~
    \begin{enumerate}
        \setcounter{enumi}{2}
    \item On note $E=\{e_G\}\cup \{x_1\cdots x_n, \quad n\in\mathbb N, x_1, \cdots, x_n\in A\cup A^{-1}\}$. On a \[
            E\subset \langle A\rangle\qquad \text{ car }A,A^{-1}\subset \langle A\rangle
        \]
        et $A\subset E$ donc $\langle A\rangle \subset E$ par 2. donc $E=\langle A\rangle$
\end{enumerate}
\end{proof}

\begin{ex}
    \begin{itemize}
        \item $\langle \exp \left( \frac{2i\pi}n \right)\rangle=\mathbb U_n$
        \item $\mathbb Z=\langle 1\rangle = \langle 2, 3\rangle$
        \item $\mathfrak S_n=\langle \tau_{1,2}, (1\;\cdots\; n)\rangle$
    \end{itemize}
\end{ex}

\begin{dfn}
    Un groupe $G$ est dit monogène\index{groupe!monogène} s'il existe $a\in G$ tel que $G=\langle a\rangle$. Si de plus $G$ est fini, on dira qu'il est cyclique\index{groupe!cyclique}
\end{dfn}


\section{Morphismes}

\begin{dfn}
    Soient $(G,\cdot), (G',\star)$ deux groupes. Une application $f:G\to G'$ est un morphisme ssi \[
        \forall x, y\in G, f(x\cdot y)=f(x)\star f(y)
    \]
\end{dfn}

\begin{notation}
    $f:G\to G'$ est un endomorphisme ssi c'est un morphisme. C'est un isomorphisme si c'est une bijection. Si $G=G'$, on dira qu'un morphisme de $G$ dans $G'$ est un endomorphisme de $G$. Un endomorphisme bijectif est appelé automorphisme. On note $\Hom (G, G')$ l'ensemble des morphisme de $G$ dans $G'$ et $\Aut(G)$ l'ensemble des automorphismes de $G$.\index{groupe!isomorphisme} \index{groupe!endomorphisme}
\end{notation}

\begin{prop}
    \Hyp $f\in\Hom(G, G')$
    \begin{concenum}
    \item $f(e_G)=e_G$
    \item $\forall x\in G, \quad f(x^{-1})=f(x)^{-1}$
    \item On note $\Ker f=f^{-1}(\{e_G\})$. \begin{enumerate}
            \item $\Ker f$ est un sous groupe de $G$
            \item $f$ est injective si et seulement si $\Ker f=\{e_G\}$
        \end{enumerate}
    \item $f^{-1}$ est un isomorphisme si $f$ est un isomorphisme.
    \item $g\circ f$ est un morphisme si $f:G\to G'$ et $g:G'\to G''$ le sont
    \end{concenum}
\end{prop}

\begin{csq}
    $(\Aut(G), \circ)$ est un groupe
\end{csq}

\begin{csq}
    Les $i_a:x\in G\longmapsto axa^{-1}$ pour $a\in G$ sont appelés automorphismes intérieurs\index{groupe!automorphismes intérieurs}. $\varphi:a\longmapsto i_a$ est un morphisme.
\end{csq}

Les morphismes ont la propriété intéressante suivante:

\begin{prop}
    L'image et la préimage d'un groupe par un morphisme est un groupe.
\end{prop}

\begin{proof}
    Facile mais pénible
\end{proof}

\section{Groupes finis}

\subsection{Théorème de Lagrange}

\index{Lagrange!théorème}

On note $G$ un groupe fini et $H$ un sous groupe de $G$. On va montrer que $|H|$ divise $|G|$. On a vu que \[
    x\mathcal Ry\iff x^{-1}y\in H
\]
définit une relation d'équivalence sur $G$, les classes d'équivalence forment donc une partition de $G$. On a ainsi ($\sqcup$ désigne l'union disjointe) \[
    G=\bigsqcup_{i=1}^p a_iH \implies |G|=p|H|
\]

\subsection{\texorpdfstring{$|G|=|\Ker f|\cdot |\Img f|$}{\#G=\#Ker f × \#Im f}}

On note $G$ un groupe fini et $f:G\to G'$ un morphisme. On note $\Img f=\{y_1, \cdots, y_p\}$ de sorte que \[
    G=\bigsqcup_{i=1}^p f^{-1}(\{y_i\}) \implies |G|=\sum_{i=1}^p \left| f^{-1}(\{y_i\}) \right|
\]
On se donne $y\in \Img f$ fixé et $x\in f^{-1}(\{y\})$. On pose \[
    \varphi: t\in \Ker f\longmapsto xt\in f^{-1}(\{y\}).
\]
Cette application est injective et si $u\in f^{-1}(\{y\})$ alors $u=\varphi(x^{-1}u)$ donc elle est aussi surjective. Elle est donc bijective, d'où \[
    |f^{-1}(\{y\})|=|\Ker f|\quad \implies \quad |G|=|\Ker f|\cdot |\Img f|
\]

\subsection{Ordre d'un élément}

\begin{defprop}
    \Hyp $G$ un groupe fini, $a\in G$
    \begin{concenum}
    \item $\varphi:\mathbb Z\longrightarrow G, n\longmapsto a^n$ est un morphisme. Il existe un unique entier naturel non nul tel que $\Ker \varphi=d\mathbb Z$, qu'on appelle \textbf{ordre} de $a$ dans $G$\index{groupe!ordre}, noté $\ord_G (a)$ ou $\ord (a)$
    \item $\ord (a)=|\langle a\rangle|=\min\{n\in\mathbb N^\star, a^n=e_G\}$
    \item $a^n=e_G \iff \ord(a)\;|\; n$ et \[
            \ord(a^p)=\frac{\ord (a)}{p\land \ord (a)}
        \]
    \end{concenum}
\end{defprop}

\begin{proof}~
    \begin{enumerate}
        \item $\varphi$ n'est pas injective car $G$ fini donc $\Ker \varphi$ est un sous-groupe non trivial de $\mathbb Z$, ce qui conclut
        \item $d=\min \Ker\varphi\cap \mathbb N^\star=\min\{n\in\mathbb N^\star, a^n=e_G\}$ puis \[
                a^k=a^{dq+r}=a^r
            \] avec $r\in\llbracket 0, d-1\rrbracket$ donc $\langle a\rangle = \{e_G, a, \cdots, a^{d-1}\}$ et pour $0\leq r_1<r_2<d$, $a^{r_1}=a^{r_2} \implies r_2-r_1\in d\mathbb Z\cap \rrbracket 0, d-1 \llbracket $ absurde donc $|\langle a \rangle |=d$
        \item $a^n=e_G\iff n\in\Ker \varphi=d\mathbb Z\iff d\;|\;n$ puis \[
                \left( a^p \right)^{\frac{\ord (a)}{p\land \ord(a)}}=a^{k\ord(a)}=e_G\quad \text{ donc }\quad \ord(a^p)\;\Big|\;\frac{\ord(a)}{p\land\ord(a)}
            \]
            et \[
                (a^p)^{\ord (a^p)}=e_G \implies \ord(a)\;\Big|\;p\ord (a^p) \implies \frac{\ord(a)}{p\land \ord(a)}\;\Big|\; \frac{p}{p\land \ord(a)}\times \ord(a^p)
            \]
            donc par le lemme de Gauss, \[
                \frac{\ord(a)}{p\land \ord(a)}\;\Big|\; \ord(a^p)
            \]
            d'où l'égalité
    \end{enumerate}
\end{proof}

\begin{rem}
    Si $G$ est fini abélien et $a, b\in G$, \[
        (ab)^{\ord(a)\lor \ord(b)}=e_G\implies \ord (ab)\;|\; \ord(a)\lor \ord(b)
    \]
    Si $\ord (a)$ et $\ord (b)$ sont premiers entre eux alors \[
        (ab)^{\ord(ab)\ord (b)}=e_G=a^{\ord(ab)\ord(b)}\implies \ord (a)\;|\;\ord (b)\ord(ab)\underset{\text{Gauss}}\implies \ord(a)\;|\;\ord(ab)
    \]
    et par symétrie, $\ord(b)\;|\;\ord (ab)$ don $\ord(ab)=\ord(a)\ord(b)$
\end{rem}

\begin{prop}
    \Hyp $(G, \cdot)$ groupe fini, $a\in G$
    \Conc $\ord(a)\;|\;\# G$
\end{prop}

\begin{proof}
    C'est évident avec le théorème de Lagrange. Sinon, \[
        \prod_{x\in G}x=\prod_{x\in G}ax=a^{|G|}\prod_{x\in G}x \implies a^{|G|}=e_G
    \]
\end{proof}

\subsection{Groupe de cardinal $p^2$}

On note $p$ un nombre premier et on suppose que $G$ est un groupe d'ordre $p^2$. Les seuls ordres possibles sont $1, p$ et $p^2$. S'il y a un élément d'ordre $p^2$, alors $G=\langle a\rangle$ est abélien et $a^p$ est d'ordre $p$. Sinon, tous les éléments sont d'ordres $p$. Dans les deux cas, il y a un élément d'ordre $p$.

\subsection{Groupe de cardinal divisible par $p$}

On note $p$ premier et $G$ un groupe d'ordre $kp$. On va montrer qu'il existe un élément d'ordre $p$. Il suffit de montrer qu'un élément est d'ordre $p$. On note \[
    E=\{(x_1, \cdots, x_p)\in G^p, \quad x_1 \cdots x_p=e_G\}\qquad \qquad \#E=\#G^{p-1}.
\]
Cet ensemble est stable par permutation circulaire (car $x_1\cdots x_p=e_G \implies x_2\cdots x_px_1=e_G$). On note $\sigma$ la permutation circulaire $(1\;\; 2 \;\; \cdots \;\; p)$. On va faire agir $\langle \sigma \rangle$ sur $E$. Pour $C\in \mathfrak S_n$, $x=(x_1, \cdots, x_p)\in E$, on note $C.x=(x_{\sigma(1)}, \cdots, x_{\sigma(p)})$.

Pour $x\in E$, on note $O_x=\{C.x, C\in\langle \sigma\rangle\}$ l'orbite de $x$. La relation $x\mathcal Ry\iff y\in O_x$ est une relation d'équivalence sur $E$ de sorte que les $O_x$ forment une partition de $E$. On note \[
    \varphi_x:c\in \langle \sigma \rangle \longmapsto c.x\in O_x.
\]
Si $\varphi_x$ est injective alors $O_x=\Img \varphi_x$ et $|O_x|=|\langle \sigma \rangle|=p$. Sinon, il existe $1\leq k_1 < k_2\leq p$ tel que $\sigma^{k_1}.x=\sigma^{k_2}.x$ donc $\sigma^{k_2-k_1}.x=x$. On a alors $\forall n\geq 0, \sigma^n.x=\sigma^{n+k_2-k_1}.x$ donc $\forall i\in\llbracket 1, p\rrbracket$, $x_i=x_{i+n(k_2-k_1)\pmod p}$ or $k_2-k_1$ engendre $(\mathbb Z_p, +)$ (car $p$ premier) donc $(n(k_2-k_1))_n$ parcourt tous les entiers modulo $p$ et $x=(x_1, \cdots, x_1)$, d'où finalement $\#O_x=1$

Ainsi, les orbites sont de cardinal $p$ ou $1$. On note $O_{x_1}, \cdots, O_{x_r}$ les orbites de cardinal $p$ et $O_{x_1'}, \cdots, O_{x_s'}$ celles de cardinal $1$. On a \[
    \#E=rp+s=\#G^{p-1}\equiv 0\pmod p \implies r\equiv 0\pmod p
\]
donc $r\geq p\geq 2$ et il existe $x=(a, \cdots, a)\neq (e_G, \cdots, e_G)$ dans $E$, donc $a^p=e_G$ et $a\neq e_G$ donc $\ord (a)=p$ (car $p$ premier n'est divisible que par $1$ et $p$).

\subsection{Moyenne dans un groupe}

On note $E$ un $K$-e.v., $G$ un sous-groupe de $\mathrm{GL}(E)$ fini d'ordre $n$. On note \[
    p= \frac{1}{|G|} \sum_{g\in G} g
\]
Si $g_0\in G$, alors $g\longmapsto g_0\circ g$ est bijective et \[
    g_0 p=\frac1{|G|}\sum_{g\in G}g_0 g=p=pg_0
\]
donc \[
    p\circ p=p\circ \left( \frac1{|G|}\sum_{g\in G} g\right)=\frac1{|G|}\sum_{g\in G}\underbrace{pg}_{=p}=\frac1{|G|}|G|p=p
\]
donc $p$ est un projecteur


\begin{exo}
    \begin{itemize}
        \item ~[3/2] On note $f\in \mathcal L(E)$, \[
                f^\star = \frac1{|G|}\sum_{g\in G}g^{-1}\circ f\circ g
            \]
            Montrer que $f^\star$ commute avec les éléments de $G$ et $f^{\star\star}=f^\star$
        \item ~[5/2] Montrer que si $F$ est un s.e.v stable par tous les éléments de $G$ alors il a un supplémentaire stable par tous les éléments de $G$.
    \end{itemize}
\end{exo}

\section{Groupes cycliques}

\begin{dfn}
    Un groupe cyclique est un groupe fini monogène\index{groupe!cyclique}
\end{dfn}

\subsection{Sous-groupes d'un groupe cyclique}

On note $G=\langle a\rangle$ cyclique d'ordre $n$. On note $H$ un sous-groupe de $G$, et $r=\min\{p\in\mathbb N^\star, a^p\in H\}$. On a \begin{itemize}
    \item $\langle a^r\rangle\subseteq H$
    \item Soit $x=a^m\in H$. On écrit $m=rq+s$, $s\in\llbracket 0, r-1\rrbracket$ de sorte que $a^s\in H$ donc $s=0$ et donc $x\in \langle a^p\rangle=H$
\end{itemize}

Ainsi, $H$ est cyclique.

\subsection{Générateurs d'un groupe cyclique}

Avec les mêmes notations, on considère\index{Euler!indicatrice} \[
    \varphi: n\longmapsto \{k\in\llbracket 1, n\rrbracket, \quad k\land n=1\}.
\]
On appelle cette application l'indicatrice d'Euler.
$g=a^p$ est un générateur de $G$ ssi $\ord(a^p)=n$ ssi $n=\frac{n}{p\land n}$ ssi $p\land n=1$. Il y a donc $\varphi(n)$ générateurs de $G$.

Plus généralement, il y a $\varphi(d)$ éléments d'ordre $d$ (on peut le voir en appliquant ce raisonnement à un sous-groupe d'ordre $d\;|\; n$). On a ainsi \[
    \sum_{d\;|\; n}\varphi(d)=n
\]

\subsection{Sous-groupes d'ordre $d$}

On note $H$ un sous-groupe de cardinal $d$ de $G$ (donc $d$ divise $n$).
\begin{itemize}
    \item $H$ a $\varphi(d)$ générateurs (donc d'ordre $d$)
    \item $G$ a $\varphi(d)$ éléments d'ordre $d$
\end{itemize}
donc $H$ contient tous les éléments d'ordre $d$, il est donc unique.

\section{Le groupe \texorpdfstring{$\mathfrak S_n$}{symétrique d'ordre n}}

Le $n$-ième groupe symétrique\index{groupe!symétrique} $(\mathfrak S_n, \circ)$ est un groupe d'ordre $n!$. On note $\tau_{i, j}$ la transposition de support $\{i, j\}$. Si $\sigma \in \mathfrak S_n$, alors $\sigma$ s'identifie au $n$-uplet \[
    (\sigma(1), \cdots, \sigma(n))
\]

On note $c=(i_1\;\;i_2\;\;\cdots \;\;i_p)$ la permutation \[ 
    c: n \in\llbracket 1, n\rrbracket \longmapsto \begin{cases}
        i_{k+1}& \text{ si }n=i_k, k<p\\
        i_1 &\text{ si }n=i_p\\
        n & \text{sinon}
    \end{cases}
\] 
On dira que $c$ est un cycle de support $\{i_1, \cdots, i_p\}$ et d'ordre\footnote{il s'agit bien de l'ordre au sens de $\ord_{\mathfrak S_n}(c)$} $p$ (on prend les $i_k$ deux à deux distincts)

\subsection{Décomposition en cycles}

On note $\sigma \in\mathfrak S_n$ et $O_1, \cdots O_p$ les orbites pour l'action de $\langle \sigma \rangle$ sur $\llbracket1, n\rrbracket$ (c'est à dire les classes d'équivalence pour $x\mathcal Ry\iff \exists k, y=\sigma^k(x)$)

Les orbites s'écrivent $O_k=\{i_k, \sigma(i_k), \cdots, \sigma^{p_k}(i_k)\}$. On note $c_k=(i_k\;\;\sigma(i_k)\;\;\cdots\;\;\sigma^{p_k}(i_k))$ de sorte que \[
    \sigma=c_1\circ \cdots \circ c_p
\]
On a alors \[
    \sigma^k=\id \iff c_1^k=\cdots =c_p^k=\id \iff \ord(c_1)\lor \cdots \lor \ord(c_p)\;|\;k
\]

\subsection{Signature}

Une inversion pour $\sigma \in \mathfrak S_n$ est un couple $(i, j)\in\llbracket 1, n\rrbracket^2$ tel que $i<j$ et $\sigma(j)<\sigma(i)$. On note $N_\sigma$ le nombre de ces inversions et \[
    \varepsilon: \sigma\longmapsto (-1)^{N_\sigma}
\]
est appelée \textbf{signature}\index{signature (d'une permutation)} de $\sigma$

\begin{prop}
    \begin{enumerate}
        \item \[
                \forall \sigma\in\mathfrak S_n, \qquad \varepsilon(\sigma)=\prod_{i<j}\frac{\sigma(i)-\sigma(j)}{i-j}=\prod_{\{i, j\}\in \mathcal P_2(\llbracket 1, n\rrbracket)}\frac{\sigma(i)-\sigma(j)}{i-j}
            \]
        \item $\varepsilon:\mathfrak S_n\to \mathbb U_2$ est un morphisme
        \item La signature d'une transposition vaut $-1$
    \end{enumerate}
\end{prop}

\begin{proof}~
    \begin{enumerate}
        \item La seconde égalité est claire. Puis, \[
                \left| \prod_{i<j}\frac{\sigma(i)-\sigma(j)}{i-j} \right|=\frac{\prod_{i<j}|\sigma(i)-\sigma(j)|}{\prod_{i<j}|i-j|}=1
            \]
            et le signe est bon.
        \item $\sigma, \sigma'\in\mathfrak S_n$. \[
                \varepsilon(\sigma \circ \sigma')=\prod_{i<j}\frac{\sigma\circ \sigma'(i)-\sigma\circ \sigma'(j)}{\sigma'(i)-\sigma'(j)}\times \frac{\sigma'(i)-\sigma'(j)}{i-j}=\varepsilon(\sigma)\varepsilon(\sigma')
            \]
        \item On compte les inversions de $\tau_{1, 2}$ et $\sigma\circ \tau_{i, j}\circ \sigma^{-1}=\tau_{1, 2}$.
    \end{enumerate}
\end{proof}

\subsection{Théorème de Cayley}

On note $G=\{g_1, \cdots, g_n\}$ un groupe fini. On a \[
    \varphi: G\to \mathfrak S(G), x\longmapsto (t\longmapsto xt)
\]
Cette application est injective car \[
    \varphi(x)=\varphi(x')\implies \varphi(x)(e_G)=\varphi(x')(e_G)\implies x=x'
\]
C'est un morphisme car $\varphi(xx')(t)=xx't=x\varphi(x')(t)=\varphi(x)\circ \varphi(x')(t)$.
Ainsi, $G$ est isomorphe à $\varphi(G)$ sous-groupe de $\mathfrak S(G)$ qui est isomorphme à $\mathfrak S_n$. Donc, $G$ est isomorphe à un sous-groupe de $\mathfrak S_n$.

\section{Anneaux et corps}

\begin{dfn}
    \index{anneau} Un ensemble $A$ muni des lois $+$ et $\times$ est un anneau si \begin{itemize}
        \item $(A, +)$ est un groupe abélien de neutre $0_A$
        \item $(A\setminus\{0\}, \times)$ est un monoïde\footnote{mêmes hypothèses que la structure de groupe sans l'inversibilité} de neutre $1_A$
        \item $\times$ est distributive par rapport à $+$
    \end{itemize}
\end{dfn}

\begin{dfn}
    Un anneau commutatif est intègre\index{anneau!intègre} si tous ses éléments sont réguliers pour $\times$
\end{dfn}

\begin{dfn}
    $(A, +, \times)$ est un corps\index{corps} si $(A\setminus\{0\}, \times)$ est un groupe abélien.
\end{dfn}

\begin{rem}
    Un anneau intègre fini est un corps, car $\forall a\in A\setminus\{0\}, x\longmapsto ax$ est injective donc bijective donc tous les éléments sont inversibles.
\end{rem}

\subsection{Calcul dans un anneau}

\begin{prop}[Binôme de Newton]
    \Hyp $a, b\in A$ un anneau, $ab=ba$.
    \Conc \index{Newton!binôme}\[
        (a+b)^n=\sum_{k=0}^n\binom nk a^kb^{n-k}
    \]
\end{prop}

\begin{proof}
    Vu en sup, facile mais long.\footnote{«Moralement, c'est vrai»}
\end{proof}

\begin{rem}
    On prend la convention \[
        \forall k\not\in\llbracket 0, n\rrbracket, \qquad \binom nk=0
    \]
\end{rem}

La formule du binôme se généralise ainsi: pour $a_1, \cdots, a_p\in A$ qui commutent $2$ à $2$ et pour $n\in\mathbb N$, \[
    (a_1+\cdots +a_p)^n=\sum_{a_1+\cdots +a_p=n}\binom{n}{n_1, \cdots, n_p}a_1^{n_1}\cdots a_p^{n_p}
\]
avec \[
    \binom{n}{n_1, \cdots, n_p}=\frac{n!}{n_1!\cdots n_p!}
\]
Pour le voir, on va développer $(a_1+\cdots +a_p)\times\cdots\times(a_1+\cdots +a_p)$. On obtient une somme de termes en $a_1^{n_1}\cdots a_p^{n_p}$ avec $n_1+\cdots +n_p=n$. \begin{itemize}
    \item Il y a $\binom n{n_1}$ blocs pour $a_1$
    \item Il y a $\binom{n-n_1}{n_2}$ pour $a_2$
    \item $\cdots$
    \item Il y a $\binom{n-n_1-\cdots -n_{p-1}}{n_p}$ blocs pour $a_p$
\end{itemize}

Donc les coefficients valent \[
    \prod_{i=1}^p\binom{n-n_1-\cdots -n_{i-1}}{n_i}=\frac{n!}{n_1!\cdots n_p!}=\binom n{n_1,\cdots, n_p}
\]

\textbf{Techniques pour les sommes de coefficients binomiaux}
\begin{itemize}
    \item $(1+X)^p(1+X)^q=(1+X)^{p+q}$ donc \[
    \binom{p+q}\ell=\sum_{i+j=\ell}\binom pi\binom qj
\]
En particulier, \[
    \binom {2n}n=\sum_{i=0}^p\binom pi^2
\]
\item Si on note $S:x\longmapsto (1+x)^n$ \[
        S'(x)=n(1+x)^{n-1}=\sum_{k=0}^n\binom nk kx^{k-1}
    \]
    donc \[
        \sum_{k=0}^n\binom nkk=n2^{n-1}
    \]
\item \textbf{Sommes lacunaires}: On note $p\in\mathbb N^\star$ fixé, on va calculer \[
        S_n=\sum_{pk\leq n}\binom n{pk}
    \]
    On note $\omega=\exp \left(  \frac{2i\pi}p\right)$ de sorte que \[
        1+\omega^k+\cdots +\omega^{(p-1)k}= \begin{cases}
            p&\text{ si } w^k=1 \text{ i.e. } p\;|\; k \\ 0 &\text{ sinon }
        \end{cases}
    \]
    On pose $f(x)=(1+x)^n$ et \[
        f(1)+f(\omega)+\cdots +f(\omega^{p-1})=\sum_{k=1}\binom nk(1+\cdots +\omega^{k(p-1)})=pS_n
    \]
    d'où \[
        S_n=\frac1p \left( f(1)+\cdots +f(\omega^{p-1}) \right)
    \]
\end{itemize}

\subsection{Élément nilpotent}

\begin{dfn}
    Un élément $a\in A$ ($A$ est un anneau) est \textbf{nilpotent}\index{nilpotence} si il existe un entier naturel $n$ tel que $a^n=0_A$. On appelle \textbf{indice de nilpotence}\index{nilpotence!indice de nilpotence} l'entier naturel minimal tel que $a^n=0_A$
\end{dfn}

Si $a, b$ sont nilpotents et commutent alors \[
    (a+b)^{n+m}=\sum_{k=0}^{n+m}\binom{n+m}k a^kb^{n+m-k}=0_A
\]
avec $n$ et $m$ les indices de nilpotence respectifs de $a$ et $b$. Ainsi, $a+b$ est nilpotent. Si $A$ est un anneau commutatif alors l'ensemble $\mathcal N$ des éléments nilpotents est un idéal de $A$. Pour $a$ nilpotent, $a^n=0_A$ et \[
    1=1-a^n=(1-a)(1+a+\cdots +a^{n-1})
\]
donc $a-1$ est inversible.

\section{Constructeurs d'anneaux, sous-anneaux, sous-corps}

Comme pour les groupes, si $A, A'$ sont des anneaux et $X$ un ensemble non vide, $A\times A'$ a une structure d'anneau et $\mathfrak F(X, A)$ aussi

\begin{defprop}
    \begin{enumerate}
        \item On dira que $A'$ est un sous-anneau de $A$ si \begin{itemize}
            \item $A'\subset A$
            \item $1_A\in A$
            \item $\forall x, y\in A', x-y\in A'$ et $xy\in A'$
        \end{itemize}
    \item On dira que $K'$ est un sous-corps de $K$ si \begin{itemize}
        \item $K'\subset K$
        \item $1_K\in K$
        \item $\forall x, y\in K', x-y\in K'$ et si $y\neq 0$, $xy^{-1}\in K'$
    \end{itemize}
    \end{enumerate}
\end{defprop}

\begin{ex}~
    \begin{itemize}
        \item $\mathbb Z[\sqrt 2]$ est un sous-anneau de $\mathbb R$
        \item $\mathbb Q[\sqrt 2]$ est un sous-corps de $\mathbb R$ qu'on note $\mathbb Q(\sqrt 2)$
        \item $\mathbb Q[\omega]$ pour $\omega$ algébrique est un sous-corps de $\mathbb R$ (c'est un espace vectoriel de dimension finie et $\varphi: t\longmapsto tx$ est injective donc bijective car linéaire ce qui montre l'inversibilité)
    \end{itemize}
\end{ex}

\begin{rem}
    Si on fixe $n$ et on note \[
        S_n=\left\{\sum_{i=1}^nx_i^2\;, \qquad x_1, \cdots, x_n\in K\right\}
    \]
    alors on peut montrer que $S_n$ est stable par $\times$ si $n=2^k$
\end{rem}

\section{Idéal d'un anneau}

\begin{dfn}
    Un idéal\index{idéal} à gauche (resp. à droite) $I\subset A$ d'un anneau $A$ est un sous-groupe de $(A, +)$ tel que \[
        \forall a\in A, x\in I,\quad  ax\in I\quad \text{(resp. $xa\in I$)}
    \]
    Un idéal bilatère (ou simplement idéal) est un idéal à gauche et à droite.
\end{dfn}


\begin{rem}
    Pour un anneau commutatif $A$ et $a\in A$, $I=aA$ est un idéal.
\end{rem}

L'ensemble des idéaux d'un anneau commutatif est stable par somme et par intersection. De plus, si $I$ et $J$ sont des idéaux, alors \[
    IJ \defeq \left\{ \sum_{k\in K}i_kj_k, \quad K\subset \mathbb N \text{ fini }, i_k\in I, j_k\in J \right\}
\]
est aussi un idéal. Le radical $\sqrt I$ de $I$, définit par \[
    \sqrt I \defeq \left\{ x\in A, \quad \exists n\in\mathbb N, x^n\in I \right\},
\]
est également un idéal (c'est un sous-groupe par binôme de Newton).

\begin{dfn}[Hors-Programme]
    Un anneau $A$ est principal si tous ses idéaux sont principaux\index{idéal!principal}, c'est à dire de la forme $I=aA$. Un anneau est noéthérien\index{anneau!noéthérien} si toute suite croissante d'idéaux est stationnaire
\end{dfn}

\begin{exo}
    Décrire les idéaux de $\mathbb K^2$ pour un corps $\mathbb K$
\end{exo}

\section{Groupe des inversibles d'un anneau}

Pour un anneau $A$, on note $A^\star$ le groupe des inversibles\index{anneau!inversibles} de $A$.

\begin{ex}
    On note $A=\mathbb Z[\sqrt 2]$, on va décrire $\mathbb Z[\sqrt 2]^\star$.

    On note $G=\mathbb Z[\sqrt 2]^\star$ de sorte que $G_+=G\cap \mathbb R_+^\star$ est un groupe pour $\times$. Le logarithme est un morphisme injectif donc $\ln(G_+)$ est un sous-groupe de $\mathbb R$, donc dense si et seulement si $a=\inf (\ln(G_+)\cap \mathbb R_+^\star)=0$.

    On a $e^a=\inf(G_+\cap ]1; +\infty[)$ et \[
        1\leq e^a< 1+\sqrt 2 \implies \exists p,q \in \mathbb Z, 1\leq e^a\leq \underbrace{p+q\sqrt 2}_{\in G_+}< 1+\sqrt 2
    \]
    Si $p>0, q<0$ alors $1\leq p+q\sqrt 2\leq p-q\sqrt 2<p^2-2q^2=1$ (la dernière égalité vient de l'inversibilité de $p+\sqrt2 q$, cf. remarque). C'est absurde. On arrive à la même conclusion si $p<0, q>0$. On a donc $e^a=1+\sqrt 2$ et $G=\langle 1+\sqrt 2 \rangle$
\end{ex}

\begin{rem}
    Dans un anneau euclidien $A$ de stathme $v$, les éléments inversibles sont exactement les éléments qui minimisent le stathme. Si $u$ minimise le stathme, $1=qu+r$ avec $r=0$ ou $v(r)<v(u)$ donc $r=0$ et $1=qu$ donc $u$ inversible. Sinon, $v(u)>v(1)$ donc $v(u)\geq 2$ donc $\forall q, 2\leq v(uq)$ donc $u$ n'est pas inversible.

    Par exemple, dans $\mathbb Z$ euclidien pour le stathme $|\,\cdot\,|$, les inversibles sont $-1$ et $1$. Dans $\mathbb Q[X]$ euclidien pour le stathme $\deg$, les inversibles sont tous les polynômes de degré $0$ (les polynômes constants).
\end{rem}

\section{Morphismes d'anneaux}

\begin{dfn}
    Pour $A, A'$ des anneaux, $\varphi:A\to A'$ est un morphisme\index{anneau!morphisme} d'anneau si \begin{itemize}
        \item $f(1_A)=1_{A'}$
        \item $\forall x, y\in A, f(x+y)=f(x)+f(y)$
        \item $\forall x, y\in A, f(xy)=f(x)f(y)$
    \end{itemize}
\end{dfn}

Comme dans les groupes ou les espaces vectoriels (dans les e.v. les applications linéaires jouent le rôle de morphismes), les morphismes d'anneaux préservent la structure (l'image et la préimage d'un anneau par un morphisme est un anneau).

\section{Structure d'algèbre}

\begin{dfn}
    On note $(A, +, \times, \cdot)$ muni de deux lois internes $+$ et $\times$ et d'une loi externe $\cdot$ sur le corps $K$. On dira que $A$ est une $K$-algèbre\index{algèbre (structure d' -- )} si $(A, +,\times)$ est un anneau, $(A, +, \cdot)$ est un $K$-e.v, et \[
        \forall \lambda\in K, \forall x, y\in A, \lambda\cdot (x\times y)=(\lambda\cdot x)\times y=x\times (\lambda\cdot y)
    \]
\end{dfn}

\begin{rem}
    Un corps $K$ est toujours une $K$-algèbre, la multiplication servant de loi interne et externe.
\end{rem}
\endchapter
