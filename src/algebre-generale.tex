
\ifsolo
~

\vspace{1cm}

\begin{center}
    \textbf{\LARGE Algèbre générale} \\[1em]
\end{center}
\tableofcontents
\else
\minitoc
\fi
\thispagestyle{empty}

\ifsolo \newpage \setcounter{page}{1} \fi
\section{Groupes}

\textbf{Rappel.} $(G, \star)$ est un groupe\index{groupe} ssi \begin{itemize}
    \item $\star$ est une loi de composition interne associative
    \item $\star$ a un neutre dans $G$
    \item Tous les éléments de $G$ ont un inverse dans $G$ pour $\star$
\end{itemize}

\begin{notation}
    Si $\star$ est commutative, on la notera en général $+$ et l'inverse sera alors noté $-x$. Sinon, on notera souvent la loi $\cdot$ et l'inverse $x^{-1}$. Le neutre de $G$ se note $e_G$ (parfois $0_G$ ou $0$ dans le cas d'une loi notée additivement)
\end{notation}

Si $G$ est de cardinal fini, alors on note $|G|$ ou $\#G$ son cardinal.

\section{Constructeurs de groupes}

\begin{prop}~
    \begin{enumerate}
        \item Si $(G, \cdot)$ est un groupe et $X$ un ensemble non vide alors on muni l'ensemble de fonction de $X$ dans $G$ de la loi $\star$ en posant \[
                \forall f, g: X\longrightarrow G, \quad f\star g:x\longmapsto f(x)\cdot g(x)
            \]
            $(\mathcal F(X, G), \star)$ est alors un groupe.
        \item Soit $(G, \land)$, $(G', \star)$ deux groupes. On munit $G\times G'$ de la loi $\cdot$ en posant \[
                (x, x')\cdot (y, y')=(x\land x', y\star y')
            \]
            $(G\times G', \cdot)$ est alors un groupe.
    \end{enumerate}
\end{prop}

\begin{proof}
    Facile mais pénible.
\end{proof}

\section{Sous-groupes}

\begin{thmdef}
    \Hyp $(G, \cdot)$ un groupe, $H\subset G$
    \Conc On dira que $H$ est un sous-groupe\index{sous-groupe} de $G$ si $H\neq \emptyset$, $\cdot$ est une l.c.i. sur $H$ et $(H, \cdot)$ est un groupe. Dans ce cas, $e_H=e_G$. Il y a équivalence entre: \begin{enumerate}[label=(\alph*)]
        \item $H$ est un sous-groupe de $G$
        \item $H\subset G$, $H\neq \emptyset$ et $\forall x, y\in H, x\cdot y^{-1}\in H$
    \end{enumerate}
\end{thmdef}

\begin{notation}
    $(G, \cdot)$ groupe, $H\subset G$, $a\in G$. \[
        aH=\{a\cdot h, \quad h\in H\}
    \]
    \[
        H^{-1}={h^{-1}, \quad h\in H}
    \]
    et si $A\subset G$,
    \[
        AH = \{a\cdot h, \quad a\in A, h\in H\}
    \]
\end{notation}

Ainsi $H\subseteq G$ est un sous-groupe de $G$ si et seulement si $H\neq \emptyset$ et $HH^{-1}\subseteq H$

\begin{exo}
    Soient $H, K$ des sous-groupes de $G$. Montret que $HK$ est un sous groupe si et seulement si $HK=KH$.
\end{exo}

\begin{proof}[Résolution]
    Si $HK$ sous-groupe de $G$ alors $HK=(HK)^{-1}=K^{-1}H^{-1}=KH$. Si $HK=KH$ alors $HK(HK)^{-1}=HKK^{-1}H=HKKH=HHKK\subseteq HK$
\end{proof}

\subsection{Centre d'un groupe}

\begin{dfn}
    Le centre d'un groupe\index{centre (d'un groupe)}\index{sous-groupe!centre} $G$ est le groupe \[
        Z(G)=\{x\in G, \quad \forall y\in G, xy=yx\}
    \]
\end{dfn}

\begin{rem}
    C'est un sous-groupe de $G$, on peut le vérifier manuellement.
\end{rem}

\subsection{Normalisateur}

On note $H$ un sous-groupe de $G$. On appelle \textbf{normalisateur}\index{normalisateur} de $H$ l'ensemble \[
    N(H)=\{x\in G, \quad xHx^{-1}=H\}
\]
On peut vérifier que c'est un sous-groupe de $G$. On dira que $H$ est un sous-groupe distingué \index{sous-groupe!distingué} de $G$ si $N(H)=G$.

\begin{rem}
    On note $\mathcal R$ la relation \[
        x\mathcal Ry\iff y^{-1}x\in H
    \]
    C'est bien une relation d'équivalence (réfléxive symétrique transitive) et si $a\in H$, $x\in \bar a \iff x\in aH$. De même, $x\mathcal R'y\iff xy^{-1}\in H$ définit une autre relation d'équivalence dont les classes sont les $Ha$. Lorsque $H$ est distingué (on notera $H\triangleleft G$), $aH=Ha$
\end{rem}

\subsection{Sous-groupe de torsion}

Soit $G$ un groupe \emph{abélien}. On note \[
    \tau(G)=\{g\in G, \quad \exists n\in\mathbb N^\star, g^n=e_G\}
\]
C'est un sous-groupe\footnote{on a besoin de l'hypothèse de commutativité de $G$, ça n'est pas vrai sinon} de $G$ appelé sous-groupe de torsion\index{sous-groupe!de torsion}.

\subsection{Sous-groupes de $\mathbb Z$ et $\mathbb R$}
