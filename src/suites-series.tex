\ifsolo
    ~

    \vspace{1cm}

    \begin{center}
        \textbf{\LARGE Suites et séries numériques} \\[1em]
    \end{center}
    \tableofcontents
\else
    \chapter{Suites et séries numériques}

    \minitoc
\fi
\thispagestyle{empty}

\ifsolo \newpage \setcounter{page}{1} \fi

\section{Rappels sur les séries numériques}

\begin{dfn}
    Soit $u\in\R^{\N}$. On appelle \textbf{série de terme général} \index{série numérique}$u_n$ la suite $(S_n(u))_n$ définie par \[
        S_n(u)\defeq \sum_{k=1}^nu_k
    \]
    On dira que cette série converge si $(S_n(u))$ converge, et on écrira \[
        \lim_{n\to+\infty}S_n(u)=\sum_{k=1}^{+\infty}u_k
    \]
    Dans ce cas, on pose \[
        R_n(u)\defeq\sum_{k=1}^{+\infty}u_k-S_n(u)
    \]
    qu'on appelle reste d'ordre $n$ de la série. La série de terme général $u_k$ sera notée $\sum u_k$
\end{dfn}


\begin{rem}
    Le terme général d'une séries convergente tend vers $0$. Si $u_n\longrightarrow \ell\in\bar{ \R}\setminus\{0\}$ alors la série diverge grossièrement. Si $\sum u_n$ converge alors $R_n(u)\longrightarrow0$ et on peut noter \[
        R_n(u)=\sum_{k=n+1}^{+\infty}u_k
    \]
\end{rem}

\begin{rem}
    On peut remplacer $\R$ par $\C$ ou n'importe quel espace vectoriel normé (traité dans un chapitre ultérieur)
\end{rem}

\begin{ex}
    Si $u_n=a^n$ pour un complexe $a$, alors \[
        S_n(u)= \begin{cases}
            n+1 &\text{ si }a=1\\[1em]
            \dfrac{1-a^{n+1}}{1-a}&\text{ sinon}
        \end{cases}
    \]
    d'où \[
        \sum u_n\text{ CV }\iff |a|<1
    \]
    et \[
        \sum_{k=0}^{+\infty}u_k=\frac1{1-a}
    \]
\end{ex}

\begin{ex}
    On a \[
        \ln \left( 1+\frac2{n(n+3)} \right)=\ln(n+1)+\ln(n+2)-\ln(n)-\ln(n+3)
    \]
    donc \[
        \sum_{k=1}^n\ln \left( 1+\frac2{k(k+3)} \right) =\ln3+\ln \left( \frac{n+1}{n+3} \right)\xrightarrow[n\to+\infty]{}\ln 3
    \]
\end{ex}

\section{Opérations}

\begin{prop}
    \Hyp $\sum u_n, \sum v_n$ deux séries numériques, $\lambda\in\C$
    \begin{concenum}
    \item $\lambda\sum u_n+\sum v_n=\sum(\lambda u_n+v_n)$
    \item Si les séries convergent, alors $\sum(\lambda u_n+v_n)$ converge (+ égalité des limites)
    \end{concenum}
\end{prop}

\begin{rem}
    On en déduit
    \begin{center}
    \begin{tabular}{ccccc}
        CV & + & CV & = & CV\\
        DV & + & CV & = & DV\\
        DV & + & DV & = & ?
    \end{tabular}
    \end{center}
\end{rem}

\begin{prop}
    \Hyp $u\in\C^{\N}$
    \Conc \[
        (u_n)\text{ CV }\iff \sum (u_{n+1}-u_n)\text{ CV }
    \]
\end{prop}

\begin{ex}[Séries de Riemann\index{série de Riemann}]
    On veut étudier $\sum\frac1{n^\alpha}$.
    \begin{itemize}
        \item Si $\alpha\leq0$ alors la série diverge grossièrement
        \item Si $\alpha> 0$ alors \[
                \frac1{(n+1)^\alpha}-\frac1{n^\alpha}=\frac1{n^\alpha} \left( -\frac\alpha n+O \left( \frac1n \right) \right)\sim -\frac\alpha{n^\alpha}
            \]
            or $(\frac1{n^{\alpha}})_n$ converge vers $0$ donc $\sum \frac1{(n+1)^\alpha}-\frac1{n^\alpha}$ aussi et donc, par équivalence\footnote{c'est une propriété qui sera vue dans la prochaine partie}, $\sum\frac1{n^{\alpha+1}}$ aussi.

        \item Pour $\alpha<0$, on trouve alors que $\sum\frac1{n^{\alpha+1}}$ diverge
    \end{itemize}
    Il ne reste que le cas $\alpha=1$ à traiter: la série diverge. On a donc \[
    \sum\frac1{n^\alpha}\text{ CV }\iff \alpha>1
\]
\end{ex}

On va maintenant s'intéresser à la série harmonique\index{série harmonique} \[
    H_n=\sum_{k=1}^n\frac1k.
\]
On a \[
    \ln(n+1)-\ln (n)\sim \frac1n \implies \sum\frac1k \text{ DV }
\]
et par équivalence des sommes partielles\footnote{vu plus tard} (ou bien par comparaison série-intégrale), \[
    H_n\sim\ln(n).
\]
On note $u_n=H_n-\ln n$. On a \[
    u_{n+1}-u_n=\frac1{n+1}-\ln \left( 1+\frac1n \right)=\frac1n\times \frac1{1+\frac1n}-\frac1n-\frac1{2n^2}+o \left( \frac1{n^2} \right)=-\frac2{n^2}+o \left( \frac1{n^2} \right)
\]
donc $\sum u_{n+1}-u_n$ converge donc $(u_n)$ converge vers un réel $\gamma$. On note \[
    v_n=H_n-\ln n-\gamma
\]
et \[
    v_{n+1}-v_n=-\frac1{2n^2}+o \left( \frac1{n^2} \right)
\]
donc $(v_n)$ converge. Par équivalence des restes (série CV), \[
    \sum_{k=n}^{+\infty}v_{k+1}-v_k\sim\sum_{k=n} -\frac12 \left( \frac1n-\frac1{n+1} \right)
\]
d'où $v_n\sim \frac1{2n}$.
On pose \[
    \omega_n=H_n-\ln n-\gamma-\frac1{2n}
\]
Après calculs (vérifiez-le), \[
    \omega_{n+1}-\omega_n=\frac1{6n^3}+o \left( \frac1{n^3} \right)\sim\frac1{6n^3}\sim\frac1{12}\cdot \left( \frac 1{n^2}-\frac1{(n+1)^2} \right)
\]
d'où $\omega_n\sim\frac1{12n^2}$

\textbf{Bilan.} (Question X) \[
    H_n=\ln n+\gamma+\frac1{2n}-\frac1{12n^2}+o \left( \frac1{n^2} \right)
\]

\begin{exo}[Mines-Ponts]
    On note \[
        k_n=\min\{k\in\N, \quad H_k>n\}
    \]
    Calculer \[
        \lim_{n\to+\infty}\frac{k_{n+1}}{k_n}
    \]
\end{exo}

\section{Séries à termes positifs}

\begin{prop}
    \Hyp $u$ est une suite positive
    \Conc $\sum u_n$ converge si et seulement si $(S_n(u))$ est majorée
\end{prop}

\begin{prop}
    \Hyp $u, v$ des suites positives
    \begin{concenum}
    \item Si $u_n\leq v_n$ à partir d'un certain rang et si $\sum v_n$ converge alors $\sum u_n$ aussi
    \item Si $u_n=O(v_n)$ et $\sum v_n$ converge alors $\sum u_n$ aussi
    \item Si $u_n\sim v_n$ alors $\sum u_n$ et $\sum v_n$ ont la même nature
    \end{concenum}
\end{prop}

\begin{proof}~
    \begin{enumerate}
        \item Si $u_n\leq v_n$ à partir du rang $N$, \[
                S_n(u)\leq \sum_{k=0}^Nu_k+\sum_{k=N+1}^{+\infty}v_k
            \]
            donc $(S_n(u))$ croissante est majorée donc converge.
        \item Idem à constante multiplicative près
        \item $u_n\sim v_n$ donne $u_n=O(v_n)$ et $v_n=O(u_n)$ et on applique $2$.
    \end{enumerate}
\end{proof}

\begin{exo}
    On note $(a_n)_{n\geq 1}$ une suite positive telle que $\sum a_n$ converge. Donner la nature de \[
        \sum a_n\sin(a_n), \qquad \sum\frac{a_n}{1+a_n^2}, \qquad \sum\frac{\sqrt{a_n}}{n}
    \]
\end{exo}

\begin{proof}[Résolution]
    À partir d'un certain rang, $0\leq a_n\sin(a_n)\leq a_n$ puis \[
        \frac{a_n}{1+a_n^2}\sim a_n
    \]
    et enfin (AM-GM) \[
        0\leq \frac{\sqrt{a_n}}n\leq\frac12 \left( a_n+\frac1{n^2} \right)
    \]
    donc les trois séries convergent.
\end{proof}

\begin{exo}[X]
    Soit $x$ une suite positive. Montrer que si $\sum x_n$ converge, alors \[
        \sum x_n^{\frac n{n+1}}
    \]
    aussi
\end{exo}

\begin{proof}[Résolution]~
    Idée: On fait un découpage. On pose \[
        I=\{n\in\N, \quad x_n\geq 0 \text{ et }x_n^{-\frac1{n+1}}>2\}
    \]
    Si $n\in I$ alors \[
        \frac12>x_n^{\frac1{n+1}}\iff \frac1{2^{n+1}}>x_n\implies \frac1{2^n}>y_n
    \]
    Sinon, \[
        x_n=0\implies y_n=0\leq 2x_n
    \]
    et \[
        x_n^{-\frac1{n+1}}\leq 2\implies y_n\leq 2x_n
    \]
    donc dans tous les cas, \[
        0\leq y_n\leq \frac1{2^n}+2x_n
    \]
\end{proof}

\section{Absolue convergence, semi-convergence}

\begin{dfn}
    Soit $a$ une suite réelle. \begin{enumerate}
        \item On dira que $\sum a_n$ est absolument convergnete\index{absolue convergence} si $\sum |a_n|$ converge
        \item On dira que $\sum a_n$ est \index{semi-convergence}semi convergente si elle est convergente mais pas absolument convergente
    \end{enumerate}
\end{dfn}

\begin{thm}
    L'absolue convergence entraîne la convergence
\end{thm}

\begin{proof}
    On écrit $a_n=a_n^+-a_n^-$ avec $a_n^+=\max(a_n, 0)$ et $a_n^-=\max(-a_n, 0)$ de sorte que $|a_n|=a_n^++a_n^-$
\end{proof}

\begin{rem}
    \begin{center}
    \begin{tabular}{ccccc}
        ACV & + & ACV & = & ACV\\
        ACV & + & SCV & = & SCV\\
        SCV & + & SCV & = & ?
    \end{tabular}
    \end{center}
\end{rem}

\section{Comparaison logarithmique}

\begin{prop}
    \Hyp $u, v$ deux suites strictement positives telles que APCR $N$ \[
    \frac{u_{n+1}}{u_n}\leq \frac{v_{n+1}}{v_n}
    \]
    \Conc $u_n=O(v_n)$
\end{prop}

\begin{proof}
    \[
        \forall n>N, \prod_{k=N}^{n-1}\frac{u_{k+1}}{u_k}=\frac{u_n}{u_N}\leq \frac{v_n}{v_N}\implies u_n\leq v_n\frac{u_N}{v_N}=O(v_n)
    \]
\end{proof}

\begin{thm}[Règle de D'Alembert\index{D'Alembert (règle de -- )}]
    \Hyp $a$ une suite positive telle que \[
        \frac{a_{n+1}}{a_n}\xrightarrow[n\to+\infty]{}\ell\in\bar{\R}
    \]
    \begin{concenum}
    \item Si $0\leq \ell <1$ alors $\sum a_n$ converge
    \item Si $\ell>1$ alors $\sum a_n$ diverge
    \end{concenum}
\end{thm}

\begin{proof}~
    \begin{itemize}
        \item $\exists \epsilon>0, \quad 0<\ell'=\ell+\epsilon<1$ et on applique la proposition précédente à $u=a$, $v=(\ell'^n)_n$.
        \item $\exists \epsilon > 0, \quad 1<\ell'=\ell-\epsilon$ et on fait pareil.
    \end{itemize}
\end{proof}

\begin{rem}
    Il faut faire attention à avoir $(a_n)$ jamais nulle à partir d'un certain rang. Le critère est intéressant lorsque $\frac{a_{n+1}}{a_n}$ a une expression simple.
\end{rem}

\begin{exo}
    Donner la nature de $\sum\frac{n^n}{n!^\alpha}$ en fonction de $\alpha$
\end{exo}

\begin{proof}[Résolution]
    On a \[
        \frac{a_{n+1}}{a_n}=\frac{(n+1)^{n+1}}{n^n}\times \frac1{(n+1)^\alpha}=\frac{ \left( 1+\frac1n \right)^n }{(n+1)^{\alpha-1}}\xrightarrow[n\to+\infty]{} \begin{cases}
            0&\text{ si }\alpha>1 \\ e &\text{ si }\alpha=1\\ +\infty& \text{ si }\alpha<1
        \end{cases}
    \]
    et on conclut avec la règle de d'Alembert
\end{proof}

\begin{exo}
    On note $a$ une suite réelle telle que \[
        |a_n|^{\frac1n}\xrightarrow[n\to+\infty]{}\ell\in\R\cup\{+\infty\}
    \]
    Donner la nature de $\sum |a_n|$ en fonction de $\ell$
\end{exo}

\begin{proof}[Résolution]
    Si $\ell>1$ alors $\exists \epsilon>0, 1<\ell'=\ell-\epsilon\leq |a_n|^{\frac1n}$ donc $\ell'^n\leq |a_n|$ et la série diverge.

    Si $\ell<1$ alors par un raisonnement similaire, la série converge.

    Si $\ell=1$ alors on ne peut pas conclure en général: \[
        \left| \frac1{n^2} \right|^{\frac1n}\xrightarrow[n\to+\infty]{}1\qquad\text{ et }\qquad \sum\frac1{n^2}\text{ CV }
    \]
    \[
        \left| \left( 1+\frac1n\right)^n \right|^{\frac1n}\xrightarrow[n\to+\infty]{}1\qquad \text{ et }\qquad \sum \left( 1+\frac1n \right)^{n}\text{ DV grossièrement }
    \]
\end{proof}

\section{Sommation des relations de comparaion}

\begin{thm}
    \Hyp $u, v$ sont des suites réelles positives et $\sum v_n$ converge.
    \begin{concenum}
    \item Si $u_n=O(v_n)$ alors $\sum u_n$ converge et \[
            R_n(u)=O(R_n(v))
        \]
    \item Si $u_n=o(v_n)$ alors $\sum u_n$ converge et \[
            R_n(u)=o(R_n(v))
        \]
    \item Si $u_n\sim v_n$ alors $\sum u_n$ converge et \[
            R_n(u)\sim R_n(v)
        \]
    \end{concenum}
\end{thm}

\begin{proof}
    La convergence est déjà vue, on ne montre que les relations de comparaison

    \begin{enumerate}
        \item Il existe un rang $N$ et une constante $C>0$ tels que $\forall n\geq N, u_n\leq Cv_n$. On a alors \[
    \forall m\geq n\geq N, \qquad \sum_{k=n}^Nu_k\leq C\sum_{k=n}^mv_k
            \]
            puis $m\to+\infty$ donnc $R_n(u)\leq CR_n(v)$ ie \conc
        \item Soit $\epsilon>0$. Il existe un rang $N$ après lequel $u_n\leq \epsilon v_n$. Par le même raisonnement, $R_n(u)\leq \epsilon R_n(v)$ après $N$ donc \conc
        \item $|u_n-v_n|=o(v_n)$ donc \[
                |R_n(u)-R_n(v)|=|R_n(u-v)|\leq R_n(|u-v|)=o(R_n(v))
            \]
    \end{enumerate}
\end{proof}

\begin{thm}
    \Hyp $u, v$ suites réelles positives et $\sum u_n$ diverge.
    \begin{concenum}
    
    \item Si $u_n=O(v_n)$ alors $\sum v_n$ diverge et \[
            S_n(u)=O(S_n(v))
        \]
    \item Si $u_n=o(v_n)$ alors $\sum v_n$ diverge et \[
            S_n(u)=o(S_n(v))
        \]
    \item Si $u_n\sim v_n$ alors $\sum v_n$ diverge et \[
            S_n(u)\sim S_n(v)
        \]
    \end{concenum}
\end{thm}

\begin{proof}
    La divergence est déjà vue, on ne montre que le reste.
    \begin{enumerate}
        \item Il existe $C>0$, et un rang $N$ à partir duquel $u_n\leq Cv_n$. \[
                \forall n\geq N, \quad \underbrace{S_n(u)}_{\longrightarrow+\infty}=\sum_{k=0}^{N-1}u_k+\sum_{k=N}^nu_k\leq \sum_{k=0}^{N-1}+C\underbrace{\sum_{k=0}^nv_k}_{S_n(v)\longrightarrow+\infty}-C\sum_{k=0}^{N-1}v_k
            \]
            donc à partir d'un certain rang, \[
                \frac{S_n(u)}{S_n(v)}\leq C+\frac{\sum_{k=0}^{N-1}u_k-C\sum_{k=0}^{N-1}v_k}{S_n(v)}\xrightarrow[n\to+\infty]{}C
            \]
            d'où \conc
        \item Idem
        \item $S_n(|u-v|)=o(S_n(v))$ (attention aux hypothèses, si $\sum |u-v|$ converge ok sinon on applique 2.)
    \end{enumerate}
\end{proof}

\begin{ex}
    On va montrer \[
        \sum_{k=n}^{+\infty}\frac{\ln(k+1)}{k^2}\sim\frac{\ln n}n
    \]
\end{ex}

\begin{proof}
    La série converge puisque $\frac{\ln n}{n^2}=o \left( \frac1{n^{3/2}} \right)$. Puis,
    on a \[
        \frac{\ln(n+1)}{n+1}-\frac{\ln n}n=\frac{\ln \left( \frac{n+1}n \right)}{n+1}+\frac{\ln n}{n+1}-\frac{\ln n}n=\ln n \left( \frac1{n+1}-\frac1n \right)+\ln \left( 1+\frac1n \right)\frac1{n+1}\sim -\frac{\ln n}{n^2}.
    \]
    d'où la conclusion.
\end{proof}

\begin{exo}
    Donner un développement asymptotique de \[
        R_n=\sum_{k=n}^{+\infty}\frac1{k^2}
    \]
\end{exo}

\begin{proof}[Résolution]
    \[
        \frac1{n^2}\sim\frac1n-\frac1{n+1}
    \]
    donc \[
        R_n\sim \frac1n
    \]
    puis \[
        R_n-\frac1n=\sum_{k=n}^{+\infty} \left( \frac1{k^2}-\frac1k+\frac1{k+1} \right)=\sum_{k=n}^{+\infty}\frac1{k^2(k+1)}
    \]
    or \[
        \frac1{n^3}\sim \frac1{2n^2}-\frac1{2(n+1)^2}
    \]
    donc \[
        R_n=\frac1n+\frac1{2n^2}+o \left( \frac1{n^2} \right)
    \]
\end{proof}

\begin{ex}[Équivalent de Stirling\index{Stirling (formule de -- )}]
    On va trouver un équivalent de $u_n=n!$. On a \[
        \ln(u_n)=\sum_{k=1}^n\ln k
    \]
    or \[
        (k+1)\ln(k+1)-k\ln k\sim \ln k
    \]
    qui est le terme général positif d'une série divergente donc \[
        \ln (u_n)\sim\sum_{k=1}^n((k+1)\ln(k+1)-k\ln k)\sim n\ln n.
    \]
    On note $v_n=\ln u_n-n\ln n$ puis \[
        v_{n+1}-v_n=\ln(n+1)-(n+1)\ln(n+1)+n\ln n=-n\ln \left( 1+\frac1n \right)\sim -1
    \]
    qui est le terme général négatif (mais de signe constant) d'une série divergente donc \[
        v_n\sim -n
    \]
    On note $w_n=v_n+n$ et \[
        w_{n+1}-w_n=-n \left( 1+\frac1n \right)+1=-n \left( \frac1n-\frac1{2n^2}+o \left( \frac1{n^2} \right) \right)+1=\frac1{2n}+o \left( \frac1n \right)\sim\frac1{2n}\sim \frac12 (\ln(n+1)-\ln n)
    \]
    donc \[
        w_n\sim \frac{\ln n}2
    \]
    On pose $y_n=w_n-\frac12\ln n$ et \[
        y_{n+1}-y_n=-n\ln \left( 1+\frac1n \right)+1-\frac12\ln \left( 1+\frac1n \right)\sim\frac1{6n^2}
    \]
    qui est le terme général d'une séries ACV donc $y_n$ converge vers un réel $\ell$.
    On a donc \[
        \ln(u_n)=n\ln n-n+\frac12\ln n+\ell+o(1)
    \]
    donc \[
        u_n\sim e^\ell\sqrt n\; n^ne^{-n}
    \]
    Dans le chapitre d'intégration \ifsolo\else(\textbf{\nameref{sec:wallis}} p.\pageref{sec:wallis})\fi{} on montrera $e^\ell=\sqrt{2\pi}$
\end{ex}

\begin{ex}[Raabe-Duhamel\index{Raabe-Duhamel (règle de -- )}]
    On note $u$ une suite réelle jamais nulle à partir d'un certain rang telle que \[
        \frac{u_{n+1}}{u_n}=1-\frac\lambda n+o \left( \frac1n \right), \qquad \lambda>0
    \]
    On va chercher la nature de $\sum u_n$ en fonction de $\lambda$

    On a \[
        \frac{u_{n+1}}{u_n}\sim 1
    \]
    donc $u_n$ est de signe constant à partir d'un certain rang, on suppose spdg que c'est positif. \[
        \ln u_{n+1}-\ln u_n\sim-\frac\lambda n
    \]
    qui est le terme général d'une série divergente (signe constant) donc \[
        \ln u_n\sim -\lambda H_n\sim -\lambda\ln n
    \]
    d'où $\ln u_n=-\lambda \ln n + o(\ln n)$ donc $u_n=n^{-\lambda+o(1)}=\dfrac1{n^{\lambda+o(1)}}$.

    Si $\lambda>1$ alors $\sum u_n$ converge, si $\lambda <1$ alors $\sum u_n$ diverge et si $\lambda=1$ on ne peut pas conclure (e.g. $u_n=1/n$ et $u_n=1/(n\ln^2n)$).
\end{ex}

\begin{rem}
    Si $\ell\neq 0$, $u_n\sim \ell$ (tg série DV) donc $\sum_{k=1}^nu_k\sim n\ell$ d'où le théorème de Cesàro. Si $\ell=0$ alors $u_n=o(1)$ et $\sum_{k=1}^nu_k=o(n)$ et on retrouve aussi Cesàro.

    Plus généralement, si \[
        \frac{a_{n+1}-a_n}{b_{n+1}-b_n}\longrightarrow\ell\neq 0
    \]
    avec $b_n$ croissante qui tend vers $+\infty$, alors \[
        a_{n+1}-a_n\sim \ell (b_{n+1}-b_n)
    \]
    qui est le terme général d'une série divergente donc \[
        \sum_{k=1}^{n-1}a_{n+1}-a_n=a_n-a_1\sim \ell (b_n-b_1) \implies a_n\sim \ell b_n\implies \frac{a_n}{b_n}\longrightarrow \ell
    \]
    d'où le théorème de Cesàro-Stolz (dans le cas $\ell\in\R^\star$)
\end{rem}

\section{Comparaison série-intégrale\texorpdfstring{\footnote{Épisode 1}}{}}

On note $f$ décroissante continue positive au voisinage de $+\infty$ (pour simplifier on prend $\R_+$ comme voisinage). On a \[
    f(k+1)\leq \int_k^{k+1}f\leq f(k)
\]

\begin{ex}
    On trouve de cette manière \[
        H_n\sim\ln n\qquad \text{ et }\qquad H_{2n}-H_n\sim \ln 2.
    \]
\end{ex}

\begin{thm}
    \Hyp $f$ est continue décroissante positive sur $\R_+$, $u_n=\displaystyle\int_{n-1}^n f-f(n)$
    \Conc $\sum u_n$ converge et $u_n\geq 0$
\end{thm}

\begin{proof}
    $u_n\leq f(n-1)-f(n)$ tg série CV car $f$ converge en $+\infty$
\end{proof}

\begin{exo}
    On note $u$ une suite positive, $S_n=\sum_{k=0}^nu_k$ et quand c'est possible, $R_n=\sum_{k=n}^{+\infty}u_k$
    \begin{enumerate}
        \item Montrer que si $\sum u_n$ diverge alors $\sum \frac{u_n}{S_n^\alpha}$ converge ssi $\alpha>1$
        \item Montrer que si $\sum u_n$ converge alors $\sum \frac{u_n}{R_n^\alpha}$ converge ssi $\alpha<1$
    \end{enumerate}
\end{exo}

\begin{proof}[Résolution (uniquement 1.)]~
    \begin{enumerate}
        \item On a $S_n\longrightarrow+\infty$, $(S_n)$ croissante. On suppose $\alpha>1$. \[
                \frac{u_n}{S_n^\alpha}=\frac{S_n-S_{n-1}}{S_n^\alpha}\leq \int_{S_{n-1}}^{S_n}\frac{\diff t}{t^\alpha}\implies \sum_{k=1}^n\frac{u_k}{S_k^\alpha}\leq \frac1{\alpha-1} \left( \frac1{S_0^{\alpha-1}}-\frac1{S_n^{\alpha-1}} \right)\leq \frac1{\alpha-1}S_0^{\1-\alpha} \text{ donc CV }
            \]
            Si $\alpha\geq 0$ alors APCR, $\frac{u_n}{S_n^\alpha}\geq u_n$ donc la série diverge.

            Si $\alpha=1$, on a \[
                \sum_{k=n}^p\frac{u_k}{S_k}\geq \frac1{S_p}\sum_{k=n}^pu_k=\frac{S_p-S_{n-1}}{S_p}\geq 1-\frac{S_{n-1}}{S_p}
            \]
            Si $\sum\frac{u_k}{S_k}$ converge alors $p\longrightarrow+\infty$ donne $\sum_{k=n}^{+\infty}\frac{u_k}{S_k}\geq 1$ absurde, donc la série diverge.
            Pour $\alpha\in ]0, 1]$, \[
                \frac{u_k}{S_k^\alpha}\underset{\text{APCR}}\geq \frac{u_k}{S_k}
            \]
            donc la série diverge.
    \end{enumerate}
\end{proof}

\section{Théorème des séries alternées}

\begin{thm}
    \Hyp $a$ une suite décroissante positive qui tend vers $0$.
    \begin{concenum}
    \item $\sum a_n(-1)^n$ converge
    \item \[
            \sum_{k=n+1}^{+\infty}(-1)^ka_k
        \]
        a le même signe que $(-1)^{n+1}a_{n+1}$ (le premier terme de la somme)
    \item \[
            \left|  \sum_{k=n+1}^{+\infty}(-1)^ka_k \right|\leq a_{n+1}
    \]
    \end{concenum}
\end{thm}

\begin{proof}~
    \begin{enumerate}
        \item Les suites $S_{2n}$ et $S_{2n+1}$ sont adjacentes donc convergentes de limite $\ell$.
        \item Vu le calcul fait en 1., \[
                \sum_{k=2p}^{+\infty}(-1)^ka_k=\ell-S_{2p-1}\geq 0
            \]
            du même signe que son premier terme. C'est la même chose pour les restes d'indices impair.
        \item \[ 0\leq \ell-S_{2p-1}\leq S_{2p}-S_{2p-1}\implies 0\leq \sum_{k=2n}(-1)^ka_k\leq a_{2n}\]
            et on fait dans l'autre sens pour les impairs.
    \end{enumerate}
\end{proof}

\section{Transformation d'Abel}

On part de deux suites $(a_n)$ et $(b_n)$ et d'une somme $\displaystyle\sum_{k=0}^na_kb_k$. On pose \[
    S_n=\sum_{k=0}^na_k
\]
On a $a_n=S_n-S_{n-1}$ pour $n\geq 0$ avec $S_{-1}=0$. Alors, \[
    \sum_{k=0}^na_kb_k=\sum_{k=0}^n(S_n-S_{n-1})b_k=\sum_{k=0}^{n-1}S_k(b_k-b_{k+1})+S_n b_n
\]

\begin{ex}
    On suppose $|S_n|$ majorée par $M$ et $(b_n)$ décroissante de limite nulle. \[
        b_nS_n\longrightarrow0
    \]
    et \[
        |S_k(b_k-b_{k+1})|\leq M(b_k-b_{k+1})
    \]
    donc \[
        \sum S_k(b_k-b_{k+1})\text{ ACV }
    \]
    donc $\sum a_kb_k$ converge.
\end{ex}

\begin{ex}
    $\sum\frac{\sin n}n$ converge car $S_n(\sin n)$ est bornée (on peut faire le calcul) et $\frac1n$ est décroissante de limite nulle.
\end{ex}

\begin{ex}
    On va chercher la nature de $\sum \frac{\sin \sqrt n}n$. On note $S_n$ les sommes partielles de sorte que \[
        \forall n\geq 1, \qquad \sin \sqrt n=S_{n}-S_{n-1}
    \]
    et \[
        \sum_{k=1}^n\frac{\sin \sqrt n}{n}=\sum_{k=1}^n\frac{S_k-S_{k-1}}k=\sum_{k=1}^{n-1}S_k \left( \frac1k-\frac1{k+1} \right)+\frac{S_n}n.
    \]
    Pour $x\in [k, k+1]$, il existe $c_x\in[k, k+1]$ tel que \[
        \sin\sqrt x-\sin \sqrt k=(x-k)\frac{\cos \sqrt {c_x}}{2\sqrt {c_x}}
    \]
    donc \[
        |\sin \sqrt x-\sin \sqrt k|\leq \frac1{2\sqrt k}
    \]
    et \[
        \left| \sin k-\int_k^{k+1}\sin\sqrt t\;\diff t \right|\leq \int_k^{k+1}|sin\sqrt k-\sin \sqrt t|\diff t\leq \frac1{2\sqrt k}
    \]
    donc \[
        \left| \sum_{k=1}^n\left(\sin \sqrt k-\int_k^{k+1}\sin\sqrt t\;\diff t\right) \right|= \left| S_n-\int_1^{n+1}\sin\sqrt t\;\diff t \right| \leq \sum_{k=1}^n\frac1{2\sqrt k}
    \]
    or \[
        \int_1^{n+1}\sin\sqrt t\;\diff t=\int_1^{\sqrt{n+1}}2u\sin u\diff u=2[-u\cos u+\sin u]_1^{\sqrt{n+1}}=O(\sqrt n).
    \]
    Puis, \[
        \frac1{2\sqrt k}\sim\sqrt {k+1}-\sqrt k \implies \sum_{k=1}^n\frac1{2\sqrt k}\sim \sqrt n
    \]
    On a donc \[
        |S_n|- \left| \int_1^{n+1}\sin\sqrt t\;\diff t \right|\leq \sum_{k=1}^n\frac1{2\sqrt k}=O(\sqrt n)\implies S_n=O(\sqrt n)
    \]
    donc \[
        S_k \left( \frac1k-\frac1{k+1} \right)=O\left(\frac1{n^{3/2}}\right)
    \]
    et la série converge.
\end{ex}

\section{Utilisation des paquets}

\begin{rem}
    Si $\sum a_n$ converge alors pour toute suite d'entiers $(u_n)$ telle que $u_n\geq n$, on a \[
        \sum_{k=n}^{u_n}a_k=S_{u_n}(a)-S_{n-1}(a)\xrightarrow[n\to+\infty]{}0
    \]
    En particulier,  si \[
        \sum_{k=n}^{u_n}a_k\xnrightarrow[n\to+\infty]{}0
    \]
    alors la série diverge.
\end{rem}

\begin{ex}[Série harmonique]
    \[
        \sum_{k=n}^{2n}\frac1k\geq \frac{n+1}{2n+1}\xrightarrow[n\to+\infty]{}\frac12\neq 0
    \]
    donc la série harmonique diverge.
\end{ex}

\begin{ex}[Oscillations lentes]
    On va déterminer la nature de \[
        \sum\frac{\cos (\ln n)}n
    \]
    Pour $N\geq 1$, \[
        \sum_{-\frac\pi3+2N\pi\leq \ln n\leq \frac\pi3+2N\pi}\frac{\cos(\ln n)}n \geq \frac12\sum_{e^{-\frac\pi3}e^{2N\pi}\leq n\leq e^{\frac\pi3}e^{2N\pi}}\frac1n\geq \frac12\frac{\floor{e^{\frac\pi3}e^{2N\pi}}-\floor{e^{-\frac\pi3}e^{2N\pi}}}{e^{\frac\pi3}e^{2N\pi}}\xrightarrow[n\to+\infty]{}\frac12(1-e^{-\frac{2\pi}3})\neq 0
    \]
    donc la série diverge.
\end{ex}

\begin{ex}[Série lacunaire]
    On note \[
        u_n=\begin{cases}\frac1{n^\alpha}&\text{ si $n$ n'a pas de $9$ dans son écriture décimale }\\ 0 & \text{ sinon }\end{cases}
    \]
    Les sommes partielles sont croissantes donc la série $\sum u_n$ converge si et seulement si $(S_n)$ a une suite extraite convergente.
    On va considérer des paquets du type \[
        \sum_{k=10^n}^{10^{n+1}-1}u_k\leq \frac1{10^{n\alpha}}\times \underbrace{8\times 9^n}_{\text{nb d'entiers sans }9}
    \]
    On a ainsi \[
        \underbrace{\left( \frac9{10^\alpha} \right)^n}_{\text{tg série CV} \iff \frac9{10^\alpha}<1}\frac8{10^\alpha}\leq \sum_{k=10^n}^{10^{n+1}-1}u_k=S_{10^{n+1}-1}-S_{10^n-1}\leq \left( \frac9{10^\alpha} \right)^n\cdot 8
    \]
    donc $(S_{10^n-1})_n$ converge ssi $\displaystyle \alpha>\frac{\ln 9}{\ln 10}$
\end{ex}

\section{Développement décimal illimité}

Pour $x\in\R$, on pose $a_0=\floor x$ et \[
    a_n=\floor{10^nx}-10\floor{10^{n-1}x}\in\llbracket 0,9\rrbracket
\]
Par récurrence, on montre \[
    \frac{\floor{10^nx}}{10^n}=\sum_{k=0}^n\frac{a_k}{10^k}=\bar{a_0,a_1\cdots a_n}
\]
Puis, $\sum \frac{a_k}{10^k}$ converge et sa somme vaut $x$.

\begin{rem}
    La suite $(a_n)$ n'est pas stationnaire égale à $9$.
\end{rem}

\begin{proof}
    On suppose que la suite est stationnaire égale à $9$: $a_{N-1}\neq 9$ et $(a_n)_{n\geq N}$ constante égale à $9$. Dans ce cas, \[
        x=\sum_{k=0}^N\frac{a_k}{10^k}+\underbrace{\sum_{k=N}^{+\infty}\frac9{10^k}}_{=\frac1{10^{N-1}}}
    \]
    et c'est absurde en calculant explicitement $a_{N-1}$
\end{proof}

On dit que le développement obtenu est le \textbf{développement décimal propre} de $x$ \index{développement décimal propre}

\begin{prop}
    $x$ est rationnel si et seulement si son développement propre est ultimement périodique (périodique à partir d'un certain rang)
\end{prop}

\begin{proof}
    On supose $(a_k)$ $T$-périodique à partir du rang $N$. On a \[
        x=\underbrace{\sum_{k=0}^{N-1}\frac{a_k}{10^k}}_{\in\Q}+\underbrace{\sum_{k=0}^{+\infty}\frac1{10^{kT}}}_{=\frac1{1-\frac1{10^T}}\in\Q} \underbrace{\left(  \frac{a_N}{10^N}+\cdots +\frac{a_{N+T-1}}{10^{N+T-1}}\right)}_{\in \Q}\in\Q
    \]
    On suppose $x=\frac pq\in\Q$. Il y a un nombre fini de restes modulo $p$ donc il existe $n<m$ tels que $10^np\equiv 10^mp\pmod q$.

On a $10^np=q_nq+r_n$ avec $q_n\in\N$, donc $10^nx=q_n+\frac{r_n}q$. On a alors, pour $n, i\in\N^\star$, \begin{align*}
                a_{n+i} &= E(10^{n+i}x)-10E(10^{n+i-1}x) \\
                        &= E\left(10^i \left( q_n+\frac{r_n}q \right)\right)-10E \left( 10^{i-1} \left(  q_n+\frac{r_n}q \right) \right) \\
                        &= E \left( 10^i\frac {r_n}q \right)+10^iq_n-10E \left( 10^{i-1}\frac {r_n}q \right) -10^iq_n \\
                        &= E \left( 10^i\frac {r_n}q \right) - 10E \left( 10^{i-1}\frac {r_n}q \right)
            \end{align*}
         On pose $T=m-n$. On a, pour tout $i\geq 1$, $a_{n+i+T}=a_{m+i}=a_{n+i}$. Donc $(a_k)_k$ est périodique à partir du rang $n$.
\end{proof}

\endchapter
