\ifsolo
    ~

    \vspace{1cm}

    \begin{center}
        \textbf{\LARGE Suites et séries numériques} \\[1em]
    \end{center}
    \tableofcontents
\else
    \minitoc
\fi
\thispagestyle{empty}

\ifsolo \newpage \setcounter{page}{1} \fi

\section{Rappels sur les séries numériques}

\begin{dfn}
    Soit $u\in\mathbb R^{\mathbb N}$. On appelle \textbf{série de terme général} \index{série numérique}$u_n$ la suite $(S_n(u))_n$ définie par \[
        S_n(u)\defeq \sum_{k=1}^nu_k
    \]
    On dira que cette série converge si $(S_n(u))$ converge, et on écrira \[
        \lim_{n\to+\infty}S_n(u)=\sum_{k=1}^{+\infty}u_k
    \]
    Dans ce cas, on pose \[
        R_n(u)\defeq\sum_{k=1}^{+\infty}u_k-S_n(u)
    \]
    qu'on appelle reste d'ordre $n$ de la série. La série de terme général $u_k$ sera notée $\sum u_k$
\end{dfn}


\begin{rem}
    Le terme général d'une séries convergente tend vers $0$. Si $u_n\longrightarrow \ell\in\bar{ \mathbb R}\setminus\{0\}$ alors la série diverge grossièrement. Si $\sum u_n$ converge alors $R_n(u)\longrightarrow0$ et on peut noter \[
        R_n(u)=\sum_{k=n+1}^{+\infty}u_k
    \]
\end{rem}

\begin{rem}
    On peut remplacer $\mathbb R$ par $\mathbb C$ ou n'importe quel espace vectoriel normé (traité dans un chapitre ultérieur)
\end{rem}

\begin{ex}
    Si $u_n=a^n$ pour un complexe $a$, alors \[
        S_n(u)= \begin{cases}
            n+1 &\text{ si }a=1\\[1em]
            \dfrac{1-a^{n+1}}{1-a}&\text{ sinon}
        \end{cases}
    \]
    d'où \[
        \sum u_n\text{ CV }\iff |a|<1
    \]
    et \[
        \sum_{k=0}^{+\infty}u_k=\frac1{1-a}
    \]
\end{ex}

\begin{ex}
    On a \[
        \ln \left( 1+\frac2{n(n+3)} \right)=\ln(n+1)+\ln(n+2)-\ln(n)-\ln(n+3)
    \]
    donc \[
        \sum_{k=1}^n\ln \left( 1+\frac2{k(k+3)} \right) =\ln3+\ln \left( \frac{n+1}{n+3} \right)\xrightarrow[n\to+\infty]{}\ln 3
    \]
\end{ex}

\needspace{5cm}
\section{Opérations}

\begin{prop}
    \Hyp $\sum u_n, \sum v_n$ deux séries numériques, $\lambda\in\mathbb C$
    \begin{concenum}
    \item $\lambda\sum u_n+\sum v_n=\sum(\lambda u_n+v_n)$
    \item Si les séries convergent, alors $\sum(\lambda u_n+v_n)$ converge (+ égalité des limites)
    \end{concenum}
\end{prop}

\begin{rem}
    On en déduit
    \begin{center}
    \begin{tabular}{ccccc}
        CV & + & CV & = & CV\\
        DV & + & CV & = & DV\\
        DV & + & DV & = & ?
    \end{tabular}
    \end{center}
\end{rem}

\begin{prop}
    \Hyp $u\in\mathbb C^{\mathbb N}$
    \Conc \[
        (u_n)\text{ CV }\iff \sum (u_{n+1}-u_n)\text{ CV }
    \]
\end{prop}

\begin{ex}[Séries de Riemann\index{série de Riemann}]
    On veut étudier $\sum\frac1{n^\alpha}$.
    \begin{itemize}
        \item Si $\alpha\leq0$ alors la série diverge grossièrement
        \item Si $\alpha> 0$ alors \[
                \frac1{(n+1)^\alpha}-\frac1{n^\alpha}=\frac1{n^\alpha} \left( -\frac\alpha n+O \left( \frac1n \right) \right)\sim -\frac\alpha{n^\alpha}
            \]
            or $(\frac1{n^{\alpha}})_n$ converge vers $0$ donc $\sum \frac1{(n+1)^\alpha}-\frac1{n^\alpha}$ aussi et donc, par équivalence\footnote{c'est une propriété qui sera vue dans la prochaine partie}, $\sum\frac1{n^{\alpha+1}}$ aussi.

        \item Pour $\alpha<0$, on trouve alors que $\sum\frac1{n^{\alpha+1}}$ diverge
    \end{itemize}
    Il ne reste que le cas $\alpha=1$ à traiter: la série diverge. On a donc \[
    \sum\frac1{n^\alpha}\text{ CV }\iff \alpha>1
\]
\end{ex}

On va maintenant s'intéresser à la série harmonique\index{série harmonique} \[
    H_n=\sum_{k=1}^n\frac1k.
\]
On a \[
    \ln(n+1)-\ln (n)\sim \frac1n \implies \sum\frac1k \text{ DV }
\]
et par équivalence des sommes partielles\footnote{vu plus tard} (ou bien par comparaison série-intégrale), \[
    H_n\sim\ln(n).
\]
On note $u_n=H_n-\ln n$. On a \[
    u_{n+1}-u_n=\frac1{n+1}-\ln \left( 1+\frac1n \right)=\frac1n\times \frac1{1+\frac1n}-\frac1n-\frac1{2n^2}+o \left( \frac1{n^2} \right)=-\frac2{n^2}+o \left( \frac1{n^2} \right)
\]
donc $\sum u_{n+1}-u_n$ converge donc $(u_n)$ converge vers un réel $\gamma$. On note \[
    v_n=H_n-\ln n-\gamma
\]
et \[
    v_{n+1}-v_n=-\frac1{2n^2}+o \left( \frac1{n^2} \right)
\]
donc $(v_n)$ converge. Par équivalence des restes (série CV), \[
    \sum_{k=n}^{+\infty}v_{k+1}-v_k\sim\sum_{k=n} -\frac12 \left( \frac1n-\frac1{n+1} \right)
\]
d'où $v_n\sim \frac1{2n}$.
On pose \[
    \omega_n=H_n-\ln n-\gamma-\frac1{2n}
\]
Après calculs (vérifiez-le), \[
    \omega_{n+1}-\omega_n=\frac1{6n^3}+o \left( \frac1{n^3} \right)\sim\frac1{6n^3}\sim\frac1{12}\cdot \left( \frac 1{n^2}-\frac1{(n+1)^2} \right)
\]
d'où $\omega_n\sim\frac1{12n^2}$

\textbf{Bilan.} (Question X) \[
    H_n=\ln n+\gamma+\frac1{2n}-\frac1{12n^2}+o \left( \frac1{n^2} \right)
\]

\begin{exo}[Mines-Ponts]
    On note \[
        k_n=\min\{k\in\mathbb N, \quad H_k>n\}
    \]
    Calculer \[
        \lim_{n\to+\infty}\frac{k_{n+1}}{k_n}
    \]
\end{exo}

\section{Séries à termes positifs}

\begin{prop}
    \Hyp $u$ est une suite positive
    \Conc $\sum u_n$ converge si et seulement si $(S_n(u))$ est majorée
\end{prop}

\begin{proof}
    \Hyp $u, v$ des suites positives
    \begin{concenum}
    \item Si $u_n\leq v_n$ à partir d'un certain rang et si $\sum v_n$ converge alors $\sum u_n$ aussi
    \item Si $u_n=O(v_n)$ et $\sum v_n$ converge alors $\sum u_n$ aussi
    \item Si $u_n\sim v_n$ alors $\sum u_n$ et $\sum v_n$ ont la même nature
    \end{concenum}
\end{proof}

\begin{proof}~
    \begin{enumerate}
        \item Si $u_n\leq v_n$ à partir du rang $N$, \[
                S_n(u)\leq \sum_{k=0}^Nu_k+\sum_{k=N+1}^{+\infty}v_k
            \]
            donc $(S_n(u))$ croissante est majorée donc converge.
        \item Idem à constante multiplicative près
        \item $u_n\sim v_n$ donne $u_n=O(v_n)$ et $v_n=O(u_n)$ et on applique $2$.
    \end{enumerate}
\end{proof}

\begin{exo}
    On note $(a_n)_{n\geq 1}$ une suite positive telle que $\sum a_n$ converge. Donner la nature de \[
        \sum a_n\sin(a_n), \qquad \sum\frac{a_n}{1+a_n^2}, \qquad \sum\frac{\sqrt{a_n}}{n}
    \]
\end{exo}

\begin{proof}[Résolution]
    À partir d'un certain rang, $0\leq a_n\sin(a_n)\leq a_n$ puis \[
        \frac{a_n}{1+a_n^2}\sim a_n
    \]
    et enfin (AM-GM) \[
        0\leq \frac{\sqrt{a_n}}n\leq\frac12 \left( a_n+\frac1{n^2} \right)
    \]
    donc les trois séries convergent.
\end{proof}

\begin{exo}[X]
    Soit $x$ une suite positive. Montrer que si $\sum x_n$ converge, alors \[
        \sum x_n^{\frac n{n+1}}
    \]
    aussi
\end{exo}

\begin{proof}[Résolution]~
    Idée: On fait un découpage. On pose \[
        I=\{n\in\mathbb N, \quad x_n\geq 0 \text{ et }x_n^{-\frac1{n+1}}>2\}
    \]
    Si $n\in I$ alors \[
        \frac12>x_n^{\frac1{n+1}}\iff \frac1{2^{n+1}}>x_n\implies \frac1{2^n}>y_n
    \]
    Sinon, \[
        x_n=0\implies y_n=0\leq 2x_n
    \]
    et \[
        x_n^{-\frac1{n+1}}\leq 2\implies y_n\leq 2x_n
    \]
    donc dans tous les cas, \[
        0\leq y_n\leq \frac1{2^n}+2x_n
    \]
\end{proof}

\section{Absolue convergence, semi-convergence}

\begin{dfn}
    Soit $a$ une suite réelle. \begin{enumerate}
        \item On dira que $\sum a_n$ est absolument convergnete\index{absolue convergence} si $\sum |a_n|$ converge
        \item On dira que $\sum a_n$ est \index{semi-convergence}semi convergente si elle est convergente mais pas absolument convergente
    \end{enumerate}
\end{dfn}

\begin{thm}
    L'absolue convergence entraîne la convergence
\end{thm}

\begin{proof}
    On écrit $a_n=a_n^+-a_n^-$ avec $a_n^+=\max(a_n, 0)$ et $a_n^-=\max(-a_n, 0)$ de sorte que $|a_n|=a_n^++a_n^-$
\end{proof}

\begin{rem}
    \begin{center}
    \begin{tabular}{ccccc}
        ACV & + & ACV & = & ACV\\
        ACV & + & SCV & = & SCV\\
        SCV & + & SCV & = & ?
    \end{tabular}
    \end{center}
\end{rem}

\section{Comparaison logarithmique}

\begin{prop}
    \Hyp $u, v$ deux suites strictement positives telles que APCR $N$ \[
    \frac{u_{n+1}}{u_n}\leq \frac{v_{n+1}}{v_n}
    \]
    \Conc $u_n=O(v_n)$
\end{prop}

\begin{proof}
    \[
        \forall n>N, \prod_{k=N}^{n-1}\frac{u_{k+1}}{u_k}=\frac{u_n}{u_N}\leq \frac{v_n}{v_N}\implies u_n\leq v_n\frac{u_N}{v_N}=O(v_n)
    \]
\end{proof}

\begin{thm}[Critère de D'Alembert\index{D'Alembert (critère de -- )}]
    \Hyp $a$ une suite positive telle que \[
        \frac{a_{n+1}}{a_n}\xrightarrow[n\to+\infty]{}\ell\in\bar{\mathbb R}
    \]
    \begin{concenum}
    \item Si $0\leq \ell <1$ alors $\sum a_n$ converge
    \item Si $\ell>1$ alors $\sum a_n$ diverge
    \end{concenum}
\end{thm}

\begin{proof}~
    \begin{itemize}
        \item $\exists \epsilon>0, \quad 0<\ell'=\ell+\epsilon<1$ et on applique la proposition précédente à $u=a$, $v=(\ell'^n)_n$.
        \item $\exists \epsilon > 0, \quad 1<\ell'=\ell-\epsilon$ et on fait pareil.
    \end{itemize}
\end{proof}

\begin{rem}
    Il faut faire attention à avoir $(a_n)$ jamais nulle à partir d'un certain rang. Le critère est intéressant lorsque $\frac{a_{n+1}}{a_n}$ a une expression simple.
\end{rem}
