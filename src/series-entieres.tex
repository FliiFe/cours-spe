\ifsolo
    ~

    \vspace{1cm}

    \begin{center}
        \textbf{\LARGE Séries entières} \\[1em]
    \end{center}
    \tableofcontents
\else
    \chapter{Séries entières}

    \minitoc
\fi
\thispagestyle{empty}

\section{Définitions}

\begin{dfn}
    Soit $(a_n)_n$ une suite complexe. La série de fonctions de terme général $(a_nz^n)_n$ est appelée série entière\index{série entière} et se note $\sum a_nz^n$. Lorsqu'elle converge, on note sa somme \[
        \sum_{n=0}^{+\infty}a_nz^n
    \]
\end{dfn}

\begin{ex}
    Le domaine de convergence de $\sum z^n$ est $\mathcal D_0(0, 1)$.

    Le somaine de convergence $\mathcal D$ de $\sum \frac{z^n}n$ vérifie $\mathcal D_o(0, 1)\subsetneq\mathcal D\subsetneq \mathcal D_f(0, 1)$
\end{ex}

\section{Rayon de convergence}
\begin{thmdef}
    \Hyp $(a_n)_{n\in\mathbb N}\in\mathbb C^{\mathbb N}$
    \begin{concenum}
    \item Les cinq nombres suivants sont égaux {\begin{align*}
                R_1 &= \sup\{|z|, \quad z\in\mathbb C, (a_nz^n)_n\text{ bornée }\} \\
                R_2 &= \sup\{|z|, \quad z\in\mathbb C, a_nz^n\longrightarrow 0\}\\
                R_3 &= \sup\{|z|, \quad z\in\mathbb C, \sum a_nz^n\text{ converge }\} \\
                R_4 &= \sup\{|z|, \quad z\in\mathbb C, \sum a_nz^n\text{ converge absolument }\} \\
                R_5 &= \inf\{|z|, \quad z\in\mathbb C, (a_nz^n)_n\text{ non bornée}\}
    \end{align*}
avec la convention $\inf\emptyset=+\infty$.}
    On appelle ce nomber \emph{rayon de convergence\index{rayon de convergence}} de $\sum a_nz^n$ et on le note $R(a_nz^n)$
\item Si $\sum a_nz^n$ admet pour rayon de convergence $R$, alors le domaine de convergence est tel que \[
        \mathcal D_0(0, R)\subseteq \mathcal D\subseteq \mathcal D_f(0, R)
    \]
    \end{concenum}
\end{thmdef}

\begin{proof}
    \begin{enumerate}
        \item $R_1\geq R_2\geq R_3\geq R_4$. Par l'absurde, on suppose $R_4<R_5$. Alors, on prend $z\in\mathbb C$ tel que $R_5>|z|>R_4$. $(a_nz^n)_n$ est bornée donc si on note $z'$ tel que $|z|>|z'|>R_4$, alors \[
                a_nz'^n=\underbrace{a_nz^n}_{\text{borné}} \underbrace{\left( \frac{z'}{z}  \right)^n}_{|\;|<1} \text{ tg série ACV absurde car }|z'|>R_4
            \]
            Donc $R_4\geq R_5$. Par l'absurde, $R_1>R_5$. On se donne $z\in\mathbb C$ tel que $R_1>|z|>R_5$ et il existe $z', z''\in\mathbb C$ tel que $R_1\geq |z''|>|z|>|z'|\geq R_5$, $(a_nz'^n)$ non bornée et $(a_nz''^n)$ bornée. \[
                \underbrace{a_nz'^n}_{\lnot \text{ borné}}=a_nz''^n \left( \frac{z'}{z''} \right)^n\xrightarrow{}_{n\to+\infty}0
            \]
            absurde donc $R_1=R_2=R_3=R_4=R_5$.
        \item Immédiat vu $1$.
    \end{enumerate}
\end{proof}
% 
\begin{rem}[Lemme d'Abel]
    Si $a\in\mathbb C^{\mathbb N}$ et $z_0\in\mathbb C^\star$. Si la suite $(a_nz_0^n)$ est bornée alors $\forall z\in\mathcal D_o(0, |z_0|), \sum a_nz^n$ ACV
\end{rem}

\section{Opération algébrique sur les séries entières}

\begin{prop}
    \Hyp $\sum a_nz^n$ de rayon $R_a$ et $\sum b_nz^n$ de rayon $R_b$
    \begin{concenum}
    \item Pour $\lambda\in\mathbb C^\star$, $R(\sum \lambda a_nz^n)=R_a$
    \item Pour $\lambda\in\mathbb C^\star$, $R(\sum(\lambda a_n+b_n)z^n)\geq \min(R_a, R_b)$ et si $R_a\neq R_b$, il y a égalité.
    \item Le produit de Cauchy\index{Cauchy!produit} des séries $\sum a_nz^n$ et $\sum b_nz^n$ est la série $\sum c_nz^n$ avec \[
            c_n=\sum_{p+q=n}a_pb_q
        \]
        et $R(\sum c_nz^n)\geq \min(R_a, R_b)$.
    \end{concenum}
\end{prop}

\begin{proof}
    \begin{enumerate}
        \item "Trivial"
        \item $\geq 0$ OK et $R_a\neq R_b\implies =$ OK
        \item Si $|z|<\min(R_a, R_b)$ alors $\sum a_nz^n, \sum b_nz^n$ ACV donc produit de Cauchy ie \conc
    \end{enumerate}
\end{proof}

\section{Régularité des séries entières}

\begin{prop}
    \Hyp $\sum a_nz^n$ de RCV $R>0$ (ou $+\infty$)
    \begin{concenum}
    \item $\forall a\in ]0, R[, \quad \sum a_nz^n$ CVN sur $\mathcal D_f(0, a)$.
    \item $z\in\mathcal D_o(0, R)\longmapsto\displaystyle \sum_{k=0}^{+\infty}a_nz^n$ est continue
    \end{concenum}
\end{prop}

\begin{proof}
    $a\in]0, R[$. \[
        \sum_{z\in\mathcal D_f(0, a)}|a_nz^n|=|a_na^n|\text{ tg série CV car } a<R \text{ donc il y a CVN sur }\mathcal D_f(0, a)
    \]
    d'où la continuité sur tous les $\mathcal D_f(0, a)$ donc sur $\mathcal D_o(0, a)$.
\end{proof}

\begin{thm}
    \Hyp $f(z)=\sum_{n\geq 0}a_nz^n$ série entière de rayon $R>0$
    \begin{concenum}
    \item On appelle série dérivée\index{série dérivée} la série $\sum_{n\geq 1}na_nz^{n-1}$
        \begin{enumerate}
            \item Cette série a le même rayon de convergence que $f$
            \item $f$ est dérivable sur $]-R, R[$ et $\forall t\in ]-R, R[,$ \[
                    f'(t)=\sum_{n\geq 1}a_nnt^{n-1}
                \]
        \end{enumerate}
        \item La série $\sum_{n\geq 0}\frac{a_n}{n+1}z^{n+1}$ a même rayon de convergence que $f$ et pour $t\in ]-R, R[$, \[
            \sum_{n\geq 0}\frac{a_n}{n+1}z^{n+1}=\int_0^tf(u)\diff u
            \]
    \end{concenum}
\end{thm}

\begin{proof}
    \begin{enumerate}
        \item $|na_nz^{n-1}|=|a_na^n|\times \left| (\frac za)^{n-1} \right|\times \frac na\xrightarrow{}0$ si $|z|<a<R$. Puis, $(a_nz^n)$ non bornée entraîne $(na_nz^{n-1})$ non bornée. On a donc $R(\sum na_nz^{n-1})=R$.

            On vérifie les hypothèses du théorème de dérivation terme à terme: \begin{itemize}
                \item $\forall n\in\mathbb N, \quad t\in]-R,R[\longmapsto a_nt^n$ est $\mathcal C^1$
                \item $\sum a_nt^n$ converge simplement sur $]-R, R[$
                \item $\sum(a_nz^n)'=\sum na_nz^{n-1}$ converge normalement donc univormément sur tout segment de $]-R, R[$.
            \end{itemize}
            Ainsi, le théorème donne bien \conc
        \item Il y a égalité des rayons de convergence (vu 1) et pour tout $t\in ]-R,R[$, la série $\sum a_nz^n$ converge normalement donc uniformément sur $[0,t]$, donc le théorème d'intégration terme à terme donne \[
                \int_0^tf(u)\diff u=\sum a_n\int_0^t u^n\diff u=\sum \frac{a_n}{n+1}t^{n+1}
            \]
    \end{enumerate}
\end{proof}

\begin{ex}
    On note $f(t)=-\ln(1-t)$ de sorte que \[
        \forall t\in ]-1, 1[, \qquad f(t)=\sum_{n\geq 0}\frac{t^{n+1}}{n+1}.
    \]
    Alors, \[
        f(-t)=\sum_{n\geq 0}(-1)^{n+1}\frac{t^{n+1}}{n+1}
    \]
    et (théorème des séries alternées\index{séries alternées (théorème des --), TSA}) \[
        \forall t\in[0, 1[, \forall n\in\mathbb N, \qquad \left|f(-t)-\sum_{k=0}^n(-1)^{k+1}\frac{t^{k+1}}{k+1}\right|\leq \frac1{n+2}\xrightarrow[n\to+\infty]{}0
    \]
    donc le reste CVU vers $0$, puis le théorème de la double limite donne \[
        f(-t)\xrightarrow[\substack{t\to1\\t<1}]{}\sum_{n\geq 0}\frac{(-1)^{n+1}}{n+1}=-\ln2
    \]
\end{ex}

\begin{rem}
    Pour $f:\mathbb C\longrightarrow \mathbb C$, on peut définir la dérivabilité au sens complexe par \[
        f \text{ dérivable en } a\iff \lim_{\substack{h\to0\\h\neq 0}}\frac{f(a+h)-f(a)}h\text{ existe }
    \]
    On note $f(z)=\sum a_nz^n$ une série entière de rayon de convergence $R>0$ et $a\in\mathcal D_o(0, R)$. On va montrer que $f$ est dérivable au sens complexe en $a$ et que \[
        f'(a)=\sum_{n\geq 0}na_na^{n-1}
    \]
    Pour $h\neq 0$ assez petit en module: 
    \begin{align*}
        \frac{f(a+h)-f(a)}h&=\sum_{n\geq 0}a_n\frac{(a+h)^n-a^n}h\\&=\sum_{n\geq 0}a_n\underbrace{\left(\binom n1a^{n-1}+\binom n2a^{n-2}h+\cdots+h^{n-1}\right)}_{\displaystyle\xrightarrow[\substack{h\to0\\h\neq 0}]{}na^{n-1}}
    \end{align*}
    Puis si on pose $\varphi:t\longmapsto (a+th)^n$, on a (Taylor-Lagrange) \[
        |\varphi(1)-\varphi(0)-\varphi'(0)|\leq \frac1{2!}\sup_{[0,1]}|\varphi''|\leq \frac{n(n-1)}2|h|^2(|a|+|h|)^{n-2}
    \]
    donc si $b$ est tel que $|a|<b<R$, il existe un voisinage $V$ de $0$ tel que $\forall h\in V, |a|+|h|<|b|$ et dans ce cas
    \begin{align*}
        \left|  \frac{f(a+h)-f(a)}h-\sum_{n\geq 0}na_na^{n-1}\right|&\leq \left| \sum_{n\geq 0}a_n \left( \frac{(a+h)^n-a^n}h-na^{n-1} \right) \right| \\
                                                                    &\leq \sum_{n\geq 0}|a_n|\frac{n(n-1)}2|h|(|a|+|h|)^{n-2}\\
                                                                    &\leq |h|\underbrace{\sum_{n\geq 0}|a_n|\frac{n(n-1)}2b^{n-2}}_{\text{CV car $b<R$, série dérivée 2\textsuperscript{nde}}}\xrightarrow[\substack{h\to0\\h\neq 0}]{}0
    \end{align*}
\end{rem}

\begin{ex}
    Calcul de $I=\displaystyle\int_0^1\ln(1-x)\ln x\diff x$. \begin{itemize}
        \item $x\longmapsto \ln(1-x)\ln x$ est intégrable car continue par prolongement
        \item $\forall x\in ]0,1[$, $\ln(1-x)=-\sum_{n\geq 0}\frac{x^{n+1}}{n+1}$ donc \[
                I=-\int_0^1\sum_{n\geq 0}\frac{x^{n+1}}{n+1}\ln x\diff x
            \]
        \item $\sum \frac{x^{n+1}}{n+1}\ln x$ converge simplement sur $]0,1[$
        \item Les $x\longmapsto \frac{x^{n+1}}{n+1}\ln x$ sont continues intégrables sur $]0, 1[$ et \[
                \int_0^1 \left| \frac{x^{n+1}}{n+1}\ln x \right|\diff x=-\int_0^1\frac{x^{n+1}}{n+1}\ln x\diff x=\frac1{(n+2)^2(n+1)}\quad \text{tg série CV}
            \]
    \end{itemize}
    Donc le théorème d'interversion série-intégrale (version convergence dominée) donne \[
        I=\sum_{n\geq 0}\int_0^1\frac{-x^{n+1}}{n+1}\ln x\diff x=\sum_{n\geq 0}\frac{1}{(n+2)^2(n+1)}
    \]
    et \[
        \frac1{X(X+1)^2}=\frac{-1}{(X+1)^2}+\frac{-1}{X+1}+\frac1X
    \]
    d'où finalement $I=2-\dfrac{\pi^2}6$
\end{ex}

\begin{thm}
    \Hyp $f(z)=\sum_{n\geq 0}a_nz^n$ avec le RCV $R>0$
    \begin{concenum}
    \item $f\in\mathcal C^\infty(]-R, R[, \mathbb C)$ et \[\forall n\in\mathbb N, \quad \frac{f^{(n)}(0)}{n!}=a_n\]
    \item \[
            \forall t\in ]-R, R[, \qquad f(t)=\sum_{n\geq 0}\frac{f^{(n)}(0)}{n!}t^n
        \]
    \item Si $\sum a_nz^n=\sum b_nz^n$ au voisinage de $0$ alors $\forall n, a_n=b_n$
    \end{concenum}
\end{thm}

\begin{proof}
    Facile.
\end{proof}

\begin{rem}
    On peut faire des DSE ailleurs qu'en $0$. Soit $f(z)=\sum_{n\geq 0}a_nz^n$ DSE avec un rayon de convergence non nul, et $a\in\mathcal D_o(0, R)$. On va montrer que pour $z\in\mathcal D_o(a, R-|a|)$, \[
        f(z)=\sum_{n\geq 0}\frac{f^{(n)}(a)}{n!}(z-a)^n
    \]
    On a: \[
        \forall z\in\mathcal D_o(0, R), \qquad f(z)=\sum_{n\geq 0}a_nz^n
    \]
    donc \[\forall k\geq 0, \qquad f^{(k)}(a)=\sum_{n\geq k}a_n\frac{n!}{(n+k)!}a^{n-k}.\]
    On va montrer que la famille \[
        \left( a_n\frac{n!}{(n+k)!}a^{n-k}\frac{(z-a)^k}{k!} \right)_{n\geq k\geq 0}
    \]
    est sommable. Pour $n\in\mathbb N$, \[
        \sum_{k=0}^n \left| a_n\frac{n!}{(n+k)!}a^{n-k}\frac{(z-a)^k}{k!} \right|=|a_n|(|a|+|z-a|)^n
    \]
    et si $z\in\mathcal D_o(a, R-|a|)$, on a $|a|+|z-a|<R$ donc c'est le terme général d'une série convergente. Le théorème de Fubini\index{Fubini (théorème discret de -- )} donne alors \[
        \sum_{n\geq 0}\sum_{k=0}^na_n\frac{n!}{(n+k)!}a^{n-k}\frac{(z-a)^k}{k!}=\sum_{n\geq 0}a_nz^n=f(z)=\sum_{k\geq 0}\sum_{n\geq k}a_n\frac{n!}{(n+k)!}a^{n-k}\frac{(z-a)^k}{k!}=\sum_{k\geq 0}\frac{f^{(k)}(a)}{k!}(z-a)^k
    \]
\end{rem}

\section{DSE usuels}

\begin{rem}
    On note $f\in\mathcal C^\infty(]-R, R[, \mathbb C)$. Pour $n\in\mathbb N$ et $x\in ]-R, R[$, \[
        f(x)=\sum_{k=0}^n\frac{f^{(k)}(0)}{k!}x^k+\underbrace{\int_0^x\frac{(x-t)^n}{n!}f^{(n+1)}(t)\diff t}_{\defeq R_n(x)}
    \]
    Si $R_n(x)\xrightarrow[n\to+\infty]{\text{CVS}}0$ sur $]-R, R[$ alors la série \[
        \sum \frac{f^{(k)}(0)}{k!}x^k
    \]
    converge et on a \[
        f(x)=\sum_{n\geq 0} \frac{f^{(k)}(0)}{k!}x^k
    \]
    On en déduit que si $|z|<R$, \[
        \frac{f^{(k)}(0)}{k!}z^k\xrightarrow{k\to+\infty}0
    \]
    et \[
        \sum_{k\geq 0}\frac{f^{(k)}(0)}{k!}z^k
    \]
    a un rayon de convergence $\geq R$. La fonction $f$ se prolonge alors naturellement sur $\mathcal D_o(0, R)$ avec son DSE.
\end{rem}

\begin{ex}
    Pour la fonction $\exp$ sur $\mathbb R$, \[
        \left| R_n(x) \right|\leq \frac{|x|^{n+1}}{(n+1)!}\max(1, e^x)\xrightarrow[n\to+\infty]{}0
    \]
    donc la fonction $\exp$ est DSE sur $\mathbb R$ et admet un prolongement $\mathcal C^\infty$ sur $\mathbb C$ donné par ce DSE.
\end{ex}

\begin{thm}
    \begin{center}
        \begin{tabular}{ccc}
            \hline \hline Expression & Terme général du DSE & Rayon de Convergence \\
            \hline
            \rule{0pt}{2em} $\exp(x)$ & $\displaystyle \frac{x^k}{k!}$ & $+\infty$ \\
            \rule{0pt}{2em} $\ch(x)$ & $\displaystyle \frac{x^{2k}}{(2k)!}$ & $+\infty$ \\
            \rule{0pt}{2em} $\sh(x)$ & $\displaystyle \frac{x^{2k+1}}{(2k+1)!}$ & $+\infty$ \\
            \rule{0pt}{2em} $\sin(x)$ & $\displaystyle \frac{(-1)x^{2k+1}}{(2k+1)!}$ & $+\infty$ \\
            \rule{0pt}{2em} $\cos(x)$ & $\displaystyle \frac{(-1)^kx^{2k}}{(2k)!}$ & $+\infty$ \\
            \rule{0pt}{2em} $\dfrac1{1-x}$ & $x^k$ & $1$ \\
            \rule{0pt}{2em} $\ln(1-x)$ & $\dfrac{x^k}k$ & $1$ \\
            \rule{0pt}{2em} $(1+x)^\alpha$ & $\displaystyle \binom \alpha kx^k$ & $1$ si $\alpha\not\in\mathbb N$, $+\infty$ sinon \\[1em]
            \hline
        \end{tabular}
    \end{center}
\end{thm}

\begin{proof}
    On ne fait la preuve que pour le dernier cas. On pose \[
        f:x\longmapsto(1+x)^\alpha
    \]
    et \[
        g:x\longmapsto \sum_{k\geq 0}\binom \alpha kx^k.
    \] La fonction $f$ vérifie $f^{(k)}(0)=\alpha(\alpha-1)\cdots (\alpha-k+1)$, et la règle de D'Alembert \index{D'Alembert (règle de -- )} montre que dans le cas $\alpha \not \in \mathbb N$, $R(g)=1$. On a $g(0)=1$ et \[
    g'(x)=\sum_{k\geq 1}k\binom \alpha kx^{k-1}=\alpha\sum_{k\geq 1}\binom {\alpha-1}{k-1}x^{k-1}
\]
donc \[
    (1+x)g'(x)=\alpha \left( \sum_{k\geq 1} \left( \binom {\alpha-1}k+\binom{\alpha-1}{k-1} \right)x^k+1 \right)=\alpha g(x)
\]
C'est un problème de Cauchy dont $f$ et $g$ sont solutions, donc $f=g$.
\end{proof}

\begin{ex}
    $f=\arctan$. \begin{itemize}
        \item $f'(x)=\dfrac1{1+x^2}$ sur $\mathbb R$
        \item Sur $]-1, 1[$, \[
                \frac1{1+x^2}=\sum_{k\geq 0}(-1)^kx^{2k}
            \]
    \end{itemize}
    Donc, \[
        f(x)=\arctan x=\sum_{k\geq 0}\frac{(-1)^kx^{2k+1}}{2k+1}
    \]
\end{ex}

\begin{ex}
    $f:x\longmapsto \sqrt{1+x}$. On a (RCV $1$) \[
        \sqrt{1+x}=\sum_{k\geq 0}\binom {\frac12}kx^k
    \]
    et après calculs, on a l'expression suivante du binôme généralisé: \[
        \binom{\frac12}k=\frac{(-1)^{k-1}}{2^{2k-1}}\cdot \frac1{2k-1}\cdot \binom{2k-1}k
    \]
\end{ex}

\section{Méthodes pour obtenir un DSE}
\subsection{Utilisation d'une équation différentielle}

\todo{Récupérer cette partie}
\subsection{Utilisation d'une primitive}
On se donne $a\in\mathbb R$ et on pose \[
    \forall x\in ]-1, 1[, \qquad S(x)=\sum_{k\geq 1}\frac{\sin(ka)}kx^k
\]
On peut observer que $S(x)=\Im(H(x))$ où \[
    H(x)=\sum_{k\geq 1}(e^{ix})^k\frac{x^k}k.
\]
$H$ est dérivable sur $]-1, 1[$ et \[
    H'(x)=\sum_{k\geq 1}(e^{ix})^kx^{k-1}=\frac{e^{ia}}{1-e^{ia}x}
\]
d'où \[
    \Im(H'(x))=\frac{x\sin a\cos a+\sin a-x\sin a\cos a}{1+x^2-2x\cos a}=\frac{\sin a}{(x-\cos a)^2+\sin^2a}
\]
Il y a deux cas: \begin{itemize}
    \item $a\in\pi\mathbb Z$. Dans ce cas, $S\equiv 0$.
    \item $\sin a\neq 0$ et une primitive de $\Im(H'(x))$ est \[
            \arctan \left( \frac{x-\cos a}{\sin a} \right)
        \]
        et $S(0)=0$ permet d'avoir \[
            S(x)=\arctan \left( \frac{x-\cos a}{\sin a} \right)+\arctan \left( \tan \left( \frac\pi2-a \right) \right)
        \]
\end{itemize}

\begin{exo}
    Application à \[
        S(x)=\sum_{k\geq 1}\frac{\cos (ka)}kx^k
    \]
\end{exo}

\section{Utilisation des DSE}

\subsection{Pour la régularité}

On pose \[
    f:t\longmapsto \begin{cases}
        \ch(\sqrt t) &\text{ si }t\geq 0 \\
        \cos(\sqrt{-t}) &\text{ sinon }
    \end{cases}
\]
On veut savoir si $f$ est $\mathcal C^\infty$. On a \[
    \forall x\in\mathbb R, \qquad f(t)=\sum_{k\geq 0}\frac{t^k}{(2k)!}
\]
donc $f$ est DSE sur $\mathbb R$ et donc bien $\mathcal C^\infty$

\subsection{Utilisation combinatoire}

On note $a_n$ le nombre d'involutions de $\mathfrak S_n$. On convient que $a_0=0$. On va montrer que $a_{n+1}=a_n+na_{n-1}$, puis on va calculer \[
    \sum_{n\geq 0}\frac{a_n}{n!}x^n
\]
et montrer \[
    \frac{a_n}{n!}\xrightarrow[n\to+\infty]{}0
\]

Soit $\sigma\in\mathfrak S_{n+1}$ une involution. Il y a deux cas: \begin{itemize}
    \item $\sigma(n+1)=n+1$, on a $a_n$ choix possibles
    \item $\sigma(n+1)\neq n+1$. On a $n$ choix pour $\sigma(n+1)$ et $\sigma_{|\llbracket 1, n\rrbracket \setminus\{\sigma(n+1)\}}$ est une involution, donc on a au total $na_{n-1}$ choix.
\end{itemize}

On sait que $\frac{a_n}{n!}$ donc la série a un RCV $\geq 1$. On note \[
    f:x\in ]-1, 1[\;\longmapsto \sum_{n\geq 0}\frac{a_n}{n!}x^n
\]
On a alors \[
    f'(x)=\sum_{n\geq 1}\frac{na_n}{n!}x^{n-1}=\sum_{n\geq 0}\frac{a_{n+1}}{n!}x^n=1+x+(1+x)f(x)
\]
Puis $f'(0)=0$ donc $f$ est solution d'un problème de Cauchy. Par unicité: \[
    f(x)=e^xe^{\frac {x^2}2}-1
\]
et cette fonction est DSE avec un RCV infini, et la convergence en $1$ conclut.

\section{Étude au bord}

\subsection{Continuité radiale}

On note $f(z)=\sum_{n\geq 1}a_nz^n$ de rayon de convergence $R>0$. On suppose qu'il existe $z_0$ tel que \[
    \sum_{n\geq 0}a_nz_0^n
\]
converge et $|z_0|=R$. On va montrer que \[
    \lim_{\substack{t\to1\\t<1}}f(tz_0)=f(z_0).
\]
On note \[
    R_q\defeq \sum_{k=q+1}^{+\infty}a_kz_0^k\xrightarrow[q\to+\infty]{}0.
\]
On a \begin{align*}
    \sum_{k=n}^ma_kt^kz_0^k&=\sum_{k=n}^mt^k(R_{k-1}-R_k)\\&=R_{n-1}t^n+R_n(t^{n+1}-t^n)+\cdots +R_{m-1}(t^m-t^{m-1})+R_mt^m
\end{align*}
Soit $\epsilon>0$. Alors, $\exists N\in\mathbb N, \forall n\geq N, \quad |R_n|\leq \epsilon'=\frac\epsilon5$.
Pour $m>n\geq N+1$ et $t\in[0, 1]$, \[
    \left| \sum_{k=n}^ma_kt^kz_0^k \right|\leq 2\epsilon' + \epsilon'(t^n-t^{n+1}+\cdots t^{m-1}-t^m)\leq 3\epsilon'
\]
On fait $m\longrightarrow +\infty$ de sorte que \[
    f(tz_0)-f(z_0)=\sum_{k=0}^{n-1}(a_kt^kz_0^k-a_kz_0^k)+\sum_{k=n}^{+\infty}(a_nt^kz_0^k-a_kt_0^k)
\]
et avec $n=N$, \[
    \left| f(tz_0)-f(z_0) \right|\leq \underbrace{\left| \sum_{k=0}^{N-1}(a_kt^kz_0^k-a_kz_0^k) \right|}_{\displaystyle \xrightarrow[t\to 1]{}0}+4\epsilon'
\]
Donc il existe $\delta>0$ tel que $\forall t\in ]1-\delta, 1[$, $|f(tz_0)-f(z_0)|\leq 5\epsilon'=\epsilon$ ce qui conclut.

\subsection{Formule utile}

On note \[
    f:t\longmapsto \sum_{n\geq 0}a_nt^n
\]
avec un RCV $R\geq 1$.

Pour $|t|<1$, si on note $S_n=a_0+\cdots +a_n$, on a \[
    \frac{f(t)}{1-t}=\sum_{n\geq 0}S_nt^n\qquad \qquad \text{(Produit de Cauchy)}
\]

Si $f(1)$ converge, alors en notant $R_n=\sum_{k\geq n+1}a_k$, \[
    (t-1)\sum_{n\geq 0}R_nt^n=-R_0+\sum_{n\geq 1}(R_{n-1}-R_n)t^n=-R_0+f(t)-a_0=f(t)-f(1)
\]
donc pour $t\neq 1$, \[
    \frac{f(t)-f(1)}{t-1}=\sum_{n\geq 0}R_nt^n
\]

\begin{ex}
    \[
        \sum_{n\geq 1}\frac{t^n}n=-\ln(1-t)\implies \sum_{k\geq 1}H_kt^k=\frac{-\ln(1-t)}{1-t}
    \]
\end{ex}

\begin{ex}
    On note \[
        f:t\longmapsto \sum_{n\geq 2}\frac{(-1)^n}{\ln n}t^n
    \]
    avec le rayon de convergence $1$ (facile). Cette fonction est continue en $1$ (on peut le voir avec le TSA uniforme ou la continuité radiale). On va vérifier qu'elle est dérivable en $1$. On a \[
        \frac{f(t)-f(1)}{t-1}=\sum_{n\geq 0}R_nt^n\qquad \qquad \text{où}\quad R_n\defeq\sum_{k=n+1}^{+\infty}\frac{(-1)^k}{\ln k}
    \]
    Le TSA donne la convergence de $\sum R_n$, de sorte que la continuité radiale donne l'existence de la limite du taux d'accroissement en $1^-$, ce qui conclut.
\end{ex}

\subsection{Recherche d'équivalents}

\subsubsection{Limite en un point non-convergent du bord}

On se donne $f(t)=\sum_{n\geq 0}a_nt^n$ avec $a_n\geq 0$ et le RCV $1$, et on suppose que $\sum a_n$ diverge. On va montrer que \[
    f(t)\xrightarrow[t\to1^-]{}+\infty
\]
Soit $A>0$. Il existe un rang $N$ tel que \[
    \sum_{k=0}^Na_k>A+1.
\]
Puis, pour $t\in [0, 1[$, \[
    f(t)\geq \sum_{k=0}^Na_kt^k\xrightarrow[t\to1^-]{}\sum_{k=0}^Na_k
\]
donc il existe $\delta>0$ tel que $\forall t\in ]1-\delta, 1[$, \[
    f(t)\geq \sum_{k=0}^Nt^k\geq A
\]
donc \conc.

\subsubsection{Sommation de relation de comparaison pour les séries entières}

On se donne $f(t)=\sum_{n\geq 0}a_nt^n$ et $g(t)=\sum_{n\geq 0}b_nt^n$ de rayons de convergence $1$ et de termes généraux positifs et équivalents ($a_n,b_n\geq 0$ et $a_n\sim b_n$). On suppose que $\sum b_n$ diverge, et on va montrer que \[
    f(t)\underset{1^-}\sim g(t)
\]
Déjà, $g(t)\xrightarrow[t\to1^-]{}+\infty$ (vu \emph i). Puis, pour $\epsilon>0$, il existe $N\in\mathbb N$ tel que pour $n\geq N$, \[
    (1-\epsilon')b_n\leq a_n\leq (1+\epsilon')b_n \implies (1-\epsilon')\sum_{n\geq N}b_nt^n\leq \sum_{n\geq N}a_nt^n\leq(1+\epsilon')\sum_{n\geq N}b_nt^n
\]
D'où \[
    (1-\epsilon')g(t)-(1-\epsilon')\sum_{k=0}^{N-1}b_kt^k\leq f(t)-\sum_{k=0}^{N-1}a_kt^k\leq (1+\epsilon')g(t)-(1+\epsilon')\sum_{k=0}^{N-1}b_kt^k
\]
En divisant par $g(t)$ et pour un $\delta>0$, $1-\delta<t<1$, \[
    1-2\epsilon'\leq \frac fg(t)\leq 1+2\epsilon'
\]
d'où \conc.

\begin{exo}
    Même exercice mais avec un rayon de convergence infini et l'équivalent en $+\infty$. En déduire un équivalent en $+\infty$ de \[
        \sum_{n=1}^{+\infty} \left( 1+ \frac{1}{n} \right)^{n^2}\frac{x^n}{n!}
    \]
\end{exo}

\section{Résultats théoriques sur les DSE}

\subsection{La formule de Cauchy}

\index{Cauchy!formule (DSE)}Soit $f$ DSE avec $R(f)=R>0$. On va montrer que pour $z \in \mathcal{D}_o(0, R)$ et pour $r$ tel que $|z|<r<R$, \[
    f(z)=\frac{1}{2\pi}\int_{0}^{2\pi} \frac{f(re^{it})re^{it}}{re^{it}-z}\diff t
\]
On écrit \[
    f(z)=\sum_{n=0}^{+\infty} a_n z^n
\]
donc\[
    \forall t\in [0, 2\pi],  \qquad \frac{f(re^{it})re^{it}}{re^{it}-z}=\frac{f(re^{it})}{1- \left( \frac{z}{re^{it}} \right) } = f(re^{it}) \sum_{n=0}^{+\infty} \left( \frac{z}{re^{it}} \right)^k=\sum_{n,m\geq 0} \frac{a_nr^nz^m}{r^m}e^{i(n-m)t}
\]
où la dernière somme est celle d'une famille sommable. Donc la convergence normale sur $\mathcal{D}_f(0,r)$ donne \[
    \int_{0}^{2\pi} \frac{f(re^{it})re^{it}}{re^{it}-z}\diff t= \sum_{n,m\geq 0}a_nr^{n-m}z^m \int_{0}^{2\pi} e^{i(n-m)t}\diff t=\sum_{n\geq 0}2\pi a_nz^n=2\pi f(z)
\]

\begin{rem}
    Pour $0<r<R$, \[
        \int_{0}^{2\pi } f(re^{it})e^{-i nt}\diff t = \int_{0}^{2\pi } \left( \sum_{k=0}^{+\infty} a_kr^ke^{ikt} \right) e^{-int}\diff t=a_nr^{n}2\pi 
    \]
    donc \[
        a_n=\frac{1}{2\pi r^n}\int_{0}^{2\pi } f(re^{it})e^{-int}\diff t 
    \]
\end{rem}

\subsection{Théorème de Liouville}

\index{Liouville (théorème de -- )}On suppose $f$ DSE avec un rayon de convergence infini, $f$ bornée. On a \[
    \forall r>0, \forall n\in\mathbb N, \qquad a_n=\frac{1}{2\pi r^n}\int_{0}^{2\pi }f(re^{it}) e^{-int}\diff t
\]
Pour $n\geq 1$, \[
    |a_n|\leq \frac{1}{2\pi r^n}\int_{0}^{2\pi } \underbrace{|f(re^{it})e^{-int}}_{\leq \sup_{\mathbb C}|f|}\diff t\leq \frac{2\pi \sup_{\mathbb C}|f|}{2\pi r^n} \xrightarrow[r\to+\infty]{}0
\]
On a donc $f=a_0$ constante.


\subsection{Principe des zéros isolés}

On note $R$ le rayon de convergence de $f$ DSE en $0$, et on suppose $R>0$. \[
    \forall z \in \mathcal{D}_o(0,R),\qquad f(z)=\sum_{k=0}^{+\infty} \frac{f^{(k)}(0)}{k!}z^k
\]
Si $f(0)=0$ et $f\not\equiv 0$ alors on note $p=\min \{k\in\mathbb N, f^{(k)}(0)\neq 0\}$. Pour $z$ non nul au voisinage de $0$, \[
    f(z)=z^p \left( \frac{f^{(p)}(0)}{p!}+ z \frac{f^{(p+1)}(0)}{(p+1)!}+\cdots \right) \neq 0
\]
Donc $0$ est un zéro isolé. On va montrer que s'il existe une suite $(z_n)$ de racines deux à deux distinctes de $f$ qui converge vers $a\in\mathcal D_o(0,R)$, alors $f$ est nulle.

Par continuité, $a$ est une racine de $f$. Cette racine n'est pas isolée donc $f$ est nulle sur un voisinage de $a$ (par le raisonnement qui précède).

On note \[
    \mathcal E=\{z\in \mathcal  D_o(0,R), \quad f\equiv 0\text{ au voisinage de }z\}
\]

Si $z_0\in \mathcal E$, alors on note $r_0>0$ le rayon d'un disque ouvert tel que $f(\mathcal D_o(z_0,r_0))=\{0\}$. Pour tout $z\in \mathcal D_o(z_0, r_0)$, il existe $r_z>0$ tel que $\mathcal D_o(z,r_z)\subset \mathcal D_o(z_0,r_0)$ et $f$ s'annule sur ce disque, donc $\mathcal D_o(z_0, r_0)\subset \mathcal E$. Ainsi, $\mathcal E$ est un ouvert.

Si $(x_n)\in \mathcal E^{\mathbb N}$ converge vers $a\in \mathcal D_o(0, R)$, alors $a$ n'est pas isolée, $f$ s'annule sur un voisinage de $a$ (déjà vu) donc $a\in \mathcal E$. L'ensemble $\mathcal E$ est donc un fermé relatif du disque ouvert de convergence.

C'est un ouvert fermé relatif, donc $\mathcal E=\mathcal D_o(0, R)$ et $f$ est nulle.
