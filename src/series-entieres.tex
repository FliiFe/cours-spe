\ifsolo
    ~

    \vspace{1cm}

    \begin{center}
        \textbf{\LARGE Séries entières} \\[1em]
    \end{center}
    \tableofcontents
\else
    \chapter{Séries entières}

    \minitoc
\fi
\thispagestyle{empty}

\section{Définitions}

\begin{dfn}
    Soit $(a_n)_n$ une suite complexe. La série de fonctions de terme général $(a_nz^n)_n$ est appelée série entière\index{série entière} et se note $\sum a_nz^n$. Lorsqu'elle converge, on note sa somme \[
        \sum_{n=0}^{+\infty}a_nz^n
    \]
\end{dfn}

\begin{ex}
    Le domaine de convergence de $\sum z^n$ est $\mathcal D_0(0, 1)$.

    Le somaine de convergence $\mathcal D$ de $\sum \frac{z^n}n$ vérifie $\mathcal D_o(0, 1)\subsetneq\mathcal D\subsetneq \mathcal D_f(0, 1)$
\end{ex}

\section{Rayon de convergence}
\begin{thmdef}
    \Hyp $(a_n)_{n\in\mathbb N}\in\mathbb C^{\mathbb N}$
    \begin{concenum}
    \item Les cinq nombres suivants sont égaux {\begin{align*}
                R_1 &= \sup\{|z|, \quad z\in\mathbb C, (a_nz^n)_n\text{ bornée }\} \\
                R_2 &= \sup\{|z|, \quad z\in\mathbb C, a_nz^n\longrightarrow 0\}\\
                R_3 &= \sup\{|z|, \quad z\in\mathbb C, \sum a_nz^n\text{ converge }\} \\
                R_4 &= \sup\{|z|, \quad z\in\mathbb C, \sum a_nz^n\text{ converge absolument }\} \\
                R_5 &= \inf\{|z|, \quad z\in\mathbb C, (a_nz^n)_n\text{ non bornée}\}
    \end{align*}
avec la convention $\inf\emptyset=+\infty$.}
    On appelle ce nomber \emph{rayon de convergence\index{rayon de convergence}} de $\sum a_nz^n$ et on le note $R(a_nz^n)$
\item Si $\sum a_nz^n$ admet pour rayon de convergence $R$, alors le domaine de convergence est tel que \[
        \mathcal D_0(0, R)\subseteq \mathcal D\subseteq \mathcal D_f(0, R)
    \]
    \end{concenum}
\end{thmdef}

\begin{proof}
    \begin{enumerate}
        \item $R_1\geq R_2\geq R_3\geq R_4$. Par l'absurde, on suppose $R_4<R_5$. Alors, on prend $z\in\mathbb C$ tel que $R_5>|z|>R_4$. $(a_nz^n)_n$ est bornée donc si on note $z'$ tel que $|z|>|z'|>R_4$, alors \[
                a_nz'^n=\underbrace{a_nz^n}_{\text{borné}} \underbrace{\left( \frac{z'}{z}  \right)^n}_{|\;|<1} \text{ tg série ACV absurde car }|z'|>R_4
            \]
            Donc $R_4\geq R_5$. Par l'absurde, $R_1>R_5$. On se donne $z\in\mathbb C$ tel que $R_1>|z|>R_5$ et il existe $z', z''\in\mathbb C$ tel que $R_1\geq |z''|>|z|>|z'|\geq R_5$, $(a_nz'^n)$ non bornée et $(a_nz''^n)$ bornée. \[
                \underbrace{a_nz'^n}_{\lnot \text{ borné}}=a_nz''^n \left( \frac{z'}{z''} \right)^n\xrightarrow{}_{n\to+\infty}0
            \]
            absurde donc $R_1=R_2=R_3=R_4=R_5$.
        \item Immédiat vu $1$.
    \end{enumerate}
\end{proof}
% 
\begin{rem}[Lemme d'Abel]
    Si $a\in\mathbb C^{\mathbb N}$ et $z_0\in\mathbb C^\star$. Si la suite $(a_nz_0^n)$ est bornée alors $\forall z\in\mathcal D_o(0, |z_0|), \sum a_nz^n$ ACV
\end{rem}

\section{Opération algébrique sur les séries entières}

\begin{prop}
    \Hyp $\sum a_nz^n$ de rayon $R_a$ et $\sum b_nz^n$ de rayon $R_b$
    \begin{concenum}
    \item Pour $\lambda\in\mathbb C^\star$, $R(\sum \lambda a_nz^n)=R_a$
    \item Pour $\lambda\in\mathbb C^\star$, $R(\sum(\lambda a_n+b_n)z^n)\geq \min(R_a, R_b)$ et si $R_a\neq R_b$, il y a égalité.
    \item Le produit de Cauchy des séries $\sum a_nz^n$ et $\sum b_nz^n$ est la série $\sum c_nz^n$ avec \[
            c_n=\sum_{p+q=n}a_pb_q
        \]
        et $R(\sum c_nz^n)\geq \min(R_a, R_b)$.
    \end{concenum}
\end{prop}

\begin{proof}
    \begin{enumerate}
        \item "Trivial"
        \item $\geq 0$ OK et $R_a\neq R_b\implies =$ OK
        \item Si $|z|<\min(R_a, R_b)$ alors $\sum a_nz^n, \sum b_nz^n$ ACV donc produit de Cauchy ie \conc
    \end{enumerate}
\end{proof}

\section{Régularité des séries entières}

\begin{prop}
    \Hyp $\sum a_nz^n$ de RCV $R>0$ (ou $+\infty$)
    \begin{concenum}
    \item $\forall a\in ]0, R[, \quad \sum a_nz^n$ CVN sur $\mathcal D_f(0, a)$.
    \item $z\in\mathcal D_o(0, R)\longmapsto\displaystyle \sum_{k=0}^{+\infty}a_nz^n$ est continue
    \end{concenum}
\end{prop}

\begin{proof}
    $a\in]0, R[$. \[
        \sum_{z\in\mathcal D_f(0, a)}|a_nz^n|=|a_na^n|\text{ tg série CV car } a<R \text{ donc il y a CVN sur }\mathcal D_f(0, a)
    \]
    d'où la continuité sur tous les $\mathcal D_f(0, a)$ donc sur $\mathcal D_o(0, a)$.
\end{proof}

\begin{thm}
    \Hyp $f(z)=\sum_{n\geq 0}a_nz^n$ série entière de rayon $R>0$
    \begin{concenum}
    \item On appelle série dérivée\index{série dérivée} la série $\sum_{n\geq 1}na_nz^{n-1}$
        \begin{enumerate}
            \item Cette série a le même rayon de convergence que $f$
            \item $f$ est dérivable sur $]-R, R[$ et $\forall t\in ]-R, R[,$ \[
                    f'(t)=\sum_{n\geq 1}a_nnt^{n-1}
                \]
        \end{enumerate}
        \item La série $\sum_{n\geq 0}\frac{a_n}{n+1}z^{n+1}$ a même rayon de convergence que $f$ et pour $t\in ]-R, R[$, \[
            \sum_{n\geq 0}\frac{a_n}{n+1}z^{n+1}=\int_0^tf(u)\diff u
            \]
    \end{concenum}
\end{thm}

\begin{proof}
    \begin{enumerate}
        \item $|na_nz^{n-1}|=|a_na^n|\times \left| (\frac za)^{n-1} \right|\times \frac na\xrightarrow{}0$ si $|z|<a<R$. Puis, $(a_nz^n)$ non bornée entraîne $(na_nz^{n-1})$ non bornée. On a donc $R(\sum na_nz^{n-1})=R$.

            On vérifie les hypothèses du théorème de dérivation terme à terme: \begin{itemize}
                \item $\forall n\in\mathbb N, \quad t\in]-R,R[\longmapsto a_nt^n$ est $\mathcal C^1$
                \item $\sum a_nt^n$ converge simplement sur $]-R, R[$
                \item $\sum(a_nz^n)'=\sum na_nz^{n-1}$ converge normalement donc univormément sur tout segment de $]-R, R[$.
            \end{itemize}
            Ainsi, le théorème donne bien \conc
        \item Il y a égalité des rayons de convergence (vu 1) et pour tout $t\in ]-R,R[$, la série $\sum a_nz^n$ converge normalement donc uniformément sur $[0,t]$, donc le théorème d'intégration terme à terme donne \[
                \int_0^tf(u)\diff u=\sum a_n\int_0^t u^n\diff u=\sum \frac{a_n}{n+1}t^{n+1}
            \]
    \end{enumerate}
\end{proof}

\begin{ex}
    On note $f(t)=-\ln(1-t)$ de sorte que \[
        \forall t\in ]-1, 1[, \qquad f(t)=\sum_{n\geq 0}\frac{t^{n+1}}{n+1}.
    \]
    Alors, \[
        f(-t)=\sum_{n\geq 0}(-1)^{n+1}\frac{t^{n+1}}{n+1}
    \]
    et (théorème des séries alternées\index{séries alternées (théorème des --), TSA}) \[
        \forall t\in[0, 1[, \forall n\in\mathbb N, \qquad \left|f(-t)-\sum_{k=0}^n(-1)^{k+1}\frac{t^{k+1}}{k+1}\right|\leq \frac1{n+2}\xrightarrow[n\to+\infty]{}0
    \]
    donc le reste CVU vers $0$, puis le théorème de la double limite donne \[
        f(-t)\xrightarrow[\substack{t\to1\\t<1}]{}\sum_{n\geq 0}\frac{(-1)^{n+1}}{n+1}=-\ln2
    \]
\end{ex}

\begin{rem}
    Pour $f:\mathbb C\longrightarrow \mathbb C$, on peut définir la dérivabilité au sens complexe par \[
        f \text{ dérivable en } a\iff \lim_{\substack{h\to0\\h\neq 0}}\frac{f(a+h)-f(a)}h\text{ existe }
    \]
    On note $f(z)=\sum a_nz^n$ une série entière de rayon de convergence $R>0$ et $a\in\mathcal D_o(0, R)$. On va montrer que $f$ est dérivable au sens complexe en $a$ et que \[
        f'(a)=\sum_{n\geq 0}na_na^{n-1}
    \]
    Pour $h\neq 0$ assez petit en module: 
    \begin{align*}
        \frac{f(a+h)-f(a)}h&=\sum_{n\geq 0}a_n\frac{(a+h)^n-a^n}h\\&=\sum_{n\geq 0}a_n\underbrace{\left(\binom n1a^{n-1}+\binom n2a^{n-2}h+\cdots+h^{n-1}\right)}_{\displaystyle\xrightarrow[\substack{h\to0\\h\neq 0}]{}na^{n-1}}
    \end{align*}
    Puis si on pose $\varphi:t\longmapsto (a+th)^n$, on a (Taylor-Lagrange) \[
        |\varphi(1)-\varphi(0)-\varphi'(0)|\leq \frac1{2!}\sup_{[0,1]}|\varphi''|\leq \frac{n(n-1)}2|h|^2(|a|+|h|)^{n-2}
    \]
    donc si $b$ est tel que $|a|<b<R$, il existe un voisinage $V$ de $0$ tel que $\forall h\in V, |a|+|h|<|b|$ et dans ce cas
    \begin{align*}
        \left|  \frac{f(a+h)-f(a)}h-\sum_{n\geq 0}na_na^{n-1}\right|&\leq \left| \sum_{n\geq 0}a_n \left( \frac{(a+h)^n-a^n}h-na^{n-1} \right) \right| \\
                                                                    &\leq \sum_{n\geq 0}|a_n|\frac{n(n-1)}2|h|(|a|+|h|)^{n-2}\\
                                                                    &\leq |h|\underbrace{\sum_{n\geq 0}|a_n|\frac{n(n-1)}2b^{n-2}}_{\text{CV car $b<R$, série dérivée 2\textsuperscript{nde}}}\xrightarrow[\substack{h\to0\\h\neq 0}]{}0
    \end{align*}
\end{rem}

\begin{ex}
    Calcul de $I=\displaystyle\int_0^1\ln(1-x)\ln x\diff x$. \begin{itemize}
        \item $x\longmapsto \ln(1-x)\ln x$ est intégrable car continue par prolongement
        \item $\forall x\in ]0,1[$, $\ln(1-x)=-\sum_{n\geq 0}\frac{x^{n+1}}{n+1}$ donc \[
                I=-\int_0^1\sum_{n\geq 0}\frac{x^{n+1}}{n+1}\ln x\diff x
            \]
        \item $\sum \frac{x^{n+1}}{n+1}\ln x$ converge simplement sur $]0,1[$
        \item Les $x\longmapsto \frac{x^{n+1}}{n+1}\ln x$ sont continues intégrables sur $]0, 1[$ et \[
                \int_0^1 \left| \frac{x^{n+1}}{n+1}\ln x \right|\diff x=-\int_0^1\frac{x^{n+1}}{n+1}\ln x\diff x=\frac1{(n+2)^2(n+1)}\quad \text{tg série CV}
            \]
    \end{itemize}
    Donc le théorème d'interversion série-intégrale (version convergence dominée) donne \[
        I=\sum_{n\geq 0}\int_0^1\frac{-x^{n+1}}{n+1}\ln x\diff x=\sum_{n\geq 0}\frac{1}{(n+2)^2(n+1)}
    \]
    et \[
        \frac1{X(X+1)^2}=\frac{-1}{(X+1)^2}+\frac{-1}{X+1}+\frac1X
    \]
    d'où finalement $I=2-\dfrac{\pi^2}6$
\end{ex}

\begin{thm}
    \Hyp $f(z)=\sum_{n\geq 0}a_nz^n$ avec le RCV $R>0$
    \begin{concenum}
    \item $f\in\mathcal C^\infty(]-R, R[, \mathbb C)$ et \[\forall n\in\mathbb N, \quad \frac{f^{(n)}(0)}{n!}=a_n\]
    \item \[
            \forall t\in ]-R, R[, \qquad f(t)=\sum_{n\geq 0}\frac{f^{(n)}(0)}{n!}t^n
        \]
    \item Si $\sum a_nz^n=\sum b_nz^n$ au voisinage de $0$ alors $\forall n, a_n=b_n$
    \end{concenum}
\end{thm}

\begin{proof}
    Facile.
\end{proof}

\begin{rem}
    On peut faire des DSE ailleurs qu'en $0$. Soit $f(z)=\sum_{n\geq 0}a_nz^n$ DSE avec un rayon de convergence non nul, et $a\in\mathcal D_o(0, R)$. On va montrer que pour $z\in\mathcal D_o(a, R-|a|)$, \[
        f(z)=\sum_{n\geq 0}\frac{f^{(n)}(a)}{n!}(z-a)^n
    \]
    On a: \[
        \forall z\in\mathcal D_o(0, R), \qquad f(z)=\sum_{n\geq 0}a_nz^n
    \]
    donc \[\forall k\geq 0, \qquad f^{(k)}(a)=\sum_{n\geq k}a_n\frac{n!}{(n+k)!}a^{n-k}.\]
    On va montrer que la famille \[
        \left( a_n\frac{n!}{(n+k)!}a^{n-k}\frac{(z-a)^k}{k!} \right)_{n\geq k\geq 0}
    \]
    est sommable. Pour $n\in\mathbb N$, \[
        \sum_{k=0}^n \left| a_n\frac{n!}{(n+k)!}a^{n-k}\frac{(z-a)^k}{k!} \right|=|a_n|(|a|+|z-a|)^n
    \]
    et si $z\in\mathcal D_o(a, R-|a|)$, on a $|a|+|z-a|<R$ donc c'est le terme général d'une série convergente. Le théorème de Fubini\index{Fubini (théorème discret de -- )} donne alors \[
        \sum_{n\geq 0}\sum_{k=0}^na_n\frac{n!}{(n+k)!}a^{n-k}\frac{(z-a)^k}{k!}=\sum_{n\geq 0}a_nz^n=f(z)=\sum_{k\geq 0}\sum_{n\geq k}a_n\frac{n!}{(n+k)!}a^{n-k}\frac{(z-a)^k}{k!}=\sum_{k\geq 0}\frac{f^{(k)}(a)}{k!}(z-a)^k
    \]
\end{rem}

\section{DSE usuels}

\begin{rem}
    On note $f\in\mathcal C^\infty(]-R, R[, \mathbb C)$. Pour $n\in\mathbb N$ et $x\in ]-R, R[$, \[
        f(x)=\sum_{k=0}^n\frac{f^{(k)}(0)}{k!}x^k+\underbrace{\int_0^x\frac{(x-t)^n}{n!}f^{(n+1)}(t)\diff t}_{\defeq R_n(x)}
    \]
    Si $R_n(x)\xrightarrow[n\to+\infty]{\text{CVS}}0$ sur $]-R, R[$ alors la série \[
        \sum \frac{f^{(k)}(0)}{k!}x^k
    \]
    converge et on a \[
        f(x)=\sum_{n\geq 0} \frac{f^{(k)}(0)}{k!}x^k
    \]
    On en déduit que si $|z|<R$, \[
        \frac{f^{(k)}(0)}{k!}z^k\xrightarrow{k\to+\infty}0
    \]
    et \[
        \sum_{k\geq 0}\frac{f^{(k)}(0)}{k!}z^k
    \]
    a un rayon de convergence $\geq R$. La fonction $f$ se prolonge alors naturellement sur $\mathcal D_o(0, R)$ avec son DSE.
\end{rem}

\begin{ex}
    Pour la fonction $\exp$ sur $\mathbb R$, \[
        \left| R_n(x) \right|\leq \frac{|x|^{n+1}}{(n+1)!}\max(1, e^x)\xrightarrow[n\to+\infty]{}0
    \]
    donc la fonction $\exp$ est DSE sur $\mathbb R$ et admet un prolongement $\mathcal C^\infty$ sur $\mathbb C$ donné par ce DSE.
\end{ex}

\begin{thm}
    \begin{center}
        \begin{tabular}{ccc}
            \hline \hline Expression & Terme général du DSE & Rayon de Convergence \\
            \hline
            \rule{0pt}{2em} $\exp(x)$ & $\displaystyle \frac{x^k}{k!}$ & $+\infty$ \\
            \rule{0pt}{2em} $\ch(x)$ & $\displaystyle \frac{x^{2k}}{(2k)!}$ & $+\infty$ \\
            \rule{0pt}{2em} $\sh(x)$ & $\displaystyle \frac{x^{2k+1}}{(2k+1)!}$ & $+\infty$ \\
            \rule{0pt}{2em} $\sin(x)$ & $\displaystyle \frac{(-1)x^{2k+1}}{(2k+1)!}$ & $+\infty$ \\
            \rule{0pt}{2em} $\cos(x)$ & $\displaystyle \frac{(-1)^kx^{2k}}{(2k)!}$ & $+\infty$ \\
            \rule{0pt}{2em} $\dfrac1{1-x}$ & $x^k$ & $1$ \\
            \rule{0pt}{2em} $\ln(1-x)$ & $\dfrac{x^k}k$ & $1$ \\
            \rule{0pt}{2em} $(1+x)^\alpha$ & $\displaystyle \binom \alpha kx^k$ & $1$ si $\alpha\not\in\mathbb N$, $+\infty$ sinon \\[1em]
            \hline
        \end{tabular}
    \end{center}
\end{thm}

\begin{proof}
    On ne fait la preuve que pour le dernier cas. On pose \[
        f:x\longmapsto(1+x)^\alpha
    \]
    et \[
        g:x\longmapsto \sum_{k\geq 0}\binom \alpha kx^k.
    \] La fonction $f$ vérifie $f^{(k)}(0)=\alpha(\alpha-1)\cdots (\alpha-k+1)$, et la règle de D'Alembert \index{D'Alembert (règle de -- )} montre que dans le cas $\alpha \not \in \mathbb N$, $R(g)=1$. On a $g(0)=1$ et \[
    g'(x)=\sum_{k\geq 1}k\binom \alpha kx^{k-1}=\alpha\sum_{k\geq 1}\binom {\alpha-1}{k-1}x^{k-1}
\]
donc \[
    (1+x)g'(x)=\alpha \left( \sum_{k\geq 1} \left( \binom {\alpha-1}k+\binom{\alpha-1}{k-1} \right)x^k+1 \right)=\alpha g(x)
\]
C'est un problème de Cauchy dont $f$ et $g$ sont solutions, donc $f=g$.
\end{proof}

\begin{ex}
    $f=\arctan$. \begin{itemize}
        \item $f'(x)=\dfrac1{1+x^2}$ sur $\mathbb R$
        \item Sur $]-1, 1[$, \[
                \frac1{1+x^2}=\sum_{k\geq 0}(-1)^kx^{2k}
            \]
    \end{itemize}
    Donc, \[
        f(x)=\arctan x=\sum_{k\geq 0}\frac{(-1)^kx^{2k+1}}{2k+1}
    \]
\end{ex}

\begin{ex}
    $f:x\longmapsto \sqrt{1+x}$. On a (RCV $1$) \[
        \sqrt{1+x}=\sum_{k\geq 0}\binom {\frac12}kx^k
    \]
    et après calculs, on a l'expression suivante du binôme généralisé: \[
        \binom{\frac12}k=\frac{(-1)^{k-1}}{2^{2k-1}}\cdot \frac1{2k-1}\cdot \binom{2k-1}k
    \]
\end{ex}

% \section{Utilisation d'une primitive}
