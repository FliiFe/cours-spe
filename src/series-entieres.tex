\ifsolo
    ~

    \vspace{1cm}

    \begin{center}
        \textbf{\LARGE Séries entières} \\[1em]
    \end{center}
    \tableofcontents
\else
    \chapter{Séries entières}

    \minitoc
\fi
\thispagestyle{empty}

\section{Définitions}

\begin{dfn}
    Soit $(a_n)_n$ une suite complexe. La série de fonctions de terme général $(a_nz^n)_n$ est appelée série entière\index{série entière} et se note $\sum a_nz^n$. Lorsqu'elle converge, on note sa somme \[
        \sum_{n=0}^{+\infty}a_nz^n
    \]
\end{dfn}

\begin{ex}
    Le domaine de convergence de $\sum z^n$ est $\mathcal D_0(0, 1)$.

    Le somaine de convergence $\mathcal D$ de $\sum \frac{z^n}n$ vérifie $\mathcal D_o(0, 1)\subsetneq\mathcal D\subsetneq \mathcal D_f(0, 1)$
\end{ex}

\section{Rayon de convergence}
\begin{thmdef}
    \Hyp $(a_n)_{n\in\mathbb N}\in\mathbb C^{\mathbb N}$
    \begin{concenum}
    \item Les cinq nombres suivants sont égaux {\begin{align*}
                R_1 &= \sup\{|z|, \quad z\in\mathbb C, (a_nz^n)_n\text{ bornée }\} \\
                R_2 &= \sup\{|z|, \quad z\in\mathbb C, a_nz^n\longrightarrow 0\}\\
                R_3 &= \sup\{|z|, \quad z\in\mathbb C, \sum a_nz^n\text{ converge }\} \\
                R_4 &= \sup\{|z|, \quad z\in\mathbb C, \sum a_nz^n\text{ converge absolument }\} \\
                R_5 &= \inf\{|z|, \quad z\in\mathbb C, (a_nz^n)_n\text{ non bornée}\}
    \end{align*}
avec la convention $\inf\emptyset=+\infty$.}
    On appelle ce nomber \emph{rayon de convergence\index{rayon de convergence}} de $\sum a_nz^n$ et on le note $R(a_nz^n)$
\item Si $\sum a_nz^n$ admet pour rayon de convergence $R$, alors le domaine de convergence est tel que \[
        \mathcal D_0(0, R)\subseteq \mathcal D\subseteq \mathcal D_f(0, R)
    \]
    \end{concenum}
\end{thmdef}

\begin{proof}
    \begin{enumerate}
        \item $R_1\geq R_2\geq R_3\geq R_4$. Par l'absurde, on suppose $R_4<R_5$. Alors, on prend $z\in\mathbb C$ tel que $R_5>|z|>R_4$. $(a_nz^n)_n$ est bornée donc si on note $z'$ tel que $|z|>|z'|>R_4$, alors \[
                a_nz'^n=\underbrace{a_nz^n}_{\text{borné}} \underbrace{\left( \frac{z'}{z}  \right)^n}_{|\;|<1} \text{ tg série ACV absurde car }|z'|>R_4
            \]
            Donc $R_4\geq R_5$. Par l'absurde, $R_1>R_5$. On se donne $z\in\mathbb C$ tel que $R_1>|z|>R_5$ et il existe $z', z''\in\mathbb C$ tel que $R_1\geq |z''|>|z|>|z'|\geq R_5$, $(a_nz'^n)$ non bornée et $(a_nz''^n)$ bornée. \[
                \underbrace{a_nz'^n}_{\lnot \text{ borné}}=a_nz''^n \left( \frac{z'}{z''} \right)^n\xrightarrow{}_{n\to+\infty}0
            \]
            absurde donc $R_1=R_2=R_3=R_4=R_5$.
        \item Immédiat vu $1$.
    \end{enumerate}
\end{proof}
% 
\begin{rem}[Lemme d'Abel]
    Si $a\in\mathbb C^{\mathbb N}$ et $z_0\in\mathbb C^\star$. Si la suite $(a_nz_0^n)$ est bornée alors $\forall z\in\mathcal D_o(0, |z_0|), \sum a_nz^n$ ACV
\end{rem}

\section{Opération algébrique sur les séries entières}

\begin{prop}
    \Hyp $\sum a_nz^n$ de rayon $R_a$ et $\sum b_nz^n$ de rayon $R_b$
    \begin{concenum}
    \item Pour $\lambda\in\mathbb C^\star$, $R(\sum \lambda a_nz^n)=R_a$
    \item Pour $\lambda\in\mathbb C^\star$, $R(\sum(\lambda a_n+b_n)z^n)\geq \min(R_a, R_b)$ et si $R_a\neq R_b$, il y a égalité.
    \item Le produit de Cauchy des séries $\sum a_nz^n$ et $\sum b_nz^n$ est la série $\sum c_nz^n$ avec \[
            c_n=\sum_{p+q=n}a_pb_q
        \]
        et $R(\sum c_nz^n)\geq \min(R_a, R_b)$.
    \end{concenum}
\end{prop}

\begin{proof}
    \begin{enumerate}
        \item "Trivial"
        \item $\geq 0$ OK et $R_a\neq R_b\implies =$ OK
        \item Si $|z|<\min(R_a, R_b)$ alors $\sum a_nz^n, \sum b_nz^n$ ACV donc produit de Cauchy ie \conc
    \end{enumerate}
\end{proof}

\section{Régularité des séries entières}

\begin{prop}
    \Hyp $\sum a_nz^n$ de RCV $R>0$ (ou $+\infty$)
    \begin{concenum}
    \item $\forall a\in ]0, R[, \quad \sum a_nz^n$ CVN sur $\mathcal D_f(0, a)$.
    \item $z\in\mathcal D_o(0, R)\longmapsto\displaystyle \sum_{k=0}^{+\infty}a_nz^n$ est continue
    \end{concenum}
\end{prop}

\begin{proof}
    $a\in]0, R[$. \[
        \sum_{z\in\mathcal D_f(0, a)}|a_nz^n|=|a_na^n|\text{ tg série CV car } a<R \text{ donc il y a CVN sur }\mathcal D_f(0, a)
    \]
    d'où la continuité sur tous les $\mathcal D_f(0, a)$ donc sur $\mathcal D_o(0, a)$.
\end{proof}

\begin{thm}
    \Hyp $f(z)=\sum_{n\geq 0}a_nz^n$ série entière de rayon $R>0$
    \begin{concenum}
    \item On appelle série dérivée la série $\sum_{n\geq 1}na_nz^{n-1}$
        \begin{enumerate}
            \item Cette série a le même rayon de convergence que $f$
            \item $f$ est dérivable sur $]-R, R[$ et $\forall t\in ]-R, R[,$ \[
                    f'(t)=\sum_{n\geq 1}a_nnt^{n-1}
                \]
        \end{enumerate}
        \item La série $\sum_{n\geq 0}\frac{a_n}{n+1}z^{n+1}$ a même rayon de convergence que $f$ et pour $t\in ]-R, R[$, \[
            \sum_{n\geq 0}\frac{a_n}{n+1}z^{n+1}=\int_0^tf(u)\diff u
            \]
    \end{concenum}
\end{thm}

\begin{proof}
    \begin{enumerate}
        \item $|na_nz^{n-1}|=|a_na^n|\times \left| (\frac za)^{n-1} \right|\xrightarrow{}0$ si $|z|<R$.
    \end{enumerate}
\end{proof}
