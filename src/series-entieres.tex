\ifsolo
    ~

    \vspace{1cm}

    \begin{center}
        \textbf{\LARGE Séries entières} \\[1em]
    \end{center}
    \tableofcontents
\else
    \chapter{Séries entières}

    \minitoc
\fi
\thispagestyle{empty}

\section{Définitions}

\begin{dfn}
    Soit $(a_n)_n$ une suite complexe. La série de fonctions de terme général $(a_nz^n)_n$ est appelée série entière\index{série entière} et se note $\sum a_nz^n$. Lorsqu'elle converge, on note sa somme \[
        \sum_{n=0}^{+\infty}a_nz^n
    \]
\end{dfn}

\begin{ex}
    Le domaine de convergence de $\sum z^n$ est $\mathcal D_0(0, 1)$.

    Le somaine de convergence $\mathcal D$ de $\sum \frac{z^n}n$ vérifie $\mathcal D_o(0, 1)\subsetneq\mathcal D\subsetneq \mathcal D_f(0, 1)$
\end{ex}

\section{Rayon de convergence}
\begin{thmdef}
    \Hyp $(a_n)_{n\in\mathbb N}\in\mathbb C^{\mathbb N}$
    \begin{concenum}
    \item Les cinq nombres suivants sont égaux {\begin{align*}
                R_1 &= \sup\{|z|, \quad z\in\mathbb C, (a_nz^n)_n\text{ bornée }\} \\
                R_2 &= \sup\{|z|, \quad z\in\mathbb C, a_nz^n\longrightarrow 0\}\\
                R_3 &= \sup\{|z|, \quad z\in\mathbb C, \sum a_nz^n\text{ converge }\} \\
                R_4 &= \sup\{|z|, \quad z\in\mathbb C, \sum a_nz^n\text{ converge absolument }\} \\
                R_5 &= \inf\{|z|, \quad z\in\mathbb C, (a_nz^n)_n\text{ non bornée}\}
    \end{align*}
avec la convention $\inf\emptyset=+\infty$.}
    On appelle ce nomber \emph{rayon de convergence\index{rayon de convergence}} de $\sum a_nz^n$ et on le note $R(a_nz^n)$
\item Si $\sum a_nz^n$ admet pour rayon de convergence $R$, alors le domaine de convergence est tel que \[
        \mathcal D_0(0, R)\subseteq \mathcal D\subseteq \mathcal D_f(0, R)
    \]
    \end{concenum}
\end{thmdef}

\begin{proof}
    
\end{proof}
