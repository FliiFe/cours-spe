\ifsolo
    ~

    \vspace{1cm}

    \begin{center}
        \textbf{\LARGE Suites et Séries de fonctions} \\[1em]
    \end{center}
    \tableofcontents
\else
    \chapter{Suites et Séries de fonctions}

    \minitoc
\fi
\thispagestyle{empty}

Dans tout le chapitre, $I$ est un intervalle non trivial, $A$ est une partie non vide de $\R$ et $K=\R$ ou $\C$. Plus tard, on pourra remplacer $A$ par une partie d'un e.v.n (de dimension finie généralement) et $K$ un e.v.n.

\section{Différents types de convergence}

\begin{dfn}
    On note $(f_n)_n$ une suite de fonctions de $I$ dans $K$. \begin{enumerate}
        \item On dit que $(f_n)_n$ \emph{converge simplement}\index{convergence simple} (CVS) s'il existe $f:I\to K$ telle que $\forall x\in I, f_n(x)\xrightarrow[n\to+\infty]{} f(x)$
        \item On dit que $(f_n)_n$ \emph{converge uniformément}\index{convergence uniforme} (CVU) s'il existe $f:I\to K$ telle que \begin{itemize}
            \item $(f_n-g)$ est bornée à partir d'un certain rang
            \item $\sup_I|f_n-g|\xrightarrow[n\to+\infty]{}0$
        \end{itemize}
    \item On dit que $(f_n)$ converge uniformément sur tout segment (CVUTS) si pour tout segment $J\subset I$, on a $f_n\xrightarrow[J]{\mathrm{CVU}}f$
    \end{enumerate}
    Dans chacun des cas, on peut remplacer $I$ par $A$.
\end{dfn}

\begin{rem}
    \begin{itemize}
        \item CVU $\implies$ CVUTS $\implies$ CVS
        \item CVS $\centernot\implies$ CVUTS $\centernot \implies$ CVU
    \end{itemize}
\end{rem}

\begin{rem}
    La limite uniforme d'une suite uniformément convergente est identique à la limite simple de cette suite. En particulier, on a unicité de la limite pour les trois convergences.
\end{rem}

\section{Opérations simples sur les convergences}

\begin{prop}
    \Hyp $(u_n)_n,(v_n)_n$ sont des suites de fonctions de $I$ dans $K$, $u,v:I\to K$, $\lambda\in K$.
    \Conc Si $u_n\xrightarrow{}u$, $v_n\xrightarrow v$ avec de la convergence simple (resp. CVUTS, CVU) alors \[
        (\lambda u_n+v_n)\xrightarrow{}\lambda u+v
    \]
    avec la convergence simple (resp. CVUTS, CVU).
\end{prop}

\begin{proof}
    Évident.
\end{proof}

\begin{rem}
Pour la convergence simple, on a $u_nv_n \longrightarrow uv$, mais pas pour la  CVU ou la CVUTS. Par exemple, \[
\begin{matrix}
    u:& \R_+ & \longrightarrow &\R  \\
    & x & \longmapsto & e^x+\frac{1}{n}
\end{matrix}
\] 
On a \[
    u_n \xrightarrow[\R_+]{\text{CVU}} u
\] 
On a aussi  \[
    u_n^2 \xrightarrow[\R_+]{\text{CVS}} u^2
\]
donc si $u_n^2$ converge uniformément, \[
    u_n^2(x)-u^2(x) = \frac{2e^x}{n}+\frac{1}{n^2} \xrightarrow[n\to+\infty]{}0
\] 
ce qui est absurde
\end{rem}

\begin{prop}[Propriétées héritées par CVS]
    \Hyp $u_n: I\to \K$, $u_n \xrightarrow[]{\text{CVS}} u$ sur $I$
    \Conc La monotonie, la convexité, le caractère $k$-lipschitzien, la positivité sont hérités.
\end{prop}

\begin{proof}
On écrit les inégalités et on passe à la limite.
\end{proof}

\section{Propriétés héritées par CVU}

\begin{thm}[Double limite\index{double limite (théorème)}]
    \Hyp $u_n \xrightarrow[]{\text{CVU}} u$ sur $A$,  $a$ point adhérent de  $A$,  $u_n(x) \xrightarrow[x\to a]{}\ell _n \in  \K$
    \begin{concenum}
    \item $(\ell _n)$ converge vers $\ell  \in \K$
    \item La fonction $u$ a pour limite  $\ell $ en $a$, i.e.  \[
            \lim_{\substack{x \to  a\\x \in  A}} \lim_{x \to  +\infty} u_n(x)=\lim_{n \to  +\infty}\lim_{\substack{x \to  a\\ x \in  A}} u_n(x)
    \] 
    \end{concenum}
\end{thm}

\begin{proof}~
\begin{enumerate}
    \item (admis) On va construire $(n_k)$ strictement croissante telle que  $ \forall  n\geq n_k, |\ell _n-\ell _{n_k}| \leq  \dfrac{1}{2^k}$. \begin{itemize}
        \item $\forall \epsilon>0, \exists  N \in  \N, \forall  n\geq N, \forall  x \in  A, |u_n(x)-u(x)|\leq \epsilon'$
        \item $\forall  n, p\geq N, |u_n(x)-u_p(x)|\leq |u_n(x)-u(x)|+|u(x)-u_p(x)|\leq 2\epsilon'=\epsilon$ et $x \longrightarrow a$ donne  $|\ell _n-\ell _p|\leq \epsilon$
        \item Pour $\epsilon=1$, il existe  $n_0 \in  \N$ tel que $\forall n\geq n_0, |\ell _n-\ell _{n_0}|\leq 1$
        \item Pour $k \in  \N$, si on suppose $n_0<\cdots <n_k$ construits, pour $\epsilon=\frac{1}{2^{k+1}}$, il existe $N \in  \N$ tel que $\forall  p \geq N, |\ell  _n-\ell _p|\leq \epsilon$. On prend $n_{k+1}=\max(N, n_0,\cdots , n_k+1)$.
    \end{itemize}
    Puis $|\ell _{n_{k+1}}-\ell _{n_k}|\leq \frac{1}{2^k}$, c'est le terme général d'une série convergente donc $(\ell _{n_k})$ converge.

     Soit $\epsilon>0$. Il existe $K$ tel que  $k\geq K$, $\frac{1}{2^k}\leq \epsilon'$ et $|\ell  -\ell _{n_k}|\leq \epsilon'$ donc pour $n\geq n_k$, \[
    |\ell _n-\ell |\leq |\ell _n-\ell _{n_k}|+|\ell _{n_k}-\ell |\leq 2\epsilon'=\epsilon
    \] 
    donc $\ell _n \longrightarrow \ell $ 
\item On suppose $a \in  \R$. Soit $\epsilon>0$.  \begin{itemize}
    \item $\exists N \in  \N, \forall  n \geq  N, \forall  x \in  A, \quad  |u_n(x)-u(x)|\leq \epsilon'$
    \item $\exists N_1 \in  \N, \forall  n\geq N_1, \quad  |\ell _n-\ell |\leq \epsilon'$
    \item $M=\max(N, N_1). \qquad  \exists \delta>0, \forall  x \in  ]a-\delta, a+\delta[\cap A, |u_M(x)-\ell _M|\leq \epsilon'$.
\end{itemize}
Donc $|u(x)-\ell |\leq |u(x)-u_M(x)|+|u_M(x)-\ell _M| + |\ell _M-\ell |\leq 3\epsilon'=\epsilon$
\end{enumerate}
\end{proof}

\begin{rem}
    $a$ est un point adhérent  si et seulement si $ \exists  (a_n) \in  A^{\N}, a_n \longrightarrow a$
\end{rem}

\begin{cor}
    \Hyp $(u_n)$ une suite de fonctions continues sur  $A$ qui converge uniformément sur  $A$ vers  $u$
    \Conc  $u$ est continue sur  $A$
\end{cor}

\begin{proof}
    $a \in  A$ donc $a$ est un point adhérent de  $A$,  $u_n \xrightarrow[A]{\text{CVU}}u$ et $\forall  n \in  \N$, \[
        \lim_{\substack{x \to  a\\x \in  A}}u_n(x)=u_n(a) \quad (=\ell _n)
    \] 
    et TDL: \[
        \lim_{\substack{x \to  a\\x \in  A}}u(x)=\lim_{n \to  +\infty}u_n(x)=u(a)
    \] 
\end{proof}

\section{Convergence uniforme, dérivation, intégration}

\begin{thm}
    \Hyp $u_n:I\to K$, $u:I\to K$ et $u_n \xrightarrow[I]{\text{CVUTS}}u$, $a \in  I$, $u_n$ et  $u$ sont $\CPM(I, \K)$. On note $U_n:x \in  I \longmapsto \displaystyle\int_a^xu_n$ et $U:x \in  I \longmapsto \displaystyle \int_a^x u$
    \Conc $U_n \xrightarrow[I]{\text{CVUTS}} U$ et en particulier $ \forall \alpha, \beta \in  I, \displaystyle \int_{\alpha}^\beta u_n \xrightarrow[n\to+\infty]{}\int_\alpha^\beta u$
\end{thm}

\begin{proof}
    $[\alpha, \beta] \subset I, J$ plus petit segment qui contient  $a, \alpha, \beta$
\end{proof}
