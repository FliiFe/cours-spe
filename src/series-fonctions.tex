\ifsolo
    ~

    \vspace{1cm}

    \begin{center}
        \textbf{\LARGE Suites et Séries de fonctions} \\[1em]
    \end{center}
    \tableofcontents
\else
    \chapter{Suites et Séries de fonctions}

    \minitoc
\fi
\thispagestyle{empty}

Dans tout le chapitre, $I$ est un intervalle non trivial, $A$ est une partie non vide de $\mathbb R$ et $K=\mathbb R$ ou $\mathbb C$. Plus tard, on pourra remplacer $A$ par une partie d'un e.v.n (de dimension finie généralement) et $K$ un e.v.n.

\section{Différents types de convergence}

\begin{dfn}
    On note $(f_n)_n$ une suite de fonctions de $I$ dans $K$. \begin{enumerate}
        \item On dit que $(f_n)_n$ \emph{converge simplement}\index{convergence simple} (CVS) s'il existe $f:I\to K$ telle que $\forall x\in I, f_n(x)\xrightarrow[n\to+\infty]{} f(x)$
        \item On dit que $(f_n)_n$ \emph{converge uniformément}\index{convergence uniforme} (CVU) s'il existe $f:I\to K$ telle que \begin{itemize}
            \item $(f_n-g)$ est bornée à partir d'un certain rang
            \item $\sup_I|f_n-g|\xrightarrow[n\to+\infty]{}0$
        \end{itemize}
    \item On dit que $(f_n)$ converge uniformément sur tout segment (CVUTS) si pour tout segment $J\subset I$, on a $f_n\xrightarrow[J]{\mathrm{CVU}}f$
    \end{enumerate}
    Dans chacun des cas, on peut remplacer $I$ par $A$.
\end{dfn}

\begin{rem}
    \begin{itemize}
        \item CVU $\implies$ CVUTS $\implies$ CVS
        \item CVS $\centernot\implies$ CVUTS $\centernot \implies$ CVU
    \end{itemize}
\end{rem}

\begin{rem}
    La limite uniforme d'une suite uniformément convergente est identique à la limite simple de cette suite. En particulier, on a unicité de la limite pour les trois convergences.
\end{rem}

\section{Opérations simples sur les convergences}

\begin{prop}
    \Hyp $(u_n)_n,(v_n)_n$ sont des suites de fonctions de $I$ dans $K$, $u,v:I\to K$, $\lambda\in K$.
    \Conc Si $u_n\xrightarrow{}u$, $v_n\xrightarrow v$ avec de la convergence simple (resp. CVUTS, CVU) alors \[
        (\lambda u_n+v_n)\xrightarrow{}\lambda u+v
    \]
    avec la convergence simple (resp. CVUTS, CVU).
\end{prop}

\begin{proof}
    Évident.
\end{proof}


