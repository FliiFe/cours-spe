\ifsolo
    ~

    \vspace{1cm}

    \begin{center}
        \textbf{\LARGE Suites et Séries de fonctions} \\[1em]
    \end{center}
    \tableofcontents
\else
    \chapter{Suites et Séries de fonctions}

    \minitoc
\fi
\thispagestyle{empty}

Dans tout le chapitre, $I$ est un intervalle non trivial, $A$ est une partie non vide de $\R$ et $K=\R$ ou $\C$. Plus tard, on pourra remplacer $A$ par une partie d'un e.v.n (de dimension finie généralement) et $K$ un e.v.n.

\section{Différents types de convergence}

\begin{dfn}
    On note $(f_n)_n$ une suite de fonctions de $I$ dans $K$. \begin{enumerate}
        \item On dit que $(f_n)_n$ \emph{converge simplement}\index{convergence simple} (CVS) s'il existe $f:I\to K$ telle que $\forall x\in I, f_n(x)\xrightarrow[n\to+\infty]{} f(x)$
        \item On dit que $(f_n)_n$ \emph{converge uniformément}\index{convergence uniforme} (CVU) s'il existe $f:I\to K$ telle que \begin{itemize}
            \item $(f_n-g)$ est bornée à partir d'un certain rang
            \item $\sup_I|f_n-g|\xrightarrow[n\to+\infty]{}0$
        \end{itemize}
    \item On dit que $(f_n)$ converge uniformément sur tout segment (CVUTS) si pour tout segment $J\subset I$, on a $f_n\xrightarrow[J]{\mathrm{CVU}}f$
    \end{enumerate}
    Dans chacun des cas, on peut remplacer $I$ par $A$.
\end{dfn}

\begin{rem}
    \begin{itemize}
        \item CVU $\implies$ CVUTS $\implies$ CVS
        \item CVS $\centernot\implies$ CVUTS $\centernot \implies$ CVU
    \end{itemize}
\end{rem}

\begin{rem}
    La limite uniforme d'une suite uniformément convergente est identique à la limite simple de cette suite. En particulier, on a unicité de la limite pour les trois convergences.
\end{rem}

\section{Opérations simples sur les convergences}

\begin{prop}
    \Hyp $(u_n)_n,(v_n)_n$ sont des suites de fonctions de $I$ dans $K$, $u,v:I\to K$, $\lambda\in K$.
    \Conc Si $u_n\xrightarrow{}u$, $v_n\xrightarrow v$ avec de la convergence simple (resp. CVUTS, CVU) alors \[
        (\lambda u_n+v_n)\xrightarrow{}\lambda u+v
    \]
    avec la convergence simple (resp. CVUTS, CVU).
\end{prop}

\begin{proof}
    Évident.
\end{proof}

\begin{rem}
Pour la convergence simple, on a $u_nv_n \longrightarrow uv$, mais pas pour la  CVU ou la CVUTS. Par exemple, \[
\begin{matrix}
    u:& \R_+ & \longrightarrow &\R  \\
    & x & \longmapsto & e^x+\frac{1}{n}
\end{matrix}
\] 
On a \[
    u_n \xrightarrow[\R_+]{\text{CVU}} u
\] 
On a aussi  \[
    u_n^2 \xrightarrow[\R_+]{\text{CVS}} u^2
\]
donc si $u_n^2$ converge uniformément, \[
    u_n^2(x)-u^2(x) = \frac{2e^x}{n}+\frac{1}{n^2} \xrightarrow[n\to+\infty]{}0
\] 
ce qui est absurde
\end{rem}

\begin{prop}[Propriétées héritées par CVS]
    \Hyp $u_n: I\to \K$, $u_n \xrightarrow[]{\text{CVS}} u$ sur $I$
    \Conc La monotonie, la convexité, le caractère $k$-lipschitzien, la positivité sont hérités.
\end{prop}

\begin{proof}
On écrit les inégalités et on passe à la limite.
\end{proof}

\section{Propriétés héritées par CVU}

\begin{thm}[Double limite\index{double limite (théorème)}]
    \Hyp $u_n \xrightarrow[]{\text{CVU}} u$ sur $A$,  $a$ point adhérent de  $A$,  $u_n(x) \xrightarrow[x\to a]{}\ell _n \in  \K$
    \begin{concenum}
    \item $(\ell _n)$ converge vers $\ell  \in \K$
    \item La fonction $u$ a pour limite  $\ell $ en $a$, i.e.  \[
            \lim_{\substack{x \to  a\\x \in  A}} \lim_{x \to  +\infty} u_n(x)=\lim_{n \to  +\infty}\lim_{\substack{x \to  a\\ x \in  A}} u_n(x)
    \] 
    \end{concenum}
\end{thm}

\begin{proof}~
\begin{enumerate}
    \item (admis) On va construire $(n_k)$ strictement croissante telle que  $ \forall  n\geq n_k, |\ell _n-\ell _{n_k}| \leq  \dfrac{1}{2^k}$. \begin{itemize}
        \item $\forall \epsilon>0, \exists  N \in  \N, \forall  n\geq N, \forall  x \in  A, |u_n(x)-u(x)|\leq \epsilon'$
        \item $\forall  n, p\geq N, |u_n(x)-u_p(x)|\leq |u_n(x)-u(x)|+|u(x)-u_p(x)|\leq 2\epsilon'=\epsilon$ et $x \longrightarrow a$ donne  $|\ell _n-\ell _p|\leq \epsilon$
        \item Pour $\epsilon=1$, il existe  $n_0 \in  \N$ tel que $\forall n\geq n_0, |\ell _n-\ell _{n_0}|\leq 1$
        \item Pour $k \in  \N$, si on suppose $n_0<\cdots <n_k$ construits, pour $\epsilon=\frac{1}{2^{k+1}}$, il existe $N \in  \N$ tel que $\forall  p \geq N, |\ell  _n-\ell _p|\leq \epsilon$. On prend $n_{k+1}=\max(N, n_0,\cdots , n_k+1)$.
    \end{itemize}
    Puis $|\ell _{n_{k+1}}-\ell _{n_k}|\leq \frac{1}{2^k}$, c'est le terme général d'une série convergente donc $(\ell _{n_k})$ converge.

     Soit $\epsilon>0$. Il existe $K$ tel que  $k\geq K$, $\frac{1}{2^k}\leq \epsilon'$ et $|\ell  -\ell _{n_k}|\leq \epsilon'$ donc pour $n\geq n_k$, \[
    |\ell _n-\ell |\leq |\ell _n-\ell _{n_k}|+|\ell _{n_k}-\ell |\leq 2\epsilon'=\epsilon
    \] 
    donc $\ell _n \longrightarrow \ell $ 
\item On suppose $a \in  \R$. Soit $\epsilon>0$.  \begin{itemize}
    \item $\exists N \in  \N, \forall  n \geq  N, \forall  x \in  A, \quad  |u_n(x)-u(x)|\leq \epsilon'$
    \item $\exists N_1 \in  \N, \forall  n\geq N_1, \quad  |\ell _n-\ell |\leq \epsilon'$
    \item $M=\max(N, N_1). \qquad  \exists \delta>0, \forall  x \in  ]a-\delta, a+\delta[\cap A, |u_M(x)-\ell _M|\leq \epsilon'$.
\end{itemize}
Donc $|u(x)-\ell |\leq |u(x)-u_M(x)|+|u_M(x)-\ell _M| + |\ell _M-\ell |\leq 3\epsilon'=\epsilon$
\end{enumerate}
\end{proof}

\begin{rem}
    $a$ est un point adhérent  si et seulement si $ \exists  (a_n) \in  A^{\N}, a_n \longrightarrow a$
\end{rem}

\begin{cor}
    \Hyp $(u_n)$ une suite de fonctions continues sur  $A$ qui converge uniformément sur  $A$ vers  $u$
    \Conc  $u$ est continue sur  $A$
\end{cor}

\begin{proof}
    $a \in  A$ donc $a$ est un point adhérent de  $A$,  $u_n \xrightarrow[A]{\text{CVU}}u$ et $\forall  n \in  \N$, \[
        \lim_{\substack{x \to  a\\x \in  A}}u_n(x)=u_n(a) \quad (=\ell _n)
    \] 
    et TDL: \[
        \lim_{\substack{x \to  a\\x \in  A}}u(x)=\lim_{n \to  +\infty}u_n(x)=u(a)
    \] 
\end{proof}

\section{Convergence uniforme, dérivation, intégration}

\begin{thm}
    \Hyp $u_n:I\to K$, $u:I\to K$ et $u_n \xrightarrow[I]{\text{CVUTS}}u$, $a \in  I$, $u_n$ et  $u$ sont $\CPM(I, \K)$. On note $U_n:x \in  I \longmapsto \displaystyle\int_a^xu_n$ et $U:x \in  I \longmapsto \displaystyle \int_a^x u$
    \Conc $U_n \xrightarrow[I]{\text{CVUTS}} U$ et en particulier $ \forall \alpha, \beta \in  I, \displaystyle \int_{\alpha}^\beta u_n \xrightarrow[n\to+\infty]{}\int_\alpha^\beta u$
\end{thm}

\begin{proof}
    $[\alpha, \beta] \subset I, J$ plus petit segment qui contient  $a, \alpha, \beta$. On a sur $J$, avec $\mu(J)=\int_J\diff x$ (constant): \[
        \|U_n-U\|_{\infty}\leq \mu(J) \|u_n-u\|_\infty \xrightarrow[n\to+\infty]{}0
    \] 
    donc $U_n \xrightarrow[J]{\text{CVU}}U$ d'où la conclusion sur la CVUTS. Puis, \[
        \int_\alpha^\beta u_n=U_n(\beta)-U_n(\alpha) \xrightarrow[n\to+\infty]{}U(\beta)-U(\alpha)
    \] 
\end{proof}

\begin{ex}
    $u_n:x \in  [0, 1] \longmapsto n^2x^n(1-x)$ converge simplement vers la fonction nulle. Puis \[
        \int_0^1 u_n = n^2 \left( \frac{1}{n+1}-\frac{1}{n+2} \right) \xrightarrow[n\to+\infty]{}1\neq \int_0^10=0
    \] 
    donc il n'y a pas CVU.
\end{ex}

\begin{thm}[Dérivation]
    \Hyp $(u_n)$ suite de fonctions $\mathcal  C^1$ sur  $I$, telles que \[
        u_n \xrightarrow[I]{\text{CVS}}u \qquad  \text{ et }\qquad  u_n' \xrightarrow[I]{\text{CVUTS}}v
    \] 
    \begin{concenum}
    \item  $u_n \xrightarrow[I]{\text{CVUTS}}u$
    \item $v$ est continue et  $u'=v$
    \end{concenum}
\end{thm}

\begin{proof}
    On se donne  $a \in  I$ fixé, on écrit \[
        u_n(x)=u_n(a)+\int_a^xu_n'
    \] 
    et $n\to +\infty$ donne \[
        u(x)=u(a)+\int_a^xv
    \] 
    donc $u$ est  $\mathcal C^1$ et $u'=v$. Puis,  \[
        \int_a^xu_n \xrightarrow[I]{\text{CVUTS}}\int_a^x u'
    \] 
    donc \[
        u_n \xrightarrow[I]{\text{CVUTS}}u
    \] 
\end{proof}

\begin{thm}[Dérivation itérée]
    \Hyp \index{derivation iteree suite@dérivation itérée (suite de fonctions)} $(u_n)$ fonctions  $\mathcal  C^p$, \[
        \forall  k \in  \llbracket 0, p-1 \rrbracket , \qquad  u_n^{(i)}\xrightarrow[I]{\text{CVS}}\varphi_i
    \] 
    et  \[
        u_n^{(p)}\xrightarrow[I]{\text{CVUTS}}\varphi_p
    \] 
    \begin{concenum}
    \item Il y a CVUTS à la place de la CVS pour toutes les dérivées d'ordre  $\leq p$
    \item $\varphi_0$ est  $\mathcal  C^p$ et $\varphi_0^{(k)}=\varphi_k$
    \end{concenum}
\end{thm}

\begin{proof}
    Admis. (Récurrence)
\end{proof}

\begin{thm}[Convergence dominée\index{convergence dominée (suites)}]
    \Hyp $(u_n), u$ des fonctions  $\CPM(I, \K)$, $u_n \xrightarrow[I]{\text{CVS}}u$\\ $ \exists \varphi \in  \mathcal  L^1(I, \R_+), \quad  \forall  x \in  I, \forall  n \in  \N, \quad  |u_n(x)|\leq \varphi(x)$
    \begin{concenum}
    \item $u_n$ et $u$ intégrable sur $I$
    \item  $\displaystyle \lim_{n\to +\infty}\int_Iu_n=\int_Iu$
    \end{concenum}
\end{thm}

\begin{proof}
Admis
\end{proof}

\begin{rem}
On note \[
\begin{matrix}
    f_n:& \R_+ & \longrightarrow &\R  \\
    & x & \longmapsto & \dfrac{x^ne^{-x}}{n!}
\end{matrix}
\] 
Les $f_n$ sont  $\mathcal  C^1$ et \[
    f_n'(x)=\frac{e^{-x}}{n!}\left( -x^n+nx^{n-1} \right) =\frac{x^{n-1}e^x(n-x)}{n!}
\] 
Le calcul donne \[
    \sup_{\R_+}|f_n| \xrightarrow[n\to+\infty]{}0 \qquad  \text{ donc } \qquad  f_n \xrightarrow[\R_+]{\text{CVU}}0
\]
Les $f_n$ sont intégrables (ce sont des $o(\frac{1}{x^2})$) et \[
    \int_0^{+\infty}f_n= \underbrace{\left[ -\frac{x^ne^{-x}}{n!} \right]_0^{+\infty}}_{=0}+\int_0^{+\infty} \frac{x^{n-1}}{(n-1)!}e^{-x}\diff x=\cdots =\int_0^{+\infty}e^{-x}\diff x=1
\]
Ainsi  \[
    \lim_{n\to +\infty} \int_0^{+\infty} \frac{x^ne^{-x}}{n!}\diff x=1 \neq  \int_0^{+\infty} \left( \lim_{n\to +\infty}f_n(x) \right) \diff x=0
\] 
Le théorème ne s'applique pas sans hypothèse de domination
\end{rem}

\begin{ex}
\[
\Gamma:x\longmapsto \int_{{0}}^{{+\infty}} {e^{-t}t^{x-1}}\diff t
\] 
Cette fonction est bien définie car l'intégrande est intégrable. On note \[
    f_n:t\longmapsto \1_{[0,n]}(t) \left( t-\frac{t}{n} \right) ^nt^{x-1}
\]
Ce sont des fonctions $\CPM$ sur  $\R_+^\star$. On a \[
    f_n \xrightarrow[\R_+^\star]{\text{CVS}}e^{-t}t^{x-1}
\]
Donc $\forall  n \in  \N^\star,$ \[ \forall  t >0, \qquad  0\leq f_n(t)\leq \1_{[0, n]}\exp\left(n \ln \left(1-\frac{t}{n}\right)\right)\leq e^{-t}t^{x-1} \]
et le membre de droite est $\CPM$ intégrable. Par convergence dominée,  \[
    \int_{0}^{+\infty} f_n \xrightarrow[t\to +\infty]{}\int_0^{+\infty}f=\Gamma(x) 
\] 
Puis (calculer) \[
    \int_0^{+\infty}f_n=\frac{n^xn!}{x(x+1)\cdots (x+n)}
\] 
donc \[
    \frac{n^xn!}{x(x+1)\cdots (x+n)} \xrightarrow[n\to+\infty]{}\Gamma(x)
\] 
\end{ex}

\section{Méthodes pour la convergence uniforme}

On suppose que l'on a une suite $(u_n)_n$ de fonctions de  $I$ dans  $\C$ et $u$ de  $I$ dans  $\C$ telle que \[
    u_n \xrightarrow[I]{\text{CVS}}u
\]
Comment établir $u_n \xrightarrow[I]{\text{CVS}}u$ ?
\begin{itemize}
    \item Étude de $|u_n-u|$ (ou $u_n-u$ pour les fonctions réelles)
    \item On trouve $(a_n)$ tel que  $\forall  x \in  I, \quad  |u_n(x)-u(x)|\leq a_n$ et $a_n\longrightarrow 0$
    \item Autre (il faut réfléchir)
\end{itemize}

Comment montrer qu'il n'y a pas CVU ?
\begin{itemize}
    \item Étudier $\sup_I|u_n-u|$ et vérifier que ça ne tend pas vers  $0$
    \item Trouver  $(x_n)$ tel que  $(u_n-u)(x_n)$ ne tend pas vers  $0$
    \item Trouver un segment  $I$ tel que  $\int_I u_n$ ne tend pas vers  $\int_Iu$
    \item Régularité de la limite simple
    \item Autre (réfléchir)
\end{itemize}

\subsection{Un premier exemple}

\[
    f_n(x)= \begin{cases}
        \dfrac{\sin(n^2x)^2}{n\sin x} &\text{ si } x \in  ]0, \frac{\pi}{2}]\\
        0 &\text{si }x=0
    \end{cases}
\] 
On a \[
    f_n \xrightarrow[]{\text{CVS}}0
\] 
mais \[
    f_n \left( \frac{1}{n} \right) \underset{}\sim \sin(n)^2 \xnrightarrow{}0
\] 
donc il n'y a pas CVU.

\subsection{Un deuxième exemple}

\[
    \begin{array}{rrcl}
    \varphi_n:& \left[ 0, \frac{\pi}{2} \right] & \longrightarrow & \R \\
    & x & \longmapsto & n\cos^n x\sin x
\end{array}
\] 
La suite $(\varphi_n)$ converge simplement vers la fonction nulle, or  \[
    \int_{0}^{\frac{\pi }{2}} \varphi_n=\frac{n}{n+1} \xrightarrow[n\to+\infty]{}1\neq 0
\] 
Donc il n'y a pas CVU.

\subsection{Approximation de l'exponentielle}

On va montrer que pour tout $R>0$,  \[
    \left( 1+\frac{z}{n} \right) ^n \xrightarrow[\displaystyle\mathcal  D_f(0, R)]{\text{CVU}}e^z
\] 
On remarque d'abord (Taylor) \[
    \forall  z \in  \mathcal  D_f(0, R), \quad  \left| e^z- \sum_{k=0}^{n} \frac{z^k}{k!} \right| = \left| z^{n+1}\int_0^1 \frac{(1-t)^n}{n!}e^{tz}\diff t \right| \leq \sup_{|z|\leq R} \left|e^z\right| \frac{R^{n+1}}{(n+1)!}=a_n \xrightarrow[n\to+\infty]{}0
\] 
donc \[
    P_n(z)\defeq \sum_{k=0}^{n} \frac{z^k}{k!} \xrightarrow[\mathcal  D_f(0, R)]{\text{CVU}} e^z.
\]
Puis, \[
    P_n(z)-\left( 1-\frac{z}{n} \right) ^n= \sum_{k=0}^{n} \underbrace{\left( \frac{1}{k!}-\frac{n(n-1)\cdots (n-k+1)}{k!n^k} \right) }_{\geq 0}z^k
\] 
donc \[
    \left| P_n(z)- \left( 1+\frac{z}{n} \right) ^n \right| \leq P_n(R)- \left( 1+\frac{R}{n} \right) ^n \xrightarrow[n\to+\infty]{}0
\] 
et finalement \[
    e^z-\left( 1+\frac{z}{n} \right) ^n= \underbrace{e^z-P_n(z)}_{\text{CVU vers } 0}+\underbrace{P_n(z)-\left( 1+\frac{z}{n} \right) ^n}_{\text{CVU vers }0} \xrightarrow[\mathcal D_f(0, R)]{\text{CVU}}0
\] 

\subsection{Suite récurrente de fonctions}

On se place sur $[0, \frac{\pi}{2}]$ et on pose \[
    u_0=\id \qquad  \qquad  \forall  n\geq 0, \quad  u_{n+1}=\sin(u_n)
\] 
On observe que les $u_n$ sont croissantes donc  \[
    0\leq u_n(x)\leq u_n\left( \frac{\pi }{2} \right) \xrightarrow[n\to+\infty]{}0
\] 
Donc $u_n \xrightarrow[]{\text{CVU}}0$ 

\section{Résultats de densité}

\subsection{Fonctions en escalier}

On a montré que $f \in  \CPM([a, b], \C)$ est limite uniforme de fonctions en escalier sur $[a, b]$ (vu en sup).

\subsection{Fonctions continues affines par morceaux}

On va montrer que toutes les fonctions continues sur un segment sont limites uniforme de fonctions continues affines par morceaux.

On prend $f \in  \mathcal  C([a, b], \C)$. Soit $\epsilon>0$. Il existe  $\varphi$ en escalier qui approche  $f$ à  $\epsilon'=\frac{\epsilon}{3}$ près. On note $\sigma=(a_0, \cdots , a_n)$ une subdivision adaptée à $\varphi$ (on note $|\sigma|$ le pas de cette subdivision, c'est à dire le minimum des écarts entre deux termes consécutifs). On note $\delta>0$ tel que  $\delta < \frac{|\sigma|}{2}$, et on note $c_i$ la valeur de  $\varphi$ sur  $]a_i, a_{i+1}[$. On a: \[
    \forall  i \in  \llbracket 1, n-1 \rrbracket , \quad  |c_i-f(a_{i+1})|\leq \epsilon' \text{ et } |c_{i+1}-f(a_{i+1})|\leq \epsilon' \qquad  \text{ donc } \qquad  |c_{i+1}-c_i| \leq 2\epsilon'
\] 
On pose $g$ l'unique fonction continue affine sur les intervalles du type $[a_i-\delta, a_i+\delta]$ et qui coïncide avec  $\varphi$ ailleurs. On a dans tous les cas $|g(x)-\varphi(x)|\leq 2\epsilon'$ et donc \[
    \forall  x \in  [a, b] , \qquad  |g(x)-f(x)|\leq 3\epsilon'
\] 

\subsection{Théorème de Weierstrass}

\begin{thm}[Weierstrass\index{Weierstrass (théorème de -- )}]
    \Hyp $f \in  \mathcal  C^0([a, b], \R)$
    \Conc $f$ est limite uniforme de polynômes:  \[
        \forall  \epsilon>0, \exists  P \in  \R[X], \forall  x \in  [a, b], \quad  |f(x)-P(x)|\leq \epsilon
    \] 
\end{thm}

\begin{proof}[Démonstration par les polynômes de Bernstein]
    On commence par le cas $[a, b]=[0, 1]$. Pour $f \in  \mathcal  C^0([0, 1], \R)$, on note \[
        B_n(f)(x)\defeq \sum_{k=0}^{n} \binom knx^k(1-x)^{n-k}f\left( \frac{k}{n} \right) 
    \] 
    Le calcul donne \begin{itemize}
        \item $B_n(1)(x)=1$
        \item  $B_n(X)(x)=x$
        \item  $B_n(X^2)(x)=\dfrac{x}{n}(nx-x+1)=x^2+\dfrac{x-x^2}{n}$ donc \[
                B_n(X^2)(x) \xrightarrow[ \detokenize{[0, 1]} ]{\text{CVU}} x^2
        \]
    \end{itemize}
    On en déduit \[
        \sum_{k=0}^{n} \binom nkx^k(1-x)^{n-k}(k-nx)^2=n^2B_n(X^2)(x)-2n^2xB_n(X)(x)+n^2B_n(1)(x)=nx(1-x)
    \] 
    On va montrer que $B_n(f)$ converge uniformément vers  $f$ sur $[0, 1]$. La fonction $f$ est continue sur un segment donc (Heine) elle est uniformément continue. Soit  $\alpha>0$, il existe  $\delta>0$ associé tel que  \[
        |x-y|\leq \delta \implies |f(x)-f(y)|\leq \alpha
    \] 
    On a \begin{align*}
        B_n(f)(x)-f(x)= &\sum_{\substack{k=0\\|\frac{k}{n}-x|\leq \delta}}^{n} \left( f\left( \frac{k}{n} \right)-f(x)  \right) \binom nkx^k(1-x)^{n-k} \\
                        &+\sum_{\substack{k=0\\|\frac{k}{n}-x|> \delta}}^n \left( f\left( \frac{k}{n} \right)-f(x)  \right) \binom nkx^k(1-x)^{n-k}
    \end{align*}
    On note $M$ un majorant de  $|f|$ sur  $[0, 1]$, et en remarquant que $ \left| \frac{k}{n}-x \right|>\delta \iff  (n\delta)^2<(k-nx)^2$, on a \begin{align*}
            (n\delta)^2 \sum_{\substack{k=0\\|\frac{k}{n}-x|> \delta}}^n \binom nk x^k(1-x)^{n-k}&\leq \sum_{\substack{k=0\\|\frac{k}{n}-x|> \delta}}^n(k-nx)^2 \binom nk x^k(1-x)^{n-k} \\
                                                                                                 &\leq  \sum_{k=0}^n(k-nx)^2 \binom nk x^k(1-x)^{n-k} \\
                                                                                                 &=nx(1-x)\leq \frac{n}{4}
        \end{align*}
        d'où finalement \[
            \sum_{\substack{k=0\\|\frac{k}{n}-x|> \delta}}^n \binom nk x^k(1-x)^{n-k}\leq \frac{1}{4n\delta^2}
        \] 
        et \[
            \left| B_n(f)(x)-f(x) \right|\leq \frac{2M}{4n\delta^2}+\alpha<\epsilon
        \] 
        d'où la CVU.

        Pour le cas général, on compose à droite par une application affine (polynômiale).
\end{proof}

\section{Séries de fonctions}

\begin{dfn}
    On note $(f_n)_{n \in  \N}$ une suite de fonctions de $I$ dans  $\K$. \begin{itemize}
        \item On appelle série de fonctions associée à $f=(f_n)$ la suite  $(S_n(f))_n$, où \[
                S_n(f): x \in  I \longmapsto  \sum_{k=0}^nf_k(x)
        \] 
        On la note $\sum f_n$
    \item On dira que  $\sum f_n$ converge simplement ou uniformément si la suite sous-jacente $(S_n(f))$ aussi. 
    \item On dira que $\sum f_n$ converge normalement si  $\|f_n\|_\infty$ est le terme général d'une série convergente.
    \end{itemize}
\end{dfn}

\begin{rem}
    La convergence normale entraine tous les autres types de convergences, c'est une contrainte plus forte. En effet, si $\sum f_n$ converge normalement, alors il y a CVS car $\sum |f_n(x)|$ converge. On note  $S$ la limite simple et on a\[
        \forall  x \in  I, \forall  n \in  N, \qquad  \left| S_n(f)(x) -S(x)\right|= \left| \sum_{k=n+1}^{+\infty} f_k(x) \right|\leq \sum_{k=n+1}^{+\infty} \|f_k\|_\infty \xrightarrow[n\to+\infty]{}0
\] 
d'où la CVU.
\end{rem}

\section{Propriétés héritées}

\begin{prop}
\Hyp $f_n:I\to \K$ continues, $\sum f_n$ CVU
\Conc  $\displaystyle S:x \in  I \longmapsto  \sum_{n=0}^{+\infty}f_n(x)$ est continue
\end{prop}

\begin{proof}
    Les $S_n(f)$ sont continues et il y a CVU.
\end{proof}

\begin{prop}[Dérivation]
    \Hyp $f_n:I\to \K \quad  \in \mathcal  C^1$, $\sum f_n$ CVU et  $\sum f_n'$ CVUTS
    \Conc  $\displaystyle S:x \in  I \longmapsto  \sum_{n=0}^{+\infty}f_n(x)$ est $\mathcal  C^1$ et \[
        S'(x)= \sum_{n=0}^{+\infty} f_n'(x)
    \] 
\end{prop}

\begin{prop}[Double limite\index{double limite (théorème)}]
    \Hyp $f_n : I \to  \K$, $a$ adhérent à  $I$,  $\sum f_n$ CVU,  $ \qquad \displaystyle\lim_{\substack{x\to a\\x \in  I}}f_n(x)=\ell _n \in  \K$
    \begin{concenum}
    \item $\sum \ell _n$ converge
    \item $\displaystyle \lim_{x\to a}\sum_{n=0}^{+\infty}f_n(x)=\sum_{n=0}^{+\infty}\ell _n$
    \end{concenum}
\end{prop}

\begin{prop}[Dérivation itérée\index{derivation iteree serie@dérivation itérée (série de fonctions)}]
    \Hyp $f_n \in  \mathcal  C^p(I, \K)$, $ \quad \sum f_n, \cdots , \sum f_n^{(p-1)}$ CVS et $\sum f_n^{(p)}$ CVUTS
    \Conc Les CVS sont des CVUTS et  \[
        \forall  \ell  \in \llbracket 0, p \rrbracket , \quad  S^{(\ell )}(x)= \sum_{n=0}^{+\infty} f_n^{\ell }(x)
    \] 
\end{prop}

\section{Exemples}

\subsection{\texorpdfstring{Étude de $\sum_{n\geq 0}\frac{1}{n!(x+n)}$}{Premier exemple}}

On a bien CVS sur $\R_+^\star$ et pour $k>0$\[
    \left| \frac{1}{k!(x+k)} \right|\leq \frac{1}{k!} \text{ tg série CV }
\] 
donc il y a CVN et CVU. La somme $S$ est continue et on a  \[
    S(x)=\frac{1}{x}+ \underbrace{\sum_{n\geq 1}f_n}_{\text{borné}} \underset{0^+}\sim \frac{1}{x}
\] 
Puis, $xS(x)$ converge normalement et  \[
    \lim_{x \to  +\infty} \frac{x}{n!(x+n)}=\frac{1}{n!}
\] 
Donc par théorème de la double limite, \[
    xS(x) \xrightarrow[x \to  +\infty]{}e \qquad  \text{donc} \qquad  S(x)\underset{+\infty}\sim \frac{e}{x}
\] 
On étudie $S(x)-\frac{e}{x}$. \[
    S(x)-\frac{e}{x}= \sum_{n=0}^{+\infty} \frac{1}{n!} \frac{-n}{x(x+n)}=-\frac{1}{x} \sum_{n=1}^{+\infty}\frac{1}{(n-1)!(x+n)}=-\frac{1}{x}S(1+x) \underset{+\infty}\sim \frac{e}{x^2}
\] 
D'où finalement \[
    S(x)=\frac{e}{x}-\frac{e}{x^2}+o_{+\infty}\left( \frac{1}{x^2} \right) 
\] 
La fonction $S$ est  $\mathcal  C^1$ sur $\R_+^\star$ car \begin{itemize}
    \item Les $f_n$ sont  $\mathcal C^1$
    \item Il y a CVS
    \item Sur un segment $[a, b]$, $\|f_n'\|_\infty\leq \frac{1}{n!(a+n)^2}$ qui est le terme général d'une série convergente, indépendant de $x$. Donc  $\sum f_n'$ CVN donc CVUTS
\end{itemize}

\subsection{\texorpdfstring{Étude de $\sum (-1)^n\ln(1+1 / n^2x^2)$}{Second exemple}}

On pose  \[
    \forall x \in  \R_+^\star, \qquad  f(x)= \sum_{n\geq 1} (-1)^n \ln \left( 1+\frac{1}{n^2x^2} \right) 
\] 
La série converge absolument donc on a bien la CVS, les $f_n$ sont continus puis il y a CVN sur chaque segment donc  $f$ est continue.

Puis, sur $[1, +\infty[$, on a $x^2\ln(1+\frac{1}{n^2x^2})\leq \frac{1}{n^2}$ donc il y a CVN et  \[
    \forall  n \in  \N^\star, \quad  \lim_{x \to +\infty} (-1)^nx^2\ln\left( 1+\frac{1}{n^2x^2} \right) = \frac{(-1)^n}{n^2}
\] 
donc (double limite): \[
    f(x)\underset{+\infty}\sim \frac{1}{x^2}\sum_{n\geq 1}\frac{(-1)^n}{n^2}
\] 

\begin{rem}
    \[
        \zeta(2)=\frac{\pi ^2}{6}=\underbrace{\sum_{n\geq 1}\frac{1}{(2n)^2}}_{\frac{1}{4}\cdot \frac{\pi ^2}{6}}+\sum_{n\geq 0} \frac{1}{(2n+1)^2}
    \] 
    donc \[
        \sum_{n=1}^{+\infty}\frac{(-1)^n}{n^2}=\frac{1}{4}\cdot \frac{\pi ^2}{6}- \frac{\pi^2}{8}=-\frac{\pi^2}{12}
    \] 
    et $f(x)\underset{+\infty}\sim -  \dfrac{\pi ^2}{12x^2}$
\end{rem}

\section{Intégration terme à terme}

L'intégration terme à terme dans le cas uniforme est identique au cas des suites.

\begin{thm}[Intégration terme à terme uniforme]
    \Hyp $f_n \in \CPM([a, b], \R)$, $\sum f_n$ CVU et  $\sum_{n\geq 0} f_n$  $\CPM$
    \Conc  $\displaystyle \int_a^b\sum f_n=\sum \int_a^bf_n$
\end{thm}

\begin{thm}[Intégration dominée]
    \Hyp $f_n \in  \CPM$ sur $I$, intégrable. $\sum f_n$ CVS,  $\sum_{n\geq 0} f_n$ $\CPM$ et  $\sum \int_I|f_n|$ converge
    \Conc  $\sum_{n\geq 0}f_n$ est intégrable sur  $I$ et  \[
        \int_I\sum_{n\geq 0}f_n=\sum_{n\geq 0}\int_If_n
    \] 
\end{thm}

\begin{ex}
    Calcul de $\displaystyle \int_0^{+\infty}\frac{t}{e^t-1}\diff t$.

Pour $t>0$,  \[
    \frac{t}{e^t-1}=\frac{t}{e^t(1-e^{-t})}=\frac{t}{e^t}\sum_{k=0}^{+\infty}e^{-tk}=\sum_{k=0}^{+\infty}te^{-(k+1)t}
\] 
On note $f_k:t\longmapsto te^{-(k+1)t} \in  \CPM(\R_+^\star)$. On a $\sum f_n$ CVS sur $\R_+^\star$ et $\sum_{n\geq 0}f_n$ est  continue sur  $ \R_+^\star$. Puis, pour $k\geq 0$,  \[
    \int_0^{+\infty}|f_k|= \int_{0}^{+\infty} te^{-(k+1)t}\diff t = \frac{1}{(k+1)^2} \text{ tg série CV }
\] 
donc on peut intégrer terme à terme: \[
    \int_{0}^{+\infty} \frac{t}{e^t-1}\diff t=\sum_{k=0}^{+\infty}\int_0^{+\infty}te^{-(k+1)t}\diff t=  \frac{\pi ^2}{6} 
\] 
\end{ex}

\begin{ex}
Calcul de $\displaystyle I= \int_{0}^{+\infty} \frac{\ln u}{1+u}\diff u $.

\emph{Première méthode.} $u=e^{-t}$ donne $\displaystyle I=- \int_{0}^{+\infty}  \frac{t}{e^t-1}\diff t$ et on a calculé cette intégrale

\emph{Seconde méthode.} $\displaystyle \frac{\ln u}{1+u}=\sum_{k\geq 0}(-1)^ku^k\ln u$ et on applique le théorème d'intégration terme à terme (version simple)
\end{ex}

\section{Méthodes pour la convergence uniforme}

Pour montrer la convergence uniforme, on peut montrer \begin{itemize}
    \item la convergence normale
    \item la convergence simple puis la convergence uniforme du reste vers $0$
    \item autre (il faut réfléchir)
\end{itemize}

\subsection{TSA uniforme}

Si $(u_n)$ est une suite de fonctions de  $I$ dans  $ \R_+$, que $\forall  x \in  I, (u_n(x))$ décroît et $u_n$ converge uniformément vers  $0$, alors  $\sum (-1)^ku_k$ converge simplement sur  $I$ par TSA et  \[
    \forall  x \in  I, \qquad , \left| \sum_{k=n+1}^{+\infty}(-1)^ku_k(x) \right|\leq u_{n+1}(x) \xrightarrow[I]{\text{CVU}}0
\] 
donc $\sum u_n$ CVU. 

\subsection{Étude de \texorpdfstring{$\zeta$}{zeta}}

On note \[
    \forall  s >1, \qquad  \zeta(s)=\sum_{k\geq 1} \frac{1}{k^s}
\] 
et \[
    \forall  x >0, \qquad  \varphi(x)=\sum_{n\geq 1}\frac{(-1)^{n-1}}{n^x}
\] 
On note $u_n: x \longmapsto \dfrac 1n$

\paragraph{Étude de régularité.} On va montrer que $\zeta$ est  $\mathcal  C^\infty$. Soit $n>1$.  \begin{itemize}
    \item Les $u_k$ sont  $\mathcal  C^\infty$
    \item Pour $i \in  \llbracket 0, n-1 \rrbracket ,$ on a \[
            u_k^{(i)}(x)=\frac{(-1)^i\ln^i(k)}{k^x}
    \] 
    donc pour $x>1$,  \[
        u_k^{(i)}(x)=o_{+\infty}\left( \frac{1}{k^{\frac{x+1}{2}}} \right) \text{ tg série CV }
    \] 
    donc $\sum u_k^{(i)}$ CVS
\item Soit $[a, b] \subset ]1, +\infty[$. Alors  \[
        \forall  x \in  [a, b], \forall  k \geq 1, \qquad  |u_k^{(n)}(x)|\leq \frac{\ln^nk}{k^a}=o_{+\infty}\left( \frac{1}{k^{\frac{1+a}2}} \right) 
\] 
d'où la CVUTS.
\end{itemize}
$\zeta$ est donc  $\mathcal  C^n$ pour tout $n$, donc  $\zeta \in \mathcal  C^\infty$ et \[
    \zeta^{(n)}(s)=\sum_{k\geq 1} \frac{(-1)^n\ln^nk}{k^s}
\] 
Des arguments identiques montrent le même résultat pour $\varphi$ sur $\R_+^\star$ (pour montrer la CVUTS on utilise le TSA à partir d'un certain rang). 

\paragraph{Étude en \boldmath $1^+$\unboldmath.}
Pour $x>1$ et  $n\geq 1$, \[
    \frac{1}{(n+1)^x}\leq \int_n^{n+1} \frac{\diff t}{t^x}
\] 
donc \[
    \zeta(x)-1\leq \left[ \frac{t^{x+1}}{1-x} \right]_1^{+\infty}=\frac{1}{x-1}\leq \zeta(x)
\] 
et finalement \[
    \zeta(x)\underset{1^+}\sim \frac{1}{x-1}
\] 
Pour aller plus loin, on note \[
    f_n:x\longmapsto \frac{1}{n^x}-\int_n^{n+1}\frac{\diff t}{t^x}
\] 
On a CVS car \[
    \forall  x \geq 1, \forall  n\geq 0, \quad  0\leq f_n(x)\leq \frac{1}{n^x}-\frac{1}{(n+1)^x}\text{ tg série CV }
\] 
et il y a CVU sur $[1, +\infty[$ car \[
    \forall  x\geq 1, n\geq 1, \quad 0\leq \sum_{k=n+1}^{+\infty}f_k(x)\leq \sum_{k=n+1}^{+\infty}\frac{1}{k^x}-\frac{1}{(k+1)^x}=\frac{1}{(n+1)^2}\leq \frac{1}{n+1} \xrightarrow[n\to+\infty]{}0
\] 
Ainsi, \[
    S:x \longmapsto \sum_{n\geq 1}f_n(x)
\] 
est continue et \[
    \zeta(s)-\frac{1}{s-1} \xrightarrow[s \to  1^+]{}S(1)=\sum_{n\geq 1}\lim _{s \to 1^+}f_n(s)=\sum_{n\geq 1}\left( \frac{1}{n}-(\ln(n+1)-\ln n) \right) =\gamma
\] 
donc finalement \[
    \zeta(s)=\frac{1}{s-1}+\gamma+o(1)
\] 

\paragraph{Étude en \boldmath $+\infty$\unboldmath.} \[
    \forall  s\geq 1, \forall  n\geq 1 , \qquad  n^s\left( \zeta(s)-\sum_{k=1}^n \frac{1}{k^s} \right) =\sum_{k=n+1}^{+\infty} \left( \frac{n}{k} \right) ^s\] 
et \[
    \forall  s\geq 2, \forall  k\geq n+1, \qquad  \left(\frac{n}{k}\right)^s\leq  \left( \frac{n}{k} \right)^2\text{ tg série CV (variable $k$) }
\]
donc $\sum(\frac{n}{k})^s$ converge normalement sur $[2, +\infty[$
Une interversion des limites donne  \[
    \lim _{s\to +\infty} \left( \frac{n}{k} \right) ^s=0 \qquad  \text{ donc } \qquad  \zeta(s)=\sum_{k=1}^n \frac{1}{k^s}+o(\frac{1}{n^s})
\] 

\paragraph{Lien entre \boldmath $\varphi$ et  $\zeta$\unboldmath.}
\[
    \forall  s \geq 1, \qquad  -\varphi(s)+\zeta(s)=\sum_{n\geq 1}\frac{1-(-1)^{n-1}}{k^s}=2\sum_{n\geq 1}\frac{1}{(2n)^s}
\] 
donc \[
    \zeta(s)= \cfrac{1}{1-\cfrac{1}{2^{s-1}}}\;\varphi(s)
\] 
On en déduit ainsi un prolongement de $\zeta$ sur  $]0, 1[$. 

\section{Méthodes pour les équivalents}

\subsection{Un développement asymptotique à trois termes}

On note \[
a_n=\int_0^1 \frac{\diff t}{1+t^n}
\] 
On a \[
    a_n-1=-\frac{1}{n} \int_0^1 \frac{u^{\frac{1}{n}}}{1+u}\diff u
\] 
On note donc \[
    f_n:u\longmapsto \frac{u^{\frac{1}{n}}}{1+u} \in  \mathcal  C([0, 1], \R)
\] 
On a la convergence simple suivante: \[
    f_n(u) \xrightarrow[\detokenize{[0, 1]}]{\text{CVS}} \begin{cases}
        \dfrac{1}{1+u} & \text{ si } u>0 \\ 0 & \text{ sinon }
    \end{cases}
\] 
Puis \[
\forall  n \in  \N, \forall  u \in  [0, 1], \qquad  |f_n(u)|\leq \frac{1}{1+u} \text{ continu intégrable sur } [0, 1]
\] 
donc par convergence dominée, \[
    \int_0^1 \frac{u^{\frac{1}{n}}}{1+u}\diff u \xrightarrow[n\to+\infty]{}\int_0^1 \frac{\diff u}{1+u}=\ln 2
\] 
donc \[
    a_n=1-\frac{\ln 2}{n}+o \left( \frac{1}{n} \right) 
\] 
Puis \[
    a_n-1+\frac{\ln 2}{n}=\frac{1}{n}\int_0^1 \frac{1-u^{\frac{1}{n}}}{1+u}\diff u
\] 
et (Taylor-Lagrange), pour $u \in  [0, 1]$,  \[
    \left| u^{\frac{1}{n}}-1-\frac{1}{n} \ln u \right| \leq  \frac{(\ln u)^2}{n^2} \sup_{[\ln(u)/ n, 0]} |\exp| \leq \frac{(\ln u)^2}{n^2}
\] 
En appliquant la même méthode, \[
    a_n=1- \frac{\ln 2}{n}-\frac{1}{n^2}\int_0^1 \frac{\ln u}{1+u} \diff u + o\left( \frac{1}{n^2} \right) 
\] 

\subsection{Étude d'une norme}

\paragraph{Étude en \boldmath $0^+$\unboldmath.}
On définit \[
    I(\alpha)= \left( \int_0^1f^{\alpha} \right) ^{\frac{1}{\alpha}}=\|f\|_{\alpha}
\] 
avec $f \in  \mathcal  C([0, 1], \R_+^\star)$. Pour $x \in  [0, 1]$ fixé, \[
    f^\alpha(x)=e^{\alpha \ln f(x)}=1+\alpha \ln(f(x))+\epsilon(\alpha, x)
\] 
donc \[
    I(\alpha)=\exp \biggl( \frac{1}{\alpha} \ln \biggl( 1+\alpha \int_0^1 \ln f + \underbrace{\int_0^1 \epsilon(\alpha, x)\diff x}_{o(\alpha) ?} \biggr)  \biggr) 
\]
Taylor-Lagrange donne, si $\alpha$ est assez petit pour avoir $[\alpha \ln (f(x)), 0]\subset [-2, 2]$, \[
    |\epsilon(\alpha, x)| \leq  \frac{(\alpha \ln f(x))^2}{2}e^2
\] 
donc \[
    \int_0^1 \epsilon(\alpha, x)\diff x=o(\alpha)
\] 
et \[
    I(\alpha) \xrightarrow[\alpha\to 0^+]{} \exp \left( \int_0^1\ln f \right) 
\] 

\paragraph{Étude en \boldmath $+\infty$\unboldmath.} On va montrer que $\|f\|_\alpha \xrightarrow[\alpha\to +\infty]{}\|f\|_{\infty}=\max |f|$ (cette valeur existe car $f$ est définie sur un segment). On note $x_0$ un antécédent du maximum de  $|f|$.

On suppose $x_0 \in  ]0, 1[$. Soit $\epsilon>0$,  $\delta>0$ associé tel que  \[
    \forall  x \in  [x_0-\delta, x_0+\delta]\subset [0, 1], \quad  M-\epsilon'\leq f(x)\leq M
\] 
de sorte que \[
    (M-\epsilon')^\alpha 2\delta\leq \int_{x_0-\delta}^{x_0+\delta}f^\alpha \leq \int_0^1f^\alpha\leq M^\alpha
\] 
et \[
    \underbrace{(M-\epsilon')(2\delta)^{\frac{1}{\alpha}}}_{\displaystyle \xrightarrow[\alpha\to +\infty]{}M-\epsilon'} \leq I(\alpha) \leq M
\] 
Donc il existe $A>0$ tel que pour  $\alpha>A$,  $M-2\epsilon'=M-\epsilon\leq I(\alpha)\leq M$ donc \[
    I(\alpha) = \|f\|_\alpha\xrightarrow[\alpha \to +\infty]{}M= \|f\|_\infty
\] 

\subsection{Une série de primitives}

On note $f_0: [a, b] \to  \R$ continue et pour $n\geq 1$, \[
\begin{array}{rrcl}
    f_n:& [a, b] & \longrightarrow & \R \\
    & x & \longmapsto & \displaystyle\int_a^xf_{n-1}
\end{array}
\] 
Que dire de $\sum _{n\geq 1}f_n$ ?

On note $M= \|f\|_\infty=\sup_{[a, b]}\limits|f|$. Par récurrence, on montre que \[
    \forall  x \in  [a, b], \qquad  |f_n(x)| \leq M \frac{(x-a)^n}{n!}
\] 
Puis \[
    \forall  x \in  [a, b], \qquad  |f_n(x)|\leq M \frac{(b-a)^n}{n!} \text{ tg série CV }
\]
donc $\sum f_n$ converge normalement sur  $[a, b]$. On note  $S$ la somme de cette série.

Les  $f_n$ dont  $\mathcal  C^1$ donc $S$ aussi (facile) et  \[
    S'=\sum_{n\geq 1}f_{n-1}=f_0+S
\] 
Avec $S(a)=0$, on déduit \[
    \forall  x \in  [a, b], \qquad  S(x)=e^x\int_a^xe^{-t}f_0(t)\diff t
\] 

\endchapter
